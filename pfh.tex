\section{Point Feature Histograms descriptors}

\subsection{Introduction}

Le Point Feature Histrogram - PFH - estime les propriétés géométries d'un seul point, dans une nuage, d'après ses plus proches voisins. 

*
The PFH is a histogram that collects the pairwise pan, tilt and
yaw angles between every pair of normals on a surface patch
(see Figure 4). In detail, for a pair of 3D points hpi, pj
i, and their estimated surface normals hni, nj i, the set of normal
angular deviations can be estimated as:

$ \alpha = v · nj $ \\
$ \phi = u ·(pj − pi)/d $ \\
$ \theta = arctan(w · nj , u · nj ) $ \\

where u, v,w represent a Darboux frame coordinate system
chosen at pi. Then, the Point Feature Histogram at a patch
of points P = {pi} with i = {1 · · · n} captures all the sets
of $ hα, \phi, \theta_i $ between all pairs of <pi, pj> from P, and bins
the results in a histogram.*


\subsection{Estimation de la normale}

Pour constituer les informations géométriques l'estimation de la normale du point est d'extrême importance. 

Sont calcul est fait de la manière suivant :
1. Un nombre de voisins est choisi 
2. Ces point *servem* à trouver des paramètres de l'équation du plan tangent et, par consequent, la normale correspondent.

Le méthode adopté pour la bibliothéque PCL correspond à prendre un certain nombre de plus proches voisins définis par un seiul. Un petit seil rendre le calcul faux et un grand prend en compte points distants que peuvent ne pas faire partie du plan estimé.

\subsection{Point Feature Histogram - PFH}

\textbf{Avantages}
Invariant à le positionement et rotation *pose*
Robustesse à des differents échélles de densité de points et de bruit.

\textbf{Inconvenients}
Dependence de la qualité de l'estimation de la normale.


L'histograme est évalué à partir des pairs de points à l'intérieur d'une ensemble prédéfinis. D'abord, un répère initial est définis d'après le vecteur distance normalisé et les deux normales. Ensuite, trois ângles, qui correspondent à la transformation angulaire entre les deux normales, et la distance euclidienne entre le deux points sont estimés. Ces quatres valeurs seront considérés comme les features qui réduisent le space initial de douze dimension - coordonnées et normales des dois point - à un space de quatre dimension.

*ADD IMAGE FROM WEBSITE*
*ADD FORMULA FROM WEBSITE*


Le prochain étape c'est de calculer l'histogramme en-soi. Un subdivision du range de valeur de chaque feature, où, les trois ângles peuvent sont normalisés pour rester dans la même intervale trigonométrique est faite et les chaque célulle du histogramme est incrémenté dès qu'une feature pour deux pairs de points tombe dans une interval spécifique. 

\url{http://pointclouds.org/documentation/tutorials/pfh_estimation.php#pfh-estimation}

\subsection{Fast Point Feature Histogram - FPFH}

L'avènement du FPFH viens de la motivation de réduire la complexité de calcule du descripteur PFH $ O(nk^2) $. Pour cela, l'algorithme au lieu de calculer la relation bidirectionnelle entre deux points voisins.

La méthode pour estimer l'histogramme est le suivant:
1. Calcule des angles selon l'algorithme décrit pour le Point Feature Histogram.
2. Reévaluation de chaque point avec la valeur de chaque point voisin pondéré par l'invers d'une mesure de la distance .

Ce procedure résult dans une complexité O(n*k). Pourtant, cet algorithme bien que le gain en vitesse, qui empêche en certains cas application temps réeles, est important, une perte d'information de quelques points voisins est aussi n'est pas négligeble, et un compromis entre ce deux méthodes est nécessaire. Une autre difference intéressant c'est que le FPFH incorpore quelques point externes au rayon de voisinage, mais que sont compris dans un rayon de taille deux fois plus important.

the resultant histogram is simplified by decorrelating the values, that is simply creating d separate feature histograms, one for each feature dimension, and concatenate them together (see figure below).

\subsection{Viewpoint Feature Histogram- VFH}

Le VFH, différemment du rapport entre PFH et FPFH, c'est une extension du descripteur deuxième où la variance de point de vue est prise en compte. De forme sucinte, un histogramme des angles entre les normales de chaque point et la direction principale est concatené au histogramme provenant du SPFH (Simplified PFH) du centroide du \textit{cluster}. Ce résultat permet, au même temps, de reconnaitre l'objet et sa orientation spatialle. 


\url{http://www.pointclouds.org/documentation/tutorials/vfh_recognition.php#vfh-recognition} \\

\url{http://pointclouds.org/documentation/tutorials/fpfh_estimation.php} \\

\url{https://github.com/PointCloudLibrary/pcl/wiki/Overview-and-Comparison-of-Features} \\
