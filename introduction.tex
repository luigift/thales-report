\chapter{Introduction au système}
	Le système de Paillier est basé sur les concepts de résiduosité modulo $n^2$ quand $n$ a exactement 
	deux facteurs premiers $p$ et $q$,
	il utilise les particularités de l'anneau $\Mgrinv{Z}{n^2}$ pour combiner un message 
	dans l'anneau \Mgr{Z}{n} et une valeur aléatoire dans \Mgrinv{Z}{n} de tel sorte que deux messages égales 
	sont chiffrés différemment, Paillier a aussi démontré que le problème de dé\-chif\-frer les donnés peut être 
	réduit en temps polynomial au problème \emph{RSA} \footnote{Trouver la clé privé de l'algorithme \emph{RSA} a partir 
	de la clé publique. c.f.: factoriser la multiplication $n = pq$.} qui est supposé difficile\footnote{Le terme {\it difficile} dans ce contexte désigne 
	une algorithme non calculable en temps polynomial.} par la communauté scientifique. 
	Premièrement on va introduire les fonctions de chiffrement et déchiffrement, nécessaires pour les conclusions sur les 
	groupes cycliques liés, après on les reprendra en prouvant chaque propriété et en montrant les mathématiques sous-jacentes. 



	\section{La fonction de chiffrement}
		Il s'agit d'un chiffrement randomisé, alors la fonction de chiffrement d'un mot $m$ reçoit une valeur aléatoire $r$:% et est un 
%		isomorphisme\footnote{le isomorphisme sera discuté dans le 
%		chapitre \ref{chap:detaille}, page \pageref{chap:detaille}} entre la couple 
%		$(m,r)$ et le chiffré $c$. 
		\begin{definition} On utilisera la notation $\mathcal{E}_g$ pour designer la fonction de chiffrement:
			\begin{equation}
					\mathcal{E}_g : \begin{array}[t]{ccl} \Mgr{Z}{n} \times \Mgrinv{Z}{n} &\rightarrow & \Mgrinv{Z}{n^2}  \\
						(m,r)&\mapsto& g^mr^n\mod{n^2}	 \end{array}
			\end{equation}
		\end{definition}

		Le système repose sur le fait que la clé publique est formée par le couple $(n,g)$ et que le déchiffrement du message
		sans la connaissance de la factorisation de $n$ est difficile.
	\section{La fonction de déchiffrement}
		\label{section:dech}
		Le déchiffrement se déroule en utilisant la fonction $L_{n}$ 
		juste après avoir élevé le mot chiffré à la valeur de la fonction de Carmichael 
		en $n$, $\lambda(n)$: % \footnote{on la définira dans le chapitre \ref{chap:detaille}}:
		\begin{equation}
			L_{m} : \begin{array}[t]{ccl} \{x \in \Mgrinv{Z}{m^2}, x \equiv 1\mod{m}\} &\rightarrow &   \Mgr{Z}{m}  \\
				x&\mapsto& \frac{x-1}{m}
			 \end{array}
		\end{equation}

		\begin{definition} On utilisera la notation $D_n$ pour designer la fonction de dé\-chif\-fre\-ment:
			\begin{equation}
				D_n : \begin{array}[t]{ccl} \Mgrinv{Z}{n^2} &\rightarrow &   \Mgr{Z}{n}  \\
					c&\mapsto& \frac{L_n(c^{\lambda(n)}\mod{n^2})}{L_n(g^{\lambda(n)}\mod{n^2})}\mod{n}
				\end{array}
			\end{equation}
		\end{definition}


%		\begin{remark}
%			Dans le rapport, la notation $L_m$ dénote $L_{n,m}$ et $L$ dénote $L_{n,n}$, où $n$ est le produit des premiers $p$ et $q$ déjà décrits.
%		\end{remark}

		Maintenant qu'on a déjà introduit les fonctions de chiffrement et dé\-chiffre\-ment, on va étudier avec plus de précision chaque aspect 
		ma\-thé\-ma\-ti\-que du système et autres simplifications et extensions proposées par des différents chercheurs.

		Dans la section \ref{sec:mecanisme}, on reprendra la fonction de déchiffrement en montrant pourquoi elle permet de retrouver 
		le message clair. %le raisonnement derrière.

