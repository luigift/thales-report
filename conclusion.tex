\chapter{Conclusion}

Le système de Paillier a été inventé en 1999 et est un sujet d'étude pour divers chercheurs, par contre
il est un système assez neuf si comparé avec le bien connu RSA, inventé en 1977.
Leur jeunesse reliée à l'homomorphisme du sys\-tè\-me donne l'espoir qu'il peut devenir 
un des principaux blocs des systèmes complètement homomorphes des %qui seront inventées dans les
pro\-chai\-nes années.

%O sistêma de paillier surgiu em 1999 e foi objeto de estudo para diversos pesquisadores, no 
%entanto é um sistêma relativamente (assez) novo, se comparado com o sistêma RSA, descrito em 1977. 
%Isto aliado ao fato de que o sistêma é homomorfico, nos leva a crer que ele pode vir a formar um
%dos blocos principais dos possiveis sistêmas homomorficos que podem vir a existir. (wow! rolou umas dorgas ai)

La recherche de chiffrements homomorphes est encouragée par les possibles applications en banque, le nuage, vote électronique et
tout calcul déporté. L'apparition d'un système homomorphe performant est capable de ré\-vo\-lu\-tion\-ner 
les applications en ligne. La sécurité d'un tel système est cruciale et des études orientées vers les problèmes 
de résiduosité d'ordre supérieur doivent être conduites. 
%homomórfico > provêm boas ferramentas para o diversos campos, bancos, cloud computing, 
%tele-sena, votacao. 

Pour le système de Paillier, 
il n'existe pas d'algorithme connu qui peut, sans la clé privée, dé\-chif\-frer en temps polynomial
un mot chiffré, en outre il n'existe pas d'algorithme polynomial d'évaluation d'égalité entre mots chiffrés.   
Même si la sécurité n'est pas complètement prouvée, ce panorama est assez courant dans le domaine du chiffrement
et ne doit pas être un obstacle pour ne pas utiliser ce genre de système, une grand partie des systèmes de chiffrement 
est basée sur l'hypothèse $P \not= NP$, sans solution depuis 1971 quand Stephen Cook a introduit l'idée. Les résultats 
de \cite{DBLP:conf/eurocrypt/Paillier99} et \cite{Damgard:2001:GSA:648118.746742} montrent qui le système de Paillier 
est moins performante que l'algorithme \verb!RSA!, par contre
un tel système constitue une grande avancée du point de vue de la recherche des systèmes homomorphes. 

%	nao existem algoritimos conhecidos para a quebra do codigo chifrado ou verificacao de 
%igualdade de valores chifrados. por outro lado a segurança nao eh provada. Esta situacao é 
%razoavelmente comum neste ramo e nao deve ser tomada como um impedimento à implantacao do sistema,
%uma gama enorme de sistemas sao baseados na premissa P diff NP, sem solucao desde 1971 quando 
%Stephen Cook introduziu a idéia. Os resultados em [paillier] e [jurik] podemos ver que a 
%eficiencia de paillier é inferior à eficiência do sistêma RSA, no entanto ele é importante do ponto 
%de vista do homomorfismo.

	





