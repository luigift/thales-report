\chapter{Conclusion}

\section{Synthèse} 
La principale contribution de ce projet est liée au traitement de la reconnaissance qui intègre un modèle temporel de transition entre vues. La méthode proposée ici pourrait être mise en place pour n'importe quel système de reconnaissance d'objets à condition qu'il soit mobile et capable de fournir une estimation de son déplacement, et que chaque élément de la base des objets à reconnaitre soit associé à une estimation de son orientation. La reconnaissance d'objets multi-vues augmente la capacité à résoudre des situations ambiguës et gère les problèmes de bruit provenant de la base de donnée (absence de vue, erreurs de segmentation). Elle se montre plus performant quand comparé à son correspondent fixe d'après une première évaluation, ayant un taux de réussit de 92\% lorsqu'un tour complète est fait pour la reconnaissance d'objet, en contraste aux 59\% du système classique. Pour le défi de reconnaître l'orientation spatial le taux diminue mais restant encore assez élévé autour de 75\%.

\section{Discussion} 
Même ayant des résultats intéressants quelques limitations apparaissent lorsqu'on se déplace dans environnement plus exigeants et qu'on veut se déplacer librement. Regarder les objets de trop près, par exemple, coupe une partie de l'image qui résulte dans un mauvais classification de l'objet ce qui restreint le zones de déplacement. De même objets placés à une distance inférieur à 1 mètre entre eux risquent d'être mélanges par le tracking. Le modèle est aussi sensible à la densité de la base de donné, car avoir une base trop discrète résulte en points de vues inexistants qui sont souvent associés à d'autres objets. 

\section{Perspectives} 
 La formulation de la chaîne de Markov cachée est suffisamment général pour incorporer des nuances plus complexes en cas d'occlusion, de changement ou d'association d'objet (humain+chaise). Explorer ce potentiel semble agrandir encore plus la puissance du modèle. Au même temps,la communication entre la reconnaissance, qui suggérait des modèle cinématiques de déplacement, et le module de suivi multi-cible peut faire en sort qu'environnements plus complexes avec des objets mobiles où la position physiques des objets, des fois, n'est pas suffisant pour les déterminer puissent être gérés. Une amélioration un peu plus immédiate serait d'ajouter des features de couleur et de texture pour lever les ambiguités de vues.  Finalement intégrer un algorithme de SLAM pour rendre plus robustes les estimations de position et avoir une meilleure répresentation de l'environnement couplé avec un méthode de planification de trajectoires afin que le robot puisse se déplacer de manière autonome. Par ailleurs, on peut envisager une extension du filtre de Kalman pour des objets en déplacement grâce à des modèles cinématiques suggérés par la reconnaissance.






