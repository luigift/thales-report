\chapter{Conclusion}

\section{Synthèse}  bien
La principale contribution de ce projet est liée au traitement de la reconnaissance qui intègre un modèle temporel de transition entre vues. La méthode proposée ici pourrait être mise en place pour n'importe quel système de reconnaissance d'objets à condition qu'il soit mobile et capable de fournir une estimation de son déplacement, et que chaque élément de la base des objets à reconnaitre soit associé à une estimation de son orientation. La reconnaissance d'objets multi-vues augmente la capacité à résoudre des situations ambiguës et gère les problèmes de bruit provenant de la base de donnée (absence de vue, erreurs de segmentation). en conséquence ...

Le robot n'a pas de restriction de déplacement pour reconnaitre les objets.

\section{Discussion} mauvais

limitations du système

\section{Perspectives} améliorer
Les possibles améliorations seraient d'intégrer un algorithme de SLAM pour rendre plus robustes les estimations de position et avoir une meilleure répresentation de l'environnement. On pourrait ajouter ensuite un méthode de planification de trajectoires afin que le robot puisse se déplacer de manière autonome. Par ailleurs, on peut envisager une extension du filtre de Kalman pour des objets en déplacement grâce à des modèles cinématiques suggérés par la reconnaissance.

\color[rgb]{1,0,0}
{
Quelques améliorations que tu n'as pas citées
\begin{itemize}
\item Dans les HMM, mettre une faible probabilité de changer d'objet (en cas d'occlusions, deux objets à côté ...). Associer à cela la possibilité d'ouvrir une nouvelle track si l'objet change
\item Renforcer le filtre de Kalman avec une observation de similarité liée aux features, pas seulement à la position x,y de l'objet.
\item Ajouter des features de couleur et de texture pour lever les ambiguités de vues
\item Faire une base incrémentale (a moins que tu ne la fasse la semaine prochaine)
\end{itemize}
 }
\color[rgb]{0,0,0}






