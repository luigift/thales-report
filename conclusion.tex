{\color{green}
\chapter{Conclusion}

L'apport de ce projet vient du traitement finale de la reconnaissance. Au remarque qu'il pourrait être mis en place pour n'importe quel système de reconnaissance avec estimation de l'orientation du objet reconnu et un système mobile capable d'informer son déplacement. La reconnaissance d'objet multi-vue augment la capacité de résoudre situations d'ambiguité et gére les problèmes provenants de l'absence de vues base de données et erreurs de segmentation, en conséquence ...
}

\section{ Améliorations }
Les possibles améliorations seraient d'un côté, intégrer un algorithme de SLAM pour robostifier les estimations de position et pour avoir une répresentation de l'environnement pour ajouter ensuite un méthode de planification de trajectoires à fin d'atteindre l'\textbf{autonomie de déplacement du robot}. Par ailleurs, un extension du filtre de Kalman avec des objets en déplassement et des mes modèles cinématiques suggéré par la reconnaissance.




