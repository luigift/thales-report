
% These are all easy differentiation problems

\newproblem{modcomp:groupeznzet}{%
Soit $n\in \mathbb{N}$, et considérons le groupe multiplicatif \Mgrinv{Z}{n} des inversibles modulo n.
\begin{enumerate}%
		\item Montrer que $\forall a \in \Mgrinv{Z}{n}$, $a^{\phi(n)}\equiv 1\mod{n}$, où $\phi$ désigne la fonction indicatrice d'Euler: $\phi(n) = \sharp \{ x,1\leq x \leq n; (n,x) = 1 \}$ (théorème d'Euler). %
		\item Montrer que $\forall a \in \Mgrinv{Z}{n}$, $a^{\lambda(n)}\equiv 1\mod{n}$, où $\lambda$ est définie sur $\mathbb{N}^*$ par: \\
			\begin{equation*}
			\lambda(n) =
				\left\{
					\begin{array}{ll}
						p^{(k-1)}(p-1)& \mbox{si } n=p^k, p\text{ premier}, k \in \mathbb{N}^*\\
						p^{(k-2)}     & \mbox{si } n=2^k, k \in \mathbb{N}^*\\
						ppcm({\lambda(p_i^{e_i}), 1\leq i \leq  k }) & n = \prod_{i=1}^{k}{p_i^{e_i}}
						\end{array}
				\right.
			\end{equation*}
			où $\prod_{i=1}^{k}{p_i^{e_i}}$ est la décomposition de $n$ en facteurs premiers (théorème de Carmichaël).\\
		 \textit{Remarque:}$\lambda(n)$ est le plus petit exposant $m$ tel que $a^m \equiv 1 \mod{n}, \forall a \in \Mgrinv{Z}{n}$.%
		\item En deduire que $w^{n\lambda{(n)}}\equiv 1 \mod{n^2}, \forall w \in \Mgrinv{Z}{n^2}$
	\end{enumerate}}{%
	\begin{enumerate}
		\item On crée la fonction $f$ tel que:
			\begin{equation*}
				f_\alpha : \begin{array}[t]{lcl} \Mgrinv{Z}{n} &\rightarrow &   \Mgrinv{Z}{n}  \\
			                                 x & \mapsto    & \alpha x
         			  \end{array}
			\end{equation*}
			come $f_\alpha$ est une bijection (car $\alpha^{-1} \in \Mgrinv{Z}{n}$), on a que:
			\begin{equation*}
				\begin{array}{rcl}
					\prod_{\Mgrinv{Z}{n}} x                  &\equiv& \prod_{\Mgrinv{Z}{n}} f_{\alpha}(x) \mod{n} \Rightarrow \\
					\Rightarrow \prod_{\Mgrinv{Z}{n}} x      &\equiv& \alpha^{|\Mgrinv{Z}{n}|}\cdot\prod_{\Mgrinv{Z}{n}}x \mod{n} \Rightarrow\\
					\stackrel{(\alpha,n) = 1}{\Rightarrow} 1 &\equiv& \alpha^{|\Mgrinv{Z}{n}|} =  \alpha^{\phi(n)}\mod{n} \\
					\Rightarrow \alpha^{\phi(n)}             &\equiv& 1 \mod{n}\text{ pour }\forall a \in \Mgrinv{Z}{n}.
				\end{array}
			\end{equation*}
		\item \begin{enumerate}[I.]
				\item pour $n = p_i^{e_i}$, $p_i\text{ premier} \neq 2$ \\
					Pour qu'un nombre soit congrue à $1 \mod{p_i^{e_i}}$, il doit avoir la même congruence avec $p_i,p_i^2,p_i^3,..,e_i^{p_i-1}$, et donc, 
					Supossant que l'on peut trouver un nombre $j \leq \phi(n)$ tel que $a^(j) \equiv 1 \mod(n)$, donc 
						
				\begin{equation*}
					\begin{array}{rcl}
						\prod_{\Mgrinv{Z}{n}} x                  &\equiv& \prod_{\Mgrinv{Z}{n}} f_{\alpha}(x) \mod{n} \Rightarrow \\
						\Rightarrow \prod_{\Mgrinv{Z}{n}} x      &\equiv& \alpha^{|\Mgrinv{Z}{n}|}\cdot\prod_{\Mgrinv{Z}{n}}x \mod{n} \Rightarrow\\
						\stackrel{(\alpha,n) = 1}{\Rightarrow} 1 &\equiv& \alpha^{|\Mgrinv{Z}{n}|} =  \alpha^{\phi(n)}\mod{n} \\
						\Rightarrow \alpha^{\phi(n)}             &\equiv& 1 \mod{n}\text{ pour }\forall a \in \Mgrinv{Z}{n}.
					\end{array}
				\end{equation*}
				
			\end{enumerate}
	\end{enumerate}}



\newproblem{modcomp:nresidus}{%
Soit $n\in \mathbb{N}$, et considérons le groupe multiplicatif \Mgrinv{Z}{n} des inversibles modulo n.
\begin{enumerate}%
		\item Montrer que $\forall a \in \Mgrinv{Z}{n}$, $a^{\phi(n)}\equiv 1\mod{n}$, où $\phi$ désigne la fonction indicatrice d'Euler: $\phi(n) = \sharp \{ x,1\leq x \leq n; (n,x) = 1 \}$ (théorème d'Euler). %
		\item Montrer que $\forall a \in \Mgrinv{Z}{n}$, $a^{\lambda(n)}\equiv 1\mod{n}$, où $\lambda$ est définie sur $\mathbb{N}^*$ par: \\
			\begin{equation*}
			\lambda(n) =
				\left\{
					\begin{array}{ll}
						p^{(k-1)}(p-1)& \mbox{si } n=p^k, p\text{ premier}, k \in \mathbb{N}^*\\
						p^{(k-2)}     & \mbox{si } n=2^k, k \in \mathbb{N}^*\\
						ppcm({\lambda(p_i^{e_i}), 1\leq i \leq  k }) & n = \prod_{i=1}^{k}{p_i^{e_i}}
						\end{array}
				\right.
			\end{equation*}
			où $\prod_{i=1}^{k}{p_i^{e_i}}$ est la décomposition de $n$ en facteurs premiers (théorème de Carmichaël).\\
		 \textit{Remarque:}$\lambda(n)$ est le plus petit exposant $m$ tel que $a^m \equiv 1 \mod{n}, \forall a \in \Mgrinv{Z}{n}$.%
		\item En deduire que $w^{n\lambda{(n)}}\equiv 1 \mod{n^2}, \forall w \in \Mgrinv{Z}{n^2}$
	\end{enumerate}}{%
solucao boa dmais}
