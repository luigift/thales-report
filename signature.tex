\chapter{Signature Numérique}

	À partir de tout système de chiffrement à clé publique bijectif, on peut construire un système de signature 
	numérique. %
%	Comme les systèmes habituels à clé publique, 
	La signature dans le sys\-tè\-me Paillier se passe en calculant la valeur hachée $h$ 
	du message $m$ et en ``dé\-chif\-frant'' cette valeur. Pour que ça soit possible 
	la fonction de chiffrement doit être utilisée complètement, c'est-à-dire la fonction de dé\-chif\-fre\-ment
	a une image d'ordre $n\phi(n)$ et la signature doit contenir le ``message déchiffré'' $\benolah{h}_g$ et la valeur 
	aléatoire $r$ utilisée pour le chiffrement pour apporter toute l'information nécessaire à calculer $h$.

	Le premier problème posé est la récupération de $r$. La solution est assez naturelle et utilise le fait que 
	l'entité qui signe possède $\lambda(n)$ et peut retrouver la racine $n$-ième facilement étant donné le résidu.

	\begin{rappel}
		$$\mathcal{E}_g(m,r) = g^mr^n\mod{n^2}$$
		$$L_n(x)  = \frac{x-1}{n}$$
	 	$$D_n(c) = \frac{L_n(c^\lambda\mod{n^2})}{L_n(g^\lambda\mod{n^2})}\mod{n}$$
	\end{rappel}



	\begin{algo}{Signature}
		\begin{itemize}\renewcommand{\labelitemi}{} \renewcommand{\labelitemii}{$\cdot$}
			\item{\bf Entrée:} 
				\begin{itemize}
					\item $m$: le message à être signé.
					\item $\lambda$: la clé privée. 
					\item $(n,g)$: la clé publique.
				\end{itemize}
			\item{\bf Calcul:}
				\begin{enumerate} %			
				\renewcommand{\theenumi}{\arabic{enumi}}
				\renewcommand{\theenumii}{\arabic{enumii}}
				\renewcommand{\theenumiii}{\arabic{enumiii}}
				\renewcommand{\labelenumi}{\theenumi.}
				\renewcommand{\labelenumii}{\theenumi.\theenumii.}
				\renewcommand{\labelenumiii}{\theenumi.\theenumii.\theenumiii.}
				\makeatletter
				\renewcommand{\p@enumii}{\theenumi.}
				\renewcommand{\p@enumiii}{\theenumi.\theenumii.}
				\makeatother

					\item Calculer $h_m$ la valeur hachée du message $m$.
					\item Calculer $\benolah{h_m}_g = D_n(h_m)$.
					\item Trouver $r$:
					\begin{enumerate}
						\item Calculer $R =  h_m\cdot g^{-\benolah{h_m}_g}\mod{n}$.\label{item:calcR}
						\item Calculer $r = R^{(n^{-1}\mod{\lambda(n)})}\mod{n}$.
					\end{enumerate}
					\item Créer la signature $(\benolah{h}_g,r)$.
				\end{enumerate}
			\item{\bf Transmettre:}
				\begin{enumerate}[]
				\item la paire $(\benolah{h}_g,r)$.
				\end{enumerate}

		\end{itemize}
	\end{algo}


	\begin{algo}{Vérification}
		\begin{itemize}\renewcommand{\labelitemi}{} \renewcommand{\labelitemii}{$\cdot$}
			\item{\bf Entrée:} 
				\begin{itemize}
					\item $m$: le message dont la signature doit être vérifié.
					\item $(c,r)$: la signature.
					\item $(n,g)$: la clé publique.
				\end{itemize}
			\item{\bf Calcul:}
				\begin{enumerate} %			
				\renewcommand{\theenumi}{\arabic{enumi}}
				\renewcommand{\theenumii}{\arabic{enumii}}
				\renewcommand{\theenumiii}{\arabic{enumiii}}
				\renewcommand{\labelenumi}{\theenumi.}
				\renewcommand{\labelenumii}{\theenumi.\theenumii.}
				\renewcommand{\labelenumiii}{\theenumi.\theenumii.\theenumiii.}
				\makeatletter
				\renewcommand{\p@enumii}{\theenumi.}
				\renewcommand{\p@enumiii}{\theenumi.\theenumii.}
				\makeatother

				\item Calculer $H = \mathcal{E}_g(c,r) = g^cr^n\mod{n^2}$
				\item Calculer $h_m$ la valeur hachée du message $m$.
				\item Vérifier l'égalité: $H = h_m$.
				\end{enumerate}

		\end{itemize}
	\end{algo}

	Comme dit avant, la validité de ce schéma est la possibilité de trouver $r$ tel que
	$h_m = g^{\benolah{h_m}_g}r^n\mod{n^2}$. Dans la signature, le pas \ref{item:calcR} attribue à $R$ la valeur
	$h_m\cdot g^{-\benolah{h_m}_g}\equiv r^n\mod{n^2} $. Pour trouver $r$ depuis $R \equiv r^n\mod{n^2} $ il 
	faut juste calculer l'inverse de $n$ par rapport à l'ordre de $R$. En effet $\text{ordre$(R)$}|\lambda(n)$ et   
	$\pgcd(\lambda(n),n) = 1$ impliquent:
		$$R^{(n^{-1}\mod{\lambda(n)})} \equiv (r^n)^{(n^{-1}\mod{\lambda(n)})} \equiv r^{k\lambda(n)+1} \equiv r\mod{n^2}, k\in\mathbb{Z} $$

%
%	Le pas \ref{item:calcR} 
%	Comme on sait que il existe $r$ tel que: 
%	$$ h \equiv g^{\benolah{h}_g}r^n\mod{n^2} $$
%	En multipliant $h$ par $g^{-\benolah{h}_g}$, on trouve $r^n$ modulo $n^2$, 
%	comme l'ordre de $r$ divise $\lambda(n)$: $$(r^n)^{(n^{-1}\mod{\lambda(n)})} \equiv r^{k\lambda(n)+1} \equiv r\mod{n^2}, k\in\mathbb{Z} $$

	\begin{remark} La possibilité de trouver $\benolah{h}_g$ et $r$ existe grâce à la connaissance de la
	valeur de $\lambda(n)$. Il n'existe pas de méthode efficace connue pour calculer $\benolah{h}_g$ et $r$ 
	avec la seule connaissance de la clé publique  (c.f. Chapitre \ref{secu}
			à la page \pageref{secu}).
	\end{remark}
%	\begin{definition} On utilisera la notation $\mathcal{E}_g$ pour designer la fonction de chiffrement:
%		\begin{equation}
%			\mathcal{E}_g : \begin{array}[t]{ccl} \mathbb{Z} &\rightarrow & \Mgr{Z}{n} \times \Mgrinv{Z}{n}  \\
%			                           n			 &\mapsto     & (s_1,s_2)	 \end{array}
%		\end{equation}
%		\begin{align}
%			\left \{
%				\begin{array}{c @{\equiv} l l}
%				    s_1 & \frac{L(h(m)^{\lambda}\mod{n^2})}{L(g^{\lambda}\mod{n^2})}&\mod{n} \\
%				    p^i & (h(m)g^{-s_1})^{1/n\mod{\lambda}} &\mod{q} \\
%			\end{array}
%			\right. 
%		\end{align}	
%	\end{definition}
