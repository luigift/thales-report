\chapter{Sécurité}
\label{secu}
Paillier a prouvé que le problème de déchiffrer un message sans la con\-naissance 
de la factorisation du nombre $n$ est réductible au problème \emph{RSA}, qui est 
considéré difficile par la communauté scientifique. 

Sont définies trois classes de problèmes (le tableau rétrospectif \ref{tab:classes} est disponible dans la page \pageref{tab:classes}):

\begin{problem} \textbf{CR[$n$]:}% décider si un entier est un $n$-résidu modulo $n^2$.
	\begin{rm}
		\begin{itemize}	
		\renewcommand{\labelitemi}{}
		\item \textbf{Entrée:} 
			\begin{itemize}
				\item[$\cdot$] $x\in\Mgr{Z}{n^2}$: l'entier à être vérifié. 
			\end{itemize}
		\item \textbf{Question:} décider si un entier est un $n$-résidu modulo $n^2$.
		\item\textbf{Sortie:}
			$vrai$ si $x\in\mathrm{R}_n$ et $faux$ si $x\not\in\mathrm{R}_n$.
		\end{itemize}
	\end{rm}
\end{problem}


\begin{problem} \textbf{Class[$n$,$g$]:} %trouver $\benolah{w}_g$ avec $w \in \Mgr{Z}{n}$.
	\begin{rm}
		\begin{itemize}	
		\renewcommand{\labelitemi}{}
		\item\textbf{Entrée:} 
			\begin{itemize}
				\item[$\cdot$] $w\in\Mgrinv{Z}{n^2}$: le message à être déchiffré.
			\end{itemize}
		\item\textbf{Question:} trouver $\benolah{w}_g$ avec $w \in \Mgrinv{Z}{n^2}$.
		\item\textbf{Sortie:}
			$x \in \Mgr{Z}{n}$ tel que $\exists y \in \Mgrinv{Z}{n}$ avec $\mathcal{E}_g(x,y) = w$.
		\end{itemize}
	\end{rm}
\end{problem}

La question du problème précédent peut être interprété comme déchif\-frer une message $w$ qui à été chiffré sur la base $g\in\mathcal{B}$ sans
avoir la factorisation de $n$ ni la valeur de $\lambda(n)$ (c.f.: seulement avec la clé publique).

\begin{remark}
	Dans la section \ref{sec:autored} il sera introduit la terminologie ``auto ré\-ducti\-bi\-lité'' et il sera prouvé
	dans le lemme \ref{lemma:reduClassg}
	que la difficulté d'un problème de la classe \textit{Class[$n$,$g$]} ne dépend pas de $g$ et alors on peut considérer la classe 
	\textit{Class[$n$]}. 
\end{remark}

\begin{problem} \textbf{Class[$n$]:} %trouver $\benolah{w}_g$ avec $w \in \Mgr{Z}{n}$.
	\begin{rm}
		\begin{itemize}	
		\renewcommand{\labelitemi}{}
		\item\textbf{Entrée:} 
			\begin{itemize}
				\item[$\cdot$] $w\in\Mgrinv{Z}{n^2}$: le message à être déchiffré.
				\item[$\cdot$] $g\in\Mgrinv{Z}{n}$: la base du chiffrement.
			\end{itemize}
		\item\textbf{Question:} trouver $\benolah{w}_g$ avec $w \in \Mgrinv{Z}{n^2}$.
		\item\textbf{Sortie:}
			$x \in \Mgr{Z}{n}$ tel que $\exists y \in \Mgrinv{Z}{n}$ et $\mathcal{E}_g(x,y) = w$.
		\end{itemize}
	\end{rm}
\end{problem}

\begin{problem} \textbf{D-Class[$n$]:}% décider si un entier donné $x$ dans $\Mgr{Z}{n}$ vérifie $w=\mathcal{E}_g(x,y)$ avec $y \in \Mgrinv{Z}{n}$. C'est-à-dire: Décider si $\exists y \in \Mgrinv{Z}{n}$ tel que $w = \mathcal{E}_g(x,y)$.
	\begin{rm}
		\begin{itemize}	
		\renewcommand{\labelitemi}{}
		\item\textbf{Entrée:} 
			\begin{itemize}
				\item[$\cdot$] $x\in\Mgr{Z}{n}$: le message d'origine.
				\item[$\cdot$] $w\in\Mgrinv{Z}{n^2}$: le message à être déchiffré.
				\item[$\cdot$] $g$ d'ordre  multiple de $n$ dans $\Mgrinv{Z}{n^2}$.
			\end{itemize}
		\item\textbf{Question:} Décider si $\exists y \in \Mgrinv{Z}{n}$ tel que $w = \mathcal{E}_g(x,y)$. Autrement dit, décider si $w$ chiffre $x$.
		\item\textbf{Sortie:}
			$vrai$ si il existe un y comme décrit et $faux$ si il n'existe pas.
		\end{itemize}
	\end{rm}
\end{problem}

Dans les preuves on comparera les problèmes énoncés et les problèmes \textit{RSA[$n$,$n$]} et \textit{Fact[$n$]}. Premièrement on les définit:


\begin{problem} \textbf{RSA[$n$,$e$]:}% Trouver le seul $m$ qui vérifie $c = m^e\mod{n}$ avec $n$ le produit de deux premiers de grand taille, $m\in\Mgr{Z}{n}$ et $e$ tel que $\pgcd(e,\phi(n))= 1$.    
\begin{rm}
		\begin{itemize}	
		\renewcommand{\labelitemi}{}
		\item\textbf{Entrée:} 
			\begin{itemize}
%				\item[$\cdot$] $(n,e)$: la clé publique formée par $n$ semi premier et $e\in\Mgrinv{Z}{\phi(n)}$ 
				\item[$\cdot$] $c$: le message à être déchiffré en \verb!RSA!.
			\end{itemize}
		\item\textbf{Question:} Trouver $m\in\Mgr{Z}{n}$ vérifiant $c = m^e\mod{n}$ avec $n$ le produit 
		de deux premiers de grand taille tel que $\pgcd(e,\phi(n))= 1$.    
		\item\textbf{Sortie:}
			$m$ la racine $e$-ième de l'entier $c$ dans $\Mgr{Z}{n}$.
		\end{itemize}
	\end{rm}
\end{problem}


\begin{problem} \textbf{Fact[$n$]:}% Factoriser $n$, la multiplication de deux grands entiers. 
	\begin{rm}
		\begin{itemize}	
		\renewcommand{\labelitemi}{}
		\item\textbf{Entrée:} 
%				Le nombre $n$ suffit pour caractériser le problème.
%			\begin{itemize}
%			\end{itemize}

		\item\textbf{Question:} Trouver un facteur non trivial\footnote{Un facteur non trivial de $n$ est un facteur différent de $\pm1$ et $\pm n$.} de $n$.%Factoriser $n$, la multiplication de deux grands entiers. 

		\item\textbf{Sortie:}
			$f$ un facteur non trivial de $n$ et $n$ dans le cas échéant.
%			$p$ ou $q$ positifs différents de $n$ et $1$ tel que $n = pq$.
		\end{itemize}
	\end{rm}	
\end{problem}

%Sont définis trois classes de problèmes, 
%la classe du problème de décider si un entier est un résidu 
%modulo $n$ carré est appelée \emph{CR[$n$]}, \emph{Class[$n$,$g$]} dénote 
%la classe du problème de trouver $\benolah{w}_g$ avec $w$, $g$ et $n$, et finalement \emph{D-Class[$n$]} pour le problème décisionnel 
%relié à \emph{Class[$n$,$g$]}. Il est énoncé aussi la \emph{DCRA}, l'assomption de difficulté de la classe \emph{CR[$n$]} et la
%\emph{CCRA}, l'assomption de difficulté de la classe \emph{Class[$n$]}. La table \ref{tab:classes} 
%est un résumé des classes qui seront discutés dans ce chapitre.

\begin{table}[!htb]
	\centering
	\caption{Table des classes de problèmes}
	\label{tab:classes}
	\begin{tabular}{|c|c|c|c|c|}
		\hline
		Classe      	& Type\footnotemark[2]	 &Relations              & Question                  & Entrée		     	\\
		\hline
		\hline
		CR[$n$]       	& D			 &                        & $x\in\mathrm{R}_n$?    & $x$	           	\\
		Class[$n$,$g$]  & C			 & $w = g^mr^n [n^2]$     & $\benolah{w}_g$        & $w$	          	\\
		Class[$n$]      & C			 & $w = g^mr^n [n^2]$     & $\benolah{w}_g$        & $w$ et $g$	          	\\
		D-Class[$n$]  	& D			 & $w = g^mr^n [n^2]$     & $x = \benolah{w}_g$?   & $x$, $w$ et $g$	   	\\
		\hline
		RSA[$n$,$e$]    & C			 & $c = m^e    [n]$       & $m$                    & $c$	 		\\
		Fact[$n$]     	& C			 & 			  & $d|n$, $d\not\in \{1,n\}$     & 	 			\\
		QRP[$n$]	& D			 & 			  & $\exists y\in\Mgr{Z}{n}, a\equiv y^2 [n]$? & $a$		\\
		\hline
	\end{tabular}
\end{table}\footnotetext[2]{`D' pour Problème Décisionnel et `C' pour Problème de Calcul.}

%CR[n] 	**	-> Deciding Composite Residuosity 	       (decide whether a number is or not a n-th residue)
%	
%DCRA		-> Decisinal Composite Residuosity Assumption  (intractability assumption)
%
%D-Class[n] **	-> decisional problem associated to Class[n]
%
%Class[n,g] **	-> n-th Residuosity Class Problem of base g    (compute [[w]]g with w,g and n)
%
%RSA[n,n]	-> find the e-th root of a number modulo n
%				C = m^e
%				d = e^(-1) mod phi(n)
%				C^d = m'
%				m' = m.
%				the problem is to find m with C, e and n.
%Fact[n]		-> factor a number n = pq with p and q being big primes.

Une définition non formelle de réduction entre problèmes est donnée par C.~Stein et al.~dans {\em Introduction To
  Algorithms} \cite{stein2001introduction}: 

{\em Un problème $A$ est dit réductible à un problème $B$ si, pour chaque instance $I_a$ du problème $A$, il est possible de trouver 
un nombre $i$ polynomialement borné des instances
du problème $B$ dont les résultats permettent de trouver la solution de $I_a$.}

La définition formelle des mêmes auteurs:

\begin{definition} (Réduction polynomial des problèmes de décision)
Un langage $A$ (pro\-blè\-me dé\-ci\-sion\-nel) est dit réductible en temps polynomial à un langage $B$, noté $A \leq_P B$, si il existe une fonction évaluable en 
temps polynomial $f:\{0,1\}^*\to\{0,1\}^*$ tel que pour tout $x\in\{0,1\}^*$, $x\in A \text{ si et seulement si } f(x) \in B$. Si $A\leq_PB$ et $B\leq_PA$,
les problèmes sont dits équivalents et on utilise la notation $A\equiv_P B$
%
%	Un problème $A$ est dit réductible à un problème $B$ si, pour chaque instance $I_a$ du problème $A$, il est possible de trouver 
%un nombre $i$ polynomialement borné des instances
%du problème $B$ dont les résultats permettent de trouver la solution de $I_a$.
\end{definition}

Par exemple: résoudre une équation de degré $1$ (on dit $2x+1=0$) est réductible a résoudre une équation de degré $2$ ( $(2x+1)^2 = 0$ ).
La transition d'un problème dans un autre sera notée ``transformation'', la transformation est faite par un algorithme qui qui implante une 
telle fonction, ils sont appelés \textbf{algorithme de réduction} et \textbf{fonction de réduction} respectivement.

%Si il existe une fonction qui permet la transformation d'une instance de $A$ dans une instance de $B$ et une fonction 
%qui permet la traduction des résultats, et les deux fonctions sont calculables en temps polynomial, on dit que le problème $A$ est 
%\textbf{réductible à $B$ en temps polynomial}. 

\section{Auto Réductibilité}	
	\label{sec:autored}
	L'auto réductibilité est une propriété des problèmes. Étant donnée une instance d'un problème, si il est possible 
	de la réduire en temps polynomial dans un ensemble d'instances aléatoires du même problème, alors la classe est appelé 
	\textbf{auto réductible}. Cette propriété garantit que le temps pour la résolution d'une instance dite difficile n'est plus distant
	que d'un facteur polynomial du temps de solution d'une instance moyenne et donc le cas moyen est aussi difficile que le pire cas.

	\begin{quotation}
		A random-self reduction maps an arbitrary, worst-case instance $x$ in the domain of $f$ to a set
		of random instances $y_1, \dots, y_k$ \dots Thus the average-case complexity of $f$, where the 
		average is taken with respect to the induced
		distribution on instances $y_i$, is the same, up to polynomial factors, as the worst-case 
		randomized complexity of $f$.

		\hfill \em Joan Feigenbaum et Lance Fortnow\cite{Feigenbaum91onthe}
	\end{quotation}

%	Avant de prouver la réductibilité de \emph{Class[$n$,$g$]} à \emph{RSA[$n$]}, on prouve que 
%	chaque instance du problème peut être transformé dans une autre instance aléatoire en temps polynomial.

	\begin{lemma} \emph{Class[$n$,$g$]} est auto réductible sur $w\in\Mgrinv{Z}{n^2}$.
		\begin{proof}
			On pose $w' = wg^\alpha\beta^n\mod{n^2}$, en faisant le choix de $\alpha$ et $\beta$ 
			uniforme sur l'espace $\Mgr{Z}{n} \times \Mgrinv{Z}{n}$, comme $\mathcal{E}_g$ est une 
			bijection pour $g\in\mathcal{B}$ (c.f. Théorème \ref{theorem:bijective}), $w'$ est distribué uniformément dans
			$\Mgrinv{Z}{n^2}$, finalement $\benolah{w}_g$ est égal à  $\benolah{w'}_g-\alpha$  et alors chaque 
			instance peut être modifiée dans n'importe quel autre instance de façon aléatoire et la 
			solution de la première peut être trouvée à partir de la solution de la deuxième.
		\end{proof}
	\end{lemma}

	\begin{lemma} \emph{Class[$n$,$g$]} est auto réductible sur $g\in\mathcal{B}$: \label{lemma:reduClassg}
				$$\forall g_1, g_2 \in \mathcal{B}, Class[n,g_1] \equiv_P Class[n,g_2].$$
		\begin{proof}
			Du Lemme \ref{lemma:equalbenolah} on a:
				$$\benolah{w}_{g_1} \equiv \benolah{w}_{g_2}\benolah{g_2}_{g_1}\mod{n}$$
				En remplaçant $w$ par $g_1$, on a:
				$$\benolah{g_1}_{g_1} \equiv \benolah{g_1}_{g_2}\benolah{g_2}_{g_1}\mod{n}$$
				Comme $\mathcal{E}_{g_1}(1,1) = g_1$, alors $\benolah{g_1}_{g_1} = 1$ et 
				$\benolah{g_1}_{g_2} \equiv \benolah{g_2}_{g_1}^{-1}\mod{n}$. Ainsi:
				$$\benolah{w}_{g_1} \equiv \benolah{w}_{g_2}\benolah{g_1}_{g_2}^{-1}\mod{n}$$
				Il est alors possible de résoudre le problème \PC{Class}{n,g_1} à partir de la résolution de \PC{Class}{n,g_2}
%				et on peut calculer $\benolah{w}_{g_2}$, le déchiffrement de $w$ sur $g_2$ 
%				avec le calcul de $\benolah{w}_{g_2}$ et $\benolah{g_1}_{g_2}$ respectivement les déchiffrements 
%				de $w$ et $g_1$ sur $g_2$, alors il est possible de réduire le problème 
%				sur $g_1$ en ayant lui résolu sur $g_2$.
		\end{proof}
	\end{lemma}

	Avec ces deux Lemmes on peut conclure que la difficulté pour chaque instance du système de Paillier est équivalente, 
	et si une solution en temps polynomial est trou\-vée pour une instance aléatoire, on peut trouver des solutions 
	pour d'autres instances.


%	Random self-reducibility (RSR) is the rule that a good algorithm for the average case implies a good algorithm for the worst case. RSR is the ability to solve all instances of a problem by solving a large fraction of the instances.

\section{Réductibilité à \emph{RSA[$x,x$]}}
	\begin{theorem} \emph{Class[$n$,$g$]} est réductible à \emph{Fact[$n$]}.
		\label{theorem:rsa}
		\begin{proof}
			Étant donné que la fonction définie en (\ref{equation:def}) permet
			le déchif\-frement avec la valeur de $\lambda(n)$
			(en faisant la puissance modulaire, l'algorithme est polynomial), et cette
			valeur est facilement calculée avec la factorisation de $n$,
			toute instance du problème \emph{Class[$n$,$g$]} peut être 
			réduite à une instance du problème \emph{Fact[n]}.
		\end{proof}
	\end{theorem} 
	\begin{theorem} \emph{Class[$n$,$g$]} est réductible à \PC{RSA}{n,n}.
		\label{theorem:fact}
		\begin{proof}
		%	On suppose l'existence d'un oracle pour la classe \emph{RSA[$n,n$]}, et qu'on a
			Soit $w$ un message chiffré avec $\mathcal{E}_g$. 

			Grâce à la bijection on sait que il existent $x$ et $y$ tels que
			$\mathcal{E}_{n+1}(x,y) = w$, alors on utilise la particularité de 
			la base $n+1$:
			$$w \equiv (1+n)^xy^n \equiv y^n\mod n$$
			et on calcule $y$ (le problème \emph{RSA}) tel que $y^n \equiv w \mod{n}$. On divise $w$ par
			$y^n$ modulo $n^2$ et on trouve: $$R_1=\frac{w}{y^n} \equiv (1+n)^x \equiv 1+nx\mod{n^2}$$

			On fait le même raisonnement tel que $g$ était le mot chiffré ($w = g$ et $y_2^n \equiv g\mod{n}$):
			$$R_2 \equiv \frac{g}{y_2^n} \equiv 1+n\benolah{g}_{n+1}\mod{n^2}$$
			En appliquant la fonction $L$ sur les deux résultats:
				$$\frac{L(R_1)}{L(R_2)} = \frac{\benolah{w}_{n+1}}{\benolah{g}_{n+1}} \equiv \benolah{x}_{g} \mod{n^2} $$ 
			on a le déchiffrement de $w$ sur la base $g$.	
		\end{proof}
	\end{theorem}
	\section{Récapitulatif et conclusion}


	%%% Class[n,g] -> Class[n] -> D-Class[n] equiv CR pareil avec Quadratique equiv Fact
%	\TODO{revisar francês}


	On a vu que le problème \PC{Class}{n,g} est équivalent au problème \PC{Class}{n} et que ils garantissent\footnote{Si une 
		personne trouve une solution polynomiale, alors elle peut déchiffrer les données sans avoir la clé privée.} la sécurité 
	du système de Paillier et maintenant on prouve que \PC{CR}{n} $\equiv_P$ \PC{D-Class}{n}. Cette preuve est très importante 
	du point de vue de la garantie théorique de la sécurité, étant donné que \PC{CR}{n} est une généralisation du problème 
	\PC{QRP}{n} (Quadratic Residuosity Problem) accepté difficile. Ce problème sous-jacent aux système de Paillier appairait 
	donc être une généralisation d'un problème réputé difficile.
	
	%Même si l'équivalence \PC{CR}{n} $\equiv_P$ \PC{QRP}{n} n'ai pas 
	%été prouvé, il parait raisonnable que la équivalence soit vraie et par conséquence: $$\text{probablement: }\PC{QRP}{n} \equiv_P \PC{D-Class}{n},$$ 
	%une situation souhaité.

\begin{problem} \textbf{\PC{QRP}{n}:}% Factoriser $n$, la multiplication de deux grands entiers. 
	\begin{rm}
		\begin{itemize}	
		\renewcommand{\labelitemi}{}
		\item\textbf{Entrée:}
			
			\begin{itemize}
				\item[$\cdot$] $a\in \Mgr{Z}{n}$.
			\end{itemize}

		\item\textbf{Question:} Décider si $a$ est un résidu quadratique modulo $n$, i.e. $\exists y \in \Mgrinv{Z}{n^2}, a \equiv y^2 \mod{n}$.

		\item\textbf{Sortie:}
			$vrai$ si il est un résidu quadratique et $faux$ sinon.
		\end{itemize}
	\end{rm}	
\end{problem}


	\begin{theorem} \emph{CR[$n$]} $\equiv_P$ \emph{D-Class[$n$]}.
		\label{theorem:cr}
		\begin{proof}
		On prouve chaque réductibilité.

			\begin{itemize} 
				\item \emph{CR[$n$]} est réductible à \emph{D-Class[$n$]}:
					
				Soit $CR_x$  l'instance $(x)$ du problème \PC{CR}{n}, on décrit un algorithme polynomial de réduction de $CR_x$
				dans une instance de \PC{D-Class}{n}:
				\begin{enumerate}
					\item \textbf{Transformation de l'instance}:
						Soit $g \in_R \mathcal{B}$, l'instance $(0,x,g)$ de \PC{D-Class}{n} est: ``décider si $\exists y \in \Mgrinv{Z}{n}$ 
						tel que $x \equiv g^0y^n \equiv y^n\mod{n^2}$''.
					\item \textbf{Solution du problème \PC{D-Class}{n}}:
						Retourne $vrai$ si il existe $y$ et $faux$ sinon.
					\item \textbf{Traduction de la solution}:
						La réponse est directe du pas 2. Parce que si il existe $y$, $x \equiv g^0y^n\equiv y^n\mod{n^2}$.
%Le argument est la équivalence entre $g^0y^n$ et $y^n$.
				\end{enumerate}
			

				\item \emph{D-Class[$n$]} est réductible à \emph{CR[$n$]}:
					
				Soit $DC_{x,w,g}$  l'instance $(x,w,g)$ du problème \PC{D-Class}{n}, on décrit un algorithme polynomial de réduction de l'instance $DC_{x,w,g}$
				dans une instance de \PC{CR}{n}:
				\begin{enumerate}
					\item \textbf{Transformation de l'instance}:
						Soit $x' \equiv g^{-x}w \mod{n^2}$, l'instance $(x')$ de \PC{CR}{n} est: ``décider si $\exists y \in \Mgrinv{Z}{n}$
						tel que $x' \equiv y^n \mod{n^2}$''.
%						Soit $g \in_R \mathcal{B}$, l'instance $(0,x,g)$ de \PC{D-Class}{n} est: `` décider si $\exists y \in \Mgrinv{Z}{n}$ 
%						tel que $x \equiv g^0y^n \equiv y^n\mod{n^2}$''.
					\item \textbf{Solution du problème \PC{CR}{n}}:
						Retourne vrai si $x' \in \mathrm{R}_n$ et faux sinon.
%						Retourne $vrai$ si il existe $y$ et $faux$ sinon.
					\item \textbf{Traduction de la solution}:
						La réponse est directe du pas $2$. L'argument est que \PC{CR}{n} est vrai si et seulement si $\exists y \in \Mgrinv{Z}{n}$
						tel que $x'\equiv y^n\mod{n^2}$ $\iff$ $x'\equiv g^{-x}w \equiv y^n\mod{n^2}$ $\iff$ $w\equiv g^xx'\equiv g^xy^n\mod{n^2}$,
						et on obtient la solution de \PC{D-Class}{n} sur l'instance $(x,w,g)$.
%						La réponse est directe du pas $2$. Le argument est que \PC{CR}{n} est vrai si et seulement si $\exists y \in \Mgrinv{Z}{n}$
%						tel que $x'=y^n\mod{n^2}$ $\iff$
%						La réponse est directe du pas 2. Le argument est que $x = \benolah{w}_g$ si et seulement si
%						il existe $r\in\Mgrinv{Z}{n}$ tel que $w \equiv g^xr^n$, et alors $x' \equiv g^xr^ng^{-x}\equiv r^n\mod{n}$. 
%						La réponse est directe du pas 2. Le argument est que $x = \benolah{w}_g$ si et seulement si
%						il existe $r\in\Mgrinv{Z}{n}$ tel que $w \equiv g^xr^n$, et alors $x' \equiv g^xr^ng^{-x}\equiv r^n\mod{n}$. 
				\end{enumerate}

%				\item \emph{CR[$n$]} est réductible à \emph{D-Class[$n$]}:
%
%					La classe \emph{CR[$n$]} désigne le problème de décider si il existe
%					un $y$ tel que $x \equiv y^n\mod{n^2}$, pour réduire à \emph{D-Class[$n$]}, on 
%					fait le choix aléatoire d'un $g\in\mathcal{B}$ et on résoudre le problème 
%					``décider si  $x=\benolah{0}_g$'' qui est bien un problème de la classe \emph{D-Class[$n$]}
%					et peut être interprété comme: ``décider si il existe un $y$ tel que $x\equiv g^0y^n\mod{n^2}$''.
%
%				\item \emph{D-Class[$n$]} est réductible à \emph{CR[$n$]}:
%	
%					De façon analogue, on pose le problème ``décider si il existe un $y$ tel que $wg^{-x}\equiv y^n\mod{n^2}$''
%					qui est clairement un problème de la classe \emph{CR[$n$]} et est solution du problème 
%					``décider si  $x=\benolah{w}_g$''.
			\end{itemize}
			Ainsi les deux classes étant réductibles entre elles, sont équivalentes. 
		\end{proof}
	\end{theorem}

%	\begin{theorem} \emph{CR[$n$]} $\equiv$ \emph{D-Class[$n$]}.
%		\label{theorem:cr}
%		\begin{proof}
%		On prouve chaque réductibilité.
%
%			\begin{itemize} 
%				\item \emph{CR[$n$]} est réductible à \emph{D-Class[$n$]}:
%
%					La classe \emph{CR[$n$]} désigne le problème de décider si il existe
%					un $y$ tel que $x \equiv y^n\mod{n^2}$, pour réduire à \emph{D-Class[$n$]}, on 
%					fait le choix aléatoire d'un $g\in\mathcal{B}$ et on résoudre le problème 
%					``décider si  $x=\benolah{0}_g$'' qui est bien un problème de la classe \emph{D-Class[$n$]}
%					et peut être interprété comme: ``décider si il existe un $y$ tel que $x\equiv g^0y^n\mod{n^2}$''.
%
%				\item \emph{D-Class[$n$]} est réductible à \emph{CR[$n$]}:
%	
%					De façon analogue, on pose le problème ``décider si il existe un $y$ tel que $wg^{-x}\equiv y^n\mod{n^2}$''
%					qui est clairement un problème de la classe \emph{CR[$n$]} et est solution du problème 
%					``décider si  $x=\benolah{w}_g$''.
%			\end{itemize}
%			Ainsi les deux classes sont réductibles entre elles, étant équivalentes. 
%		\end{proof}
%	\end{theorem}

	Depuis les théorèmes  \ref{theorem:rsa}, \ref{theorem:fact} et  \ref{theorem:cr} et le fait que 
	\emph{D-Class[$n$]} est évidemment réductible\footnote{On calcule $\benolah{w}_g$ et compare avec $x$.} a \emph{Class[n]},
	une hiérarchie peut être écrite: %en utilisant le symbole ``$\Leftarrow$'' pour la phrase ``est réductible à'':

	%$$\PC{QRP}{n} \stackrel{?}{\equiv}_P
	$$\emph{CR[n]} \equiv_P \emph{D-Class[n]} \leq_P \emph{Class[n]} \leq_P \emph{RSA[n,n]} \leq_P \emph{Fact[n]} $$
