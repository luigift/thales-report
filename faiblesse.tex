	\section{Une faiblesse du système}
		Au cours de la familiarisation avec le système de Paillier, on a trouvé, par des méthodes expérimentales, 
		des divergences entre la théorie introduite et le comportement réel des groupes formés. 

		Dans la suite, on montrera comment le mauvais choix de $p$ et $q$ peut affaiblir le chiffrement des données. 
		
		%%%%%%%%%%% b^n = a^n
		\begin{lemma}{Soient $b \in \Mgrinv{Z}{n^2}$ et $a \in \Mgrinv{Z}{n}$, si $b \equiv a \mod{n}$, alors $b^n \equiv a^n \mod{n^2}$.} 
			\label{lemma:equivncarre}
			\begin{proof}
			Si $b = a+kn$, $0 \leq k < n$, alors:
			\begin{align}
			%	b &\equiv a+kn \Rightarrow \nonumber\\
			%\Rightarrow	
			b^n &\equiv  \sum_{i = 0}^{n}\binom{n}{i} a^{n-i} (kn)^{i}  \nonumber\\
				    &\equiv  a^n+n\cdot a^{n-1}kn +(kn)^2\cdot A, A\in \mathbb{Z}\nonumber\\
			            &\equiv  a^n+n^2A' , A'\in \mathbb{Z}\nonumber\\
				    &\equiv  a^n \mod{n^2}
			\end{align}
			
			\end{proof}
		\end{lemma}
		%%%%%%%%%%% En = {z^n = 1}
		\begin{definition} On note ${\mathbb E}_n$ le sous ensemble de $\Mgrinv{Z}{n}$ formé par:
%			\begin{equation}
%				\mathbb{E}_n = \left\{z, z \in \Mgrinv{Z}{n}, z^n \equiv 1 \mod{n^2} \right\}
%			\end{equation}
			\begin{equation}
				\mathbb{E}_n = \left\{z \in \Mgrinv{Z}{n}, z^n \equiv 1 \mod{n^2} \right\}
			\end{equation}
		\end{definition}
		
		\begin{lemma} $\mathbb{E}_n$ est un sous groupe cyclique de $\Mgrinv{Z}{n}$.
			\begin{proof}
				Soient $a$ et $b$ tels que $a,b \in \mathbb{E}_n$, grâce au Lemme \ref{lemma:equivncarre} on prouve que leur produit $c \equiv ab\mod{n}$ appartient à $\mathbb{E}_n$ (fermeture):
				\begin{equation}
				c^n \equiv (ab)^n \equiv a^nb^n \equiv 1 \cdot 1 \equiv 1\mod{n^2}.
				\end{equation}
				D'où $c \in \mathbb{E}_n$.
%					\begin{align}
%						c^n &\equiv (ab)^n \mod{n} \nonumber\\
%								&\equiv a^nb^n\mod{n} \nonumber\\
%								&\equiv 1\cdot1 \mod{n} \Rightarrow \nonumber\\
%						\Rightarrow c &\in \mathbb{E}_n
%					\end{align}
%
			L'associativité est héritée de $\Mgrinv{Z}{n}$, le élément $1$ est élément neutre et il reste 
			à prouver que l'inverse dans $\Mgrinv{Z}{n}$ fait partie aussi de $\mathbb{E}_n$:

				Soient $a\in\mathbb{E}_n$ et $d$ l'inverse de $a$ dans $\Mgrinv{Z}{n}$, on prouve que $d \in \mathbb{E}_n$:
				\begin{equation}
					(a\cdot d)^n = (1+nA)^n = 1+n\cdot nA+ (nA)^2C \equiv 1\mod{n^2}, (A,C) \in \mathbb{Z}^2
				\end{equation}
				Comme $(a\cdot d)^n = a^nd^n$ et $a^n \equiv 1\mod{n^2}$:
				\begin{equation}
					d^n = 1\mod{n^2}
				\end{equation}
				D'où $d\in\mathbb{E}_n$
%					\begin{align}
%						a\cdot d        &\equiv 1 \mod{n} \Rightarrow \nonumber\\
%						\Rightarrow ad &= 1 + nA, A\in \mathbb{Z} \Rightarrow\nonumber\\
%						\Rightarrow (ad)^n &= 1 + n\cdot nA + (nA)^2\cdot C, C\in \mathbb{Z} \Rightarrow\nonumber\\
%								&\equiv 1 \mod{n^2} \Rightarrow \nonumber\\
%						\Rightarrow (a)^n(d)^n        &\equiv 1 \mod{n^2} \Rightarrow \nonumber\\
%					 	\Rightarrow	(d)^n        &\equiv 1 \mod{n^2} \Rightarrow  \nonumber\\
%						\Rightarrow d &\in \mathbb{E}_n
%					\end{align}
			\end{proof}
		\end{lemma}
		On introduit $\lambda(n)$.
		\begin{definition}{\bf La fonction de Carmichael\cite{carmichael1914theory}}:
		
		\begin{equation}
				\lambda : \begin{array}[t]{lcl} \mathbb{Z} &\rightarrow &   \mathbb{Z}  \\
					n&\mapsto& \begin{cases}
						\phi(n) &\text{si } n = 1,2\text{ ou } 4;\\
						\frac{\phi(n)}{2} &\text{si } n = 2^k, k>2; \\
						\phi(n) &\text{si } n = p^k, p\text{ premier et } k \in \mathbb{N^*}; \\
						ppcm(\{\lambda(p_i^{e_i})\})& \text{si } n = \prod_{i = 0}^{k} p_i^{e_i}.
					\end{cases}
				 \end{array}
		\end{equation}
		où $\phi$ désigne la fonction indicatrice d'Euler, $p_0 \dots p_k$ sont premiers distincts et $e_i \in \mathbb{N^*}$.
	
%		
%		
%		Soit $p$ un nombre premier impair et $p_i$ des nombres premiers distincts, $k \in \mathbb{N}^+$, 
%			$n = \prod p_i^{e_i}$ la factorisation canonique de n, où $e_i \in \mathbb{Z}$ et $\phi(n)$ la fonction indicatrice de Euler:
%%			\begin{equation}
%%				\lambda : \begin{array}[t]{lcl} \mathbb{Z} &\rightarrow &   \mathbb{Z}  \\
%%					n&\mapsto& \begin{cases}
%%						2^{k-2}      &\text{si }n = 2^k, k >2;\\
%%						(p-1)p^{k-1} &\text{si }n = p^k;\\
%%						ppcm(\{\lambda(p_i^{e_i})\})& sinon.
%%					\end{cases}
%%				 \end{array}
%%			\end{equation}.
%			\begin{equation}
%				\lambda : \begin{array}[t]{lcl} \mathbb{Z} &\rightarrow &   \mathbb{Z}  \\
%					n&\mapsto& \begin{cases}
%						\phi(n) &\text{si } n = 1,2\text{ ou } 4;\\
%						\frac{\phi(n)}{2} &\text{si } n = 2^k, k>2; \\
%						\phi(n) &\text{si } n = p^k; \\
%						ppcm(\{\lambda(p_i^{e_i})\})& sinon.
%					\end{cases}
%				 \end{array}
%			\end{equation}
%
%			
%	%	$\forall a \in \Mgrinv{Z}{n}$, 	$\lambda(n)$ est le plus petite nombre positif tel que $a^\lambda(n) \equiv 1 \mod(n)$}
%
		\end{definition}
		\begin{theorem}{$\lambda(n)$ est le plus petit nombre positif tel que $a^{\lambda(n)} \equiv 1 \mod{n}$, pour tout $a\in \Mgrinv{Z}{n}$.}
%		               { $\forall a \in \Mgrinv{Z}{n}$, $\lambda(n)$ est le plus petit nombre positif tel que $a^{\lambda(n)} \equiv 1 \mod{n}$.}
			\label{lambda:theo}
			\begin{proof}
				On prouve cas par cas:
				\begin{itemize}
					\item Pour le premier cas ($n =$ 1, 2 ou 4):

					        Les éléments des anneaux $\Mgrinv{Z}{n}$ sont $\{1\}$, $\{1\}$ et $\{1,3\}$ respectivement et il 
						est évident que 1, 1 et 2 sont les plus 
						petits nombres qui conviennent. 
					\item $n = 2^k$, avec $k > 2$, on a dans ce cas $\phi(n) = 2^{k-1}$:

						Tout élément $a$ de $\Mgrinv{Z}{_{2^n}}$ est de la forme $4l\pm 1, l \in \mathbb{N}$ ($4l$ et $4l+2$ sont paires),
						on prouve que le théorème est vrai pour $k = 3$, on continue par récurrence et 
						finalement on prouve que $\exists a \in \Mgrinv{Z}{_{2^k}}, a^{2^{k-3}} \not\equiv 1 \mod{n}$:
						\begin{itemize}
							\item $k = 3,~ \lambda(2^3) = \lambda(8) = 2$\\
								$\Mgrinv{Z}{_8} = \{1,3,5,7\} = \{4l\pm 1\}$:
								\begin{align*}
									a^{\lambda(8)}&=(4l\pm 1)^2 = 16l^2\pm8l+1\\
										&\equiv 1 \mod{8}
								\end{align*}
							\item on suppose vrai pour $n = 2^k, k \geq 3$:

								\begin{equation} \label{lambda:l}
									a^{2^{m-2}} = 1+2^mA, A\in\mathbb{Z}
%									\lambda(2^m) = 2^{m-2}
								\end{equation}
							\item pour $n = 2^{k+1}$:

								en passant \ref{lambda:l} au carré:
%								depuis \ref{lambda:l} on sait que $a^{2^{m-2}} = 1+2^{m}A, A\in\mathbb{Z}$, alors:
								\begin{align}
									(a^{2^{m-2}})^2 &= 1+2\cdot 2^{m}A+2^{2m}A^2 \nonumber \\
											&\equiv 1 \mod{2^{m+1}}
								\end{align}
						\end{itemize}


						$2^{k-2}$ est le plus petit, parce que pour $a=5$, $5^{2^{k-3}}\not\equiv 1\mod{2^k}$:
						
						\begin{align*}
							5^{2^{k-3}}&=(1+2^2)^{2^{k-3}} \\ 
								   &= 1+2^{k-1}+A\cdot 2^k{k}, A\in\mathbb{Z} \\
								   &\equiv 1+2^{k-1} \mod{2^{k}}
						\end{align*}



					\item $n = p^k$, $p$ nombre premier impair. Dans ce cas, $\phi(n) = p^{k-1}(p-1)$:

						Les résultats de Gauss en 1798 permettent de montrer qu'il existe des éléments d'ordre $\phi(n)$ modulo $p^k$:
						\begin{quotation}
							If the greatest common divisor of the numbers $t$ and 
							$p^{n-1}\cdot(p-1)$ is $e$, the congruence $x^t\equiv 1\mod p^n$
							will have $e$ different roots.
		
							\hfill \em Carl Friedrich Gauss, DA art. 85\cite{gauss1966disquisitiones}
						\end{quotation}	
						On utilise un raisonnement par l'absurde assez simple pour ce cas. On suppose qu'il n'existent pas 
						de nombres d'ordre $\phi(n)$, dans ce cas tout élément est dans $\mathbb{A}$ ou $\mathbb{B}$:
						$$ \mathbb{A} = \{y, y^{p^{k-1}}\equiv 1 \mod{p^k}\} \text{ et } \mathbb{B} = \{y, y^{p^{k-2}(p-1)}\equiv 1 \mod{p^k}\} $$
						depuis Gauss, l'ordre de $\mathbb{A}$ et $\mathbb{B}$ est $p^{k-1}$ et $p^{k-2}(p-1)$, par contre
						$\sharp (\mathbb{A}) + \sharp (\mathbb{B}) = p^{k-2}(2p-1)$ est moins grand que le nombre 
						d'éléments $\phi(n) = p^{k-1}(p-1)$, alors il y a des éléments qui n'appartient pas à $\mathbb{A} \cup \mathbb{B}$.
%						$\lambda(n) = \phi(n)$, et comme $\Mgrinv{Z}{n}$ est un Groupe Cyclique, % \cite{ Gauss, DA, arts.52–56, 82–89 Disquisitiones arithmeticae } \\ Multiplicative group of integers modulo n
%						il y a toujours un générateur d'ordre $\phi(n)$
				\end{itemize}
			\end{proof}
		\end{theorem}
			%%%%%%%%%%% E != {1} => #En = pgcd(phi,n)
			%%%%%%%%%%% #En = pgcd(phi,n) ###  Si $\mathbb{E}_{pq} \neq \{1\}$, 
		\begin{lemma}{Soient $p$ et $q$ nombres premiers\footnote{Depuis ce lemme, $p$ et $q$ sont réservés pour des nombres premiers}
		, l'ordre de $\mathbb{E}_{pq}$ est $\pgcd( \phi(pq) ,pq)$.}
			\begin{proof}
			Soient $n = pq$, $d$ l'ordre de $\mathbb{E}_n$ et $\lambda(n)$ la fonction de Carmichael en $n$. 
			Par définition si $x \in \mathbb{E}_{pq}$ alors $x^n \equiv 1 \bmod{n^2}$ et $x^n \equiv 1 \bmod{n}$. On a aussi que 
			si $x\in \Mgrinv{Z}{n}$, $x^{\lambda(n)} \equiv 1\bmod{n}$.
%			tout élément de $\mathbb{E}_{pq}$ vaut $1$ quand élevé 
%			à la $n$-ième puissance et tout élément de $\Mgrinv{Z}{n}$ vaut 1 quand élevé a la $\lambda(n)$-ième puissance:
			\begin{align}
			d | n,~ d | \lambda(n) \Rightarrow d | \pgcd(n,\lambda(n))
			\end{align}
			Or $\lambda(n)$ et $\phi(n)$ ont les mêmes facteurs premiers et $n = pq$, par conséquent:  
			$$\pgcd(\lambda(n),n) = \pgcd(\phi(n),n).$$ On a aussi que:
			$$\pgcd(\phi(n),n) \in \{1,p,q,pq\}$$ or $\phi(n) < n$, alors le $\pgcd$ ne peut pas prendre la valeur $pq$. Maintenant, 
			sans perte de généralité, on suppose\footnote{Cette supposition sera adoptée dans le 
				reste du chapitre.} $q < p$, alors: 
				$$ \pgcd(\phi(n),n) = \pgcd(\phi(pq),pq) = \pgcd((p-1)(q-1),pq) $$
				comme les facteurs de $\phi(n)$ sont plus petits que $p$, le $\pgcd$ peut prendre
				seulement les valeurs $1$ et $q$:
				\begin{equation}
					\pgcd(\phi(n),n) \in \{1,q\} \label{pgcd:options}
				\end{equation}
			Étant donné que $d$ divise le $\pgcd$ précédent, $d$ vaut $q$ si $\mathbb{E}_n \neq \{1\}$, 
%(il y a d'autres éléments différents de $1$)
			sinon $\mathbb{E}_n = \{1\}$ et $d$ vaut $1$.
			\end{proof}
		\end{lemma}
		
		\begin{lemma}{$\pgcd(\phi(pq),pq)) = q$ si et seulement si $q|p-1$.}
			\begin{proof}
				Grâce au résultat \ref{pgcd:options}, on a que, soit $q$ divise $\phi(pq)$, soit $pq$ et $\phi(pq)$ n'ont pas de facteur en commun.
				On analyse les deux possibilités:
				\begin{itemize}
					\item  $q\mid\phi(pq)$
						$$ q \mid (p-1)(q-1) \stackrel{q\text{ premier}}{\Rightarrow} q \mid p-1$$
					\item  $q\nmid \phi(pq)$ 
						$$ q \nmid (p-1) (q-1) \Rightarrow q \nmid p-1 $$
				\end{itemize}
			\end{proof}
		\end{lemma}

		On introduit la notation $ord_m(a)$ l'ordre de l'élément $a$ dans $\Mgrinv{Z}{m}$.
		\begin{lemma}{Une condition nécessaire à l'existence d'un élément $a \neq 1, a \in E_n$ est $q \mid p-1$.}
		\label{aneq1:necessaire}
		\begin{proof}
			On suppose l'existence d'un $a \neq 1$ tel que $a^n \equiv 1 \mod{n^2}$, alors $a^n\equiv1\mod{p}$, ou plus précisément, $(a^p)^q \equiv 1\mod{p}$.
			En appliquant le Petit Théorème de Fermat ($a^p \equiv a \mod{p}$):
			\begin{equation*}
				(a^p)^q \equiv a^q \equiv 1\mod{p} \Rightarrow ord_p(a) \mid q
			\end{equation*}
				Étant donné que $q$ est premier et $a \neq 1$:
			\begin{align}
				q \mid p-1 %\\
%				\text{et~} pgcd(\phi(n),n) = q
			\end{align}
%				Alors, si $q\nmid p-1$, on a une contradiction et il n'existe pas un $a \neq 1$ tel que $a \in \mathbb{E}_n$.
				
		\end{proof}
		\end{lemma}
		%%%%%%%%%%% Z/nZ* admet un seule sous-groupe cyclique d'ordre q
		%%pagina 43.. remarque.
		On utilisera la notation $C_m$ pour designer le groupe cyclique d'ordre $m$ dans la suite.

		\begin{lemma}{Quand  $p \equiv 1 \mod{q}$, il existe un seul groupe cyclique d'ordre $q$ dans $\Mgrinv{Z}{pq}$.}
			\begin{proof}
				Depuis les résultats de  \cite{carmichael1914theory,Wei:wolfram:MMG} (voir bibliographie), on sait qu'il existe un sous 
				groupe cyclique d'ordre $\lambda(n)$ dans $\Mgrinv{Z}{n}$.
				
				On prouve que le seul sous groupe cyclique d'ordre $q$ est le groupe contenu dans $C_{\lambda(n)}$:
				\begin{enumerate}
					\item $C_{\lambda(n)}$ contient un sous groupe cyclique d'ordre $q$:\\
						Comme $q$ est diviseur de $\lambda(n)$ et $C_{\lambda(n)}$ est un groupe cyclique, il y a un seul 
						sous-groupe cyclique de $C_{\lambda(n)}$ d'ordre $q$.
					\item Dans $\Mgrinv{Z}{n}\setminus{C_{\lambda(n)}}$ il n'existe pas de élément d'ordre $q$:\\
%						Dans $\Mgrinv{Z}{pq}$ on a $\phi(pq)$ éléments et un groupe cyclique d'ordre 
%						$\lambda$ (comme vu avant), alors 
						Les ordres possibles pour les groupes sont diviseurs de:
%						\begin{itemize}
%							\item considerant $q \neq 2$:
								\begin{align}
									\text{ordre possible } &| \frac{\phi}{\lambda(n)} = \frac{(p-1)(q-1)}{\ppmc(p-1,q-1)}   \nonumber\\
%							 			 &= \frac{(p-1)(q-1)}{ \frac{(p-1)(q-1)}{pgcd(p-1,q-1)}}  \nonumber\\
%								 		 &= \frac{(p-1)(q-1)\cdot pgcd(p-1,q-1)}{(p-1)(q-1)}  \nonumber\\
								 		 &= \pgcd(p-1,q-1) < q   
								\end{align}
%							\item considerant $q = 2$:
%								\begin{align}
%									\text{ordre possible } &= \frac{\phi}{\lambda}  \nonumber\\
%								 		 &= \frac{(p-1)(q-1)}{\ppmc(p-1,q-1)}   \nonumber\\
%								 		 &= \frac{(p-1)(2-1)}{p-1}  \nonumber\\
%								 		 &= 1  \nonumber\\
%							 			 &< 2  
%								\end{align}
%						\end{itemize}
				\end{enumerate}

				Donc, comme l'ordre possible des autres groupes est limité, il y a un seul sous groupe cyclique d'ordre $q$ dans 
				$\Mgrinv{Z}{pq}$
			\end{proof}
		\end{lemma}









%		\begin{lemma}{$\Mgrinv{Z}{pq} \cong C_2 \times C_{\lambda/q} \times C_{q}$}
%			\begin{proof}
%				\begin{align}
%					n &= pq  \nonumber\\
%					  &= (\alpha q+1)q \Rightarrow \nonumber\\
%					  \Rightarrow \phi(n) &= \alpha q(q-1) = 2\lambda(n)
%				\end{align}
%				On sait qu'il existe un sous groupe cyclique d'ordre $\lambda$ \cite{carmichael1914theory,Wei:wolfram:MMG} 
%				et que $q|\lambda$, alors:
%				\begin{align}
%					\Mgrinv{Z}{n} &\cong C_2 \times C_{\lambda} \nonumber\\
%						      &\cong C_2 \times C_{q} \times C_{\lambda/q} 
%				\end{align}
%
%			\end{proof}
%		\end{lemma}
			
%On utilise la notation C_m le groupe cyclique d'ordre m

%On sait que \phi(n) = \lambda(n)*2 => Z/nZ* est isomorphe a C_2 x C_{\lambda}  (on sait qui il existe un groupe cyclique d'ordre \lambda, parce qu'il existe un élément de Z/nZ* que est d'ordre \lambda, après il faut juste montrer que le groupe engendré est multiplicatif et on a un groupe cyclique). [http://mathworld.wolfram.com/ModuloMultiplicationGroup.html]

%Si on a un groupe cyclique d'ordre \lambda(n), on a aussi un (et seulement un) sous-groupe d'ordre q (parce que q | \lambda(n)). Un élément de ce groupe peut être trouvé en faisant b^(\lambda/q) quand b est un élément générateur du C_{\lambda}. 

%J'ai aussi prouvé que si on a un élément 'c' d'ordre q sous Z/nZ*, il est d'ordre n sous n². Je organise les preuves dans le rapport.

		\begin{theorem}{Un élément $a$ d'ordre $q$ dans $\Mgrinv{Z}{n}$ appartient à $\mathbb{E}_n$. }
		\label{aneq1:sufisante}
		\begin{proof}Soit $a$ un élément d'ordre $q$ dans \Mgrinv{Z}{n}.
			Premièrement on montre que $a^q \equiv 1 \mod{n^2}$:

			On a $a^q \equiv 1 \mod{n}$ donc $a^q \equiv 1 \mod{q}$. Le Petit Théorème de Fermat donne $a \equiv 1 \mod{q}$, 
			ou bien $a = qA+1, A\in \mathbb{Z}$:
			\begin{align}
				a^q &= \sum_{i=0}^{q}\left( {q \choose i} (qA)^i \right) \nonumber\\
				&= 1+q\cdot(qA) + (qA)^2C, C\in\mathbb{Z}\nonumber
			\end{align}
			d'où	$a^q\equiv 1 \mod{q^2}$.
			
			Maintenant, on montre que $p^2|(a^n -1)$, on part de l'hypothèse que $ord_{n}(a) = q$, 
			ou bien $a^q-1 = pqA, A\in \mathbb{Z}$:
			$$ 	a^{qp} =(pqA+1)^p =1+(pqA)p+(pqA)^2B, B \in \mathbb{Z} $$
			d'où $a^n \equiv 1\mod{p^2}$.

			Finalement, si $a^n\equiv 1 \mod{q^2}$ et $a^n \equiv 1\mod{p^2}$, ors $\pgcd(p^2,q^2) = 1$, alors $n^2 | a^n-1$. Autrement dit
			$a^n \equiv 1 \mod{n^2}$ et donc  $a \in \mathbb{E}_n$.
%			
%
%			Finalement, utilisant le Théorème du Reste Chinois, le système 
%			\begin{align}
%				\left \{
%					\begin{array}{c @{\equiv} c}
%					    x^n & 1 \mod{q^2} \\
%					    x^n & 1 \mod{p^2} \\
%				\end{array}
%				\right. 
%			\end{align}
%			possède seulement une solution $\bmod{q^2p^2}$. Ors l'élément $a$ vérifie le système, alors $a^n \equiv 1 \bmod{n^2}$ et donc $a \in \mathbb{E}_n$
			%, la solution $a^n \equiv 1 \mod{n^2}$ vé\-ri\-fie le système. Alors, $a$ appartient
			%a $\mathbb{E}_n$.
		\end{proof}
		\end{theorem}

		Ici, on utilisera la même définition énoncée dans l'article de Paillier:
		\begin{definition} Un nombre $z$ est appelé le $n$-ième résidu modulo $n^2$ si il existe un nombre $y$ appartient à $\Mgrinv{Z}{n^2}$ tel que:
			$$z = y^n \mod{n^2}$$
		\end{definition}
		\begin{definition} On utilisera le symbole $\mathrm{R}_n$ pour désigner le groupe formé par les $n$-ième résidus modulo $n^2$.
		\end{definition}
		\begin{theorem} $\mathrm{R}_n$ est un groupe multiplicatif.
			\label{theorem:rGrMul}
			\begin{proof}
				 	Soient $a$ et $b$ tels que $a,b \in \mathrm{R}_n$, soient $a'$ et $b'$ racines $n$-ièmes de $a$ et $b$ $\bmod{n^2}$, 
					on prouve la fermeture du groupe multiplicatif: 
					$$c = a\cdot b \equiv (a')^n(b')^n \equiv (a'\cdot b')^n\mod{n^2}$$
					alors, $c$ possède au moins une racine $n$-ième, le produit $a'b'\mod{n^2}$.

			L'associativité est héritée de $\Mgrinv{Z}{n^2}$, l'élément $1$ est élément neutre et il reste 
			à prouver que l'inverse dans $\Mgrinv{Z}{n^2}$ fait partie aussi de $\mathrm{R}_n$:

				Soit $a\in\mathbb{Z}_n$, $a'$ une racine $n$-ième de $a$ et $d$ l'inverse de $a'$ dans $\Mgrinv{Z}{n^2}$. 
				Comme $a'\cdot d \equiv 1\mod{n^2}$, alors $ (a'd)^n = 1+n^2C, C\in\mathbb{Z}$ et:
				\begin{equation}
					a\cdot d^n \equiv (a')^nd^n \equiv (a'd)^n \equiv 1\mod{n^2}
				\end{equation}
%					\begin{align}
%						a'\cdot d        &\equiv 1 \mod{n^2} \Rightarrow \nonumber\\
%						\Rightarrow a'd &= 1 + n^2A, A\in \mathbb{Z} \Rightarrow\nonumber\\
%						\Rightarrow (a'd)^n &= 1 + n^2 C, C\in \mathbb{Z}\nonumber\\
%								&\equiv 1 \mod{n^2} \Rightarrow \nonumber\\
%						\Rightarrow (a')^n(d)^n        &\equiv 1 \mod{n^2} \Rightarrow \nonumber\\
%					 	\Rightarrow	a(d)^n        &\equiv 1 \mod{n^2} 
%					\end{align}
				clairement $d^n$ fait partie de $\mathrm{R}_n$ et est l'inverse de $a$ dans ce groupe.
			\end{proof}
		\end{theorem}

		En revanche, Paillier dit aussi que $\mathrm{R}_n$ est d'ordre $\phi(n)$, et si on regarde les résultats ici trouvés, 
		on voit que l'ordre est plutôt $\phi(n)/Ker(\psi)$ où $\psi$ est le morphisme multiplicatif:
		\begin{equation}
			\psi : \begin{array}[t]{ccl} \Mgrinv{Z}{n} &\rightarrow &   \Mgrinv{Z}{n}  \\
				y&\mapsto& y^n \mod{n^2}
			 \end{array}
		\end{equation}

%$\phi(n)/Ker(\mathrm{R}_n)$. 
		
		Avant on a introduit $\mathbb{E}_n$, maintenant, on voit clairement que $Ker(\psi) = \mathbb{E}_n $ 
		et alors, $\sharp(Ker(\psi)) = pgcd(\phi(n),n) = q$ si et seulement si $p \equiv 1\mod{q}$ (Lemme \ref{aneq1:necessaire} et 
		Théorème \ref{aneq1:sufisante}). Donc $\sharp(\mathrm{R}_n) = \frac{\phi(n)}{q}$ si et seulement si $p \equiv 1 \mod{q}$,
%		l'ordre de $\mathrm{R}_n$ est $\phi(n)/q$ si et seulement si $p \equiv 1 \mod{q}$ (avec le Lemme \ref{aneq1:necessaire}, 
%		dans la page \pageref{aneq1:necessaire}, on a prouvé que elle est nécessaire et avec le théorème 
%		\ref{aneq1:sufisante}, dans la page \pageref{aneq1:sufisante}, on a prouvé que elle est suffisante). 
		ce qui préjudicie le isomorphisme de la fonction de chiffrement: plusieurs valeurs aléatoires de $y$ peuvent 
		chiffrer la même valeur $x$ dans le même mot.
		\begin{rappel}
			Pour $x\in\Mgr{Z}{n}$, $y\in\Mgrinv{Z}{n}$, 
			$${\mathcal{E}}_g(x,y) \equiv g^xy^n\mod{n^2}$$
		\end{rappel}
%		quand appliquées sur le même message.

		Un exemple pratique peut montrer comment le choix de la paire $(p,q)$ peut affaiblir le résultat:

		\begin{equation*}
			\begin{array}{l}
				q = 53	\\
				p = 2\times 53 +1 = 107 \\
				n = pq = 5671\\
				n^2 = 32160241.\\
			\end{array}
		\end{equation*}
		On calcule $\mathbb{E}_n$:
		\begin{equation*}
			\begin{array}[t]{ll} 
				\mathbb{E}_n = &\{1, 160, 266, 319, 584, 637, 690, 743, 796, 849, 955, \\
				&1114, 1326, 1432, 1538, 1644, 1697, 1856, 1909, 1962, \\
				&2015, 2068, 2174, 2227, 2280, 2333, 2439, 2598, 2651, \\
				&2704, 2863, 2916, 3075, 3128, 3340, 3393, 3499, 3764, \\
				&4082, 4135, 4294, 4400, 4506, 4612, 4665, 4718, 4824,\\
				&\multicolumn{1}{r}{4877, 4983, 5354, 5407, 5460, 5513\}}
			\end{array}
		\end{equation*}
		chaque possible valeur chiffrée correspond à $q = $cinquante-trois valeurs de $y$:


		Pour $r\in\mathbb{E}_n$:
			$${\mathcal{E}}_g(1,1) \equiv r^n \equiv 1^{n} \equiv 160^{n} \equiv 266^{n} \equiv \dots \equiv 1 \mod{n^2}$$
			$${\mathcal{E}}_g(x,1) \equiv (g^x)r^n \equiv (g^x)1^{n} \equiv (g^x)160^{n} \equiv (g^x)266^{n} \equiv \dots \equiv (g^x) \mod{n^2}$$

			On veut chiffrer le message $450$, en utilisant $g = n+1 = 5672$ et supposant qu'on a tiré au hasard 
			$892$ pour $y$:
			\begin{align*}
				{\mathcal{E}}_{5672}(450,892)= 5672^{450}(\overset{y}{\overbrace{892}})^n &\equiv 5672^{450}(892\cdot160)^n \equiv 20893249\\
						  &\equiv 5672^{450}(892\cdot266)^n \equiv 20893249\\
						  &\equiv 5672^{450}(892\cdot319)^n \equiv 20893249\\
						  &\equiv 5672^{450}(892\cdot584)^n \equiv 20893249\\
						  &\equiv 5672^{450}(892\cdot637)^n \equiv 20893249\\
						  & \vdots
			\end{align*}

			L'espace effectif des aléas n'est plus $\Mgrinv{Z}{n}$ mais $(\Mgrinv{Z}{n}) / \mathbb{E}_n$, i.e. on réduit la randomisation 
			du chiffrement.

		\section{Autres Particularités}
		
		Au cours de la caractérisation du groupe $\mathbb{E}_n$, on a trouvé d'autres particularités que restent sans application. 
		
		\begin{theorem}{$p^i \equiv p \mod{n}, \forall i \in \mathbb{Z}^+, p\equiv1 \mod{q}$, $p^i$ est un point fixe sur l'anneau $\Mgrinv{Z}{n}$.}
			\begin{proof}
				\begin{align}
					\left \{
						\begin{array}{c @{\equiv} c}
						    p^i & 0 \mod{p} \\
					    p^i & 1 \mod{q} \\
					\end{array}
					\right. 
				\end{align}
				Du Théorème du Reste Chinois:
				\begin{align}
					p^i &\equiv 0\cdot q \cdot q^{p-2}+1\cdot p \cdot (p)^{-1}_{q}  \nonumber\\
					&\equiv 0+1\cdot p \cdot 1 \nonumber\\
					&\equiv p \mod{n}
				\end{align}
			\end{proof}
		\end{theorem}
