\section {Expériment}

D'abord, vingt objets de tailles et formes diverses ont été choisi pour évaluer les capacités *récognitif* du robot. Ils sont objets typiques qui peuvent être faciliment retrouvés dans un laboratoire. Une liste avec tous est incorporé aux annexes. Ensuite, le feature VFH était calculé pour huit positions différents écartées de 45 dégréés. La position correspondent au angle zero, était choisi de manière aléatoire en alignant un des axes de l'objet avec celui du capteur. 

\begin{figure}[H]
	\subfloat{\includegraphics[width=\textwidth]{chair_db.jpg}}		
\end{figure}

Une première évaluation proposée consiste à faire un tour complèt autour de l'objet à être reconnu en quatre positions angulaires différentes : $0, 45, 90$ et une dernière choisi de manière aléatoire pour chaque objet. Le robot fait le tour à une vitesse de $0.35 \pm 0.1 m/s$ à une distance de $1.5m$, en enregistrant des images à $1hz$, ansi, une expérience typique consiste d'environ $25$  images d'angle différent et prendre $25 \pm 3$ seconds. Ensuite, trois différentes ratios sont calculés pour exprimer la reconnaissance d'objets, la reconnaissance de vue et la suivi des reconnaissances par la chaîne de Markov cachée.

Un expérience typique est illustrée dans l'image {\color{green} ref} :

\begin{figure}[H]
	\subfloat{\includegraphics[width=\textwidth]{hmm_example.png}}		
\end{figure}

La première ligne correspond à l'image vue par le robot à chaque instant de temps et, donc, l'objet à être reconnu. Pendent que la deuxième, donnée par l'algorithme de reconnaissance, équivaut à la vue plus probable de l'objet reconnu par le K-plus proches voisins. Il est intéressant remarquer que l'invariance à rotation trompe l'estimation de l'orientation en prennat son correspond enénantiomorphe dans le premier carré rouge. Autrement, la définition de murs enlève une grande partie du dos du pinguin dans l'étape de segmentation, ainsi, la définition de ce point de vu n'est pas suffisament précis pour être differentié des autres objets. Au total, on remarque que le traitement fait par la chaîne de Markov cachée, surpasse les problèmes d'une base de donnée relativement sparse avec des possibles erreus de segmentation, pour atteindre la correction simultanée de la reconnaissance et de l'orientation. 

cela correspond à avoir une base

Finalement,  est affichée

\begin{figure}[H]
	\subfloat{\includegraphics[width=\textwidth]{hmm_mov.png}}		
\end{figure}


{\color{green}
Dans le premier tableaux on retrouve le résultats de la reconnaissance donné par la comparaison des histogrammes provenant du *plus proche voisin*. Ce résultat estime la capacité de distinguer deux objets quelconques, en autre mots, cette capacité viens de la représentativité des descripteurs utilisés et l'efficacité de la mesure de similarité entre histogrammes.
}
