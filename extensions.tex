\chapter{Extensions de Damgård–Jurik}
%	\TODO{revisar o franc\^es}
	\label{chap:jurik}

	Ivan B. Damgård et Mads J. Jurik ont proposé en 2001 une extension\cite{Damgard:2001:GSA:648118.746742} au système de 
	Paillier. Cette extension met l'accent sur les systèmes de vote électroniques
	en reposant sur la Propriété multiplicative du système (Propriété \ref{prop:mult}). Une particularité de cette 
	extension est l'utilisation des anneaux $\Mgrinv{Z}{n^{s+1}}$ 
	où $s\geq 1$, dans le système de Paillier $s=1$. Sous réserve de la limitation $s<p,q$ ,
	ils ont prouvé que l'ordre de l'élé\-ment $n+1$ est $n^s$, cette propriété ouvre la possibilité d'introduire des informations
	dans $\Mgr{Z}{n^{s}}$ en gardant l'isomorphisme de la fonction $\mathcal{E}$ et la propriété multiplicative 
	déjà présents dans le système de Paillier. Ainsi 
	augmentant le taux d'information dans un message chiffré. Dans le système de Paillier la valeur aléatoire occupe l'image de la fonction
	de chiffrement sans ajouter d'information, alors la taille des mots chiffrés est deux fois plus grande que l'information 
	transférée. Le système proposé ici accède à un taux d'information de $1 -1/s$, clairement, proche de $1$ pour $s$ grand.
	%si $s \to \infty $, le taux vaut $1$.

	L'utilisation de la base $g=1+n$ permet aussi le calcul du logarithme modulaire avec complexité $O(s^2)$, 
	l'algorithme utilisé par Damgård et Jurik est décrit plus loin.

	\section{Chiffrement dans \Mgrinv{Z}{n^{s+1}}}

	On introduit la fonction de Chiffrement pour le système étendu:

	\begin{definition} La fonction de chiffrement étendue.
		\begin{equation}
			\mathcal{E'}_{s} : \begin{array}[t]{ccl} \Mgr{Z}{n^s} \times \Mgrinv{Z}{n} &\rightarrow & \Mgrinv{Z}{n^{s+1}}  \\
			(m,r)&\mapsto& (1+n)^mr^{n^s}\mod{n^{s+1}}	 \end{array}
		\end{equation}
	\end{definition}

	Étant donnée la fonction de chiffrement il est clair que le déchiffrement se repose sur le pro\-blè\-me de trouver le 
	logarithme modulaire de base $n+1$ dans \Mgrinv{Z}{n^{s+1}}, par contre, comme dit avant, il existe un algorithme polynomial
	pour résoudre ce problème lorsque l'on connaît $\lambda(n\lambda(n))$ :

	De manière itérative sur $j$ il récupère le message modulo $n^j$: Soit $m_j$ le message modulo $n^j$ et $t$ un entier strictement positif,
	l'égalité $\binom{m_{j}}{t+1}n^t = \binom{m_{j-1}}{t+1}n^t + n^j\cdot k$ est vraie pour un entier $k$ et permet d'engendrer un 
	algorithme linéaire pour calculer $m_j$ connaissant  $m_{j-1}$. L'égalité est facilement prouvée en utilisant le fait que 
	$\binom{m_j}{t+1}$ et  $\binom{m_{j-1}}{t+1}$ sont congrus modulo $n^{j}$ sauf pour un facteur de $n^{j-1}$, alors, en multipliant les deux 
	côtés par $n^t$, $t\geq1$:
		\begin{equation}
			\label{eq:mj} \binom{m_j}{t+1}n^t \equiv \binom{m_{j-1}}{t+1}n^t+n^{j-1}\cdot k\cdot n^t \equiv \binom{m_{j-1}}{t+1}n^t \mod{n^j} 
		\end{equation}
	On note $\mathcal{L}$ la fonction:$$\mathcal{L}(m_{j-1},j) = \sum_{k = 2}^{j}\binom{m_{j-1}}{k}n^{k-1} \mod n^j$$ 
	et on reprend la fonction $L_m$ introduite dans la section   \ref{section:dech} en prouvant que la différence 
	$L_{n}((1+n)^m\mod{n^{j+1}}) - \mathcal{L}(m_{j-1},j)$ vaut $m_j$

	\begin{lemma} $L_{n}((1+n)^m\mod{n^{j+1}}) - \mathcal{L}(m_{j-1},j) \equiv m_j \mod{n^j}$:
		\label{lemma:jurik}
		\begin{proof}
			\begin{eqnarray*}
				 \text{Soit }L &=& L_{n}((1+n)^m\mod{n^{j+1}}) \\
				     &=& \frac{1+\binom{m_{j+1}}{1}n +  \dots +\binom{m_{j+1}}{j}n^{j} -1  \mod{n^{j+1}}  }{n} \\
							 &\equiv& \binom{m_{j}}{1}n^0 + \dots +\binom{m_{j}}{j}n^{j-1} \mod{n^{j}}\\
							 &\equiv& \sum_{k=1}^{j}\binom{m_{j}}{k}n^{k-1} \mod{n^{j}}
			\end{eqnarray*}
			Si on soustrait $\mathcal{L}(m_{j-1},j)$:
			$$L-\mathcal{L}(m_{j-1},j)  = \sum_{k=1}^{j}\binom{m_{j}}{k}n^{k-1} - \sum_{k = 2}^{j}\binom{m_{j-1}}{k}n^{k-1} \mod n^j $$
			Depuis l'équation \ref{eq:mj}, on peut combiner les sommations et annuler les valeurs quand $k>1$:
			$$L-\mathcal{L}(m_{j-1},j) \equiv m_j + \sum_{k=2}^{j}\left(\binom{m_{j}}{k}n^{k-1}-\binom{m_{j-1}}{k}n^{k-1} \right) \equiv m_j \mod{n^j} $$
		\end{proof}
	\end{lemma}


	Pour appliquer le lemme on définit la fonction $Rec_{n,j}$:

	\begin{equation}\label{equation:rec}
			{Rec}_{n,j} : \begin{array}[t]{ccl} \Mgr{Z}{n^{j+1}} &\rightarrow &   \Mgr{Z}{n^{j}}  \\
                               	c&\mapsto&\begin{cases} L_n(c\mod{n^{j+1}}) - \mathcal{L}({Rec}_{n,j-1}(c),j)&\text{si }j>1 \\%
				L_n(c\mod{n^{2}}) &\text{si }j=1\end{cases} 
			\end{array}
	\end{equation}

	Maintenant, on sait résoudre le problème de trouver l'exposant quand la base est $n+1$, par contre la valeur aléatoire qui multiplie la 
	puissance existe encore dans ce schéma  et on doit en tenir compte. 
	


	
	
%	On a introduit une fonction auxiliaire $Rec$ (\ref{equation:rec}) qui applique l'idée du Lemme \ref{lemma:jurik} et alors est capable de calculer l'exposant 
%	de $(1+n)^m$. Cependant la valeur résultat de la fonction de chiffrement contient $r^{n^s}$. Il faut le retirer: On efface $r^{n^s}$ de 
%	la même façon qu'on a fait pour Paillier. Comme $\lambda(n^{s+1}) = n^s\lambda(n)$, il faut juste élever le 
%	message chiffré à une valeur multiple de $\lambda(n)$, 
	Damgård et Jurik ont trouvé une façon créative de utiliser le Lemme \ref{lemma:jurik} même quand on a élevé le message chiffré à un multiple de 
	$\lambda(n)$. Comme la méthode vue permet de trouver $m \bmod n^s$, il faut juste que $d\equiv 1 \bmod n^s$ et $d\equiv 0 \bmod{\lambda(n)}$ et on peut s'en servir:
	\begin{equation} 
		\label{eq:lambdad} 
		c^d = \left((1+n)^mr^{n^s}\right)^{d} \equiv  (1+n)^{(md \bmod{n^s})}r^{(n^sd \bmod{\lambda(n)})} \equiv  (1+n)^{m}\bmod{n^{s+1}}\text{,}
	\end{equation}
	alors l'expression de déchiffrement devient:

	\begin{definition} La fonction de Déchiffrement étendue.
		\begin{equation}\label{eq:defd}
		d\text{ vérifiant: }
			\begin{cases}%
				d \equiv 1 \mod{n^s} \\%
				d \equiv 0 \mod{\lambda(n)}%
			\end{cases}
		\end{equation}


		\begin{equation}
			{D'}_{n,s} : \begin{array}[t]{ccl} \Mgrinv{Z}{n^{s+1}} &\rightarrow &   \Mgr{Z}{n^s}  \\
			c&\mapsto& Rec_{n,s}(c^{d})
			\end{array}
		\end{equation}

	\end{definition}


	
	Quelques pré calculs et l'implantation itérative de la récursivité de $Rec$ permettent d'éviter des opérations supplémentaires. 
	L'algorithme \ref{algo:logm} dans la page \pageref{algo:logm} montre comment les implanter.


	\section{Une variation à seuil}

	Dans un système à seuil il existe $N_s > k$ entités, chacune avec une partie de 
	la ``solution'' d'un problème global, le seuil $k$ définit le plus petit nombre de entités nécessaires 
	pour effectuer la tâche, avec un nombre strictement moins grand que $k$ les entités n'ont pas d'information suffisante 
	pour la solution et la tâche devient difficile. 

	Damgård et Jurik ont utilisé le travail de Fouque, Poupard et Stern dans \cite{Fouque00sharingdecryption} pour compléter l'extension 
	en engendrant un système avec un serveur fiable et des serveurs tiers. Les protocoles introduits dans \cite{Fouque00sharingdecryption} sont
	capables, avec les fonctions de Paillier, de déchiffrer de manière sûre un message 
	avec un seuil dans le nombre de serveurs utilisés, c'est-à-dire il existe un nombre $k$ de serveurs nécessaires
	pour déchiffrer la valeur, et le résultat contient une preuve de la validité du calcul effectué par chaque serveur, 
	$k-1$ serveurs ne peuvent pas déchiffrer.

	Le travail de Fouque, Poupard et Stern est basé sur le système à seuil de Victor Shoup \cite{Shoup99practicalthreshold}. Dans 
	l'original le système de chiffrement était RSA. L'avan\-cée de Fouque, Poupard et Stern a été l'adaptation au système de Paillier et 
	finalement Damgård et Jurik ont adapté l'idée au système étendu de Paillier.

	Afin d'atteindre l'impossibilité de déchiffrement des données avec moins de $k$ serveurs, une phase initiale de création de clés 
	partielles est nécessaire. Avec moins de $k$ clés il doit être impossible de trouver $\lambda(n)$ ou $d$. Ici $l\geq k$ serveurs 
	collaborent pour déchiffrer $c$. Chacun se voit attribuer une clé partielle. En collaborant les $l$ serveurs reconstruisent $d$ et peuvent 
	déchiffrer le message.
	
	Shoup a défini des restrictions pour les serveurs:

	\begin{itemize}  \label{item:restr}
		\item Pour la sécurité de la clé secrète: Le nombre de serveurs corruptibles doit être $<k$.
		\item Pour le calcul du mot déchiffré: Le nombre de serveurs corruptibles ne peut dépasser $N_s-k$.
	\end{itemize}

	Le protocole repose sur le fait qu'un polynôme de degré $m$ ne peut être défini par un nombre de points plus petit que $m+1$. 
	Si l'on considère les polynômes de degré $2$ dans $\mathbb{R}$, il y a une infinité polynômes qui passent par $(1,7)$ et $(2,16)$, alors
	la valeur de ces polynômes dans n'importe quel point sauf $1$ et $2$ peut pendre n'importe quel valeur dans $\mathbb{R}$. Par contre
	si on introduit un autre point $(3,29)$, le polynôme $2x^2+3x+2$ est complètement déterminé (voir l'image \ref{graph:polyboth}).

	\begin{figure}[htl]
		\centering%
		\subfloat[Deux Points]{\label{graph:polyall}% GNUPLOT: LaTeX picture
\setlength{\unitlength}{0.240900pt}
\ifx\plotpoint\undefined\newsavebox{\plotpoint}\fi
\sbox{\plotpoint}{\rule[-0.200pt]{0.400pt}{0.400pt}}%
\begin{picture}(675,405)(0,0)
\sbox{\plotpoint}{\rule[-0.200pt]{0.400pt}{0.400pt}}%
\put(110.0,82.0){\rule[-0.200pt]{4.818pt}{0.400pt}}
\put(90,82){\makebox(0,0)[r]{-10}}
\put(594.0,82.0){\rule[-0.200pt]{4.818pt}{0.400pt}}
\put(110.0,139.0){\rule[-0.200pt]{4.818pt}{0.400pt}}
\put(90,139){\makebox(0,0)[r]{ 0}}
\put(594.0,139.0){\rule[-0.200pt]{4.818pt}{0.400pt}}
\put(110.0,195.0){\rule[-0.200pt]{4.818pt}{0.400pt}}
\put(90,195){\makebox(0,0)[r]{ 10}}
\put(594.0,195.0){\rule[-0.200pt]{4.818pt}{0.400pt}}
\put(110.0,252.0){\rule[-0.200pt]{4.818pt}{0.400pt}}
\put(90,252){\makebox(0,0)[r]{ 20}}
\put(594.0,252.0){\rule[-0.200pt]{4.818pt}{0.400pt}}
\put(110.0,308.0){\rule[-0.200pt]{4.818pt}{0.400pt}}
\put(90,308){\makebox(0,0)[r]{ 30}}
\put(594.0,308.0){\rule[-0.200pt]{4.818pt}{0.400pt}}
\put(110.0,365.0){\rule[-0.200pt]{4.818pt}{0.400pt}}
\put(90,365){\makebox(0,0)[r]{ 40}}
\put(594.0,365.0){\rule[-0.200pt]{4.818pt}{0.400pt}}
\put(110.0,82.0){\rule[-0.200pt]{0.400pt}{4.818pt}}
\put(110,41){\makebox(0,0){-1}}
\put(110.0,345.0){\rule[-0.200pt]{0.400pt}{4.818pt}}
\put(211.0,82.0){\rule[-0.200pt]{0.400pt}{4.818pt}}
\put(211,41){\makebox(0,0){ 0}}
\put(211.0,345.0){\rule[-0.200pt]{0.400pt}{4.818pt}}
\put(312.0,82.0){\rule[-0.200pt]{0.400pt}{4.818pt}}
\put(312,41){\makebox(0,0){ 1}}
\put(312.0,345.0){\rule[-0.200pt]{0.400pt}{4.818pt}}
\put(412.0,82.0){\rule[-0.200pt]{0.400pt}{4.818pt}}
\put(412,41){\makebox(0,0){ 2}}
\put(412.0,345.0){\rule[-0.200pt]{0.400pt}{4.818pt}}
\put(513.0,82.0){\rule[-0.200pt]{0.400pt}{4.818pt}}
\put(513,41){\makebox(0,0){ 3}}
\put(513.0,345.0){\rule[-0.200pt]{0.400pt}{4.818pt}}
\put(614.0,82.0){\rule[-0.200pt]{0.400pt}{4.818pt}}
\put(614,41){\makebox(0,0){ 4}}
\put(614.0,345.0){\rule[-0.200pt]{0.400pt}{4.818pt}}
\put(110.0,82.0){\rule[-0.200pt]{0.400pt}{68.175pt}}
\put(110.0,82.0){\rule[-0.200pt]{121.414pt}{0.400pt}}
\put(614.0,82.0){\rule[-0.200pt]{0.400pt}{68.175pt}}
\put(110.0,365.0){\rule[-0.200pt]{121.414pt}{0.400pt}}
\put(240.67,82){\rule{0.400pt}{0.482pt}}
\multiput(240.17,82.00)(1.000,1.000){2}{\rule{0.400pt}{0.241pt}}
\multiput(242.59,84.00)(0.477,0.933){7}{\rule{0.115pt}{0.820pt}}
\multiput(241.17,84.00)(5.000,7.298){2}{\rule{0.400pt}{0.410pt}}
\multiput(247.59,93.00)(0.482,0.671){9}{\rule{0.116pt}{0.633pt}}
\multiput(246.17,93.00)(6.000,6.685){2}{\rule{0.400pt}{0.317pt}}
\multiput(253.59,101.00)(0.477,0.821){7}{\rule{0.115pt}{0.740pt}}
\multiput(252.17,101.00)(5.000,6.464){2}{\rule{0.400pt}{0.370pt}}
\multiput(258.59,109.00)(0.477,0.821){7}{\rule{0.115pt}{0.740pt}}
\multiput(257.17,109.00)(5.000,6.464){2}{\rule{0.400pt}{0.370pt}}
\multiput(263.59,117.00)(0.477,0.710){7}{\rule{0.115pt}{0.660pt}}
\multiput(262.17,117.00)(5.000,5.630){2}{\rule{0.400pt}{0.330pt}}
\multiput(268.59,124.00)(0.477,0.821){7}{\rule{0.115pt}{0.740pt}}
\multiput(267.17,124.00)(5.000,6.464){2}{\rule{0.400pt}{0.370pt}}
\multiput(273.59,132.00)(0.477,0.710){7}{\rule{0.115pt}{0.660pt}}
\multiput(272.17,132.00)(5.000,5.630){2}{\rule{0.400pt}{0.330pt}}
\multiput(278.59,139.00)(0.477,0.599){7}{\rule{0.115pt}{0.580pt}}
\multiput(277.17,139.00)(5.000,4.796){2}{\rule{0.400pt}{0.290pt}}
\multiput(283.59,145.00)(0.477,0.710){7}{\rule{0.115pt}{0.660pt}}
\multiput(282.17,145.00)(5.000,5.630){2}{\rule{0.400pt}{0.330pt}}
\multiput(288.59,152.00)(0.477,0.599){7}{\rule{0.115pt}{0.580pt}}
\multiput(287.17,152.00)(5.000,4.796){2}{\rule{0.400pt}{0.290pt}}
\multiput(293.59,158.00)(0.477,0.599){7}{\rule{0.115pt}{0.580pt}}
\multiput(292.17,158.00)(5.000,4.796){2}{\rule{0.400pt}{0.290pt}}
\multiput(298.59,164.00)(0.477,0.599){7}{\rule{0.115pt}{0.580pt}}
\multiput(297.17,164.00)(5.000,4.796){2}{\rule{0.400pt}{0.290pt}}
\multiput(303.00,170.59)(0.599,0.477){7}{\rule{0.580pt}{0.115pt}}
\multiput(303.00,169.17)(4.796,5.000){2}{\rule{0.290pt}{0.400pt}}
\multiput(309.00,175.59)(0.487,0.477){7}{\rule{0.500pt}{0.115pt}}
\multiput(309.00,174.17)(3.962,5.000){2}{\rule{0.250pt}{0.400pt}}
\multiput(314.00,180.59)(0.487,0.477){7}{\rule{0.500pt}{0.115pt}}
\multiput(314.00,179.17)(3.962,5.000){2}{\rule{0.250pt}{0.400pt}}
\multiput(319.00,185.59)(0.487,0.477){7}{\rule{0.500pt}{0.115pt}}
\multiput(319.00,184.17)(3.962,5.000){2}{\rule{0.250pt}{0.400pt}}
\multiput(324.00,190.60)(0.627,0.468){5}{\rule{0.600pt}{0.113pt}}
\multiput(324.00,189.17)(3.755,4.000){2}{\rule{0.300pt}{0.400pt}}
\multiput(329.00,194.60)(0.627,0.468){5}{\rule{0.600pt}{0.113pt}}
\multiput(329.00,193.17)(3.755,4.000){2}{\rule{0.300pt}{0.400pt}}
\multiput(334.00,198.60)(0.627,0.468){5}{\rule{0.600pt}{0.113pt}}
\multiput(334.00,197.17)(3.755,4.000){2}{\rule{0.300pt}{0.400pt}}
\multiput(339.00,202.60)(0.627,0.468){5}{\rule{0.600pt}{0.113pt}}
\multiput(339.00,201.17)(3.755,4.000){2}{\rule{0.300pt}{0.400pt}}
\multiput(344.00,206.61)(0.909,0.447){3}{\rule{0.767pt}{0.108pt}}
\multiput(344.00,205.17)(3.409,3.000){2}{\rule{0.383pt}{0.400pt}}
\multiput(349.00,209.61)(0.909,0.447){3}{\rule{0.767pt}{0.108pt}}
\multiput(349.00,208.17)(3.409,3.000){2}{\rule{0.383pt}{0.400pt}}
\multiput(354.00,212.61)(0.909,0.447){3}{\rule{0.767pt}{0.108pt}}
\multiput(354.00,211.17)(3.409,3.000){2}{\rule{0.383pt}{0.400pt}}
\multiput(359.00,215.61)(1.132,0.447){3}{\rule{0.900pt}{0.108pt}}
\multiput(359.00,214.17)(4.132,3.000){2}{\rule{0.450pt}{0.400pt}}
\put(365,218.17){\rule{1.100pt}{0.400pt}}
\multiput(365.00,217.17)(2.717,2.000){2}{\rule{0.550pt}{0.400pt}}
\put(370,220.17){\rule{1.100pt}{0.400pt}}
\multiput(370.00,219.17)(2.717,2.000){2}{\rule{0.550pt}{0.400pt}}
\put(375,222.17){\rule{1.100pt}{0.400pt}}
\multiput(375.00,221.17)(2.717,2.000){2}{\rule{0.550pt}{0.400pt}}
\put(380,223.67){\rule{1.204pt}{0.400pt}}
\multiput(380.00,223.17)(2.500,1.000){2}{\rule{0.602pt}{0.400pt}}
\put(385,225.17){\rule{1.100pt}{0.400pt}}
\multiput(385.00,224.17)(2.717,2.000){2}{\rule{0.550pt}{0.400pt}}
\put(390,226.67){\rule{1.204pt}{0.400pt}}
\multiput(390.00,226.17)(2.500,1.000){2}{\rule{0.602pt}{0.400pt}}
\put(400,227.67){\rule{1.204pt}{0.400pt}}
\multiput(400.00,227.17)(2.500,1.000){2}{\rule{0.602pt}{0.400pt}}
\put(395.0,228.0){\rule[-0.200pt]{1.204pt}{0.400pt}}
\put(421,227.67){\rule{1.204pt}{0.400pt}}
\multiput(421.00,228.17)(2.500,-1.000){2}{\rule{0.602pt}{0.400pt}}
\put(426,226.67){\rule{1.204pt}{0.400pt}}
\multiput(426.00,227.17)(2.500,-1.000){2}{\rule{0.602pt}{0.400pt}}
\put(431,225.67){\rule{1.204pt}{0.400pt}}
\multiput(431.00,226.17)(2.500,-1.000){2}{\rule{0.602pt}{0.400pt}}
\put(436,224.67){\rule{1.204pt}{0.400pt}}
\multiput(436.00,225.17)(2.500,-1.000){2}{\rule{0.602pt}{0.400pt}}
\put(441,223.17){\rule{1.100pt}{0.400pt}}
\multiput(441.00,224.17)(2.717,-2.000){2}{\rule{0.550pt}{0.400pt}}
\put(446,221.67){\rule{1.204pt}{0.400pt}}
\multiput(446.00,222.17)(2.500,-1.000){2}{\rule{0.602pt}{0.400pt}}
\put(451,220.17){\rule{1.100pt}{0.400pt}}
\multiput(451.00,221.17)(2.717,-2.000){2}{\rule{0.550pt}{0.400pt}}
\multiput(456.00,218.95)(0.909,-0.447){3}{\rule{0.767pt}{0.108pt}}
\multiput(456.00,219.17)(3.409,-3.000){2}{\rule{0.383pt}{0.400pt}}
\put(461,215.17){\rule{1.100pt}{0.400pt}}
\multiput(461.00,216.17)(2.717,-2.000){2}{\rule{0.550pt}{0.400pt}}
\multiput(466.00,213.95)(0.909,-0.447){3}{\rule{0.767pt}{0.108pt}}
\multiput(466.00,214.17)(3.409,-3.000){2}{\rule{0.383pt}{0.400pt}}
\multiput(471.00,210.95)(1.132,-0.447){3}{\rule{0.900pt}{0.108pt}}
\multiput(471.00,211.17)(4.132,-3.000){2}{\rule{0.450pt}{0.400pt}}
\multiput(477.00,207.94)(0.627,-0.468){5}{\rule{0.600pt}{0.113pt}}
\multiput(477.00,208.17)(3.755,-4.000){2}{\rule{0.300pt}{0.400pt}}
\multiput(482.00,203.94)(0.627,-0.468){5}{\rule{0.600pt}{0.113pt}}
\multiput(482.00,204.17)(3.755,-4.000){2}{\rule{0.300pt}{0.400pt}}
\multiput(487.00,199.95)(0.909,-0.447){3}{\rule{0.767pt}{0.108pt}}
\multiput(487.00,200.17)(3.409,-3.000){2}{\rule{0.383pt}{0.400pt}}
\multiput(492.00,196.93)(0.487,-0.477){7}{\rule{0.500pt}{0.115pt}}
\multiput(492.00,197.17)(3.962,-5.000){2}{\rule{0.250pt}{0.400pt}}
\multiput(497.00,191.94)(0.627,-0.468){5}{\rule{0.600pt}{0.113pt}}
\multiput(497.00,192.17)(3.755,-4.000){2}{\rule{0.300pt}{0.400pt}}
\multiput(502.00,187.93)(0.487,-0.477){7}{\rule{0.500pt}{0.115pt}}
\multiput(502.00,188.17)(3.962,-5.000){2}{\rule{0.250pt}{0.400pt}}
\multiput(507.00,182.93)(0.487,-0.477){7}{\rule{0.500pt}{0.115pt}}
\multiput(507.00,183.17)(3.962,-5.000){2}{\rule{0.250pt}{0.400pt}}
\multiput(512.00,177.93)(0.487,-0.477){7}{\rule{0.500pt}{0.115pt}}
\multiput(512.00,178.17)(3.962,-5.000){2}{\rule{0.250pt}{0.400pt}}
\multiput(517.00,172.93)(0.487,-0.477){7}{\rule{0.500pt}{0.115pt}}
\multiput(517.00,173.17)(3.962,-5.000){2}{\rule{0.250pt}{0.400pt}}
\multiput(522.59,166.59)(0.477,-0.599){7}{\rule{0.115pt}{0.580pt}}
\multiput(521.17,167.80)(5.000,-4.796){2}{\rule{0.400pt}{0.290pt}}
\multiput(527.00,161.93)(0.491,-0.482){9}{\rule{0.500pt}{0.116pt}}
\multiput(527.00,162.17)(4.962,-6.000){2}{\rule{0.250pt}{0.400pt}}
\multiput(533.59,154.59)(0.477,-0.599){7}{\rule{0.115pt}{0.580pt}}
\multiput(532.17,155.80)(5.000,-4.796){2}{\rule{0.400pt}{0.290pt}}
\multiput(538.59,148.26)(0.477,-0.710){7}{\rule{0.115pt}{0.660pt}}
\multiput(537.17,149.63)(5.000,-5.630){2}{\rule{0.400pt}{0.330pt}}
\multiput(543.59,141.26)(0.477,-0.710){7}{\rule{0.115pt}{0.660pt}}
\multiput(542.17,142.63)(5.000,-5.630){2}{\rule{0.400pt}{0.330pt}}
\multiput(548.59,134.26)(0.477,-0.710){7}{\rule{0.115pt}{0.660pt}}
\multiput(547.17,135.63)(5.000,-5.630){2}{\rule{0.400pt}{0.330pt}}
\multiput(553.59,127.26)(0.477,-0.710){7}{\rule{0.115pt}{0.660pt}}
\multiput(552.17,128.63)(5.000,-5.630){2}{\rule{0.400pt}{0.330pt}}
\multiput(558.59,119.93)(0.477,-0.821){7}{\rule{0.115pt}{0.740pt}}
\multiput(557.17,121.46)(5.000,-6.464){2}{\rule{0.400pt}{0.370pt}}
\multiput(563.59,111.93)(0.477,-0.821){7}{\rule{0.115pt}{0.740pt}}
\multiput(562.17,113.46)(5.000,-6.464){2}{\rule{0.400pt}{0.370pt}}
\multiput(568.59,103.93)(0.477,-0.821){7}{\rule{0.115pt}{0.740pt}}
\multiput(567.17,105.46)(5.000,-6.464){2}{\rule{0.400pt}{0.370pt}}
\multiput(573.59,95.93)(0.477,-0.821){7}{\rule{0.115pt}{0.740pt}}
\multiput(572.17,97.46)(5.000,-6.464){2}{\rule{0.400pt}{0.370pt}}
\multiput(578.59,87.60)(0.477,-0.933){7}{\rule{0.115pt}{0.820pt}}
\multiput(577.17,89.30)(5.000,-7.298){2}{\rule{0.400pt}{0.410pt}}
\put(405.0,229.0){\rule[-0.200pt]{3.854pt}{0.400pt}}
\put(583.0,82.0){\usebox{\plotpoint}}
\put(237.00,82.00){\usebox{\plotpoint}}
\put(247.58,99.67){\usebox{\plotpoint}}
\put(259.74,116.43){\usebox{\plotpoint}}
\put(271.80,133.32){\usebox{\plotpoint}}
\put(284.97,149.36){\usebox{\plotpoint}}
\put(298.28,165.28){\usebox{\plotpoint}}
\put(313.43,179.43){\usebox{\plotpoint}}
\put(329.06,193.05){\usebox{\plotpoint}}
\put(345.39,205.84){\usebox{\plotpoint}}
\put(363.64,215.55){\usebox{\plotpoint}}
\put(382.66,223.53){\usebox{\plotpoint}}
\put(402.74,228.55){\usebox{\plotpoint}}
\put(423.45,229.00){\usebox{\plotpoint}}
\put(443.95,226.41){\usebox{\plotpoint}}
\put(463.59,219.96){\usebox{\plotpoint}}
\put(482.59,211.65){\usebox{\plotpoint}}
\put(499.63,199.89){\usebox{\plotpoint}}
\put(515.77,186.99){\usebox{\plotpoint}}
\put(531.38,173.35){\usebox{\plotpoint}}
\put(545.41,158.11){\usebox{\plotpoint}}
\put(558.63,142.11){\usebox{\plotpoint}}
\put(571.16,125.58){\usebox{\plotpoint}}
\put(582.76,108.38){\usebox{\plotpoint}}
\put(594.88,91.60){\usebox{\plotpoint}}
\put(601,82){\usebox{\plotpoint}}
\sbox{\plotpoint}{\rule[-0.400pt]{0.800pt}{0.800pt}}%
\put(232,82){\usebox{\plotpoint}}
\multiput(233.38,82.00)(0.560,0.928){3}{\rule{0.135pt}{1.480pt}}
\multiput(230.34,82.00)(5.000,4.928){2}{\rule{0.800pt}{0.740pt}}
\multiput(238.38,90.00)(0.560,0.760){3}{\rule{0.135pt}{1.320pt}}
\multiput(235.34,90.00)(5.000,4.260){2}{\rule{0.800pt}{0.660pt}}
\multiput(243.38,97.00)(0.560,0.928){3}{\rule{0.135pt}{1.480pt}}
\multiput(240.34,97.00)(5.000,4.928){2}{\rule{0.800pt}{0.740pt}}
\multiput(248.39,105.00)(0.536,0.574){5}{\rule{0.129pt}{1.133pt}}
\multiput(245.34,105.00)(6.000,4.648){2}{\rule{0.800pt}{0.567pt}}
\multiput(254.38,112.00)(0.560,0.592){3}{\rule{0.135pt}{1.160pt}}
\multiput(251.34,112.00)(5.000,3.592){2}{\rule{0.800pt}{0.580pt}}
\multiput(259.38,118.00)(0.560,0.760){3}{\rule{0.135pt}{1.320pt}}
\multiput(256.34,118.00)(5.000,4.260){2}{\rule{0.800pt}{0.660pt}}
\multiput(264.38,125.00)(0.560,0.592){3}{\rule{0.135pt}{1.160pt}}
\multiput(261.34,125.00)(5.000,3.592){2}{\rule{0.800pt}{0.580pt}}
\multiput(269.38,131.00)(0.560,0.760){3}{\rule{0.135pt}{1.320pt}}
\multiput(266.34,131.00)(5.000,4.260){2}{\rule{0.800pt}{0.660pt}}
\multiput(274.38,138.00)(0.560,0.592){3}{\rule{0.135pt}{1.160pt}}
\multiput(271.34,138.00)(5.000,3.592){2}{\rule{0.800pt}{0.580pt}}
\multiput(278.00,145.38)(0.424,0.560){3}{\rule{1.000pt}{0.135pt}}
\multiput(278.00,142.34)(2.924,5.000){2}{\rule{0.500pt}{0.800pt}}
\multiput(284.38,149.00)(0.560,0.592){3}{\rule{0.135pt}{1.160pt}}
\multiput(281.34,149.00)(5.000,3.592){2}{\rule{0.800pt}{0.580pt}}
\multiput(288.00,156.38)(0.424,0.560){3}{\rule{1.000pt}{0.135pt}}
\multiput(288.00,153.34)(2.924,5.000){2}{\rule{0.500pt}{0.800pt}}
\multiput(294.38,160.00)(0.560,0.592){3}{\rule{0.135pt}{1.160pt}}
\multiput(291.34,160.00)(5.000,3.592){2}{\rule{0.800pt}{0.580pt}}
\multiput(298.00,167.38)(0.424,0.560){3}{\rule{1.000pt}{0.135pt}}
\multiput(298.00,164.34)(2.924,5.000){2}{\rule{0.500pt}{0.800pt}}
\put(303,171.34){\rule{1.400pt}{0.800pt}}
\multiput(303.00,169.34)(3.094,4.000){2}{\rule{0.700pt}{0.800pt}}
\multiput(309.00,176.38)(0.424,0.560){3}{\rule{1.000pt}{0.135pt}}
\multiput(309.00,173.34)(2.924,5.000){2}{\rule{0.500pt}{0.800pt}}
\put(314,180.34){\rule{1.200pt}{0.800pt}}
\multiput(314.00,178.34)(2.509,4.000){2}{\rule{0.600pt}{0.800pt}}
\multiput(319.00,185.38)(0.424,0.560){3}{\rule{1.000pt}{0.135pt}}
\multiput(319.00,182.34)(2.924,5.000){2}{\rule{0.500pt}{0.800pt}}
\put(324,189.34){\rule{1.200pt}{0.800pt}}
\multiput(324.00,187.34)(2.509,4.000){2}{\rule{0.600pt}{0.800pt}}
\put(329,192.84){\rule{1.204pt}{0.800pt}}
\multiput(329.00,191.34)(2.500,3.000){2}{\rule{0.602pt}{0.800pt}}
\put(334,196.34){\rule{1.200pt}{0.800pt}}
\multiput(334.00,194.34)(2.509,4.000){2}{\rule{0.600pt}{0.800pt}}
\put(339,199.84){\rule{1.204pt}{0.800pt}}
\multiput(339.00,198.34)(2.500,3.000){2}{\rule{0.602pt}{0.800pt}}
\put(344,203.34){\rule{1.200pt}{0.800pt}}
\multiput(344.00,201.34)(2.509,4.000){2}{\rule{0.600pt}{0.800pt}}
\put(349,206.84){\rule{1.204pt}{0.800pt}}
\multiput(349.00,205.34)(2.500,3.000){2}{\rule{0.602pt}{0.800pt}}
\put(354,209.34){\rule{1.204pt}{0.800pt}}
\multiput(354.00,208.34)(2.500,2.000){2}{\rule{0.602pt}{0.800pt}}
\put(359,211.84){\rule{1.445pt}{0.800pt}}
\multiput(359.00,210.34)(3.000,3.000){2}{\rule{0.723pt}{0.800pt}}
\put(365,214.34){\rule{1.204pt}{0.800pt}}
\multiput(365.00,213.34)(2.500,2.000){2}{\rule{0.602pt}{0.800pt}}
\put(370,216.34){\rule{1.204pt}{0.800pt}}
\multiput(370.00,215.34)(2.500,2.000){2}{\rule{0.602pt}{0.800pt}}
\put(375,218.34){\rule{1.204pt}{0.800pt}}
\multiput(375.00,217.34)(2.500,2.000){2}{\rule{0.602pt}{0.800pt}}
\put(380,220.34){\rule{1.204pt}{0.800pt}}
\multiput(380.00,219.34)(2.500,2.000){2}{\rule{0.602pt}{0.800pt}}
\put(385,222.34){\rule{1.204pt}{0.800pt}}
\multiput(385.00,221.34)(2.500,2.000){2}{\rule{0.602pt}{0.800pt}}
\put(390,223.84){\rule{1.204pt}{0.800pt}}
\multiput(390.00,223.34)(2.500,1.000){2}{\rule{0.602pt}{0.800pt}}
\put(395,224.84){\rule{1.204pt}{0.800pt}}
\multiput(395.00,224.34)(2.500,1.000){2}{\rule{0.602pt}{0.800pt}}
\put(400,225.84){\rule{1.204pt}{0.800pt}}
\multiput(400.00,225.34)(2.500,1.000){2}{\rule{0.602pt}{0.800pt}}
\put(405,226.84){\rule{1.204pt}{0.800pt}}
\multiput(405.00,226.34)(2.500,1.000){2}{\rule{0.602pt}{0.800pt}}
\put(415,227.84){\rule{1.445pt}{0.800pt}}
\multiput(415.00,227.34)(3.000,1.000){2}{\rule{0.723pt}{0.800pt}}
\put(410.0,229.0){\rule[-0.400pt]{1.204pt}{0.800pt}}
\put(436,227.84){\rule{1.204pt}{0.800pt}}
\multiput(436.00,228.34)(2.500,-1.000){2}{\rule{0.602pt}{0.800pt}}
\put(421.0,230.0){\rule[-0.400pt]{3.613pt}{0.800pt}}
\put(446,226.84){\rule{1.204pt}{0.800pt}}
\multiput(446.00,227.34)(2.500,-1.000){2}{\rule{0.602pt}{0.800pt}}
\put(451,225.84){\rule{1.204pt}{0.800pt}}
\multiput(451.00,226.34)(2.500,-1.000){2}{\rule{0.602pt}{0.800pt}}
\put(456,224.34){\rule{1.204pt}{0.800pt}}
\multiput(456.00,225.34)(2.500,-2.000){2}{\rule{0.602pt}{0.800pt}}
\put(461,222.84){\rule{1.204pt}{0.800pt}}
\multiput(461.00,223.34)(2.500,-1.000){2}{\rule{0.602pt}{0.800pt}}
\put(466,221.34){\rule{1.204pt}{0.800pt}}
\multiput(466.00,222.34)(2.500,-2.000){2}{\rule{0.602pt}{0.800pt}}
\put(471,219.34){\rule{1.445pt}{0.800pt}}
\multiput(471.00,220.34)(3.000,-2.000){2}{\rule{0.723pt}{0.800pt}}
\put(477,217.34){\rule{1.204pt}{0.800pt}}
\multiput(477.00,218.34)(2.500,-2.000){2}{\rule{0.602pt}{0.800pt}}
\put(482,215.34){\rule{1.204pt}{0.800pt}}
\multiput(482.00,216.34)(2.500,-2.000){2}{\rule{0.602pt}{0.800pt}}
\put(487,212.84){\rule{1.204pt}{0.800pt}}
\multiput(487.00,214.34)(2.500,-3.000){2}{\rule{0.602pt}{0.800pt}}
\put(492,210.34){\rule{1.204pt}{0.800pt}}
\multiput(492.00,211.34)(2.500,-2.000){2}{\rule{0.602pt}{0.800pt}}
\put(497,207.84){\rule{1.204pt}{0.800pt}}
\multiput(497.00,209.34)(2.500,-3.000){2}{\rule{0.602pt}{0.800pt}}
\put(502,204.84){\rule{1.204pt}{0.800pt}}
\multiput(502.00,206.34)(2.500,-3.000){2}{\rule{0.602pt}{0.800pt}}
\put(507,201.84){\rule{1.204pt}{0.800pt}}
\multiput(507.00,203.34)(2.500,-3.000){2}{\rule{0.602pt}{0.800pt}}
\put(512,198.34){\rule{1.200pt}{0.800pt}}
\multiput(512.00,200.34)(2.509,-4.000){2}{\rule{0.600pt}{0.800pt}}
\put(517,194.34){\rule{1.200pt}{0.800pt}}
\multiput(517.00,196.34)(2.509,-4.000){2}{\rule{0.600pt}{0.800pt}}
\put(522,190.34){\rule{1.200pt}{0.800pt}}
\multiput(522.00,192.34)(2.509,-4.000){2}{\rule{0.600pt}{0.800pt}}
\put(527,186.34){\rule{1.400pt}{0.800pt}}
\multiput(527.00,188.34)(3.094,-4.000){2}{\rule{0.700pt}{0.800pt}}
\put(533,182.34){\rule{1.200pt}{0.800pt}}
\multiput(533.00,184.34)(2.509,-4.000){2}{\rule{0.600pt}{0.800pt}}
\put(538,178.34){\rule{1.200pt}{0.800pt}}
\multiput(538.00,180.34)(2.509,-4.000){2}{\rule{0.600pt}{0.800pt}}
\multiput(543.00,176.06)(0.424,-0.560){3}{\rule{1.000pt}{0.135pt}}
\multiput(543.00,176.34)(2.924,-5.000){2}{\rule{0.500pt}{0.800pt}}
\multiput(548.00,171.06)(0.424,-0.560){3}{\rule{1.000pt}{0.135pt}}
\multiput(548.00,171.34)(2.924,-5.000){2}{\rule{0.500pt}{0.800pt}}
\multiput(553.00,166.06)(0.424,-0.560){3}{\rule{1.000pt}{0.135pt}}
\multiput(553.00,166.34)(2.924,-5.000){2}{\rule{0.500pt}{0.800pt}}
\multiput(558.00,161.06)(0.424,-0.560){3}{\rule{1.000pt}{0.135pt}}
\multiput(558.00,161.34)(2.924,-5.000){2}{\rule{0.500pt}{0.800pt}}
\multiput(564.38,153.18)(0.560,-0.592){3}{\rule{0.135pt}{1.160pt}}
\multiput(561.34,155.59)(5.000,-3.592){2}{\rule{0.800pt}{0.580pt}}
\multiput(569.38,147.18)(0.560,-0.592){3}{\rule{0.135pt}{1.160pt}}
\multiput(566.34,149.59)(5.000,-3.592){2}{\rule{0.800pt}{0.580pt}}
\multiput(574.38,141.18)(0.560,-0.592){3}{\rule{0.135pt}{1.160pt}}
\multiput(571.34,143.59)(5.000,-3.592){2}{\rule{0.800pt}{0.580pt}}
\multiput(579.38,135.18)(0.560,-0.592){3}{\rule{0.135pt}{1.160pt}}
\multiput(576.34,137.59)(5.000,-3.592){2}{\rule{0.800pt}{0.580pt}}
\multiput(583.00,132.07)(0.462,-0.536){5}{\rule{1.000pt}{0.129pt}}
\multiput(583.00,132.34)(3.924,-6.000){2}{\rule{0.500pt}{0.800pt}}
\multiput(590.38,122.52)(0.560,-0.760){3}{\rule{0.135pt}{1.320pt}}
\multiput(587.34,125.26)(5.000,-4.260){2}{\rule{0.800pt}{0.660pt}}
\multiput(595.38,116.18)(0.560,-0.592){3}{\rule{0.135pt}{1.160pt}}
\multiput(592.34,118.59)(5.000,-3.592){2}{\rule{0.800pt}{0.580pt}}
\multiput(600.38,109.52)(0.560,-0.760){3}{\rule{0.135pt}{1.320pt}}
\multiput(597.34,112.26)(5.000,-4.260){2}{\rule{0.800pt}{0.660pt}}
\multiput(605.38,102.52)(0.560,-0.760){3}{\rule{0.135pt}{1.320pt}}
\multiput(602.34,105.26)(5.000,-4.260){2}{\rule{0.800pt}{0.660pt}}
\multiput(610.38,94.86)(0.560,-0.928){3}{\rule{0.135pt}{1.480pt}}
\multiput(607.34,97.93)(5.000,-4.928){2}{\rule{0.800pt}{0.740pt}}
\put(441.0,229.0){\rule[-0.400pt]{1.204pt}{0.800pt}}
\sbox{\plotpoint}{\rule[-0.500pt]{1.000pt}{1.000pt}}%
\put(226.00,82.00){\usebox{\plotpoint}}
\put(238.24,98.74){\usebox{\plotpoint}}
\put(251.22,114.92){\usebox{\plotpoint}}
\put(264.53,130.84){\usebox{\plotpoint}}
\put(278.32,146.32){\usebox{\plotpoint}}
\put(292.48,161.48){\usebox{\plotpoint}}
\put(308.03,175.19){\usebox{\plotpoint}}
\put(324.22,188.17){\usebox{\plotpoint}}
\put(341.06,200.23){\usebox{\plotpoint}}
\put(358.85,210.91){\usebox{\plotpoint}}
\put(377.83,219.13){\usebox{\plotpoint}}
\put(397.36,225.95){\usebox{\plotpoint}}
\put(417.62,230.00){\usebox{\plotpoint}}
\put(438.27,231.00){\usebox{\plotpoint}}
\put(458.87,229.43){\usebox{\plotpoint}}
\put(479.08,225.58){\usebox{\plotpoint}}
\put(498.64,219.02){\usebox{\plotpoint}}
\put(517.20,209.88){\usebox{\plotpoint}}
\put(535.24,199.65){\usebox{\plotpoint}}
\put(552.14,187.69){\usebox{\plotpoint}}
\put(567.85,174.15){\usebox{\plotpoint}}
\put(582.99,160.01){\usebox{\plotpoint}}
\put(598.15,145.85){\usebox{\plotpoint}}
\put(611.52,129.98){\usebox{\plotpoint}}
\put(614,127){\usebox{\plotpoint}}
\sbox{\plotpoint}{\rule[-0.600pt]{1.200pt}{1.200pt}}%
\put(219,81.01){\rule{0.723pt}{1.200pt}}
\multiput(219.00,79.51)(1.500,3.000){2}{\rule{0.361pt}{1.200pt}}
\put(222.01,85){\rule{1.200pt}{1.686pt}}
\multiput(219.51,85.00)(5.000,3.500){2}{\rule{1.200pt}{0.843pt}}
\put(227.01,92){\rule{1.200pt}{1.445pt}}
\multiput(224.51,92.00)(5.000,3.000){2}{\rule{1.200pt}{0.723pt}}
\put(232.01,98){\rule{1.200pt}{1.445pt}}
\multiput(229.51,98.00)(5.000,3.000){2}{\rule{1.200pt}{0.723pt}}
\put(237.01,104){\rule{1.200pt}{1.445pt}}
\multiput(234.51,104.00)(5.000,3.000){2}{\rule{1.200pt}{0.723pt}}
\put(242.01,110){\rule{1.200pt}{1.445pt}}
\multiput(239.51,110.00)(5.000,3.000){2}{\rule{1.200pt}{0.723pt}}
\put(247,116.51){\rule{1.445pt}{1.200pt}}
\multiput(247.00,113.51)(3.000,6.000){2}{\rule{0.723pt}{1.200pt}}
\put(253.01,122){\rule{1.200pt}{1.445pt}}
\multiput(250.51,122.00)(5.000,3.000){2}{\rule{1.200pt}{0.723pt}}
\put(258,128.01){\rule{1.204pt}{1.200pt}}
\multiput(258.00,125.51)(2.500,5.000){2}{\rule{0.602pt}{1.200pt}}
\put(263,133.01){\rule{1.204pt}{1.200pt}}
\multiput(263.00,130.51)(2.500,5.000){2}{\rule{0.602pt}{1.200pt}}
\put(268.01,138){\rule{1.200pt}{1.445pt}}
\multiput(265.51,138.00)(5.000,3.000){2}{\rule{1.200pt}{0.723pt}}
\put(273,144.01){\rule{1.204pt}{1.200pt}}
\multiput(273.00,141.51)(2.500,5.000){2}{\rule{0.602pt}{1.200pt}}
\put(278,149.01){\rule{1.204pt}{1.200pt}}
\multiput(278.00,146.51)(2.500,5.000){2}{\rule{0.602pt}{1.200pt}}
\put(283,153.51){\rule{1.204pt}{1.200pt}}
\multiput(283.00,151.51)(2.500,4.000){2}{\rule{0.602pt}{1.200pt}}
\put(288,158.01){\rule{1.204pt}{1.200pt}}
\multiput(288.00,155.51)(2.500,5.000){2}{\rule{0.602pt}{1.200pt}}
\put(293,162.51){\rule{1.204pt}{1.200pt}}
\multiput(293.00,160.51)(2.500,4.000){2}{\rule{0.602pt}{1.200pt}}
\put(298,167.01){\rule{1.204pt}{1.200pt}}
\multiput(298.00,164.51)(2.500,5.000){2}{\rule{0.602pt}{1.200pt}}
\put(303,171.51){\rule{1.445pt}{1.200pt}}
\multiput(303.00,169.51)(3.000,4.000){2}{\rule{0.723pt}{1.200pt}}
\put(309,175.51){\rule{1.204pt}{1.200pt}}
\multiput(309.00,173.51)(2.500,4.000){2}{\rule{0.602pt}{1.200pt}}
\put(314,179.51){\rule{1.204pt}{1.200pt}}
\multiput(314.00,177.51)(2.500,4.000){2}{\rule{0.602pt}{1.200pt}}
\put(319,183.01){\rule{1.204pt}{1.200pt}}
\multiput(319.00,181.51)(2.500,3.000){2}{\rule{0.602pt}{1.200pt}}
\put(324,186.51){\rule{1.204pt}{1.200pt}}
\multiput(324.00,184.51)(2.500,4.000){2}{\rule{0.602pt}{1.200pt}}
\put(329,190.01){\rule{1.204pt}{1.200pt}}
\multiput(329.00,188.51)(2.500,3.000){2}{\rule{0.602pt}{1.200pt}}
\put(334,193.51){\rule{1.204pt}{1.200pt}}
\multiput(334.00,191.51)(2.500,4.000){2}{\rule{0.602pt}{1.200pt}}
\put(339,197.01){\rule{1.204pt}{1.200pt}}
\multiput(339.00,195.51)(2.500,3.000){2}{\rule{0.602pt}{1.200pt}}
\put(344,200.01){\rule{1.204pt}{1.200pt}}
\multiput(344.00,198.51)(2.500,3.000){2}{\rule{0.602pt}{1.200pt}}
\put(349,203.01){\rule{1.204pt}{1.200pt}}
\multiput(349.00,201.51)(2.500,3.000){2}{\rule{0.602pt}{1.200pt}}
\put(354,205.51){\rule{1.204pt}{1.200pt}}
\multiput(354.00,204.51)(2.500,2.000){2}{\rule{0.602pt}{1.200pt}}
\put(359,208.01){\rule{1.445pt}{1.200pt}}
\multiput(359.00,206.51)(3.000,3.000){2}{\rule{0.723pt}{1.200pt}}
\put(365,210.51){\rule{1.204pt}{1.200pt}}
\multiput(365.00,209.51)(2.500,2.000){2}{\rule{0.602pt}{1.200pt}}
\put(370,213.01){\rule{1.204pt}{1.200pt}}
\multiput(370.00,211.51)(2.500,3.000){2}{\rule{0.602pt}{1.200pt}}
\put(375,215.51){\rule{1.204pt}{1.200pt}}
\multiput(375.00,214.51)(2.500,2.000){2}{\rule{0.602pt}{1.200pt}}
\put(380,217.51){\rule{1.204pt}{1.200pt}}
\multiput(380.00,216.51)(2.500,2.000){2}{\rule{0.602pt}{1.200pt}}
\put(385,219.51){\rule{1.204pt}{1.200pt}}
\multiput(385.00,218.51)(2.500,2.000){2}{\rule{0.602pt}{1.200pt}}
\put(390,221.01){\rule{1.204pt}{1.200pt}}
\multiput(390.00,220.51)(2.500,1.000){2}{\rule{0.602pt}{1.200pt}}
\put(395,222.51){\rule{1.204pt}{1.200pt}}
\multiput(395.00,221.51)(2.500,2.000){2}{\rule{0.602pt}{1.200pt}}
\put(400,224.01){\rule{1.204pt}{1.200pt}}
\multiput(400.00,223.51)(2.500,1.000){2}{\rule{0.602pt}{1.200pt}}
\put(405,225.51){\rule{1.204pt}{1.200pt}}
\multiput(405.00,224.51)(2.500,2.000){2}{\rule{0.602pt}{1.200pt}}
\put(410,227.01){\rule{1.204pt}{1.200pt}}
\multiput(410.00,226.51)(2.500,1.000){2}{\rule{0.602pt}{1.200pt}}
\put(415,228.01){\rule{1.445pt}{1.200pt}}
\multiput(415.00,227.51)(3.000,1.000){2}{\rule{0.723pt}{1.200pt}}
\put(421,229.01){\rule{1.204pt}{1.200pt}}
\multiput(421.00,228.51)(2.500,1.000){2}{\rule{0.602pt}{1.200pt}}
\put(431,230.01){\rule{1.204pt}{1.200pt}}
\multiput(431.00,229.51)(2.500,1.000){2}{\rule{0.602pt}{1.200pt}}
\put(426.0,232.0){\rule[-0.600pt]{1.204pt}{1.200pt}}
\put(441,231.01){\rule{1.204pt}{1.200pt}}
\multiput(441.00,230.51)(2.500,1.000){2}{\rule{0.602pt}{1.200pt}}
\put(436.0,233.0){\rule[-0.600pt]{1.204pt}{1.200pt}}
\put(456,231.01){\rule{1.204pt}{1.200pt}}
\multiput(456.00,231.51)(2.500,-1.000){2}{\rule{0.602pt}{1.200pt}}
\put(446.0,234.0){\rule[-0.600pt]{2.409pt}{1.200pt}}
\put(471,230.01){\rule{1.445pt}{1.200pt}}
\multiput(471.00,230.51)(3.000,-1.000){2}{\rule{0.723pt}{1.200pt}}
\put(477,229.01){\rule{1.204pt}{1.200pt}}
\multiput(477.00,229.51)(2.500,-1.000){2}{\rule{0.602pt}{1.200pt}}
\put(482,228.01){\rule{1.204pt}{1.200pt}}
\multiput(482.00,228.51)(2.500,-1.000){2}{\rule{0.602pt}{1.200pt}}
\put(487,227.01){\rule{1.204pt}{1.200pt}}
\multiput(487.00,227.51)(2.500,-1.000){2}{\rule{0.602pt}{1.200pt}}
\put(492,226.01){\rule{1.204pt}{1.200pt}}
\multiput(492.00,226.51)(2.500,-1.000){2}{\rule{0.602pt}{1.200pt}}
\put(497,225.01){\rule{1.204pt}{1.200pt}}
\multiput(497.00,225.51)(2.500,-1.000){2}{\rule{0.602pt}{1.200pt}}
\put(502,223.51){\rule{1.204pt}{1.200pt}}
\multiput(502.00,224.51)(2.500,-2.000){2}{\rule{0.602pt}{1.200pt}}
\put(507,222.01){\rule{1.204pt}{1.200pt}}
\multiput(507.00,222.51)(2.500,-1.000){2}{\rule{0.602pt}{1.200pt}}
\put(512,220.51){\rule{1.204pt}{1.200pt}}
\multiput(512.00,221.51)(2.500,-2.000){2}{\rule{0.602pt}{1.200pt}}
\put(517,218.51){\rule{1.204pt}{1.200pt}}
\multiput(517.00,219.51)(2.500,-2.000){2}{\rule{0.602pt}{1.200pt}}
\put(522,216.51){\rule{1.204pt}{1.200pt}}
\multiput(522.00,217.51)(2.500,-2.000){2}{\rule{0.602pt}{1.200pt}}
\put(527,214.51){\rule{1.445pt}{1.200pt}}
\multiput(527.00,215.51)(3.000,-2.000){2}{\rule{0.723pt}{1.200pt}}
\put(533,212.51){\rule{1.204pt}{1.200pt}}
\multiput(533.00,213.51)(2.500,-2.000){2}{\rule{0.602pt}{1.200pt}}
\put(538,210.01){\rule{1.204pt}{1.200pt}}
\multiput(538.00,211.51)(2.500,-3.000){2}{\rule{0.602pt}{1.200pt}}
\put(543,207.01){\rule{1.204pt}{1.200pt}}
\multiput(543.00,208.51)(2.500,-3.000){2}{\rule{0.602pt}{1.200pt}}
\put(548,204.51){\rule{1.204pt}{1.200pt}}
\multiput(548.00,205.51)(2.500,-2.000){2}{\rule{0.602pt}{1.200pt}}
\put(553,202.01){\rule{1.204pt}{1.200pt}}
\multiput(553.00,203.51)(2.500,-3.000){2}{\rule{0.602pt}{1.200pt}}
\put(558,199.01){\rule{1.204pt}{1.200pt}}
\multiput(558.00,200.51)(2.500,-3.000){2}{\rule{0.602pt}{1.200pt}}
\put(563,196.01){\rule{1.204pt}{1.200pt}}
\multiput(563.00,197.51)(2.500,-3.000){2}{\rule{0.602pt}{1.200pt}}
\put(568,192.51){\rule{1.204pt}{1.200pt}}
\multiput(568.00,194.51)(2.500,-4.000){2}{\rule{0.602pt}{1.200pt}}
\put(573,189.01){\rule{1.204pt}{1.200pt}}
\multiput(573.00,190.51)(2.500,-3.000){2}{\rule{0.602pt}{1.200pt}}
\put(578,185.51){\rule{1.204pt}{1.200pt}}
\multiput(578.00,187.51)(2.500,-4.000){2}{\rule{0.602pt}{1.200pt}}
\put(583,181.51){\rule{1.445pt}{1.200pt}}
\multiput(583.00,183.51)(3.000,-4.000){2}{\rule{0.723pt}{1.200pt}}
\put(589,177.51){\rule{1.204pt}{1.200pt}}
\multiput(589.00,179.51)(2.500,-4.000){2}{\rule{0.602pt}{1.200pt}}
\put(594,173.51){\rule{1.204pt}{1.200pt}}
\multiput(594.00,175.51)(2.500,-4.000){2}{\rule{0.602pt}{1.200pt}}
\put(599,169.51){\rule{1.204pt}{1.200pt}}
\multiput(599.00,171.51)(2.500,-4.000){2}{\rule{0.602pt}{1.200pt}}
\put(604,165.51){\rule{1.204pt}{1.200pt}}
\multiput(604.00,167.51)(2.500,-4.000){2}{\rule{0.602pt}{1.200pt}}
\put(609,161.01){\rule{1.204pt}{1.200pt}}
\multiput(609.00,163.51)(2.500,-5.000){2}{\rule{0.602pt}{1.200pt}}
\put(461.0,233.0){\rule[-0.600pt]{2.409pt}{1.200pt}}
\sbox{\plotpoint}{\rule[-0.500pt]{1.000pt}{1.000pt}}%
\put(211.00,82.00){\usebox{\plotpoint}}
\put(238.26,113.26){\usebox{\plotpoint}}
\put(268.09,142.09){\usebox{\plotpoint}}
\put(299.47,169.18){\usebox{\plotpoint}}
\put(333.56,192.73){\usebox{\plotpoint}}
\put(369.65,212.86){\usebox{\plotpoint}}
\put(408.04,228.22){\usebox{\plotpoint}}
\put(448.49,236.50){\usebox{\plotpoint}}
\put(489.80,237.44){\usebox{\plotpoint}}
\put(530.72,231.38){\usebox{\plotpoint}}
\put(569.92,218.23){\usebox{\plotpoint}}
\put(606.77,199.34){\usebox{\plotpoint}}
\put(614,195){\usebox{\plotpoint}}
\sbox{\plotpoint}{\rule[-0.200pt]{0.400pt}{0.400pt}}%
\put(200,82.17){\rule{0.482pt}{0.400pt}}
\multiput(200.00,81.17)(1.000,2.000){2}{\rule{0.241pt}{0.400pt}}
\multiput(202.00,84.59)(0.487,0.477){7}{\rule{0.500pt}{0.115pt}}
\multiput(202.00,83.17)(3.962,5.000){2}{\rule{0.250pt}{0.400pt}}
\multiput(207.00,89.59)(0.487,0.477){7}{\rule{0.500pt}{0.115pt}}
\multiput(207.00,88.17)(3.962,5.000){2}{\rule{0.250pt}{0.400pt}}
\multiput(212.00,94.59)(0.487,0.477){7}{\rule{0.500pt}{0.115pt}}
\multiput(212.00,93.17)(3.962,5.000){2}{\rule{0.250pt}{0.400pt}}
\multiput(217.00,99.59)(0.487,0.477){7}{\rule{0.500pt}{0.115pt}}
\multiput(217.00,98.17)(3.962,5.000){2}{\rule{0.250pt}{0.400pt}}
\multiput(222.00,104.59)(0.487,0.477){7}{\rule{0.500pt}{0.115pt}}
\multiput(222.00,103.17)(3.962,5.000){2}{\rule{0.250pt}{0.400pt}}
\multiput(227.00,109.59)(0.487,0.477){7}{\rule{0.500pt}{0.115pt}}
\multiput(227.00,108.17)(3.962,5.000){2}{\rule{0.250pt}{0.400pt}}
\multiput(232.00,114.59)(0.487,0.477){7}{\rule{0.500pt}{0.115pt}}
\multiput(232.00,113.17)(3.962,5.000){2}{\rule{0.250pt}{0.400pt}}
\multiput(237.00,119.59)(0.487,0.477){7}{\rule{0.500pt}{0.115pt}}
\multiput(237.00,118.17)(3.962,5.000){2}{\rule{0.250pt}{0.400pt}}
\multiput(242.00,124.60)(0.627,0.468){5}{\rule{0.600pt}{0.113pt}}
\multiput(242.00,123.17)(3.755,4.000){2}{\rule{0.300pt}{0.400pt}}
\multiput(247.00,128.59)(0.599,0.477){7}{\rule{0.580pt}{0.115pt}}
\multiput(247.00,127.17)(4.796,5.000){2}{\rule{0.290pt}{0.400pt}}
\multiput(253.00,133.60)(0.627,0.468){5}{\rule{0.600pt}{0.113pt}}
\multiput(253.00,132.17)(3.755,4.000){2}{\rule{0.300pt}{0.400pt}}
\multiput(258.00,137.60)(0.627,0.468){5}{\rule{0.600pt}{0.113pt}}
\multiput(258.00,136.17)(3.755,4.000){2}{\rule{0.300pt}{0.400pt}}
\multiput(263.00,141.59)(0.487,0.477){7}{\rule{0.500pt}{0.115pt}}
\multiput(263.00,140.17)(3.962,5.000){2}{\rule{0.250pt}{0.400pt}}
\multiput(268.00,146.60)(0.627,0.468){5}{\rule{0.600pt}{0.113pt}}
\multiput(268.00,145.17)(3.755,4.000){2}{\rule{0.300pt}{0.400pt}}
\multiput(273.00,150.60)(0.627,0.468){5}{\rule{0.600pt}{0.113pt}}
\multiput(273.00,149.17)(3.755,4.000){2}{\rule{0.300pt}{0.400pt}}
\multiput(278.00,154.60)(0.627,0.468){5}{\rule{0.600pt}{0.113pt}}
\multiput(278.00,153.17)(3.755,4.000){2}{\rule{0.300pt}{0.400pt}}
\multiput(283.00,158.60)(0.627,0.468){5}{\rule{0.600pt}{0.113pt}}
\multiput(283.00,157.17)(3.755,4.000){2}{\rule{0.300pt}{0.400pt}}
\multiput(288.00,162.61)(0.909,0.447){3}{\rule{0.767pt}{0.108pt}}
\multiput(288.00,161.17)(3.409,3.000){2}{\rule{0.383pt}{0.400pt}}
\multiput(293.00,165.60)(0.627,0.468){5}{\rule{0.600pt}{0.113pt}}
\multiput(293.00,164.17)(3.755,4.000){2}{\rule{0.300pt}{0.400pt}}
\multiput(298.00,169.60)(0.627,0.468){5}{\rule{0.600pt}{0.113pt}}
\multiput(298.00,168.17)(3.755,4.000){2}{\rule{0.300pt}{0.400pt}}
\multiput(303.00,173.61)(1.132,0.447){3}{\rule{0.900pt}{0.108pt}}
\multiput(303.00,172.17)(4.132,3.000){2}{\rule{0.450pt}{0.400pt}}
\multiput(309.00,176.60)(0.627,0.468){5}{\rule{0.600pt}{0.113pt}}
\multiput(309.00,175.17)(3.755,4.000){2}{\rule{0.300pt}{0.400pt}}
\multiput(314.00,180.61)(0.909,0.447){3}{\rule{0.767pt}{0.108pt}}
\multiput(314.00,179.17)(3.409,3.000){2}{\rule{0.383pt}{0.400pt}}
\multiput(319.00,183.61)(0.909,0.447){3}{\rule{0.767pt}{0.108pt}}
\multiput(319.00,182.17)(3.409,3.000){2}{\rule{0.383pt}{0.400pt}}
\multiput(324.00,186.61)(0.909,0.447){3}{\rule{0.767pt}{0.108pt}}
\multiput(324.00,185.17)(3.409,3.000){2}{\rule{0.383pt}{0.400pt}}
\multiput(329.00,189.61)(0.909,0.447){3}{\rule{0.767pt}{0.108pt}}
\multiput(329.00,188.17)(3.409,3.000){2}{\rule{0.383pt}{0.400pt}}
\multiput(334.00,192.61)(0.909,0.447){3}{\rule{0.767pt}{0.108pt}}
\multiput(334.00,191.17)(3.409,3.000){2}{\rule{0.383pt}{0.400pt}}
\multiput(339.00,195.61)(0.909,0.447){3}{\rule{0.767pt}{0.108pt}}
\multiput(339.00,194.17)(3.409,3.000){2}{\rule{0.383pt}{0.400pt}}
\multiput(344.00,198.61)(0.909,0.447){3}{\rule{0.767pt}{0.108pt}}
\multiput(344.00,197.17)(3.409,3.000){2}{\rule{0.383pt}{0.400pt}}
\multiput(349.00,201.61)(0.909,0.447){3}{\rule{0.767pt}{0.108pt}}
\multiput(349.00,200.17)(3.409,3.000){2}{\rule{0.383pt}{0.400pt}}
\multiput(354.00,204.61)(0.909,0.447){3}{\rule{0.767pt}{0.108pt}}
\multiput(354.00,203.17)(3.409,3.000){2}{\rule{0.383pt}{0.400pt}}
\put(359,207.17){\rule{1.300pt}{0.400pt}}
\multiput(359.00,206.17)(3.302,2.000){2}{\rule{0.650pt}{0.400pt}}
\multiput(365.00,209.61)(0.909,0.447){3}{\rule{0.767pt}{0.108pt}}
\multiput(365.00,208.17)(3.409,3.000){2}{\rule{0.383pt}{0.400pt}}
\put(370,212.17){\rule{1.100pt}{0.400pt}}
\multiput(370.00,211.17)(2.717,2.000){2}{\rule{0.550pt}{0.400pt}}
\put(375,214.17){\rule{1.100pt}{0.400pt}}
\multiput(375.00,213.17)(2.717,2.000){2}{\rule{0.550pt}{0.400pt}}
\multiput(380.00,216.61)(0.909,0.447){3}{\rule{0.767pt}{0.108pt}}
\multiput(380.00,215.17)(3.409,3.000){2}{\rule{0.383pt}{0.400pt}}
\put(385,219.17){\rule{1.100pt}{0.400pt}}
\multiput(385.00,218.17)(2.717,2.000){2}{\rule{0.550pt}{0.400pt}}
\put(390,221.17){\rule{1.100pt}{0.400pt}}
\multiput(390.00,220.17)(2.717,2.000){2}{\rule{0.550pt}{0.400pt}}
\put(395,223.17){\rule{1.100pt}{0.400pt}}
\multiput(395.00,222.17)(2.717,2.000){2}{\rule{0.550pt}{0.400pt}}
\put(400,225.17){\rule{1.100pt}{0.400pt}}
\multiput(400.00,224.17)(2.717,2.000){2}{\rule{0.550pt}{0.400pt}}
\put(405,226.67){\rule{1.204pt}{0.400pt}}
\multiput(405.00,226.17)(2.500,1.000){2}{\rule{0.602pt}{0.400pt}}
\put(410,228.17){\rule{1.100pt}{0.400pt}}
\multiput(410.00,227.17)(2.717,2.000){2}{\rule{0.550pt}{0.400pt}}
\put(415,230.17){\rule{1.300pt}{0.400pt}}
\multiput(415.00,229.17)(3.302,2.000){2}{\rule{0.650pt}{0.400pt}}
\put(421,231.67){\rule{1.204pt}{0.400pt}}
\multiput(421.00,231.17)(2.500,1.000){2}{\rule{0.602pt}{0.400pt}}
\put(426,233.17){\rule{1.100pt}{0.400pt}}
\multiput(426.00,232.17)(2.717,2.000){2}{\rule{0.550pt}{0.400pt}}
\put(431,234.67){\rule{1.204pt}{0.400pt}}
\multiput(431.00,234.17)(2.500,1.000){2}{\rule{0.602pt}{0.400pt}}
\put(436,235.67){\rule{1.204pt}{0.400pt}}
\multiput(436.00,235.17)(2.500,1.000){2}{\rule{0.602pt}{0.400pt}}
\put(441,237.17){\rule{1.100pt}{0.400pt}}
\multiput(441.00,236.17)(2.717,2.000){2}{\rule{0.550pt}{0.400pt}}
\put(446,238.67){\rule{1.204pt}{0.400pt}}
\multiput(446.00,238.17)(2.500,1.000){2}{\rule{0.602pt}{0.400pt}}
\put(451,239.67){\rule{1.204pt}{0.400pt}}
\multiput(451.00,239.17)(2.500,1.000){2}{\rule{0.602pt}{0.400pt}}
\put(456,240.67){\rule{1.204pt}{0.400pt}}
\multiput(456.00,240.17)(2.500,1.000){2}{\rule{0.602pt}{0.400pt}}
\put(466,241.67){\rule{1.204pt}{0.400pt}}
\multiput(466.00,241.17)(2.500,1.000){2}{\rule{0.602pt}{0.400pt}}
\put(471,242.67){\rule{1.445pt}{0.400pt}}
\multiput(471.00,242.17)(3.000,1.000){2}{\rule{0.723pt}{0.400pt}}
\put(461.0,242.0){\rule[-0.200pt]{1.204pt}{0.400pt}}
\put(482,243.67){\rule{1.204pt}{0.400pt}}
\multiput(482.00,243.17)(2.500,1.000){2}{\rule{0.602pt}{0.400pt}}
\put(477.0,244.0){\rule[-0.200pt]{1.204pt}{0.400pt}}
\put(492,244.67){\rule{1.204pt}{0.400pt}}
\multiput(492.00,244.17)(2.500,1.000){2}{\rule{0.602pt}{0.400pt}}
\put(487.0,245.0){\rule[-0.200pt]{1.204pt}{0.400pt}}
\put(533,244.67){\rule{1.204pt}{0.400pt}}
\multiput(533.00,245.17)(2.500,-1.000){2}{\rule{0.602pt}{0.400pt}}
\put(497.0,246.0){\rule[-0.200pt]{8.672pt}{0.400pt}}
\put(543,243.67){\rule{1.204pt}{0.400pt}}
\multiput(543.00,244.17)(2.500,-1.000){2}{\rule{0.602pt}{0.400pt}}
\put(538.0,245.0){\rule[-0.200pt]{1.204pt}{0.400pt}}
\put(553,242.67){\rule{1.204pt}{0.400pt}}
\multiput(553.00,243.17)(2.500,-1.000){2}{\rule{0.602pt}{0.400pt}}
\put(558,241.67){\rule{1.204pt}{0.400pt}}
\multiput(558.00,242.17)(2.500,-1.000){2}{\rule{0.602pt}{0.400pt}}
\put(563,240.67){\rule{1.204pt}{0.400pt}}
\multiput(563.00,241.17)(2.500,-1.000){2}{\rule{0.602pt}{0.400pt}}
\put(568,239.67){\rule{1.204pt}{0.400pt}}
\multiput(568.00,240.17)(2.500,-1.000){2}{\rule{0.602pt}{0.400pt}}
\put(573,238.67){\rule{1.204pt}{0.400pt}}
\multiput(573.00,239.17)(2.500,-1.000){2}{\rule{0.602pt}{0.400pt}}
\put(578,237.67){\rule{1.204pt}{0.400pt}}
\multiput(578.00,238.17)(2.500,-1.000){2}{\rule{0.602pt}{0.400pt}}
\put(583,236.67){\rule{1.445pt}{0.400pt}}
\multiput(583.00,237.17)(3.000,-1.000){2}{\rule{0.723pt}{0.400pt}}
\put(589,235.17){\rule{1.100pt}{0.400pt}}
\multiput(589.00,236.17)(2.717,-2.000){2}{\rule{0.550pt}{0.400pt}}
\put(594,233.67){\rule{1.204pt}{0.400pt}}
\multiput(594.00,234.17)(2.500,-1.000){2}{\rule{0.602pt}{0.400pt}}
\put(599,232.17){\rule{1.100pt}{0.400pt}}
\multiput(599.00,233.17)(2.717,-2.000){2}{\rule{0.550pt}{0.400pt}}
\put(604,230.67){\rule{1.204pt}{0.400pt}}
\multiput(604.00,231.17)(2.500,-1.000){2}{\rule{0.602pt}{0.400pt}}
\put(609,229.17){\rule{1.100pt}{0.400pt}}
\multiput(609.00,230.17)(2.717,-2.000){2}{\rule{0.550pt}{0.400pt}}
\put(548.0,244.0){\rule[-0.200pt]{1.204pt}{0.400pt}}
\put(185.00,82.00){\usebox{\plotpoint}}
\put(200.58,95.58){\usebox{\plotpoint}}
\put(216.20,109.20){\usebox{\plotpoint}}
\put(232.32,122.26){\usebox{\plotpoint}}
\put(248.63,135.09){\usebox{\plotpoint}}
\put(265.55,147.04){\usebox{\plotpoint}}
\put(282.21,159.37){\usebox{\plotpoint}}
\put(299.44,170.86){\usebox{\plotpoint}}
\put(317.17,181.54){\usebox{\plotpoint}}
\put(335.17,191.70){\usebox{\plotpoint}}
\put(352.97,202.38){\usebox{\plotpoint}}
\put(371.78,211.07){\usebox{\plotpoint}}
\put(390.37,220.15){\usebox{\plotpoint}}
\put(409.64,227.86){\usebox{\plotpoint}}
\put(429.04,235.22){\usebox{\plotpoint}}
\put(448.58,242.03){\usebox{\plotpoint}}
\put(468.51,247.50){\usebox{\plotpoint}}
\put(488.66,252.33){\usebox{\plotpoint}}
\put(509.01,256.40){\usebox{\plotpoint}}
\put(529.41,260.00){\usebox{\plotpoint}}
\put(549.97,262.00){\usebox{\plotpoint}}
\put(570.58,263.52){\usebox{\plotpoint}}
\put(591.28,264.00){\usebox{\plotpoint}}
\put(611.94,263.00){\usebox{\plotpoint}}
\put(614,263){\usebox{\plotpoint}}
\sbox{\plotpoint}{\rule[-0.400pt]{0.800pt}{0.800pt}}%
\put(162,81.84){\rule{0.964pt}{0.800pt}}
\multiput(162.00,80.34)(2.000,3.000){2}{\rule{0.482pt}{0.800pt}}
\put(166,84.84){\rule{1.204pt}{0.800pt}}
\multiput(166.00,83.34)(2.500,3.000){2}{\rule{0.602pt}{0.800pt}}
\put(171,88.34){\rule{1.200pt}{0.800pt}}
\multiput(171.00,86.34)(2.509,4.000){2}{\rule{0.600pt}{0.800pt}}
\put(176,92.34){\rule{1.200pt}{0.800pt}}
\multiput(176.00,90.34)(2.509,4.000){2}{\rule{0.600pt}{0.800pt}}
\put(181,95.84){\rule{1.204pt}{0.800pt}}
\multiput(181.00,94.34)(2.500,3.000){2}{\rule{0.602pt}{0.800pt}}
\put(186,99.34){\rule{1.200pt}{0.800pt}}
\multiput(186.00,97.34)(2.509,4.000){2}{\rule{0.600pt}{0.800pt}}
\put(191,102.84){\rule{1.445pt}{0.800pt}}
\multiput(191.00,101.34)(3.000,3.000){2}{\rule{0.723pt}{0.800pt}}
\put(197,106.34){\rule{1.200pt}{0.800pt}}
\multiput(197.00,104.34)(2.509,4.000){2}{\rule{0.600pt}{0.800pt}}
\put(202,109.84){\rule{1.204pt}{0.800pt}}
\multiput(202.00,108.34)(2.500,3.000){2}{\rule{0.602pt}{0.800pt}}
\put(207,113.34){\rule{1.200pt}{0.800pt}}
\multiput(207.00,111.34)(2.509,4.000){2}{\rule{0.600pt}{0.800pt}}
\put(212,116.84){\rule{1.204pt}{0.800pt}}
\multiput(212.00,115.34)(2.500,3.000){2}{\rule{0.602pt}{0.800pt}}
\put(217,119.84){\rule{1.204pt}{0.800pt}}
\multiput(217.00,118.34)(2.500,3.000){2}{\rule{0.602pt}{0.800pt}}
\put(222,123.34){\rule{1.200pt}{0.800pt}}
\multiput(222.00,121.34)(2.509,4.000){2}{\rule{0.600pt}{0.800pt}}
\put(227,126.84){\rule{1.204pt}{0.800pt}}
\multiput(227.00,125.34)(2.500,3.000){2}{\rule{0.602pt}{0.800pt}}
\put(232,129.84){\rule{1.204pt}{0.800pt}}
\multiput(232.00,128.34)(2.500,3.000){2}{\rule{0.602pt}{0.800pt}}
\put(237,133.34){\rule{1.200pt}{0.800pt}}
\multiput(237.00,131.34)(2.509,4.000){2}{\rule{0.600pt}{0.800pt}}
\put(242,136.84){\rule{1.204pt}{0.800pt}}
\multiput(242.00,135.34)(2.500,3.000){2}{\rule{0.602pt}{0.800pt}}
\put(247,139.84){\rule{1.445pt}{0.800pt}}
\multiput(247.00,138.34)(3.000,3.000){2}{\rule{0.723pt}{0.800pt}}
\put(253,142.84){\rule{1.204pt}{0.800pt}}
\multiput(253.00,141.34)(2.500,3.000){2}{\rule{0.602pt}{0.800pt}}
\put(258,145.84){\rule{1.204pt}{0.800pt}}
\multiput(258.00,144.34)(2.500,3.000){2}{\rule{0.602pt}{0.800pt}}
\put(263,149.34){\rule{1.200pt}{0.800pt}}
\multiput(263.00,147.34)(2.509,4.000){2}{\rule{0.600pt}{0.800pt}}
\put(268,152.84){\rule{1.204pt}{0.800pt}}
\multiput(268.00,151.34)(2.500,3.000){2}{\rule{0.602pt}{0.800pt}}
\put(273,155.84){\rule{1.204pt}{0.800pt}}
\multiput(273.00,154.34)(2.500,3.000){2}{\rule{0.602pt}{0.800pt}}
\put(278,158.84){\rule{1.204pt}{0.800pt}}
\multiput(278.00,157.34)(2.500,3.000){2}{\rule{0.602pt}{0.800pt}}
\put(283,161.84){\rule{1.204pt}{0.800pt}}
\multiput(283.00,160.34)(2.500,3.000){2}{\rule{0.602pt}{0.800pt}}
\put(288,164.84){\rule{1.204pt}{0.800pt}}
\multiput(288.00,163.34)(2.500,3.000){2}{\rule{0.602pt}{0.800pt}}
\put(293,167.84){\rule{1.204pt}{0.800pt}}
\multiput(293.00,166.34)(2.500,3.000){2}{\rule{0.602pt}{0.800pt}}
\put(298,170.84){\rule{1.204pt}{0.800pt}}
\multiput(298.00,169.34)(2.500,3.000){2}{\rule{0.602pt}{0.800pt}}
\put(303,173.34){\rule{1.445pt}{0.800pt}}
\multiput(303.00,172.34)(3.000,2.000){2}{\rule{0.723pt}{0.800pt}}
\put(309,175.84){\rule{1.204pt}{0.800pt}}
\multiput(309.00,174.34)(2.500,3.000){2}{\rule{0.602pt}{0.800pt}}
\put(314,178.84){\rule{1.204pt}{0.800pt}}
\multiput(314.00,177.34)(2.500,3.000){2}{\rule{0.602pt}{0.800pt}}
\put(319,181.84){\rule{1.204pt}{0.800pt}}
\multiput(319.00,180.34)(2.500,3.000){2}{\rule{0.602pt}{0.800pt}}
\put(324,184.84){\rule{1.204pt}{0.800pt}}
\multiput(324.00,183.34)(2.500,3.000){2}{\rule{0.602pt}{0.800pt}}
\put(329,187.84){\rule{1.204pt}{0.800pt}}
\multiput(329.00,186.34)(2.500,3.000){2}{\rule{0.602pt}{0.800pt}}
\put(334,190.34){\rule{1.204pt}{0.800pt}}
\multiput(334.00,189.34)(2.500,2.000){2}{\rule{0.602pt}{0.800pt}}
\put(339,192.84){\rule{1.204pt}{0.800pt}}
\multiput(339.00,191.34)(2.500,3.000){2}{\rule{0.602pt}{0.800pt}}
\put(344,195.84){\rule{1.204pt}{0.800pt}}
\multiput(344.00,194.34)(2.500,3.000){2}{\rule{0.602pt}{0.800pt}}
\put(349,198.34){\rule{1.204pt}{0.800pt}}
\multiput(349.00,197.34)(2.500,2.000){2}{\rule{0.602pt}{0.800pt}}
\put(354,200.84){\rule{1.204pt}{0.800pt}}
\multiput(354.00,199.34)(2.500,3.000){2}{\rule{0.602pt}{0.800pt}}
\put(359,203.34){\rule{1.445pt}{0.800pt}}
\multiput(359.00,202.34)(3.000,2.000){2}{\rule{0.723pt}{0.800pt}}
\put(365,205.84){\rule{1.204pt}{0.800pt}}
\multiput(365.00,204.34)(2.500,3.000){2}{\rule{0.602pt}{0.800pt}}
\put(370,208.34){\rule{1.204pt}{0.800pt}}
\multiput(370.00,207.34)(2.500,2.000){2}{\rule{0.602pt}{0.800pt}}
\put(375,210.84){\rule{1.204pt}{0.800pt}}
\multiput(375.00,209.34)(2.500,3.000){2}{\rule{0.602pt}{0.800pt}}
\put(380,213.34){\rule{1.204pt}{0.800pt}}
\multiput(380.00,212.34)(2.500,2.000){2}{\rule{0.602pt}{0.800pt}}
\put(385,215.84){\rule{1.204pt}{0.800pt}}
\multiput(385.00,214.34)(2.500,3.000){2}{\rule{0.602pt}{0.800pt}}
\put(390,218.34){\rule{1.204pt}{0.800pt}}
\multiput(390.00,217.34)(2.500,2.000){2}{\rule{0.602pt}{0.800pt}}
\put(395,220.84){\rule{1.204pt}{0.800pt}}
\multiput(395.00,219.34)(2.500,3.000){2}{\rule{0.602pt}{0.800pt}}
\put(400,223.34){\rule{1.204pt}{0.800pt}}
\multiput(400.00,222.34)(2.500,2.000){2}{\rule{0.602pt}{0.800pt}}
\put(405,225.34){\rule{1.204pt}{0.800pt}}
\multiput(405.00,224.34)(2.500,2.000){2}{\rule{0.602pt}{0.800pt}}
\put(410,227.84){\rule{1.204pt}{0.800pt}}
\multiput(410.00,226.34)(2.500,3.000){2}{\rule{0.602pt}{0.800pt}}
\put(415,230.34){\rule{1.445pt}{0.800pt}}
\multiput(415.00,229.34)(3.000,2.000){2}{\rule{0.723pt}{0.800pt}}
\put(421,232.34){\rule{1.204pt}{0.800pt}}
\multiput(421.00,231.34)(2.500,2.000){2}{\rule{0.602pt}{0.800pt}}
\put(426,234.34){\rule{1.204pt}{0.800pt}}
\multiput(426.00,233.34)(2.500,2.000){2}{\rule{0.602pt}{0.800pt}}
\put(431,236.34){\rule{1.204pt}{0.800pt}}
\multiput(431.00,235.34)(2.500,2.000){2}{\rule{0.602pt}{0.800pt}}
\put(436,238.84){\rule{1.204pt}{0.800pt}}
\multiput(436.00,237.34)(2.500,3.000){2}{\rule{0.602pt}{0.800pt}}
\put(441,241.34){\rule{1.204pt}{0.800pt}}
\multiput(441.00,240.34)(2.500,2.000){2}{\rule{0.602pt}{0.800pt}}
\put(446,243.34){\rule{1.204pt}{0.800pt}}
\multiput(446.00,242.34)(2.500,2.000){2}{\rule{0.602pt}{0.800pt}}
\put(451,245.34){\rule{1.204pt}{0.800pt}}
\multiput(451.00,244.34)(2.500,2.000){2}{\rule{0.602pt}{0.800pt}}
\put(456,247.34){\rule{1.204pt}{0.800pt}}
\multiput(456.00,246.34)(2.500,2.000){2}{\rule{0.602pt}{0.800pt}}
\put(461,249.34){\rule{1.204pt}{0.800pt}}
\multiput(461.00,248.34)(2.500,2.000){2}{\rule{0.602pt}{0.800pt}}
\put(466,251.34){\rule{1.204pt}{0.800pt}}
\multiput(466.00,250.34)(2.500,2.000){2}{\rule{0.602pt}{0.800pt}}
\put(471,253.34){\rule{1.445pt}{0.800pt}}
\multiput(471.00,252.34)(3.000,2.000){2}{\rule{0.723pt}{0.800pt}}
\put(477,255.34){\rule{1.204pt}{0.800pt}}
\multiput(477.00,254.34)(2.500,2.000){2}{\rule{0.602pt}{0.800pt}}
\put(482,256.84){\rule{1.204pt}{0.800pt}}
\multiput(482.00,256.34)(2.500,1.000){2}{\rule{0.602pt}{0.800pt}}
\put(487,258.34){\rule{1.204pt}{0.800pt}}
\multiput(487.00,257.34)(2.500,2.000){2}{\rule{0.602pt}{0.800pt}}
\put(492,260.34){\rule{1.204pt}{0.800pt}}
\multiput(492.00,259.34)(2.500,2.000){2}{\rule{0.602pt}{0.800pt}}
\put(497,262.34){\rule{1.204pt}{0.800pt}}
\multiput(497.00,261.34)(2.500,2.000){2}{\rule{0.602pt}{0.800pt}}
\put(502,264.34){\rule{1.204pt}{0.800pt}}
\multiput(502.00,263.34)(2.500,2.000){2}{\rule{0.602pt}{0.800pt}}
\put(507,265.84){\rule{1.204pt}{0.800pt}}
\multiput(507.00,265.34)(2.500,1.000){2}{\rule{0.602pt}{0.800pt}}
\put(512,267.34){\rule{1.204pt}{0.800pt}}
\multiput(512.00,266.34)(2.500,2.000){2}{\rule{0.602pt}{0.800pt}}
\put(517,269.34){\rule{1.204pt}{0.800pt}}
\multiput(517.00,268.34)(2.500,2.000){2}{\rule{0.602pt}{0.800pt}}
\put(522,270.84){\rule{1.204pt}{0.800pt}}
\multiput(522.00,270.34)(2.500,1.000){2}{\rule{0.602pt}{0.800pt}}
\put(527,272.34){\rule{1.445pt}{0.800pt}}
\multiput(527.00,271.34)(3.000,2.000){2}{\rule{0.723pt}{0.800pt}}
\put(533,274.34){\rule{1.204pt}{0.800pt}}
\multiput(533.00,273.34)(2.500,2.000){2}{\rule{0.602pt}{0.800pt}}
\put(538,275.84){\rule{1.204pt}{0.800pt}}
\multiput(538.00,275.34)(2.500,1.000){2}{\rule{0.602pt}{0.800pt}}
\put(543,277.34){\rule{1.204pt}{0.800pt}}
\multiput(543.00,276.34)(2.500,2.000){2}{\rule{0.602pt}{0.800pt}}
\put(548,278.84){\rule{1.204pt}{0.800pt}}
\multiput(548.00,278.34)(2.500,1.000){2}{\rule{0.602pt}{0.800pt}}
\put(553,280.34){\rule{1.204pt}{0.800pt}}
\multiput(553.00,279.34)(2.500,2.000){2}{\rule{0.602pt}{0.800pt}}
\put(558,281.84){\rule{1.204pt}{0.800pt}}
\multiput(558.00,281.34)(2.500,1.000){2}{\rule{0.602pt}{0.800pt}}
\put(563,283.34){\rule{1.204pt}{0.800pt}}
\multiput(563.00,282.34)(2.500,2.000){2}{\rule{0.602pt}{0.800pt}}
\put(568,284.84){\rule{1.204pt}{0.800pt}}
\multiput(568.00,284.34)(2.500,1.000){2}{\rule{0.602pt}{0.800pt}}
\put(573,285.84){\rule{1.204pt}{0.800pt}}
\multiput(573.00,285.34)(2.500,1.000){2}{\rule{0.602pt}{0.800pt}}
\put(578,287.34){\rule{1.204pt}{0.800pt}}
\multiput(578.00,286.34)(2.500,2.000){2}{\rule{0.602pt}{0.800pt}}
\put(583,288.84){\rule{1.445pt}{0.800pt}}
\multiput(583.00,288.34)(3.000,1.000){2}{\rule{0.723pt}{0.800pt}}
\put(589,289.84){\rule{1.204pt}{0.800pt}}
\multiput(589.00,289.34)(2.500,1.000){2}{\rule{0.602pt}{0.800pt}}
\put(594,291.34){\rule{1.204pt}{0.800pt}}
\multiput(594.00,290.34)(2.500,2.000){2}{\rule{0.602pt}{0.800pt}}
\put(599,292.84){\rule{1.204pt}{0.800pt}}
\multiput(599.00,292.34)(2.500,1.000){2}{\rule{0.602pt}{0.800pt}}
\put(604,293.84){\rule{1.204pt}{0.800pt}}
\multiput(604.00,293.34)(2.500,1.000){2}{\rule{0.602pt}{0.800pt}}
\put(609,294.84){\rule{1.204pt}{0.800pt}}
\multiput(609.00,294.34)(2.500,1.000){2}{\rule{0.602pt}{0.800pt}}
\sbox{\plotpoint}{\rule[-0.500pt]{1.000pt}{1.000pt}}%
\put(121.00,82.00){\usebox{\plotpoint}}
\put(139.53,91.27){\usebox{\plotpoint}}
\put(158.16,100.29){\usebox{\plotpoint}}
\put(176.36,110.15){\usebox{\plotpoint}}
\put(194.89,119.30){\usebox{\plotpoint}}
\put(213.38,128.55){\usebox{\plotpoint}}
\put(231.82,137.93){\usebox{\plotpoint}}
\put(250.33,147.11){\usebox{\plotpoint}}
\put(268.77,156.46){\usebox{\plotpoint}}
\put(287.28,165.71){\usebox{\plotpoint}}
\put(305.62,175.31){\usebox{\plotpoint}}
\put(324.32,184.19){\usebox{\plotpoint}}
\put(342.50,194.10){\usebox{\plotpoint}}
\put(361.15,203.08){\usebox{\plotpoint}}
\put(379.49,212.70){\usebox{\plotpoint}}
\put(398.05,221.83){\usebox{\plotpoint}}
\put(416.37,231.46){\usebox{\plotpoint}}
\put(434.99,240.39){\usebox{\plotpoint}}
\put(453.35,249.94){\usebox{\plotpoint}}
\put(471.77,259.38){\usebox{\plotpoint}}
\put(490.42,268.37){\usebox{\plotpoint}}
\put(508.72,278.03){\usebox{\plotpoint}}
\put(527.30,287.15){\usebox{\plotpoint}}
\put(545.71,296.63){\usebox{\plotpoint}}
\put(564.28,305.77){\usebox{\plotpoint}}
\put(582.46,315.67){\usebox{\plotpoint}}
\put(601.21,324.33){\usebox{\plotpoint}}
\put(614,331){\usebox{\plotpoint}}
\sbox{\plotpoint}{\rule[-0.600pt]{1.200pt}{1.200pt}}%
\put(110,110){\usebox{\plotpoint}}
\put(110,108.01){\rule{1.204pt}{1.200pt}}
\multiput(110.00,107.51)(2.500,1.000){2}{\rule{0.602pt}{1.200pt}}
\put(115,109.51){\rule{1.204pt}{1.200pt}}
\multiput(115.00,108.51)(2.500,2.000){2}{\rule{0.602pt}{1.200pt}}
\put(120,111.01){\rule{1.204pt}{1.200pt}}
\multiput(120.00,110.51)(2.500,1.000){2}{\rule{0.602pt}{1.200pt}}
\put(125,112.01){\rule{1.204pt}{1.200pt}}
\multiput(125.00,111.51)(2.500,1.000){2}{\rule{0.602pt}{1.200pt}}
\put(130,113.01){\rule{1.204pt}{1.200pt}}
\multiput(130.00,112.51)(2.500,1.000){2}{\rule{0.602pt}{1.200pt}}
\put(135,114.51){\rule{1.445pt}{1.200pt}}
\multiput(135.00,113.51)(3.000,2.000){2}{\rule{0.723pt}{1.200pt}}
\put(141,116.01){\rule{1.204pt}{1.200pt}}
\multiput(141.00,115.51)(2.500,1.000){2}{\rule{0.602pt}{1.200pt}}
\put(146,117.01){\rule{1.204pt}{1.200pt}}
\multiput(146.00,116.51)(2.500,1.000){2}{\rule{0.602pt}{1.200pt}}
\put(151,118.51){\rule{1.204pt}{1.200pt}}
\multiput(151.00,117.51)(2.500,2.000){2}{\rule{0.602pt}{1.200pt}}
\put(156,120.01){\rule{1.204pt}{1.200pt}}
\multiput(156.00,119.51)(2.500,1.000){2}{\rule{0.602pt}{1.200pt}}
\put(161,121.51){\rule{1.204pt}{1.200pt}}
\multiput(161.00,120.51)(2.500,2.000){2}{\rule{0.602pt}{1.200pt}}
\put(166,123.01){\rule{1.204pt}{1.200pt}}
\multiput(166.00,122.51)(2.500,1.000){2}{\rule{0.602pt}{1.200pt}}
\put(171,124.51){\rule{1.204pt}{1.200pt}}
\multiput(171.00,123.51)(2.500,2.000){2}{\rule{0.602pt}{1.200pt}}
\put(176,126.01){\rule{1.204pt}{1.200pt}}
\multiput(176.00,125.51)(2.500,1.000){2}{\rule{0.602pt}{1.200pt}}
\put(181,127.51){\rule{1.204pt}{1.200pt}}
\multiput(181.00,126.51)(2.500,2.000){2}{\rule{0.602pt}{1.200pt}}
\put(186,129.01){\rule{1.204pt}{1.200pt}}
\multiput(186.00,128.51)(2.500,1.000){2}{\rule{0.602pt}{1.200pt}}
\put(191,130.51){\rule{1.445pt}{1.200pt}}
\multiput(191.00,129.51)(3.000,2.000){2}{\rule{0.723pt}{1.200pt}}
\put(197,132.51){\rule{1.204pt}{1.200pt}}
\multiput(197.00,131.51)(2.500,2.000){2}{\rule{0.602pt}{1.200pt}}
\put(202,134.01){\rule{1.204pt}{1.200pt}}
\multiput(202.00,133.51)(2.500,1.000){2}{\rule{0.602pt}{1.200pt}}
\put(207,135.51){\rule{1.204pt}{1.200pt}}
\multiput(207.00,134.51)(2.500,2.000){2}{\rule{0.602pt}{1.200pt}}
\put(212,137.51){\rule{1.204pt}{1.200pt}}
\multiput(212.00,136.51)(2.500,2.000){2}{\rule{0.602pt}{1.200pt}}
\put(217,139.01){\rule{1.204pt}{1.200pt}}
\multiput(217.00,138.51)(2.500,1.000){2}{\rule{0.602pt}{1.200pt}}
\put(222,140.51){\rule{1.204pt}{1.200pt}}
\multiput(222.00,139.51)(2.500,2.000){2}{\rule{0.602pt}{1.200pt}}
\put(227,142.51){\rule{1.204pt}{1.200pt}}
\multiput(227.00,141.51)(2.500,2.000){2}{\rule{0.602pt}{1.200pt}}
\put(232,144.51){\rule{1.204pt}{1.200pt}}
\multiput(232.00,143.51)(2.500,2.000){2}{\rule{0.602pt}{1.200pt}}
\put(237,146.51){\rule{1.204pt}{1.200pt}}
\multiput(237.00,145.51)(2.500,2.000){2}{\rule{0.602pt}{1.200pt}}
\put(242,148.51){\rule{1.204pt}{1.200pt}}
\multiput(242.00,147.51)(2.500,2.000){2}{\rule{0.602pt}{1.200pt}}
\put(247,150.51){\rule{1.445pt}{1.200pt}}
\multiput(247.00,149.51)(3.000,2.000){2}{\rule{0.723pt}{1.200pt}}
\put(253,152.51){\rule{1.204pt}{1.200pt}}
\multiput(253.00,151.51)(2.500,2.000){2}{\rule{0.602pt}{1.200pt}}
\put(258,154.51){\rule{1.204pt}{1.200pt}}
\multiput(258.00,153.51)(2.500,2.000){2}{\rule{0.602pt}{1.200pt}}
\put(263,156.51){\rule{1.204pt}{1.200pt}}
\multiput(263.00,155.51)(2.500,2.000){2}{\rule{0.602pt}{1.200pt}}
\put(268,158.51){\rule{1.204pt}{1.200pt}}
\multiput(268.00,157.51)(2.500,2.000){2}{\rule{0.602pt}{1.200pt}}
\put(273,160.51){\rule{1.204pt}{1.200pt}}
\multiput(273.00,159.51)(2.500,2.000){2}{\rule{0.602pt}{1.200pt}}
\put(278,162.51){\rule{1.204pt}{1.200pt}}
\multiput(278.00,161.51)(2.500,2.000){2}{\rule{0.602pt}{1.200pt}}
\put(283,164.51){\rule{1.204pt}{1.200pt}}
\multiput(283.00,163.51)(2.500,2.000){2}{\rule{0.602pt}{1.200pt}}
\put(288,166.51){\rule{1.204pt}{1.200pt}}
\multiput(288.00,165.51)(2.500,2.000){2}{\rule{0.602pt}{1.200pt}}
\put(293,168.51){\rule{1.204pt}{1.200pt}}
\multiput(293.00,167.51)(2.500,2.000){2}{\rule{0.602pt}{1.200pt}}
\put(298,171.01){\rule{1.204pt}{1.200pt}}
\multiput(298.00,169.51)(2.500,3.000){2}{\rule{0.602pt}{1.200pt}}
\put(303,173.51){\rule{1.445pt}{1.200pt}}
\multiput(303.00,172.51)(3.000,2.000){2}{\rule{0.723pt}{1.200pt}}
\put(309,175.51){\rule{1.204pt}{1.200pt}}
\multiput(309.00,174.51)(2.500,2.000){2}{\rule{0.602pt}{1.200pt}}
\put(314,177.51){\rule{1.204pt}{1.200pt}}
\multiput(314.00,176.51)(2.500,2.000){2}{\rule{0.602pt}{1.200pt}}
\put(319,180.01){\rule{1.204pt}{1.200pt}}
\multiput(319.00,178.51)(2.500,3.000){2}{\rule{0.602pt}{1.200pt}}
\put(324,182.51){\rule{1.204pt}{1.200pt}}
\multiput(324.00,181.51)(2.500,2.000){2}{\rule{0.602pt}{1.200pt}}
\put(329,185.01){\rule{1.204pt}{1.200pt}}
\multiput(329.00,183.51)(2.500,3.000){2}{\rule{0.602pt}{1.200pt}}
\put(334,187.51){\rule{1.204pt}{1.200pt}}
\multiput(334.00,186.51)(2.500,2.000){2}{\rule{0.602pt}{1.200pt}}
\put(339,189.51){\rule{1.204pt}{1.200pt}}
\multiput(339.00,188.51)(2.500,2.000){2}{\rule{0.602pt}{1.200pt}}
\put(344,192.01){\rule{1.204pt}{1.200pt}}
\multiput(344.00,190.51)(2.500,3.000){2}{\rule{0.602pt}{1.200pt}}
\put(349,194.51){\rule{1.204pt}{1.200pt}}
\multiput(349.00,193.51)(2.500,2.000){2}{\rule{0.602pt}{1.200pt}}
\put(354,197.01){\rule{1.204pt}{1.200pt}}
\multiput(354.00,195.51)(2.500,3.000){2}{\rule{0.602pt}{1.200pt}}
\put(359,200.01){\rule{1.445pt}{1.200pt}}
\multiput(359.00,198.51)(3.000,3.000){2}{\rule{0.723pt}{1.200pt}}
\put(365,202.51){\rule{1.204pt}{1.200pt}}
\multiput(365.00,201.51)(2.500,2.000){2}{\rule{0.602pt}{1.200pt}}
\put(370,205.01){\rule{1.204pt}{1.200pt}}
\multiput(370.00,203.51)(2.500,3.000){2}{\rule{0.602pt}{1.200pt}}
\put(375,207.51){\rule{1.204pt}{1.200pt}}
\multiput(375.00,206.51)(2.500,2.000){2}{\rule{0.602pt}{1.200pt}}
\put(380,210.01){\rule{1.204pt}{1.200pt}}
\multiput(380.00,208.51)(2.500,3.000){2}{\rule{0.602pt}{1.200pt}}
\put(385,213.01){\rule{1.204pt}{1.200pt}}
\multiput(385.00,211.51)(2.500,3.000){2}{\rule{0.602pt}{1.200pt}}
\put(390,216.01){\rule{1.204pt}{1.200pt}}
\multiput(390.00,214.51)(2.500,3.000){2}{\rule{0.602pt}{1.200pt}}
\put(395,218.51){\rule{1.204pt}{1.200pt}}
\multiput(395.00,217.51)(2.500,2.000){2}{\rule{0.602pt}{1.200pt}}
\put(400,221.01){\rule{1.204pt}{1.200pt}}
\multiput(400.00,219.51)(2.500,3.000){2}{\rule{0.602pt}{1.200pt}}
\put(405,224.01){\rule{1.204pt}{1.200pt}}
\multiput(405.00,222.51)(2.500,3.000){2}{\rule{0.602pt}{1.200pt}}
\put(410,227.01){\rule{1.204pt}{1.200pt}}
\multiput(410.00,225.51)(2.500,3.000){2}{\rule{0.602pt}{1.200pt}}
\put(415,230.01){\rule{1.445pt}{1.200pt}}
\multiput(415.00,228.51)(3.000,3.000){2}{\rule{0.723pt}{1.200pt}}
\put(421,233.01){\rule{1.204pt}{1.200pt}}
\multiput(421.00,231.51)(2.500,3.000){2}{\rule{0.602pt}{1.200pt}}
\put(426,236.01){\rule{1.204pt}{1.200pt}}
\multiput(426.00,234.51)(2.500,3.000){2}{\rule{0.602pt}{1.200pt}}
\put(431,239.01){\rule{1.204pt}{1.200pt}}
\multiput(431.00,237.51)(2.500,3.000){2}{\rule{0.602pt}{1.200pt}}
\put(436,242.01){\rule{1.204pt}{1.200pt}}
\multiput(436.00,240.51)(2.500,3.000){2}{\rule{0.602pt}{1.200pt}}
\put(441,245.01){\rule{1.204pt}{1.200pt}}
\multiput(441.00,243.51)(2.500,3.000){2}{\rule{0.602pt}{1.200pt}}
\put(446,248.01){\rule{1.204pt}{1.200pt}}
\multiput(446.00,246.51)(2.500,3.000){2}{\rule{0.602pt}{1.200pt}}
\put(451,251.01){\rule{1.204pt}{1.200pt}}
\multiput(451.00,249.51)(2.500,3.000){2}{\rule{0.602pt}{1.200pt}}
\put(456,254.01){\rule{1.204pt}{1.200pt}}
\multiput(456.00,252.51)(2.500,3.000){2}{\rule{0.602pt}{1.200pt}}
\put(461,257.01){\rule{1.204pt}{1.200pt}}
\multiput(461.00,255.51)(2.500,3.000){2}{\rule{0.602pt}{1.200pt}}
\put(466,260.01){\rule{1.204pt}{1.200pt}}
\multiput(466.00,258.51)(2.500,3.000){2}{\rule{0.602pt}{1.200pt}}
\put(471,263.01){\rule{1.445pt}{1.200pt}}
\multiput(471.00,261.51)(3.000,3.000){2}{\rule{0.723pt}{1.200pt}}
\put(477,266.51){\rule{1.204pt}{1.200pt}}
\multiput(477.00,264.51)(2.500,4.000){2}{\rule{0.602pt}{1.200pt}}
\put(482,270.01){\rule{1.204pt}{1.200pt}}
\multiput(482.00,268.51)(2.500,3.000){2}{\rule{0.602pt}{1.200pt}}
\put(487,273.01){\rule{1.204pt}{1.200pt}}
\multiput(487.00,271.51)(2.500,3.000){2}{\rule{0.602pt}{1.200pt}}
\put(492,276.51){\rule{1.204pt}{1.200pt}}
\multiput(492.00,274.51)(2.500,4.000){2}{\rule{0.602pt}{1.200pt}}
\put(497,280.01){\rule{1.204pt}{1.200pt}}
\multiput(497.00,278.51)(2.500,3.000){2}{\rule{0.602pt}{1.200pt}}
\put(502,283.01){\rule{1.204pt}{1.200pt}}
\multiput(502.00,281.51)(2.500,3.000){2}{\rule{0.602pt}{1.200pt}}
\put(507,286.51){\rule{1.204pt}{1.200pt}}
\multiput(507.00,284.51)(2.500,4.000){2}{\rule{0.602pt}{1.200pt}}
\put(512,290.01){\rule{1.204pt}{1.200pt}}
\multiput(512.00,288.51)(2.500,3.000){2}{\rule{0.602pt}{1.200pt}}
\put(517,293.51){\rule{1.204pt}{1.200pt}}
\multiput(517.00,291.51)(2.500,4.000){2}{\rule{0.602pt}{1.200pt}}
\put(522,297.01){\rule{1.204pt}{1.200pt}}
\multiput(522.00,295.51)(2.500,3.000){2}{\rule{0.602pt}{1.200pt}}
\put(527,300.51){\rule{1.445pt}{1.200pt}}
\multiput(527.00,298.51)(3.000,4.000){2}{\rule{0.723pt}{1.200pt}}
\put(533,304.01){\rule{1.204pt}{1.200pt}}
\multiput(533.00,302.51)(2.500,3.000){2}{\rule{0.602pt}{1.200pt}}
\put(538,307.51){\rule{1.204pt}{1.200pt}}
\multiput(538.00,305.51)(2.500,4.000){2}{\rule{0.602pt}{1.200pt}}
\put(543,311.01){\rule{1.204pt}{1.200pt}}
\multiput(543.00,309.51)(2.500,3.000){2}{\rule{0.602pt}{1.200pt}}
\put(548,314.51){\rule{1.204pt}{1.200pt}}
\multiput(548.00,312.51)(2.500,4.000){2}{\rule{0.602pt}{1.200pt}}
\put(553,318.51){\rule{1.204pt}{1.200pt}}
\multiput(553.00,316.51)(2.500,4.000){2}{\rule{0.602pt}{1.200pt}}
\put(558,322.01){\rule{1.204pt}{1.200pt}}
\multiput(558.00,320.51)(2.500,3.000){2}{\rule{0.602pt}{1.200pt}}
\put(563,325.51){\rule{1.204pt}{1.200pt}}
\multiput(563.00,323.51)(2.500,4.000){2}{\rule{0.602pt}{1.200pt}}
\put(568,329.51){\rule{1.204pt}{1.200pt}}
\multiput(568.00,327.51)(2.500,4.000){2}{\rule{0.602pt}{1.200pt}}
\put(573,333.51){\rule{1.204pt}{1.200pt}}
\multiput(573.00,331.51)(2.500,4.000){2}{\rule{0.602pt}{1.200pt}}
\put(578,337.51){\rule{1.204pt}{1.200pt}}
\multiput(578.00,335.51)(2.500,4.000){2}{\rule{0.602pt}{1.200pt}}
\put(583,341.01){\rule{1.445pt}{1.200pt}}
\multiput(583.00,339.51)(3.000,3.000){2}{\rule{0.723pt}{1.200pt}}
\put(589,344.51){\rule{1.204pt}{1.200pt}}
\multiput(589.00,342.51)(2.500,4.000){2}{\rule{0.602pt}{1.200pt}}
\put(594,348.51){\rule{1.204pt}{1.200pt}}
\multiput(594.00,346.51)(2.500,4.000){2}{\rule{0.602pt}{1.200pt}}
\put(599,352.51){\rule{1.204pt}{1.200pt}}
\multiput(599.00,350.51)(2.500,4.000){2}{\rule{0.602pt}{1.200pt}}
\put(604,356.51){\rule{1.204pt}{1.200pt}}
\multiput(604.00,354.51)(2.500,4.000){2}{\rule{0.602pt}{1.200pt}}
\put(609,360.51){\rule{1.204pt}{1.200pt}}
\multiput(609.00,358.51)(2.500,4.000){2}{\rule{0.602pt}{1.200pt}}
\sbox{\plotpoint}{\rule[-0.500pt]{1.000pt}{1.000pt}}%
\put(110,144){\usebox{\plotpoint}}
\put(110.00,144.00){\usebox{\plotpoint}}
\put(151.51,144.00){\usebox{\plotpoint}}
\put(192.70,147.28){\usebox{\plotpoint}}
\put(233.53,154.31){\usebox{\plotpoint}}
\put(273.43,165.09){\usebox{\plotpoint}}
\put(312.61,178.44){\usebox{\plotpoint}}
\put(350.32,195.53){\usebox{\plotpoint}}
\put(387.16,214.29){\usebox{\plotpoint}}
\put(422.82,235.46){\usebox{\plotpoint}}
\put(457.02,258.82){\usebox{\plotpoint}}
\put(490.20,283.56){\usebox{\plotpoint}}
\put(521.61,310.61){\usebox{\plotpoint}}
\put(552.81,337.81){\usebox{\plotpoint}}
\put(581,365){\usebox{\plotpoint}}
\sbox{\plotpoint}{\rule[-0.200pt]{0.400pt}{0.400pt}}%
\put(110,178){\usebox{\plotpoint}}
\put(110,176.67){\rule{1.204pt}{0.400pt}}
\multiput(110.00,177.17)(2.500,-1.000){2}{\rule{0.602pt}{0.400pt}}
\put(115,175.17){\rule{1.100pt}{0.400pt}}
\multiput(115.00,176.17)(2.717,-2.000){2}{\rule{0.550pt}{0.400pt}}
\put(120,173.17){\rule{1.100pt}{0.400pt}}
\multiput(120.00,174.17)(2.717,-2.000){2}{\rule{0.550pt}{0.400pt}}
\put(125,171.67){\rule{1.204pt}{0.400pt}}
\multiput(125.00,172.17)(2.500,-1.000){2}{\rule{0.602pt}{0.400pt}}
\put(130,170.67){\rule{1.204pt}{0.400pt}}
\multiput(130.00,171.17)(2.500,-1.000){2}{\rule{0.602pt}{0.400pt}}
\put(135,169.17){\rule{1.300pt}{0.400pt}}
\multiput(135.00,170.17)(3.302,-2.000){2}{\rule{0.650pt}{0.400pt}}
\put(141,167.67){\rule{1.204pt}{0.400pt}}
\multiput(141.00,168.17)(2.500,-1.000){2}{\rule{0.602pt}{0.400pt}}
\put(146,166.67){\rule{1.204pt}{0.400pt}}
\multiput(146.00,167.17)(2.500,-1.000){2}{\rule{0.602pt}{0.400pt}}
\put(151,165.67){\rule{1.204pt}{0.400pt}}
\multiput(151.00,166.17)(2.500,-1.000){2}{\rule{0.602pt}{0.400pt}}
\put(156,164.67){\rule{1.204pt}{0.400pt}}
\multiput(156.00,165.17)(2.500,-1.000){2}{\rule{0.602pt}{0.400pt}}
\put(166,163.67){\rule{1.204pt}{0.400pt}}
\multiput(166.00,164.17)(2.500,-1.000){2}{\rule{0.602pt}{0.400pt}}
\put(171,162.67){\rule{1.204pt}{0.400pt}}
\multiput(171.00,163.17)(2.500,-1.000){2}{\rule{0.602pt}{0.400pt}}
\put(161.0,165.0){\rule[-0.200pt]{1.204pt}{0.400pt}}
\put(181,161.67){\rule{1.204pt}{0.400pt}}
\multiput(181.00,162.17)(2.500,-1.000){2}{\rule{0.602pt}{0.400pt}}
\put(176.0,163.0){\rule[-0.200pt]{1.204pt}{0.400pt}}
\put(197,160.67){\rule{1.204pt}{0.400pt}}
\multiput(197.00,161.17)(2.500,-1.000){2}{\rule{0.602pt}{0.400pt}}
\put(186.0,162.0){\rule[-0.200pt]{2.650pt}{0.400pt}}
\put(222,160.67){\rule{1.204pt}{0.400pt}}
\multiput(222.00,160.17)(2.500,1.000){2}{\rule{0.602pt}{0.400pt}}
\put(202.0,161.0){\rule[-0.200pt]{4.818pt}{0.400pt}}
\put(237,161.67){\rule{1.204pt}{0.400pt}}
\multiput(237.00,161.17)(2.500,1.000){2}{\rule{0.602pt}{0.400pt}}
\put(227.0,162.0){\rule[-0.200pt]{2.409pt}{0.400pt}}
\put(247,162.67){\rule{1.445pt}{0.400pt}}
\multiput(247.00,162.17)(3.000,1.000){2}{\rule{0.723pt}{0.400pt}}
\put(253,163.67){\rule{1.204pt}{0.400pt}}
\multiput(253.00,163.17)(2.500,1.000){2}{\rule{0.602pt}{0.400pt}}
\put(258,164.67){\rule{1.204pt}{0.400pt}}
\multiput(258.00,164.17)(2.500,1.000){2}{\rule{0.602pt}{0.400pt}}
\put(263,165.67){\rule{1.204pt}{0.400pt}}
\multiput(263.00,165.17)(2.500,1.000){2}{\rule{0.602pt}{0.400pt}}
\put(268,166.67){\rule{1.204pt}{0.400pt}}
\multiput(268.00,166.17)(2.500,1.000){2}{\rule{0.602pt}{0.400pt}}
\put(273,167.67){\rule{1.204pt}{0.400pt}}
\multiput(273.00,167.17)(2.500,1.000){2}{\rule{0.602pt}{0.400pt}}
\put(278,168.67){\rule{1.204pt}{0.400pt}}
\multiput(278.00,168.17)(2.500,1.000){2}{\rule{0.602pt}{0.400pt}}
\put(283,169.67){\rule{1.204pt}{0.400pt}}
\multiput(283.00,169.17)(2.500,1.000){2}{\rule{0.602pt}{0.400pt}}
\put(288,171.17){\rule{1.100pt}{0.400pt}}
\multiput(288.00,170.17)(2.717,2.000){2}{\rule{0.550pt}{0.400pt}}
\put(293,172.67){\rule{1.204pt}{0.400pt}}
\multiput(293.00,172.17)(2.500,1.000){2}{\rule{0.602pt}{0.400pt}}
\put(298,174.17){\rule{1.100pt}{0.400pt}}
\multiput(298.00,173.17)(2.717,2.000){2}{\rule{0.550pt}{0.400pt}}
\put(303,175.67){\rule{1.445pt}{0.400pt}}
\multiput(303.00,175.17)(3.000,1.000){2}{\rule{0.723pt}{0.400pt}}
\put(309,177.17){\rule{1.100pt}{0.400pt}}
\multiput(309.00,176.17)(2.717,2.000){2}{\rule{0.550pt}{0.400pt}}
\put(314,179.17){\rule{1.100pt}{0.400pt}}
\multiput(314.00,178.17)(2.717,2.000){2}{\rule{0.550pt}{0.400pt}}
\put(319,181.17){\rule{1.100pt}{0.400pt}}
\multiput(319.00,180.17)(2.717,2.000){2}{\rule{0.550pt}{0.400pt}}
\put(324,183.17){\rule{1.100pt}{0.400pt}}
\multiput(324.00,182.17)(2.717,2.000){2}{\rule{0.550pt}{0.400pt}}
\put(329,185.17){\rule{1.100pt}{0.400pt}}
\multiput(329.00,184.17)(2.717,2.000){2}{\rule{0.550pt}{0.400pt}}
\put(334,187.17){\rule{1.100pt}{0.400pt}}
\multiput(334.00,186.17)(2.717,2.000){2}{\rule{0.550pt}{0.400pt}}
\put(339,189.17){\rule{1.100pt}{0.400pt}}
\multiput(339.00,188.17)(2.717,2.000){2}{\rule{0.550pt}{0.400pt}}
\put(344,191.17){\rule{1.100pt}{0.400pt}}
\multiput(344.00,190.17)(2.717,2.000){2}{\rule{0.550pt}{0.400pt}}
\multiput(349.00,193.61)(0.909,0.447){3}{\rule{0.767pt}{0.108pt}}
\multiput(349.00,192.17)(3.409,3.000){2}{\rule{0.383pt}{0.400pt}}
\put(354,196.17){\rule{1.100pt}{0.400pt}}
\multiput(354.00,195.17)(2.717,2.000){2}{\rule{0.550pt}{0.400pt}}
\multiput(359.00,198.61)(1.132,0.447){3}{\rule{0.900pt}{0.108pt}}
\multiput(359.00,197.17)(4.132,3.000){2}{\rule{0.450pt}{0.400pt}}
\put(365,201.17){\rule{1.100pt}{0.400pt}}
\multiput(365.00,200.17)(2.717,2.000){2}{\rule{0.550pt}{0.400pt}}
\multiput(370.00,203.61)(0.909,0.447){3}{\rule{0.767pt}{0.108pt}}
\multiput(370.00,202.17)(3.409,3.000){2}{\rule{0.383pt}{0.400pt}}
\multiput(375.00,206.61)(0.909,0.447){3}{\rule{0.767pt}{0.108pt}}
\multiput(375.00,205.17)(3.409,3.000){2}{\rule{0.383pt}{0.400pt}}
\multiput(380.00,209.61)(0.909,0.447){3}{\rule{0.767pt}{0.108pt}}
\multiput(380.00,208.17)(3.409,3.000){2}{\rule{0.383pt}{0.400pt}}
\multiput(385.00,212.61)(0.909,0.447){3}{\rule{0.767pt}{0.108pt}}
\multiput(385.00,211.17)(3.409,3.000){2}{\rule{0.383pt}{0.400pt}}
\multiput(390.00,215.61)(0.909,0.447){3}{\rule{0.767pt}{0.108pt}}
\multiput(390.00,214.17)(3.409,3.000){2}{\rule{0.383pt}{0.400pt}}
\multiput(395.00,218.61)(0.909,0.447){3}{\rule{0.767pt}{0.108pt}}
\multiput(395.00,217.17)(3.409,3.000){2}{\rule{0.383pt}{0.400pt}}
\multiput(400.00,221.61)(0.909,0.447){3}{\rule{0.767pt}{0.108pt}}
\multiput(400.00,220.17)(3.409,3.000){2}{\rule{0.383pt}{0.400pt}}
\multiput(405.00,224.60)(0.627,0.468){5}{\rule{0.600pt}{0.113pt}}
\multiput(405.00,223.17)(3.755,4.000){2}{\rule{0.300pt}{0.400pt}}
\multiput(410.00,228.61)(0.909,0.447){3}{\rule{0.767pt}{0.108pt}}
\multiput(410.00,227.17)(3.409,3.000){2}{\rule{0.383pt}{0.400pt}}
\multiput(415.00,231.60)(0.774,0.468){5}{\rule{0.700pt}{0.113pt}}
\multiput(415.00,230.17)(4.547,4.000){2}{\rule{0.350pt}{0.400pt}}
\multiput(421.00,235.61)(0.909,0.447){3}{\rule{0.767pt}{0.108pt}}
\multiput(421.00,234.17)(3.409,3.000){2}{\rule{0.383pt}{0.400pt}}
\multiput(426.00,238.60)(0.627,0.468){5}{\rule{0.600pt}{0.113pt}}
\multiput(426.00,237.17)(3.755,4.000){2}{\rule{0.300pt}{0.400pt}}
\multiput(431.00,242.60)(0.627,0.468){5}{\rule{0.600pt}{0.113pt}}
\multiput(431.00,241.17)(3.755,4.000){2}{\rule{0.300pt}{0.400pt}}
\multiput(436.00,246.60)(0.627,0.468){5}{\rule{0.600pt}{0.113pt}}
\multiput(436.00,245.17)(3.755,4.000){2}{\rule{0.300pt}{0.400pt}}
\multiput(441.00,250.60)(0.627,0.468){5}{\rule{0.600pt}{0.113pt}}
\multiput(441.00,249.17)(3.755,4.000){2}{\rule{0.300pt}{0.400pt}}
\multiput(446.00,254.60)(0.627,0.468){5}{\rule{0.600pt}{0.113pt}}
\multiput(446.00,253.17)(3.755,4.000){2}{\rule{0.300pt}{0.400pt}}
\multiput(451.00,258.60)(0.627,0.468){5}{\rule{0.600pt}{0.113pt}}
\multiput(451.00,257.17)(3.755,4.000){2}{\rule{0.300pt}{0.400pt}}
\multiput(456.00,262.60)(0.627,0.468){5}{\rule{0.600pt}{0.113pt}}
\multiput(456.00,261.17)(3.755,4.000){2}{\rule{0.300pt}{0.400pt}}
\multiput(461.00,266.60)(0.627,0.468){5}{\rule{0.600pt}{0.113pt}}
\multiput(461.00,265.17)(3.755,4.000){2}{\rule{0.300pt}{0.400pt}}
\multiput(466.00,270.59)(0.487,0.477){7}{\rule{0.500pt}{0.115pt}}
\multiput(466.00,269.17)(3.962,5.000){2}{\rule{0.250pt}{0.400pt}}
\multiput(471.00,275.60)(0.774,0.468){5}{\rule{0.700pt}{0.113pt}}
\multiput(471.00,274.17)(4.547,4.000){2}{\rule{0.350pt}{0.400pt}}
\multiput(477.00,279.59)(0.487,0.477){7}{\rule{0.500pt}{0.115pt}}
\multiput(477.00,278.17)(3.962,5.000){2}{\rule{0.250pt}{0.400pt}}
\multiput(482.00,284.60)(0.627,0.468){5}{\rule{0.600pt}{0.113pt}}
\multiput(482.00,283.17)(3.755,4.000){2}{\rule{0.300pt}{0.400pt}}
\multiput(487.00,288.59)(0.487,0.477){7}{\rule{0.500pt}{0.115pt}}
\multiput(487.00,287.17)(3.962,5.000){2}{\rule{0.250pt}{0.400pt}}
\multiput(492.00,293.59)(0.487,0.477){7}{\rule{0.500pt}{0.115pt}}
\multiput(492.00,292.17)(3.962,5.000){2}{\rule{0.250pt}{0.400pt}}
\multiput(497.00,298.59)(0.487,0.477){7}{\rule{0.500pt}{0.115pt}}
\multiput(497.00,297.17)(3.962,5.000){2}{\rule{0.250pt}{0.400pt}}
\multiput(502.00,303.59)(0.487,0.477){7}{\rule{0.500pt}{0.115pt}}
\multiput(502.00,302.17)(3.962,5.000){2}{\rule{0.250pt}{0.400pt}}
\multiput(507.00,308.59)(0.487,0.477){7}{\rule{0.500pt}{0.115pt}}
\multiput(507.00,307.17)(3.962,5.000){2}{\rule{0.250pt}{0.400pt}}
\multiput(512.00,313.59)(0.487,0.477){7}{\rule{0.500pt}{0.115pt}}
\multiput(512.00,312.17)(3.962,5.000){2}{\rule{0.250pt}{0.400pt}}
\multiput(517.00,318.59)(0.487,0.477){7}{\rule{0.500pt}{0.115pt}}
\multiput(517.00,317.17)(3.962,5.000){2}{\rule{0.250pt}{0.400pt}}
\multiput(522.59,323.00)(0.477,0.599){7}{\rule{0.115pt}{0.580pt}}
\multiput(521.17,323.00)(5.000,4.796){2}{\rule{0.400pt}{0.290pt}}
\multiput(527.00,329.59)(0.599,0.477){7}{\rule{0.580pt}{0.115pt}}
\multiput(527.00,328.17)(4.796,5.000){2}{\rule{0.290pt}{0.400pt}}
\multiput(533.59,334.00)(0.477,0.599){7}{\rule{0.115pt}{0.580pt}}
\multiput(532.17,334.00)(5.000,4.796){2}{\rule{0.400pt}{0.290pt}}
\multiput(538.00,340.59)(0.487,0.477){7}{\rule{0.500pt}{0.115pt}}
\multiput(538.00,339.17)(3.962,5.000){2}{\rule{0.250pt}{0.400pt}}
\multiput(543.59,345.00)(0.477,0.599){7}{\rule{0.115pt}{0.580pt}}
\multiput(542.17,345.00)(5.000,4.796){2}{\rule{0.400pt}{0.290pt}}
\multiput(548.59,351.00)(0.477,0.599){7}{\rule{0.115pt}{0.580pt}}
\multiput(547.17,351.00)(5.000,4.796){2}{\rule{0.400pt}{0.290pt}}
\multiput(553.59,357.00)(0.477,0.599){7}{\rule{0.115pt}{0.580pt}}
\multiput(552.17,357.00)(5.000,4.796){2}{\rule{0.400pt}{0.290pt}}
\put(558,363.17){\rule{0.482pt}{0.400pt}}
\multiput(558.00,362.17)(1.000,2.000){2}{\rule{0.241pt}{0.400pt}}
\put(242.0,163.0){\rule[-0.200pt]{1.204pt}{0.400pt}}
\put(110,212){\usebox{\plotpoint}}
\put(110.00,212.00){\usebox{\plotpoint}}
\put(128.03,201.79){\usebox{\plotpoint}}
\put(146.66,192.74){\usebox{\plotpoint}}
\put(165.93,185.03){\usebox{\plotpoint}}
\put(185.44,178.11){\usebox{\plotpoint}}
\put(205.59,173.28){\usebox{\plotpoint}}
\put(226.04,170.19){\usebox{\plotpoint}}
\put(246.67,169.00){\usebox{\plotpoint}}
\put(267.33,170.00){\usebox{\plotpoint}}
\put(287.79,172.96){\usebox{\plotpoint}}
\put(308.18,176.86){\usebox{\plotpoint}}
\put(327.76,183.50){\usebox{\plotpoint}}
\put(347.03,191.21){\usebox{\plotpoint}}
\put(365.94,199.56){\usebox{\plotpoint}}
\put(383.74,210.24){\usebox{\plotpoint}}
\put(401.04,221.63){\usebox{\plotpoint}}
\put(418.25,233.17){\usebox{\plotpoint}}
\put(434.63,245.90){\usebox{\plotpoint}}
\put(449.91,259.91){\usebox{\plotpoint}}
\put(465.05,274.05){\usebox{\plotpoint}}
\put(480.21,288.21){\usebox{\plotpoint}}
\put(494.14,303.57){\usebox{\plotpoint}}
\put(507.90,319.08){\usebox{\plotpoint}}
\put(521.19,335.02){\usebox{\plotpoint}}
\put(534.39,350.95){\usebox{\plotpoint}}
\put(545,365){\usebox{\plotpoint}}
\sbox{\plotpoint}{\rule[-0.400pt]{0.800pt}{0.800pt}}%
\put(110,246){\usebox{\plotpoint}}
\put(110,242.34){\rule{1.200pt}{0.800pt}}
\multiput(110.00,244.34)(2.509,-4.000){2}{\rule{0.600pt}{0.800pt}}
\multiput(115.00,240.06)(0.424,-0.560){3}{\rule{1.000pt}{0.135pt}}
\multiput(115.00,240.34)(2.924,-5.000){2}{\rule{0.500pt}{0.800pt}}
\put(120,233.34){\rule{1.200pt}{0.800pt}}
\multiput(120.00,235.34)(2.509,-4.000){2}{\rule{0.600pt}{0.800pt}}
\put(125,229.34){\rule{1.200pt}{0.800pt}}
\multiput(125.00,231.34)(2.509,-4.000){2}{\rule{0.600pt}{0.800pt}}
\put(130,225.34){\rule{1.200pt}{0.800pt}}
\multiput(130.00,227.34)(2.509,-4.000){2}{\rule{0.600pt}{0.800pt}}
\put(135,221.34){\rule{1.400pt}{0.800pt}}
\multiput(135.00,223.34)(3.094,-4.000){2}{\rule{0.700pt}{0.800pt}}
\put(141,217.84){\rule{1.204pt}{0.800pt}}
\multiput(141.00,219.34)(2.500,-3.000){2}{\rule{0.602pt}{0.800pt}}
\put(146,214.34){\rule{1.200pt}{0.800pt}}
\multiput(146.00,216.34)(2.509,-4.000){2}{\rule{0.600pt}{0.800pt}}
\put(151,210.84){\rule{1.204pt}{0.800pt}}
\multiput(151.00,212.34)(2.500,-3.000){2}{\rule{0.602pt}{0.800pt}}
\put(156,207.84){\rule{1.204pt}{0.800pt}}
\multiput(156.00,209.34)(2.500,-3.000){2}{\rule{0.602pt}{0.800pt}}
\put(161,204.84){\rule{1.204pt}{0.800pt}}
\multiput(161.00,206.34)(2.500,-3.000){2}{\rule{0.602pt}{0.800pt}}
\put(166,201.84){\rule{1.204pt}{0.800pt}}
\multiput(166.00,203.34)(2.500,-3.000){2}{\rule{0.602pt}{0.800pt}}
\put(171,198.84){\rule{1.204pt}{0.800pt}}
\multiput(171.00,200.34)(2.500,-3.000){2}{\rule{0.602pt}{0.800pt}}
\put(176,195.84){\rule{1.204pt}{0.800pt}}
\multiput(176.00,197.34)(2.500,-3.000){2}{\rule{0.602pt}{0.800pt}}
\put(181,193.34){\rule{1.204pt}{0.800pt}}
\multiput(181.00,194.34)(2.500,-2.000){2}{\rule{0.602pt}{0.800pt}}
\put(186,190.84){\rule{1.204pt}{0.800pt}}
\multiput(186.00,192.34)(2.500,-3.000){2}{\rule{0.602pt}{0.800pt}}
\put(191,188.34){\rule{1.445pt}{0.800pt}}
\multiput(191.00,189.34)(3.000,-2.000){2}{\rule{0.723pt}{0.800pt}}
\put(197,186.34){\rule{1.204pt}{0.800pt}}
\multiput(197.00,187.34)(2.500,-2.000){2}{\rule{0.602pt}{0.800pt}}
\put(202,184.34){\rule{1.204pt}{0.800pt}}
\multiput(202.00,185.34)(2.500,-2.000){2}{\rule{0.602pt}{0.800pt}}
\put(207,182.84){\rule{1.204pt}{0.800pt}}
\multiput(207.00,183.34)(2.500,-1.000){2}{\rule{0.602pt}{0.800pt}}
\put(212,181.34){\rule{1.204pt}{0.800pt}}
\multiput(212.00,182.34)(2.500,-2.000){2}{\rule{0.602pt}{0.800pt}}
\put(217,179.34){\rule{1.204pt}{0.800pt}}
\multiput(217.00,180.34)(2.500,-2.000){2}{\rule{0.602pt}{0.800pt}}
\put(222,177.84){\rule{1.204pt}{0.800pt}}
\multiput(222.00,178.34)(2.500,-1.000){2}{\rule{0.602pt}{0.800pt}}
\put(227,176.84){\rule{1.204pt}{0.800pt}}
\multiput(227.00,177.34)(2.500,-1.000){2}{\rule{0.602pt}{0.800pt}}
\put(232,175.84){\rule{1.204pt}{0.800pt}}
\multiput(232.00,176.34)(2.500,-1.000){2}{\rule{0.602pt}{0.800pt}}
\put(237,174.84){\rule{1.204pt}{0.800pt}}
\multiput(237.00,175.34)(2.500,-1.000){2}{\rule{0.602pt}{0.800pt}}
\put(242,173.84){\rule{1.204pt}{0.800pt}}
\multiput(242.00,174.34)(2.500,-1.000){2}{\rule{0.602pt}{0.800pt}}
\put(253,172.84){\rule{1.204pt}{0.800pt}}
\multiput(253.00,173.34)(2.500,-1.000){2}{\rule{0.602pt}{0.800pt}}
\put(247.0,175.0){\rule[-0.400pt]{1.445pt}{0.800pt}}
\put(288,172.84){\rule{1.204pt}{0.800pt}}
\multiput(288.00,172.34)(2.500,1.000){2}{\rule{0.602pt}{0.800pt}}
\put(293,173.84){\rule{1.204pt}{0.800pt}}
\multiput(293.00,173.34)(2.500,1.000){2}{\rule{0.602pt}{0.800pt}}
\put(298,174.84){\rule{1.204pt}{0.800pt}}
\multiput(298.00,174.34)(2.500,1.000){2}{\rule{0.602pt}{0.800pt}}
\put(303,175.84){\rule{1.445pt}{0.800pt}}
\multiput(303.00,175.34)(3.000,1.000){2}{\rule{0.723pt}{0.800pt}}
\put(309,176.84){\rule{1.204pt}{0.800pt}}
\multiput(309.00,176.34)(2.500,1.000){2}{\rule{0.602pt}{0.800pt}}
\put(314,177.84){\rule{1.204pt}{0.800pt}}
\multiput(314.00,177.34)(2.500,1.000){2}{\rule{0.602pt}{0.800pt}}
\put(319,178.84){\rule{1.204pt}{0.800pt}}
\multiput(319.00,178.34)(2.500,1.000){2}{\rule{0.602pt}{0.800pt}}
\put(324,180.34){\rule{1.204pt}{0.800pt}}
\multiput(324.00,179.34)(2.500,2.000){2}{\rule{0.602pt}{0.800pt}}
\put(329,182.34){\rule{1.204pt}{0.800pt}}
\multiput(329.00,181.34)(2.500,2.000){2}{\rule{0.602pt}{0.800pt}}
\put(334,183.84){\rule{1.204pt}{0.800pt}}
\multiput(334.00,183.34)(2.500,1.000){2}{\rule{0.602pt}{0.800pt}}
\put(339,185.34){\rule{1.204pt}{0.800pt}}
\multiput(339.00,184.34)(2.500,2.000){2}{\rule{0.602pt}{0.800pt}}
\put(344,187.84){\rule{1.204pt}{0.800pt}}
\multiput(344.00,186.34)(2.500,3.000){2}{\rule{0.602pt}{0.800pt}}
\put(349,190.34){\rule{1.204pt}{0.800pt}}
\multiput(349.00,189.34)(2.500,2.000){2}{\rule{0.602pt}{0.800pt}}
\put(354,192.34){\rule{1.204pt}{0.800pt}}
\multiput(354.00,191.34)(2.500,2.000){2}{\rule{0.602pt}{0.800pt}}
\put(359,194.84){\rule{1.445pt}{0.800pt}}
\multiput(359.00,193.34)(3.000,3.000){2}{\rule{0.723pt}{0.800pt}}
\put(365,197.84){\rule{1.204pt}{0.800pt}}
\multiput(365.00,196.34)(2.500,3.000){2}{\rule{0.602pt}{0.800pt}}
\put(370,200.34){\rule{1.204pt}{0.800pt}}
\multiput(370.00,199.34)(2.500,2.000){2}{\rule{0.602pt}{0.800pt}}
\put(375,203.34){\rule{1.200pt}{0.800pt}}
\multiput(375.00,201.34)(2.509,4.000){2}{\rule{0.600pt}{0.800pt}}
\put(380,206.84){\rule{1.204pt}{0.800pt}}
\multiput(380.00,205.34)(2.500,3.000){2}{\rule{0.602pt}{0.800pt}}
\put(385,209.84){\rule{1.204pt}{0.800pt}}
\multiput(385.00,208.34)(2.500,3.000){2}{\rule{0.602pt}{0.800pt}}
\put(390,212.84){\rule{1.204pt}{0.800pt}}
\multiput(390.00,211.34)(2.500,3.000){2}{\rule{0.602pt}{0.800pt}}
\put(395,216.34){\rule{1.200pt}{0.800pt}}
\multiput(395.00,214.34)(2.509,4.000){2}{\rule{0.600pt}{0.800pt}}
\put(400,220.34){\rule{1.200pt}{0.800pt}}
\multiput(400.00,218.34)(2.509,4.000){2}{\rule{0.600pt}{0.800pt}}
\put(405,224.34){\rule{1.200pt}{0.800pt}}
\multiput(405.00,222.34)(2.509,4.000){2}{\rule{0.600pt}{0.800pt}}
\put(410,228.34){\rule{1.200pt}{0.800pt}}
\multiput(410.00,226.34)(2.509,4.000){2}{\rule{0.600pt}{0.800pt}}
\put(415,232.34){\rule{1.400pt}{0.800pt}}
\multiput(415.00,230.34)(3.094,4.000){2}{\rule{0.700pt}{0.800pt}}
\put(421,236.34){\rule{1.200pt}{0.800pt}}
\multiput(421.00,234.34)(2.509,4.000){2}{\rule{0.600pt}{0.800pt}}
\multiput(426.00,241.38)(0.424,0.560){3}{\rule{1.000pt}{0.135pt}}
\multiput(426.00,238.34)(2.924,5.000){2}{\rule{0.500pt}{0.800pt}}
\put(431,245.34){\rule{1.200pt}{0.800pt}}
\multiput(431.00,243.34)(2.509,4.000){2}{\rule{0.600pt}{0.800pt}}
\multiput(436.00,250.38)(0.424,0.560){3}{\rule{1.000pt}{0.135pt}}
\multiput(436.00,247.34)(2.924,5.000){2}{\rule{0.500pt}{0.800pt}}
\multiput(441.00,255.38)(0.424,0.560){3}{\rule{1.000pt}{0.135pt}}
\multiput(441.00,252.34)(2.924,5.000){2}{\rule{0.500pt}{0.800pt}}
\multiput(446.00,260.38)(0.424,0.560){3}{\rule{1.000pt}{0.135pt}}
\multiput(446.00,257.34)(2.924,5.000){2}{\rule{0.500pt}{0.800pt}}
\multiput(451.00,265.38)(0.424,0.560){3}{\rule{1.000pt}{0.135pt}}
\multiput(451.00,262.34)(2.924,5.000){2}{\rule{0.500pt}{0.800pt}}
\multiput(456.00,270.38)(0.424,0.560){3}{\rule{1.000pt}{0.135pt}}
\multiput(456.00,267.34)(2.924,5.000){2}{\rule{0.500pt}{0.800pt}}
\multiput(462.38,274.00)(0.560,0.592){3}{\rule{0.135pt}{1.160pt}}
\multiput(459.34,274.00)(5.000,3.592){2}{\rule{0.800pt}{0.580pt}}
\multiput(466.00,281.38)(0.424,0.560){3}{\rule{1.000pt}{0.135pt}}
\multiput(466.00,278.34)(2.924,5.000){2}{\rule{0.500pt}{0.800pt}}
\multiput(471.00,286.39)(0.462,0.536){5}{\rule{1.000pt}{0.129pt}}
\multiput(471.00,283.34)(3.924,6.000){2}{\rule{0.500pt}{0.800pt}}
\multiput(478.38,291.00)(0.560,0.592){3}{\rule{0.135pt}{1.160pt}}
\multiput(475.34,291.00)(5.000,3.592){2}{\rule{0.800pt}{0.580pt}}
\multiput(483.38,297.00)(0.560,0.592){3}{\rule{0.135pt}{1.160pt}}
\multiput(480.34,297.00)(5.000,3.592){2}{\rule{0.800pt}{0.580pt}}
\multiput(488.38,303.00)(0.560,0.592){3}{\rule{0.135pt}{1.160pt}}
\multiput(485.34,303.00)(5.000,3.592){2}{\rule{0.800pt}{0.580pt}}
\multiput(493.38,309.00)(0.560,0.592){3}{\rule{0.135pt}{1.160pt}}
\multiput(490.34,309.00)(5.000,3.592){2}{\rule{0.800pt}{0.580pt}}
\multiput(498.38,315.00)(0.560,0.760){3}{\rule{0.135pt}{1.320pt}}
\multiput(495.34,315.00)(5.000,4.260){2}{\rule{0.800pt}{0.660pt}}
\multiput(503.38,322.00)(0.560,0.760){3}{\rule{0.135pt}{1.320pt}}
\multiput(500.34,322.00)(5.000,4.260){2}{\rule{0.800pt}{0.660pt}}
\multiput(508.38,329.00)(0.560,0.592){3}{\rule{0.135pt}{1.160pt}}
\multiput(505.34,329.00)(5.000,3.592){2}{\rule{0.800pt}{0.580pt}}
\multiput(513.38,335.00)(0.560,0.760){3}{\rule{0.135pt}{1.320pt}}
\multiput(510.34,335.00)(5.000,4.260){2}{\rule{0.800pt}{0.660pt}}
\multiput(518.38,342.00)(0.560,0.760){3}{\rule{0.135pt}{1.320pt}}
\multiput(515.34,342.00)(5.000,4.260){2}{\rule{0.800pt}{0.660pt}}
\multiput(523.38,349.00)(0.560,0.760){3}{\rule{0.135pt}{1.320pt}}
\multiput(520.34,349.00)(5.000,4.260){2}{\rule{0.800pt}{0.660pt}}
\multiput(528.39,356.00)(0.536,0.685){5}{\rule{0.129pt}{1.267pt}}
\multiput(525.34,356.00)(6.000,5.371){2}{\rule{0.800pt}{0.633pt}}
\put(258.0,174.0){\rule[-0.400pt]{7.227pt}{0.800pt}}
\put(533.0,364.0){\usebox{\plotpoint}}
\sbox{\plotpoint}{\rule[-0.500pt]{1.000pt}{1.000pt}}%
\put(110,280){\usebox{\plotpoint}}
\put(110.00,280.00){\usebox{\plotpoint}}
\put(123.63,264.37){\usebox{\plotpoint}}
\put(138.03,249.48){\usebox{\plotpoint}}
\put(153.41,235.59){\usebox{\plotpoint}}
\put(169.35,222.32){\usebox{\plotpoint}}
\put(186.00,210.00){\usebox{\plotpoint}}
\put(203.56,199.07){\usebox{\plotpoint}}
\put(222.13,189.95){\usebox{\plotpoint}}
\put(241.65,183.07){\usebox{\plotpoint}}
\put(261.75,178.25){\usebox{\plotpoint}}
\put(282.29,176.00){\usebox{\plotpoint}}
\put(302.94,177.00){\usebox{\plotpoint}}
\put(323.33,180.87){\usebox{\plotpoint}}
\put(343.17,186.67){\usebox{\plotpoint}}
\put(361.92,195.46){\usebox{\plotpoint}}
\put(380.20,205.16){\usebox{\plotpoint}}
\put(397.04,217.22){\usebox{\plotpoint}}
\put(413.18,230.18){\usebox{\plotpoint}}
\put(428.76,243.76){\usebox{\plotpoint}}
\put(443.43,258.43){\usebox{\plotpoint}}
\put(457.44,273.72){\usebox{\plotpoint}}
\put(470.29,290.01){\usebox{\plotpoint}}
\put(483.88,305.63){\usebox{\plotpoint}}
\put(495.95,322.52){\usebox{\plotpoint}}
\put(507.53,339.74){\usebox{\plotpoint}}
\put(518.92,357.07){\usebox{\plotpoint}}
\put(524,365){\usebox{\plotpoint}}
\sbox{\plotpoint}{\rule[-0.600pt]{1.200pt}{1.200pt}}%
\put(110,314){\usebox{\plotpoint}}
\put(110.01,307){\rule{1.200pt}{1.686pt}}
\multiput(107.51,310.50)(5.000,-3.500){2}{\rule{1.200pt}{0.843pt}}
\put(115.01,300){\rule{1.200pt}{1.686pt}}
\multiput(112.51,303.50)(5.000,-3.500){2}{\rule{1.200pt}{0.843pt}}
\put(120.01,293){\rule{1.200pt}{1.686pt}}
\multiput(117.51,296.50)(5.000,-3.500){2}{\rule{1.200pt}{0.843pt}}
\put(125.01,286){\rule{1.200pt}{1.686pt}}
\multiput(122.51,289.50)(5.000,-3.500){2}{\rule{1.200pt}{0.843pt}}
\put(130.01,279){\rule{1.200pt}{1.686pt}}
\multiput(127.51,282.50)(5.000,-3.500){2}{\rule{1.200pt}{0.843pt}}
\put(135,273.51){\rule{1.445pt}{1.200pt}}
\multiput(135.00,276.51)(3.000,-6.000){2}{\rule{0.723pt}{1.200pt}}
\put(141.01,267){\rule{1.200pt}{1.445pt}}
\multiput(138.51,270.00)(5.000,-3.000){2}{\rule{1.200pt}{0.723pt}}
\put(146.01,261){\rule{1.200pt}{1.445pt}}
\multiput(143.51,264.00)(5.000,-3.000){2}{\rule{1.200pt}{0.723pt}}
\put(151.01,255){\rule{1.200pt}{1.445pt}}
\multiput(148.51,258.00)(5.000,-3.000){2}{\rule{1.200pt}{0.723pt}}
\put(156,250.01){\rule{1.204pt}{1.200pt}}
\multiput(156.00,252.51)(2.500,-5.000){2}{\rule{0.602pt}{1.200pt}}
\put(161,245.01){\rule{1.204pt}{1.200pt}}
\multiput(161.00,247.51)(2.500,-5.000){2}{\rule{0.602pt}{1.200pt}}
\put(166.01,239){\rule{1.200pt}{1.445pt}}
\multiput(163.51,242.00)(5.000,-3.000){2}{\rule{1.200pt}{0.723pt}}
\put(171,234.51){\rule{1.204pt}{1.200pt}}
\multiput(171.00,236.51)(2.500,-4.000){2}{\rule{0.602pt}{1.200pt}}
\put(176,230.01){\rule{1.204pt}{1.200pt}}
\multiput(176.00,232.51)(2.500,-5.000){2}{\rule{0.602pt}{1.200pt}}
\put(181,225.01){\rule{1.204pt}{1.200pt}}
\multiput(181.00,227.51)(2.500,-5.000){2}{\rule{0.602pt}{1.200pt}}
\put(186,220.51){\rule{1.204pt}{1.200pt}}
\multiput(186.00,222.51)(2.500,-4.000){2}{\rule{0.602pt}{1.200pt}}
\put(191,216.51){\rule{1.445pt}{1.200pt}}
\multiput(191.00,218.51)(3.000,-4.000){2}{\rule{0.723pt}{1.200pt}}
\put(197,212.51){\rule{1.204pt}{1.200pt}}
\multiput(197.00,214.51)(2.500,-4.000){2}{\rule{0.602pt}{1.200pt}}
\put(202,208.51){\rule{1.204pt}{1.200pt}}
\multiput(202.00,210.51)(2.500,-4.000){2}{\rule{0.602pt}{1.200pt}}
\put(207,205.01){\rule{1.204pt}{1.200pt}}
\multiput(207.00,206.51)(2.500,-3.000){2}{\rule{0.602pt}{1.200pt}}
\put(212,202.01){\rule{1.204pt}{1.200pt}}
\multiput(212.00,203.51)(2.500,-3.000){2}{\rule{0.602pt}{1.200pt}}
\put(217,198.51){\rule{1.204pt}{1.200pt}}
\multiput(217.00,200.51)(2.500,-4.000){2}{\rule{0.602pt}{1.200pt}}
\put(222,195.51){\rule{1.204pt}{1.200pt}}
\multiput(222.00,196.51)(2.500,-2.000){2}{\rule{0.602pt}{1.200pt}}
\put(227,193.01){\rule{1.204pt}{1.200pt}}
\multiput(227.00,194.51)(2.500,-3.000){2}{\rule{0.602pt}{1.200pt}}
\put(232,190.01){\rule{1.204pt}{1.200pt}}
\multiput(232.00,191.51)(2.500,-3.000){2}{\rule{0.602pt}{1.200pt}}
\put(237,187.51){\rule{1.204pt}{1.200pt}}
\multiput(237.00,188.51)(2.500,-2.000){2}{\rule{0.602pt}{1.200pt}}
\put(242,185.51){\rule{1.204pt}{1.200pt}}
\multiput(242.00,186.51)(2.500,-2.000){2}{\rule{0.602pt}{1.200pt}}
\put(247,183.51){\rule{1.445pt}{1.200pt}}
\multiput(247.00,184.51)(3.000,-2.000){2}{\rule{0.723pt}{1.200pt}}
\put(253,182.01){\rule{1.204pt}{1.200pt}}
\multiput(253.00,182.51)(2.500,-1.000){2}{\rule{0.602pt}{1.200pt}}
\put(258,180.51){\rule{1.204pt}{1.200pt}}
\multiput(258.00,181.51)(2.500,-2.000){2}{\rule{0.602pt}{1.200pt}}
\put(263,179.01){\rule{1.204pt}{1.200pt}}
\multiput(263.00,179.51)(2.500,-1.000){2}{\rule{0.602pt}{1.200pt}}
\put(268,178.01){\rule{1.204pt}{1.200pt}}
\multiput(268.00,178.51)(2.500,-1.000){2}{\rule{0.602pt}{1.200pt}}
\put(273,177.01){\rule{1.204pt}{1.200pt}}
\multiput(273.00,177.51)(2.500,-1.000){2}{\rule{0.602pt}{1.200pt}}
\put(278,176.01){\rule{1.204pt}{1.200pt}}
\multiput(278.00,176.51)(2.500,-1.000){2}{\rule{0.602pt}{1.200pt}}
\put(288,175.01){\rule{1.204pt}{1.200pt}}
\multiput(288.00,175.51)(2.500,-1.000){2}{\rule{0.602pt}{1.200pt}}
\put(283.0,178.0){\rule[-0.600pt]{1.204pt}{1.200pt}}
\put(298,175.01){\rule{1.204pt}{1.200pt}}
\multiput(298.00,174.51)(2.500,1.000){2}{\rule{0.602pt}{1.200pt}}
\put(293.0,177.0){\rule[-0.600pt]{1.204pt}{1.200pt}}
\put(314,176.01){\rule{1.204pt}{1.200pt}}
\multiput(314.00,175.51)(2.500,1.000){2}{\rule{0.602pt}{1.200pt}}
\put(319,177.01){\rule{1.204pt}{1.200pt}}
\multiput(319.00,176.51)(2.500,1.000){2}{\rule{0.602pt}{1.200pt}}
\put(324,178.01){\rule{1.204pt}{1.200pt}}
\multiput(324.00,177.51)(2.500,1.000){2}{\rule{0.602pt}{1.200pt}}
\put(329,179.51){\rule{1.204pt}{1.200pt}}
\multiput(329.00,178.51)(2.500,2.000){2}{\rule{0.602pt}{1.200pt}}
\put(334,181.01){\rule{1.204pt}{1.200pt}}
\multiput(334.00,180.51)(2.500,1.000){2}{\rule{0.602pt}{1.200pt}}
\put(339,182.51){\rule{1.204pt}{1.200pt}}
\multiput(339.00,181.51)(2.500,2.000){2}{\rule{0.602pt}{1.200pt}}
\put(344,184.51){\rule{1.204pt}{1.200pt}}
\multiput(344.00,183.51)(2.500,2.000){2}{\rule{0.602pt}{1.200pt}}
\put(349,186.51){\rule{1.204pt}{1.200pt}}
\multiput(349.00,185.51)(2.500,2.000){2}{\rule{0.602pt}{1.200pt}}
\put(354,189.01){\rule{1.204pt}{1.200pt}}
\multiput(354.00,187.51)(2.500,3.000){2}{\rule{0.602pt}{1.200pt}}
\put(359,191.51){\rule{1.445pt}{1.200pt}}
\multiput(359.00,190.51)(3.000,2.000){2}{\rule{0.723pt}{1.200pt}}
\put(365,194.01){\rule{1.204pt}{1.200pt}}
\multiput(365.00,192.51)(2.500,3.000){2}{\rule{0.602pt}{1.200pt}}
\put(370,197.01){\rule{1.204pt}{1.200pt}}
\multiput(370.00,195.51)(2.500,3.000){2}{\rule{0.602pt}{1.200pt}}
\put(375,200.01){\rule{1.204pt}{1.200pt}}
\multiput(375.00,198.51)(2.500,3.000){2}{\rule{0.602pt}{1.200pt}}
\put(380,203.01){\rule{1.204pt}{1.200pt}}
\multiput(380.00,201.51)(2.500,3.000){2}{\rule{0.602pt}{1.200pt}}
\put(385,206.51){\rule{1.204pt}{1.200pt}}
\multiput(385.00,204.51)(2.500,4.000){2}{\rule{0.602pt}{1.200pt}}
\put(390,210.51){\rule{1.204pt}{1.200pt}}
\multiput(390.00,208.51)(2.500,4.000){2}{\rule{0.602pt}{1.200pt}}
\put(395,214.51){\rule{1.204pt}{1.200pt}}
\multiput(395.00,212.51)(2.500,4.000){2}{\rule{0.602pt}{1.200pt}}
\put(400,218.51){\rule{1.204pt}{1.200pt}}
\multiput(400.00,216.51)(2.500,4.000){2}{\rule{0.602pt}{1.200pt}}
\put(405,222.51){\rule{1.204pt}{1.200pt}}
\multiput(405.00,220.51)(2.500,4.000){2}{\rule{0.602pt}{1.200pt}}
\put(410,227.01){\rule{1.204pt}{1.200pt}}
\multiput(410.00,224.51)(2.500,5.000){2}{\rule{0.602pt}{1.200pt}}
\put(415,232.01){\rule{1.445pt}{1.200pt}}
\multiput(415.00,229.51)(3.000,5.000){2}{\rule{0.723pt}{1.200pt}}
\put(421,237.01){\rule{1.204pt}{1.200pt}}
\multiput(421.00,234.51)(2.500,5.000){2}{\rule{0.602pt}{1.200pt}}
\put(426,242.01){\rule{1.204pt}{1.200pt}}
\multiput(426.00,239.51)(2.500,5.000){2}{\rule{0.602pt}{1.200pt}}
\put(431,247.01){\rule{1.204pt}{1.200pt}}
\multiput(431.00,244.51)(2.500,5.000){2}{\rule{0.602pt}{1.200pt}}
\put(436.01,252){\rule{1.200pt}{1.445pt}}
\multiput(433.51,252.00)(5.000,3.000){2}{\rule{1.200pt}{0.723pt}}
\put(441.01,258){\rule{1.200pt}{1.445pt}}
\multiput(438.51,258.00)(5.000,3.000){2}{\rule{1.200pt}{0.723pt}}
\put(446.01,264){\rule{1.200pt}{1.445pt}}
\multiput(443.51,264.00)(5.000,3.000){2}{\rule{1.200pt}{0.723pt}}
\put(451.01,270){\rule{1.200pt}{1.445pt}}
\multiput(448.51,270.00)(5.000,3.000){2}{\rule{1.200pt}{0.723pt}}
\put(456.01,276){\rule{1.200pt}{1.445pt}}
\multiput(453.51,276.00)(5.000,3.000){2}{\rule{1.200pt}{0.723pt}}
\put(461.01,282){\rule{1.200pt}{1.686pt}}
\multiput(458.51,282.00)(5.000,3.500){2}{\rule{1.200pt}{0.843pt}}
\put(466.01,289){\rule{1.200pt}{1.686pt}}
\multiput(463.51,289.00)(5.000,3.500){2}{\rule{1.200pt}{0.843pt}}
\put(471.51,296){\rule{1.200pt}{1.686pt}}
\multiput(468.51,296.00)(6.000,3.500){2}{\rule{1.200pt}{0.843pt}}
\put(477.01,303){\rule{1.200pt}{1.686pt}}
\multiput(474.51,303.00)(5.000,3.500){2}{\rule{1.200pt}{0.843pt}}
\put(482.01,310){\rule{1.200pt}{1.686pt}}
\multiput(479.51,310.00)(5.000,3.500){2}{\rule{1.200pt}{0.843pt}}
\put(487.01,317){\rule{1.200pt}{1.927pt}}
\multiput(484.51,317.00)(5.000,4.000){2}{\rule{1.200pt}{0.964pt}}
\put(492.01,325){\rule{1.200pt}{1.927pt}}
\multiput(489.51,325.00)(5.000,4.000){2}{\rule{1.200pt}{0.964pt}}
\put(497.01,333){\rule{1.200pt}{1.927pt}}
\multiput(494.51,333.00)(5.000,4.000){2}{\rule{1.200pt}{0.964pt}}
\put(502.01,341){\rule{1.200pt}{1.927pt}}
\multiput(499.51,341.00)(5.000,4.000){2}{\rule{1.200pt}{0.964pt}}
\put(507.01,349){\rule{1.200pt}{2.168pt}}
\multiput(504.51,349.00)(5.000,4.500){2}{\rule{1.200pt}{1.084pt}}
\put(512.01,358){\rule{1.200pt}{1.686pt}}
\multiput(509.51,358.00)(5.000,3.500){2}{\rule{1.200pt}{0.843pt}}
\put(303.0,178.0){\rule[-0.600pt]{2.650pt}{1.200pt}}
\sbox{\plotpoint}{\rule[-0.500pt]{1.000pt}{1.000pt}}%
\put(110,348){\usebox{\plotpoint}}
\put(110.00,348.00){\usebox{\plotpoint}}
\put(131.19,312.33){\usebox{\plotpoint}}
\put(155.51,278.69){\usebox{\plotpoint}}
\put(181.49,246.41){\usebox{\plotpoint}}
\put(211.25,217.60){\usebox{\plotpoint}}
\put(245.30,194.02){\usebox{\plotpoint}}
\put(284.06,179.79){\usebox{\plotpoint}}
\put(325.18,180.24){\usebox{\plotpoint}}
\put(364.25,193.63){\usebox{\plotpoint}}
\put(398.45,216.76){\usebox{\plotpoint}}
\put(428.38,245.38){\usebox{\plotpoint}}
\put(454.69,277.43){\usebox{\plotpoint}}
\put(478.50,311.40){\usebox{\plotpoint}}
\put(500.04,346.87){\usebox{\plotpoint}}
\put(510,365){\usebox{\plotpoint}}
\sbox{\plotpoint}{\rule[-0.200pt]{0.400pt}{0.400pt}}%
\put(118.67,362){\rule{0.400pt}{0.723pt}}
\multiput(118.17,363.50)(1.000,-1.500){2}{\rule{0.400pt}{0.361pt}}
\multiput(120.59,358.26)(0.477,-1.044){7}{\rule{0.115pt}{0.900pt}}
\multiput(119.17,360.13)(5.000,-8.132){2}{\rule{0.400pt}{0.450pt}}
\multiput(125.59,348.60)(0.477,-0.933){7}{\rule{0.115pt}{0.820pt}}
\multiput(124.17,350.30)(5.000,-7.298){2}{\rule{0.400pt}{0.410pt}}
\multiput(130.59,339.60)(0.477,-0.933){7}{\rule{0.115pt}{0.820pt}}
\multiput(129.17,341.30)(5.000,-7.298){2}{\rule{0.400pt}{0.410pt}}
\multiput(135.59,331.09)(0.482,-0.762){9}{\rule{0.116pt}{0.700pt}}
\multiput(134.17,332.55)(6.000,-7.547){2}{\rule{0.400pt}{0.350pt}}
\multiput(141.59,321.60)(0.477,-0.933){7}{\rule{0.115pt}{0.820pt}}
\multiput(140.17,323.30)(5.000,-7.298){2}{\rule{0.400pt}{0.410pt}}
\multiput(146.59,312.93)(0.477,-0.821){7}{\rule{0.115pt}{0.740pt}}
\multiput(145.17,314.46)(5.000,-6.464){2}{\rule{0.400pt}{0.370pt}}
\multiput(151.59,304.93)(0.477,-0.821){7}{\rule{0.115pt}{0.740pt}}
\multiput(150.17,306.46)(5.000,-6.464){2}{\rule{0.400pt}{0.370pt}}
\multiput(156.59,296.93)(0.477,-0.821){7}{\rule{0.115pt}{0.740pt}}
\multiput(155.17,298.46)(5.000,-6.464){2}{\rule{0.400pt}{0.370pt}}
\multiput(161.59,289.26)(0.477,-0.710){7}{\rule{0.115pt}{0.660pt}}
\multiput(160.17,290.63)(5.000,-5.630){2}{\rule{0.400pt}{0.330pt}}
\multiput(166.59,281.93)(0.477,-0.821){7}{\rule{0.115pt}{0.740pt}}
\multiput(165.17,283.46)(5.000,-6.464){2}{\rule{0.400pt}{0.370pt}}
\multiput(171.59,274.26)(0.477,-0.710){7}{\rule{0.115pt}{0.660pt}}
\multiput(170.17,275.63)(5.000,-5.630){2}{\rule{0.400pt}{0.330pt}}
\multiput(176.59,267.26)(0.477,-0.710){7}{\rule{0.115pt}{0.660pt}}
\multiput(175.17,268.63)(5.000,-5.630){2}{\rule{0.400pt}{0.330pt}}
\multiput(181.59,260.59)(0.477,-0.599){7}{\rule{0.115pt}{0.580pt}}
\multiput(180.17,261.80)(5.000,-4.796){2}{\rule{0.400pt}{0.290pt}}
\multiput(186.59,254.59)(0.477,-0.599){7}{\rule{0.115pt}{0.580pt}}
\multiput(185.17,255.80)(5.000,-4.796){2}{\rule{0.400pt}{0.290pt}}
\multiput(191.00,249.93)(0.491,-0.482){9}{\rule{0.500pt}{0.116pt}}
\multiput(191.00,250.17)(4.962,-6.000){2}{\rule{0.250pt}{0.400pt}}
\multiput(197.59,242.59)(0.477,-0.599){7}{\rule{0.115pt}{0.580pt}}
\multiput(196.17,243.80)(5.000,-4.796){2}{\rule{0.400pt}{0.290pt}}
\multiput(202.59,236.59)(0.477,-0.599){7}{\rule{0.115pt}{0.580pt}}
\multiput(201.17,237.80)(5.000,-4.796){2}{\rule{0.400pt}{0.290pt}}
\multiput(207.00,231.93)(0.487,-0.477){7}{\rule{0.500pt}{0.115pt}}
\multiput(207.00,232.17)(3.962,-5.000){2}{\rule{0.250pt}{0.400pt}}
\multiput(212.00,226.93)(0.487,-0.477){7}{\rule{0.500pt}{0.115pt}}
\multiput(212.00,227.17)(3.962,-5.000){2}{\rule{0.250pt}{0.400pt}}
\multiput(217.00,221.93)(0.487,-0.477){7}{\rule{0.500pt}{0.115pt}}
\multiput(217.00,222.17)(3.962,-5.000){2}{\rule{0.250pt}{0.400pt}}
\multiput(222.00,216.94)(0.627,-0.468){5}{\rule{0.600pt}{0.113pt}}
\multiput(222.00,217.17)(3.755,-4.000){2}{\rule{0.300pt}{0.400pt}}
\multiput(227.00,212.94)(0.627,-0.468){5}{\rule{0.600pt}{0.113pt}}
\multiput(227.00,213.17)(3.755,-4.000){2}{\rule{0.300pt}{0.400pt}}
\multiput(232.00,208.94)(0.627,-0.468){5}{\rule{0.600pt}{0.113pt}}
\multiput(232.00,209.17)(3.755,-4.000){2}{\rule{0.300pt}{0.400pt}}
\multiput(237.00,204.94)(0.627,-0.468){5}{\rule{0.600pt}{0.113pt}}
\multiput(237.00,205.17)(3.755,-4.000){2}{\rule{0.300pt}{0.400pt}}
\multiput(242.00,200.95)(0.909,-0.447){3}{\rule{0.767pt}{0.108pt}}
\multiput(242.00,201.17)(3.409,-3.000){2}{\rule{0.383pt}{0.400pt}}
\multiput(247.00,197.95)(1.132,-0.447){3}{\rule{0.900pt}{0.108pt}}
\multiput(247.00,198.17)(4.132,-3.000){2}{\rule{0.450pt}{0.400pt}}
\multiput(253.00,194.95)(0.909,-0.447){3}{\rule{0.767pt}{0.108pt}}
\multiput(253.00,195.17)(3.409,-3.000){2}{\rule{0.383pt}{0.400pt}}
\multiput(258.00,191.95)(0.909,-0.447){3}{\rule{0.767pt}{0.108pt}}
\multiput(258.00,192.17)(3.409,-3.000){2}{\rule{0.383pt}{0.400pt}}
\put(263,188.17){\rule{1.100pt}{0.400pt}}
\multiput(263.00,189.17)(2.717,-2.000){2}{\rule{0.550pt}{0.400pt}}
\put(268,186.17){\rule{1.100pt}{0.400pt}}
\multiput(268.00,187.17)(2.717,-2.000){2}{\rule{0.550pt}{0.400pt}}
\put(273,184.17){\rule{1.100pt}{0.400pt}}
\multiput(273.00,185.17)(2.717,-2.000){2}{\rule{0.550pt}{0.400pt}}
\put(278,182.17){\rule{1.100pt}{0.400pt}}
\multiput(278.00,183.17)(2.717,-2.000){2}{\rule{0.550pt}{0.400pt}}
\put(283,180.67){\rule{1.204pt}{0.400pt}}
\multiput(283.00,181.17)(2.500,-1.000){2}{\rule{0.602pt}{0.400pt}}
\put(288,179.67){\rule{1.204pt}{0.400pt}}
\multiput(288.00,180.17)(2.500,-1.000){2}{\rule{0.602pt}{0.400pt}}
\put(293,178.67){\rule{1.204pt}{0.400pt}}
\multiput(293.00,179.17)(2.500,-1.000){2}{\rule{0.602pt}{0.400pt}}
\put(303,177.67){\rule{1.445pt}{0.400pt}}
\multiput(303.00,178.17)(3.000,-1.000){2}{\rule{0.723pt}{0.400pt}}
\put(298.0,179.0){\rule[-0.200pt]{1.204pt}{0.400pt}}
\put(319,177.67){\rule{1.204pt}{0.400pt}}
\multiput(319.00,177.17)(2.500,1.000){2}{\rule{0.602pt}{0.400pt}}
\put(324,178.67){\rule{1.204pt}{0.400pt}}
\multiput(324.00,178.17)(2.500,1.000){2}{\rule{0.602pt}{0.400pt}}
\put(329,179.67){\rule{1.204pt}{0.400pt}}
\multiput(329.00,179.17)(2.500,1.000){2}{\rule{0.602pt}{0.400pt}}
\put(334,180.67){\rule{1.204pt}{0.400pt}}
\multiput(334.00,180.17)(2.500,1.000){2}{\rule{0.602pt}{0.400pt}}
\put(339,182.17){\rule{1.100pt}{0.400pt}}
\multiput(339.00,181.17)(2.717,2.000){2}{\rule{0.550pt}{0.400pt}}
\put(344,183.67){\rule{1.204pt}{0.400pt}}
\multiput(344.00,183.17)(2.500,1.000){2}{\rule{0.602pt}{0.400pt}}
\put(349,185.17){\rule{1.100pt}{0.400pt}}
\multiput(349.00,184.17)(2.717,2.000){2}{\rule{0.550pt}{0.400pt}}
\multiput(354.00,187.61)(0.909,0.447){3}{\rule{0.767pt}{0.108pt}}
\multiput(354.00,186.17)(3.409,3.000){2}{\rule{0.383pt}{0.400pt}}
\put(359,190.17){\rule{1.300pt}{0.400pt}}
\multiput(359.00,189.17)(3.302,2.000){2}{\rule{0.650pt}{0.400pt}}
\multiput(365.00,192.61)(0.909,0.447){3}{\rule{0.767pt}{0.108pt}}
\multiput(365.00,191.17)(3.409,3.000){2}{\rule{0.383pt}{0.400pt}}
\multiput(370.00,195.61)(0.909,0.447){3}{\rule{0.767pt}{0.108pt}}
\multiput(370.00,194.17)(3.409,3.000){2}{\rule{0.383pt}{0.400pt}}
\multiput(375.00,198.60)(0.627,0.468){5}{\rule{0.600pt}{0.113pt}}
\multiput(375.00,197.17)(3.755,4.000){2}{\rule{0.300pt}{0.400pt}}
\multiput(380.00,202.61)(0.909,0.447){3}{\rule{0.767pt}{0.108pt}}
\multiput(380.00,201.17)(3.409,3.000){2}{\rule{0.383pt}{0.400pt}}
\multiput(385.00,205.60)(0.627,0.468){5}{\rule{0.600pt}{0.113pt}}
\multiput(385.00,204.17)(3.755,4.000){2}{\rule{0.300pt}{0.400pt}}
\multiput(390.00,209.60)(0.627,0.468){5}{\rule{0.600pt}{0.113pt}}
\multiput(390.00,208.17)(3.755,4.000){2}{\rule{0.300pt}{0.400pt}}
\multiput(395.00,213.59)(0.487,0.477){7}{\rule{0.500pt}{0.115pt}}
\multiput(395.00,212.17)(3.962,5.000){2}{\rule{0.250pt}{0.400pt}}
\multiput(400.00,218.60)(0.627,0.468){5}{\rule{0.600pt}{0.113pt}}
\multiput(400.00,217.17)(3.755,4.000){2}{\rule{0.300pt}{0.400pt}}
\multiput(405.00,222.59)(0.487,0.477){7}{\rule{0.500pt}{0.115pt}}
\multiput(405.00,221.17)(3.962,5.000){2}{\rule{0.250pt}{0.400pt}}
\multiput(410.00,227.59)(0.487,0.477){7}{\rule{0.500pt}{0.115pt}}
\multiput(410.00,226.17)(3.962,5.000){2}{\rule{0.250pt}{0.400pt}}
\multiput(415.00,232.59)(0.491,0.482){9}{\rule{0.500pt}{0.116pt}}
\multiput(415.00,231.17)(4.962,6.000){2}{\rule{0.250pt}{0.400pt}}
\multiput(421.00,238.59)(0.487,0.477){7}{\rule{0.500pt}{0.115pt}}
\multiput(421.00,237.17)(3.962,5.000){2}{\rule{0.250pt}{0.400pt}}
\multiput(426.59,243.00)(0.477,0.599){7}{\rule{0.115pt}{0.580pt}}
\multiput(425.17,243.00)(5.000,4.796){2}{\rule{0.400pt}{0.290pt}}
\multiput(431.59,249.00)(0.477,0.710){7}{\rule{0.115pt}{0.660pt}}
\multiput(430.17,249.00)(5.000,5.630){2}{\rule{0.400pt}{0.330pt}}
\multiput(436.59,256.00)(0.477,0.599){7}{\rule{0.115pt}{0.580pt}}
\multiput(435.17,256.00)(5.000,4.796){2}{\rule{0.400pt}{0.290pt}}
\multiput(441.59,262.00)(0.477,0.710){7}{\rule{0.115pt}{0.660pt}}
\multiput(440.17,262.00)(5.000,5.630){2}{\rule{0.400pt}{0.330pt}}
\multiput(446.59,269.00)(0.477,0.710){7}{\rule{0.115pt}{0.660pt}}
\multiput(445.17,269.00)(5.000,5.630){2}{\rule{0.400pt}{0.330pt}}
\multiput(451.59,276.00)(0.477,0.710){7}{\rule{0.115pt}{0.660pt}}
\multiput(450.17,276.00)(5.000,5.630){2}{\rule{0.400pt}{0.330pt}}
\multiput(456.59,283.00)(0.477,0.821){7}{\rule{0.115pt}{0.740pt}}
\multiput(455.17,283.00)(5.000,6.464){2}{\rule{0.400pt}{0.370pt}}
\multiput(461.59,291.00)(0.477,0.710){7}{\rule{0.115pt}{0.660pt}}
\multiput(460.17,291.00)(5.000,5.630){2}{\rule{0.400pt}{0.330pt}}
\multiput(466.59,298.00)(0.477,0.821){7}{\rule{0.115pt}{0.740pt}}
\multiput(465.17,298.00)(5.000,6.464){2}{\rule{0.400pt}{0.370pt}}
\multiput(471.59,306.00)(0.482,0.762){9}{\rule{0.116pt}{0.700pt}}
\multiput(470.17,306.00)(6.000,7.547){2}{\rule{0.400pt}{0.350pt}}
\multiput(477.59,315.00)(0.477,0.821){7}{\rule{0.115pt}{0.740pt}}
\multiput(476.17,315.00)(5.000,6.464){2}{\rule{0.400pt}{0.370pt}}
\multiput(482.59,323.00)(0.477,0.933){7}{\rule{0.115pt}{0.820pt}}
\multiput(481.17,323.00)(5.000,7.298){2}{\rule{0.400pt}{0.410pt}}
\multiput(487.59,332.00)(0.477,0.933){7}{\rule{0.115pt}{0.820pt}}
\multiput(486.17,332.00)(5.000,7.298){2}{\rule{0.400pt}{0.410pt}}
\multiput(492.59,341.00)(0.477,0.933){7}{\rule{0.115pt}{0.820pt}}
\multiput(491.17,341.00)(5.000,7.298){2}{\rule{0.400pt}{0.410pt}}
\multiput(497.59,350.00)(0.477,1.044){7}{\rule{0.115pt}{0.900pt}}
\multiput(496.17,350.00)(5.000,8.132){2}{\rule{0.400pt}{0.450pt}}
\multiput(502.61,360.00)(0.447,0.909){3}{\rule{0.108pt}{0.767pt}}
\multiput(501.17,360.00)(3.000,3.409){2}{\rule{0.400pt}{0.383pt}}
\put(309.0,178.0){\rule[-0.200pt]{2.409pt}{0.400pt}}
\put(133.00,365.00){\usebox{\plotpoint}}
\put(143.07,346.87){\usebox{\plotpoint}}
\put(152.46,328.36){\usebox{\plotpoint}}
\put(162.54,310.22){\usebox{\plotpoint}}
\put(173.23,292.43){\usebox{\plotpoint}}
\put(184.54,275.04){\usebox{\plotpoint}}
\put(196.74,258.31){\usebox{\plotpoint}}
\put(209.52,241.98){\usebox{\plotpoint}}
\put(223.41,226.59){\usebox{\plotpoint}}
\put(238.20,212.04){\usebox{\plotpoint}}
\put(254.78,199.58){\usebox{\plotpoint}}
\put(272.62,189.15){\usebox{\plotpoint}}
\put(291.74,181.50){\usebox{\plotpoint}}
\put(312.12,178.00){\usebox{\plotpoint}}
\put(332.70,179.74){\usebox{\plotpoint}}
\put(352.57,185.43){\usebox{\plotpoint}}
\put(371.11,194.66){\usebox{\plotpoint}}
\put(388.11,206.49){\usebox{\plotpoint}}
\put(403.44,220.44){\usebox{\plotpoint}}
\put(418.11,235.11){\usebox{\plotpoint}}
\put(431.17,251.20){\usebox{\plotpoint}}
\put(443.68,267.75){\usebox{\plotpoint}}
\put(454.88,285.21){\usebox{\plotpoint}}
\put(465.88,302.81){\usebox{\plotpoint}}
\put(476.68,320.52){\usebox{\plotpoint}}
\put(486.80,338.64){\usebox{\plotpoint}}
\put(496.10,357.19){\usebox{\plotpoint}}
\put(500,365){\usebox{\plotpoint}}
\sbox{\plotpoint}{\rule[-0.400pt]{0.800pt}{0.800pt}}%
\multiput(147.38,357.53)(0.560,-1.264){3}{\rule{0.135pt}{1.800pt}}
\multiput(144.34,361.26)(5.000,-6.264){2}{\rule{0.800pt}{0.900pt}}
\multiput(152.38,346.86)(0.560,-1.432){3}{\rule{0.135pt}{1.960pt}}
\multiput(149.34,350.93)(5.000,-6.932){2}{\rule{0.800pt}{0.980pt}}
\multiput(157.38,336.53)(0.560,-1.264){3}{\rule{0.135pt}{1.800pt}}
\multiput(154.34,340.26)(5.000,-6.264){2}{\rule{0.800pt}{0.900pt}}
\multiput(162.38,326.53)(0.560,-1.264){3}{\rule{0.135pt}{1.800pt}}
\multiput(159.34,330.26)(5.000,-6.264){2}{\rule{0.800pt}{0.900pt}}
\multiput(167.38,317.19)(0.560,-1.096){3}{\rule{0.135pt}{1.640pt}}
\multiput(164.34,320.60)(5.000,-5.596){2}{\rule{0.800pt}{0.820pt}}
\multiput(172.38,308.19)(0.560,-1.096){3}{\rule{0.135pt}{1.640pt}}
\multiput(169.34,311.60)(5.000,-5.596){2}{\rule{0.800pt}{0.820pt}}
\multiput(177.38,299.19)(0.560,-1.096){3}{\rule{0.135pt}{1.640pt}}
\multiput(174.34,302.60)(5.000,-5.596){2}{\rule{0.800pt}{0.820pt}}
\multiput(182.38,290.19)(0.560,-1.096){3}{\rule{0.135pt}{1.640pt}}
\multiput(179.34,293.60)(5.000,-5.596){2}{\rule{0.800pt}{0.820pt}}
\multiput(187.38,281.86)(0.560,-0.928){3}{\rule{0.135pt}{1.480pt}}
\multiput(184.34,284.93)(5.000,-4.928){2}{\rule{0.800pt}{0.740pt}}
\multiput(192.39,274.74)(0.536,-0.685){5}{\rule{0.129pt}{1.267pt}}
\multiput(189.34,277.37)(6.000,-5.371){2}{\rule{0.800pt}{0.633pt}}
\multiput(198.38,266.52)(0.560,-0.760){3}{\rule{0.135pt}{1.320pt}}
\multiput(195.34,269.26)(5.000,-4.260){2}{\rule{0.800pt}{0.660pt}}
\multiput(203.38,258.86)(0.560,-0.928){3}{\rule{0.135pt}{1.480pt}}
\multiput(200.34,261.93)(5.000,-4.928){2}{\rule{0.800pt}{0.740pt}}
\multiput(208.38,251.52)(0.560,-0.760){3}{\rule{0.135pt}{1.320pt}}
\multiput(205.34,254.26)(5.000,-4.260){2}{\rule{0.800pt}{0.660pt}}
\multiput(213.38,245.18)(0.560,-0.592){3}{\rule{0.135pt}{1.160pt}}
\multiput(210.34,247.59)(5.000,-3.592){2}{\rule{0.800pt}{0.580pt}}
\multiput(218.38,238.52)(0.560,-0.760){3}{\rule{0.135pt}{1.320pt}}
\multiput(215.34,241.26)(5.000,-4.260){2}{\rule{0.800pt}{0.660pt}}
\multiput(223.38,232.18)(0.560,-0.592){3}{\rule{0.135pt}{1.160pt}}
\multiput(220.34,234.59)(5.000,-3.592){2}{\rule{0.800pt}{0.580pt}}
\multiput(227.00,229.06)(0.424,-0.560){3}{\rule{1.000pt}{0.135pt}}
\multiput(227.00,229.34)(2.924,-5.000){2}{\rule{0.500pt}{0.800pt}}
\multiput(233.38,221.18)(0.560,-0.592){3}{\rule{0.135pt}{1.160pt}}
\multiput(230.34,223.59)(5.000,-3.592){2}{\rule{0.800pt}{0.580pt}}
\multiput(237.00,218.06)(0.424,-0.560){3}{\rule{1.000pt}{0.135pt}}
\multiput(237.00,218.34)(2.924,-5.000){2}{\rule{0.500pt}{0.800pt}}
\put(242,211.34){\rule{1.200pt}{0.800pt}}
\multiput(242.00,213.34)(2.509,-4.000){2}{\rule{0.600pt}{0.800pt}}
\multiput(247.00,209.06)(0.592,-0.560){3}{\rule{1.160pt}{0.135pt}}
\multiput(247.00,209.34)(3.592,-5.000){2}{\rule{0.580pt}{0.800pt}}
\put(253,202.34){\rule{1.200pt}{0.800pt}}
\multiput(253.00,204.34)(2.509,-4.000){2}{\rule{0.600pt}{0.800pt}}
\put(258,198.34){\rule{1.200pt}{0.800pt}}
\multiput(258.00,200.34)(2.509,-4.000){2}{\rule{0.600pt}{0.800pt}}
\put(263,194.84){\rule{1.204pt}{0.800pt}}
\multiput(263.00,196.34)(2.500,-3.000){2}{\rule{0.602pt}{0.800pt}}
\put(268,191.84){\rule{1.204pt}{0.800pt}}
\multiput(268.00,193.34)(2.500,-3.000){2}{\rule{0.602pt}{0.800pt}}
\put(273,188.84){\rule{1.204pt}{0.800pt}}
\multiput(273.00,190.34)(2.500,-3.000){2}{\rule{0.602pt}{0.800pt}}
\put(278,185.84){\rule{1.204pt}{0.800pt}}
\multiput(278.00,187.34)(2.500,-3.000){2}{\rule{0.602pt}{0.800pt}}
\put(283,183.34){\rule{1.204pt}{0.800pt}}
\multiput(283.00,184.34)(2.500,-2.000){2}{\rule{0.602pt}{0.800pt}}
\put(288,181.34){\rule{1.204pt}{0.800pt}}
\multiput(288.00,182.34)(2.500,-2.000){2}{\rule{0.602pt}{0.800pt}}
\put(293,179.84){\rule{1.204pt}{0.800pt}}
\multiput(293.00,180.34)(2.500,-1.000){2}{\rule{0.602pt}{0.800pt}}
\put(298,178.84){\rule{1.204pt}{0.800pt}}
\multiput(298.00,179.34)(2.500,-1.000){2}{\rule{0.602pt}{0.800pt}}
\put(303,177.84){\rule{1.445pt}{0.800pt}}
\multiput(303.00,178.34)(3.000,-1.000){2}{\rule{0.723pt}{0.800pt}}
\put(309,176.84){\rule{1.204pt}{0.800pt}}
\multiput(309.00,177.34)(2.500,-1.000){2}{\rule{0.602pt}{0.800pt}}
\put(329,176.84){\rule{1.204pt}{0.800pt}}
\multiput(329.00,176.34)(2.500,1.000){2}{\rule{0.602pt}{0.800pt}}
\put(334,177.84){\rule{1.204pt}{0.800pt}}
\multiput(334.00,177.34)(2.500,1.000){2}{\rule{0.602pt}{0.800pt}}
\put(339,178.84){\rule{1.204pt}{0.800pt}}
\multiput(339.00,178.34)(2.500,1.000){2}{\rule{0.602pt}{0.800pt}}
\put(344,180.34){\rule{1.204pt}{0.800pt}}
\multiput(344.00,179.34)(2.500,2.000){2}{\rule{0.602pt}{0.800pt}}
\put(349,182.34){\rule{1.204pt}{0.800pt}}
\multiput(349.00,181.34)(2.500,2.000){2}{\rule{0.602pt}{0.800pt}}
\put(354,184.34){\rule{1.204pt}{0.800pt}}
\multiput(354.00,183.34)(2.500,2.000){2}{\rule{0.602pt}{0.800pt}}
\put(359,186.34){\rule{1.445pt}{0.800pt}}
\multiput(359.00,185.34)(3.000,2.000){2}{\rule{0.723pt}{0.800pt}}
\put(365,188.84){\rule{1.204pt}{0.800pt}}
\multiput(365.00,187.34)(2.500,3.000){2}{\rule{0.602pt}{0.800pt}}
\put(370,192.34){\rule{1.200pt}{0.800pt}}
\multiput(370.00,190.34)(2.509,4.000){2}{\rule{0.600pt}{0.800pt}}
\put(375,195.84){\rule{1.204pt}{0.800pt}}
\multiput(375.00,194.34)(2.500,3.000){2}{\rule{0.602pt}{0.800pt}}
\put(380,199.34){\rule{1.200pt}{0.800pt}}
\multiput(380.00,197.34)(2.509,4.000){2}{\rule{0.600pt}{0.800pt}}
\put(385,203.34){\rule{1.200pt}{0.800pt}}
\multiput(385.00,201.34)(2.509,4.000){2}{\rule{0.600pt}{0.800pt}}
\multiput(390.00,208.38)(0.424,0.560){3}{\rule{1.000pt}{0.135pt}}
\multiput(390.00,205.34)(2.924,5.000){2}{\rule{0.500pt}{0.800pt}}
\put(395,212.34){\rule{1.200pt}{0.800pt}}
\multiput(395.00,210.34)(2.509,4.000){2}{\rule{0.600pt}{0.800pt}}
\multiput(400.00,217.38)(0.424,0.560){3}{\rule{1.000pt}{0.135pt}}
\multiput(400.00,214.34)(2.924,5.000){2}{\rule{0.500pt}{0.800pt}}
\multiput(406.38,221.00)(0.560,0.592){3}{\rule{0.135pt}{1.160pt}}
\multiput(403.34,221.00)(5.000,3.592){2}{\rule{0.800pt}{0.580pt}}
\multiput(411.38,227.00)(0.560,0.592){3}{\rule{0.135pt}{1.160pt}}
\multiput(408.34,227.00)(5.000,3.592){2}{\rule{0.800pt}{0.580pt}}
\multiput(415.00,234.39)(0.462,0.536){5}{\rule{1.000pt}{0.129pt}}
\multiput(415.00,231.34)(3.924,6.000){2}{\rule{0.500pt}{0.800pt}}
\multiput(422.38,239.00)(0.560,0.592){3}{\rule{0.135pt}{1.160pt}}
\multiput(419.34,239.00)(5.000,3.592){2}{\rule{0.800pt}{0.580pt}}
\multiput(427.38,245.00)(0.560,0.760){3}{\rule{0.135pt}{1.320pt}}
\multiput(424.34,245.00)(5.000,4.260){2}{\rule{0.800pt}{0.660pt}}
\multiput(432.38,252.00)(0.560,0.760){3}{\rule{0.135pt}{1.320pt}}
\multiput(429.34,252.00)(5.000,4.260){2}{\rule{0.800pt}{0.660pt}}
\multiput(437.38,259.00)(0.560,0.760){3}{\rule{0.135pt}{1.320pt}}
\multiput(434.34,259.00)(5.000,4.260){2}{\rule{0.800pt}{0.660pt}}
\multiput(442.38,266.00)(0.560,0.928){3}{\rule{0.135pt}{1.480pt}}
\multiput(439.34,266.00)(5.000,4.928){2}{\rule{0.800pt}{0.740pt}}
\multiput(447.38,274.00)(0.560,0.928){3}{\rule{0.135pt}{1.480pt}}
\multiput(444.34,274.00)(5.000,4.928){2}{\rule{0.800pt}{0.740pt}}
\multiput(452.38,282.00)(0.560,0.928){3}{\rule{0.135pt}{1.480pt}}
\multiput(449.34,282.00)(5.000,4.928){2}{\rule{0.800pt}{0.740pt}}
\multiput(457.38,290.00)(0.560,1.096){3}{\rule{0.135pt}{1.640pt}}
\multiput(454.34,290.00)(5.000,5.596){2}{\rule{0.800pt}{0.820pt}}
\multiput(462.38,299.00)(0.560,1.096){3}{\rule{0.135pt}{1.640pt}}
\multiput(459.34,299.00)(5.000,5.596){2}{\rule{0.800pt}{0.820pt}}
\multiput(467.38,308.00)(0.560,1.096){3}{\rule{0.135pt}{1.640pt}}
\multiput(464.34,308.00)(5.000,5.596){2}{\rule{0.800pt}{0.820pt}}
\multiput(472.39,317.00)(0.536,0.797){5}{\rule{0.129pt}{1.400pt}}
\multiput(469.34,317.00)(6.000,6.094){2}{\rule{0.800pt}{0.700pt}}
\multiput(478.38,326.00)(0.560,1.264){3}{\rule{0.135pt}{1.800pt}}
\multiput(475.34,326.00)(5.000,6.264){2}{\rule{0.800pt}{0.900pt}}
\multiput(483.38,336.00)(0.560,1.264){3}{\rule{0.135pt}{1.800pt}}
\multiput(480.34,336.00)(5.000,6.264){2}{\rule{0.800pt}{0.900pt}}
\multiput(488.38,346.00)(0.560,1.432){3}{\rule{0.135pt}{1.960pt}}
\multiput(485.34,346.00)(5.000,6.932){2}{\rule{0.800pt}{0.980pt}}
\put(492.34,357){\rule{0.800pt}{1.800pt}}
\multiput(490.34,357.00)(4.000,4.264){2}{\rule{0.800pt}{0.900pt}}
\put(314.0,178.0){\rule[-0.400pt]{3.613pt}{0.800pt}}
\sbox{\plotpoint}{\rule[-0.200pt]{0.400pt}{0.400pt}}%
\put(110.0,82.0){\rule[-0.200pt]{0.400pt}{68.175pt}}
\put(110.0,82.0){\rule[-0.200pt]{121.414pt}{0.400pt}}
\put(614.0,82.0){\rule[-0.200pt]{0.400pt}{68.175pt}}
\put(110.0,365.0){\rule[-0.200pt]{121.414pt}{0.400pt}}
\end{picture}
}%
		\subfloat[Trois Points]{\label{graph:polyfixed}% GNUPLOT: LaTeX picture
\setlength{\unitlength}{0.240900pt}
\ifx\plotpoint\undefined\newsavebox{\plotpoint}\fi
\sbox{\plotpoint}{\rule[-0.200pt]{0.400pt}{0.400pt}}%
\begin{picture}(675,405)(0,0)
\sbox{\plotpoint}{\rule[-0.200pt]{0.400pt}{0.400pt}}%
\put(110.0,82.0){\rule[-0.200pt]{4.818pt}{0.400pt}}
\put(90,82){\makebox(0,0)[r]{-10}}
\put(594.0,82.0){\rule[-0.200pt]{4.818pt}{0.400pt}}
\put(110.0,139.0){\rule[-0.200pt]{4.818pt}{0.400pt}}
\put(90,139){\makebox(0,0)[r]{ 0}}
\put(594.0,139.0){\rule[-0.200pt]{4.818pt}{0.400pt}}
\put(110.0,195.0){\rule[-0.200pt]{4.818pt}{0.400pt}}
\put(90,195){\makebox(0,0)[r]{ 10}}
\put(594.0,195.0){\rule[-0.200pt]{4.818pt}{0.400pt}}
\put(110.0,252.0){\rule[-0.200pt]{4.818pt}{0.400pt}}
\put(90,252){\makebox(0,0)[r]{ 20}}
\put(594.0,252.0){\rule[-0.200pt]{4.818pt}{0.400pt}}
\put(110.0,308.0){\rule[-0.200pt]{4.818pt}{0.400pt}}
\put(90,308){\makebox(0,0)[r]{ 30}}
\put(594.0,308.0){\rule[-0.200pt]{4.818pt}{0.400pt}}
\put(110.0,365.0){\rule[-0.200pt]{4.818pt}{0.400pt}}
\put(90,365){\makebox(0,0)[r]{ 40}}
\put(594.0,365.0){\rule[-0.200pt]{4.818pt}{0.400pt}}
\put(110.0,82.0){\rule[-0.200pt]{0.400pt}{4.818pt}}
\put(110,41){\makebox(0,0){-1}}
\put(110.0,345.0){\rule[-0.200pt]{0.400pt}{4.818pt}}
\put(211.0,82.0){\rule[-0.200pt]{0.400pt}{4.818pt}}
\put(211,41){\makebox(0,0){ 0}}
\put(211.0,345.0){\rule[-0.200pt]{0.400pt}{4.818pt}}
\put(312.0,82.0){\rule[-0.200pt]{0.400pt}{4.818pt}}
\put(312,41){\makebox(0,0){ 1}}
\put(312.0,345.0){\rule[-0.200pt]{0.400pt}{4.818pt}}
\put(412.0,82.0){\rule[-0.200pt]{0.400pt}{4.818pt}}
\put(412,41){\makebox(0,0){ 2}}
\put(412.0,345.0){\rule[-0.200pt]{0.400pt}{4.818pt}}
\put(513.0,82.0){\rule[-0.200pt]{0.400pt}{4.818pt}}
\put(513,41){\makebox(0,0){ 3}}
\put(513.0,345.0){\rule[-0.200pt]{0.400pt}{4.818pt}}
\put(614.0,82.0){\rule[-0.200pt]{0.400pt}{4.818pt}}
\put(614,41){\makebox(0,0){ 4}}
\put(614.0,345.0){\rule[-0.200pt]{0.400pt}{4.818pt}}
\put(110.0,82.0){\rule[-0.200pt]{0.400pt}{68.175pt}}
\put(110.0,82.0){\rule[-0.200pt]{121.414pt}{0.400pt}}
\put(614.0,82.0){\rule[-0.200pt]{0.400pt}{68.175pt}}
\put(110.0,365.0){\rule[-0.200pt]{121.414pt}{0.400pt}}
\put(110,144){\usebox{\plotpoint}}
\put(161,143.67){\rule{1.204pt}{0.400pt}}
\multiput(161.00,143.17)(2.500,1.000){2}{\rule{0.602pt}{0.400pt}}
\put(110.0,144.0){\rule[-0.200pt]{12.286pt}{0.400pt}}
\put(176,144.67){\rule{1.204pt}{0.400pt}}
\multiput(176.00,144.17)(2.500,1.000){2}{\rule{0.602pt}{0.400pt}}
\put(166.0,145.0){\rule[-0.200pt]{2.409pt}{0.400pt}}
\put(186,145.67){\rule{1.204pt}{0.400pt}}
\multiput(186.00,145.17)(2.500,1.000){2}{\rule{0.602pt}{0.400pt}}
\put(191,146.67){\rule{1.445pt}{0.400pt}}
\multiput(191.00,146.17)(3.000,1.000){2}{\rule{0.723pt}{0.400pt}}
\put(181.0,146.0){\rule[-0.200pt]{1.204pt}{0.400pt}}
\put(202,147.67){\rule{1.204pt}{0.400pt}}
\multiput(202.00,147.17)(2.500,1.000){2}{\rule{0.602pt}{0.400pt}}
\put(207,148.67){\rule{1.204pt}{0.400pt}}
\multiput(207.00,148.17)(2.500,1.000){2}{\rule{0.602pt}{0.400pt}}
\put(212,149.67){\rule{1.204pt}{0.400pt}}
\multiput(212.00,149.17)(2.500,1.000){2}{\rule{0.602pt}{0.400pt}}
\put(217,150.67){\rule{1.204pt}{0.400pt}}
\multiput(217.00,150.17)(2.500,1.000){2}{\rule{0.602pt}{0.400pt}}
\put(222,151.67){\rule{1.204pt}{0.400pt}}
\multiput(222.00,151.17)(2.500,1.000){2}{\rule{0.602pt}{0.400pt}}
\put(227,152.67){\rule{1.204pt}{0.400pt}}
\multiput(227.00,152.17)(2.500,1.000){2}{\rule{0.602pt}{0.400pt}}
\put(232,153.67){\rule{1.204pt}{0.400pt}}
\multiput(232.00,153.17)(2.500,1.000){2}{\rule{0.602pt}{0.400pt}}
\put(237,154.67){\rule{1.204pt}{0.400pt}}
\multiput(237.00,154.17)(2.500,1.000){2}{\rule{0.602pt}{0.400pt}}
\put(242,156.17){\rule{1.100pt}{0.400pt}}
\multiput(242.00,155.17)(2.717,2.000){2}{\rule{0.550pt}{0.400pt}}
\put(247,157.67){\rule{1.445pt}{0.400pt}}
\multiput(247.00,157.17)(3.000,1.000){2}{\rule{0.723pt}{0.400pt}}
\put(253,158.67){\rule{1.204pt}{0.400pt}}
\multiput(253.00,158.17)(2.500,1.000){2}{\rule{0.602pt}{0.400pt}}
\put(258,160.17){\rule{1.100pt}{0.400pt}}
\multiput(258.00,159.17)(2.717,2.000){2}{\rule{0.550pt}{0.400pt}}
\put(263,161.67){\rule{1.204pt}{0.400pt}}
\multiput(263.00,161.17)(2.500,1.000){2}{\rule{0.602pt}{0.400pt}}
\put(268,163.17){\rule{1.100pt}{0.400pt}}
\multiput(268.00,162.17)(2.717,2.000){2}{\rule{0.550pt}{0.400pt}}
\put(273,164.67){\rule{1.204pt}{0.400pt}}
\multiput(273.00,164.17)(2.500,1.000){2}{\rule{0.602pt}{0.400pt}}
\put(278,166.17){\rule{1.100pt}{0.400pt}}
\multiput(278.00,165.17)(2.717,2.000){2}{\rule{0.550pt}{0.400pt}}
\put(283,168.17){\rule{1.100pt}{0.400pt}}
\multiput(283.00,167.17)(2.717,2.000){2}{\rule{0.550pt}{0.400pt}}
\put(288,169.67){\rule{1.204pt}{0.400pt}}
\multiput(288.00,169.17)(2.500,1.000){2}{\rule{0.602pt}{0.400pt}}
\put(293,171.17){\rule{1.100pt}{0.400pt}}
\multiput(293.00,170.17)(2.717,2.000){2}{\rule{0.550pt}{0.400pt}}
\put(298,173.17){\rule{1.100pt}{0.400pt}}
\multiput(298.00,172.17)(2.717,2.000){2}{\rule{0.550pt}{0.400pt}}
\put(303,175.17){\rule{1.300pt}{0.400pt}}
\multiput(303.00,174.17)(3.302,2.000){2}{\rule{0.650pt}{0.400pt}}
\put(309,177.17){\rule{1.100pt}{0.400pt}}
\multiput(309.00,176.17)(2.717,2.000){2}{\rule{0.550pt}{0.400pt}}
\put(314,179.17){\rule{1.100pt}{0.400pt}}
\multiput(314.00,178.17)(2.717,2.000){2}{\rule{0.550pt}{0.400pt}}
\put(319,181.17){\rule{1.100pt}{0.400pt}}
\multiput(319.00,180.17)(2.717,2.000){2}{\rule{0.550pt}{0.400pt}}
\put(324,183.17){\rule{1.100pt}{0.400pt}}
\multiput(324.00,182.17)(2.717,2.000){2}{\rule{0.550pt}{0.400pt}}
\multiput(329.00,185.61)(0.909,0.447){3}{\rule{0.767pt}{0.108pt}}
\multiput(329.00,184.17)(3.409,3.000){2}{\rule{0.383pt}{0.400pt}}
\put(334,188.17){\rule{1.100pt}{0.400pt}}
\multiput(334.00,187.17)(2.717,2.000){2}{\rule{0.550pt}{0.400pt}}
\put(339,190.17){\rule{1.100pt}{0.400pt}}
\multiput(339.00,189.17)(2.717,2.000){2}{\rule{0.550pt}{0.400pt}}
\multiput(344.00,192.61)(0.909,0.447){3}{\rule{0.767pt}{0.108pt}}
\multiput(344.00,191.17)(3.409,3.000){2}{\rule{0.383pt}{0.400pt}}
\put(349,195.17){\rule{1.100pt}{0.400pt}}
\multiput(349.00,194.17)(2.717,2.000){2}{\rule{0.550pt}{0.400pt}}
\multiput(354.00,197.61)(0.909,0.447){3}{\rule{0.767pt}{0.108pt}}
\multiput(354.00,196.17)(3.409,3.000){2}{\rule{0.383pt}{0.400pt}}
\put(359,200.17){\rule{1.300pt}{0.400pt}}
\multiput(359.00,199.17)(3.302,2.000){2}{\rule{0.650pt}{0.400pt}}
\multiput(365.00,202.61)(0.909,0.447){3}{\rule{0.767pt}{0.108pt}}
\multiput(365.00,201.17)(3.409,3.000){2}{\rule{0.383pt}{0.400pt}}
\put(370,205.17){\rule{1.100pt}{0.400pt}}
\multiput(370.00,204.17)(2.717,2.000){2}{\rule{0.550pt}{0.400pt}}
\multiput(375.00,207.61)(0.909,0.447){3}{\rule{0.767pt}{0.108pt}}
\multiput(375.00,206.17)(3.409,3.000){2}{\rule{0.383pt}{0.400pt}}
\multiput(380.00,210.61)(0.909,0.447){3}{\rule{0.767pt}{0.108pt}}
\multiput(380.00,209.17)(3.409,3.000){2}{\rule{0.383pt}{0.400pt}}
\multiput(385.00,213.61)(0.909,0.447){3}{\rule{0.767pt}{0.108pt}}
\multiput(385.00,212.17)(3.409,3.000){2}{\rule{0.383pt}{0.400pt}}
\multiput(390.00,216.61)(0.909,0.447){3}{\rule{0.767pt}{0.108pt}}
\multiput(390.00,215.17)(3.409,3.000){2}{\rule{0.383pt}{0.400pt}}
\multiput(395.00,219.61)(0.909,0.447){3}{\rule{0.767pt}{0.108pt}}
\multiput(395.00,218.17)(3.409,3.000){2}{\rule{0.383pt}{0.400pt}}
\multiput(400.00,222.61)(0.909,0.447){3}{\rule{0.767pt}{0.108pt}}
\multiput(400.00,221.17)(3.409,3.000){2}{\rule{0.383pt}{0.400pt}}
\multiput(405.00,225.61)(0.909,0.447){3}{\rule{0.767pt}{0.108pt}}
\multiput(405.00,224.17)(3.409,3.000){2}{\rule{0.383pt}{0.400pt}}
\multiput(410.00,228.61)(0.909,0.447){3}{\rule{0.767pt}{0.108pt}}
\multiput(410.00,227.17)(3.409,3.000){2}{\rule{0.383pt}{0.400pt}}
\multiput(415.00,231.61)(1.132,0.447){3}{\rule{0.900pt}{0.108pt}}
\multiput(415.00,230.17)(4.132,3.000){2}{\rule{0.450pt}{0.400pt}}
\multiput(421.00,234.60)(0.627,0.468){5}{\rule{0.600pt}{0.113pt}}
\multiput(421.00,233.17)(3.755,4.000){2}{\rule{0.300pt}{0.400pt}}
\multiput(426.00,238.61)(0.909,0.447){3}{\rule{0.767pt}{0.108pt}}
\multiput(426.00,237.17)(3.409,3.000){2}{\rule{0.383pt}{0.400pt}}
\multiput(431.00,241.61)(0.909,0.447){3}{\rule{0.767pt}{0.108pt}}
\multiput(431.00,240.17)(3.409,3.000){2}{\rule{0.383pt}{0.400pt}}
\multiput(436.00,244.60)(0.627,0.468){5}{\rule{0.600pt}{0.113pt}}
\multiput(436.00,243.17)(3.755,4.000){2}{\rule{0.300pt}{0.400pt}}
\multiput(441.00,248.61)(0.909,0.447){3}{\rule{0.767pt}{0.108pt}}
\multiput(441.00,247.17)(3.409,3.000){2}{\rule{0.383pt}{0.400pt}}
\multiput(446.00,251.60)(0.627,0.468){5}{\rule{0.600pt}{0.113pt}}
\multiput(446.00,250.17)(3.755,4.000){2}{\rule{0.300pt}{0.400pt}}
\multiput(451.00,255.61)(0.909,0.447){3}{\rule{0.767pt}{0.108pt}}
\multiput(451.00,254.17)(3.409,3.000){2}{\rule{0.383pt}{0.400pt}}
\multiput(456.00,258.60)(0.627,0.468){5}{\rule{0.600pt}{0.113pt}}
\multiput(456.00,257.17)(3.755,4.000){2}{\rule{0.300pt}{0.400pt}}
\multiput(461.00,262.60)(0.627,0.468){5}{\rule{0.600pt}{0.113pt}}
\multiput(461.00,261.17)(3.755,4.000){2}{\rule{0.300pt}{0.400pt}}
\multiput(466.00,266.60)(0.627,0.468){5}{\rule{0.600pt}{0.113pt}}
\multiput(466.00,265.17)(3.755,4.000){2}{\rule{0.300pt}{0.400pt}}
\multiput(471.00,270.61)(1.132,0.447){3}{\rule{0.900pt}{0.108pt}}
\multiput(471.00,269.17)(4.132,3.000){2}{\rule{0.450pt}{0.400pt}}
\multiput(477.00,273.60)(0.627,0.468){5}{\rule{0.600pt}{0.113pt}}
\multiput(477.00,272.17)(3.755,4.000){2}{\rule{0.300pt}{0.400pt}}
\multiput(482.00,277.60)(0.627,0.468){5}{\rule{0.600pt}{0.113pt}}
\multiput(482.00,276.17)(3.755,4.000){2}{\rule{0.300pt}{0.400pt}}
\multiput(487.00,281.60)(0.627,0.468){5}{\rule{0.600pt}{0.113pt}}
\multiput(487.00,280.17)(3.755,4.000){2}{\rule{0.300pt}{0.400pt}}
\multiput(492.00,285.60)(0.627,0.468){5}{\rule{0.600pt}{0.113pt}}
\multiput(492.00,284.17)(3.755,4.000){2}{\rule{0.300pt}{0.400pt}}
\multiput(497.00,289.60)(0.627,0.468){5}{\rule{0.600pt}{0.113pt}}
\multiput(497.00,288.17)(3.755,4.000){2}{\rule{0.300pt}{0.400pt}}
\multiput(502.00,293.59)(0.487,0.477){7}{\rule{0.500pt}{0.115pt}}
\multiput(502.00,292.17)(3.962,5.000){2}{\rule{0.250pt}{0.400pt}}
\multiput(507.00,298.60)(0.627,0.468){5}{\rule{0.600pt}{0.113pt}}
\multiput(507.00,297.17)(3.755,4.000){2}{\rule{0.300pt}{0.400pt}}
\multiput(512.00,302.60)(0.627,0.468){5}{\rule{0.600pt}{0.113pt}}
\multiput(512.00,301.17)(3.755,4.000){2}{\rule{0.300pt}{0.400pt}}
\multiput(517.00,306.59)(0.487,0.477){7}{\rule{0.500pt}{0.115pt}}
\multiput(517.00,305.17)(3.962,5.000){2}{\rule{0.250pt}{0.400pt}}
\multiput(522.00,311.60)(0.627,0.468){5}{\rule{0.600pt}{0.113pt}}
\multiput(522.00,310.17)(3.755,4.000){2}{\rule{0.300pt}{0.400pt}}
\multiput(527.00,315.60)(0.774,0.468){5}{\rule{0.700pt}{0.113pt}}
\multiput(527.00,314.17)(4.547,4.000){2}{\rule{0.350pt}{0.400pt}}
\multiput(533.00,319.59)(0.487,0.477){7}{\rule{0.500pt}{0.115pt}}
\multiput(533.00,318.17)(3.962,5.000){2}{\rule{0.250pt}{0.400pt}}
\multiput(538.00,324.59)(0.487,0.477){7}{\rule{0.500pt}{0.115pt}}
\multiput(538.00,323.17)(3.962,5.000){2}{\rule{0.250pt}{0.400pt}}
\multiput(543.00,329.60)(0.627,0.468){5}{\rule{0.600pt}{0.113pt}}
\multiput(543.00,328.17)(3.755,4.000){2}{\rule{0.300pt}{0.400pt}}
\multiput(548.00,333.59)(0.487,0.477){7}{\rule{0.500pt}{0.115pt}}
\multiput(548.00,332.17)(3.962,5.000){2}{\rule{0.250pt}{0.400pt}}
\multiput(553.00,338.59)(0.487,0.477){7}{\rule{0.500pt}{0.115pt}}
\multiput(553.00,337.17)(3.962,5.000){2}{\rule{0.250pt}{0.400pt}}
\multiput(558.00,343.59)(0.487,0.477){7}{\rule{0.500pt}{0.115pt}}
\multiput(558.00,342.17)(3.962,5.000){2}{\rule{0.250pt}{0.400pt}}
\multiput(563.00,348.60)(0.627,0.468){5}{\rule{0.600pt}{0.113pt}}
\multiput(563.00,347.17)(3.755,4.000){2}{\rule{0.300pt}{0.400pt}}
\multiput(568.00,352.59)(0.487,0.477){7}{\rule{0.500pt}{0.115pt}}
\multiput(568.00,351.17)(3.962,5.000){2}{\rule{0.250pt}{0.400pt}}
\multiput(573.00,357.59)(0.487,0.477){7}{\rule{0.500pt}{0.115pt}}
\multiput(573.00,356.17)(3.962,5.000){2}{\rule{0.250pt}{0.400pt}}
\multiput(578.00,362.61)(0.462,0.447){3}{\rule{0.500pt}{0.108pt}}
\multiput(578.00,361.17)(1.962,3.000){2}{\rule{0.250pt}{0.400pt}}
\put(197.0,148.0){\rule[-0.200pt]{1.204pt}{0.400pt}}
\put(110.0,82.0){\rule[-0.200pt]{0.400pt}{68.175pt}}
\put(110.0,82.0){\rule[-0.200pt]{121.414pt}{0.400pt}}
\put(614.0,82.0){\rule[-0.200pt]{0.400pt}{68.175pt}}
\put(110.0,365.0){\rule[-0.200pt]{121.414pt}{0.400pt}}
\end{picture}
}% 
		\caption{Diverses résultats avec deux et trois points dans $\mathbb{R}$.}
		\label{graph:polyboth}
	\end{figure}

	De façon équivalente cette idée fonctionne pour les anneaux $\Mgr{Z}{n^{s+1}}$ et on peut créer des clés distribuées.
	L'autorité génère un polynôme $f$ de degré $k-1$ tel que $f(0) = 2N_s!d$, distribue à chacun des serveurs $f(i)$. 
	Alors, si $k$ serveurs collaborent, ils retrouvent $f$ donc $f(0)$. La clé est $\frac{f(0)}{2N_s!}$. 
	%Ici on parle de exposants, alors $f$ est calculé dans \Mgrinv{}

	\begin{algo}{Création des Clés}
		\begin{itemize}\renewcommand{\labelitemi}{} \renewcommand{\labelitemii}{$\cdot$}
			\item{\bf Entrée:} 
				\begin{itemize}
					\item $N_s$: le nombre de serveurs.
					\item $k$: le nombre de serveurs nécessaires pour le déchiffrement.
					\item $d$: la clé privé.
				\end{itemize}
			\item{\bf Calcul:}
				\begin{enumerate} %			
				\renewcommand{\theenumi}{\arabic{enumi}}
				\renewcommand{\theenumii}{\arabic{enumii}}
				\renewcommand{\theenumiii}{\arabic{enumiii}}
				\renewcommand{\labelenumi}{\theenumi.}
				\renewcommand{\labelenumii}{\theenumi.\theenumii.}
				\renewcommand{\labelenumiii}{\theenumi.\theenumii.\theenumiii.}
				\makeatletter
				\renewcommand{\p@enumii}{\theenumi.}
				\renewcommand{\p@enumiii}{\theenumi.\theenumii.}
				\makeatother
				\item Choisir $p$ et $q$ premiers tels que $p = 2p'+1$ et $q = 2q'+1$ avec $p'$ et $q'$ premiers.
				\item Poser $n = pq$ et $m = p'q'$
				\item Choisir $k-1$ valeurs: $r_i\in_R \Mgr{Z}{n^sm}$, $1 \leq i \leq k-1$
				\item Pour $j$ dans $[1,N_s]$:
					\begin{enumerate}
						\item Envoyer la clé $f(j) = 2N_s!(d+\sum_{i = 1}^{k-1}r_ij^i)\mod{n^sm}$ au serveur $j$.
					\end{enumerate}
				\end{enumerate}
		\end{itemize}
	\end{algo}

	Grâce à l'interpolation de Lagrange on peut trouver facilement les coefficients du polynôme et finalement sa valeur 
	en $0$: $2dN_s!$.

%	\begin{theorem} Interpolation Lagrangienne: Soient les paires $(x_i,y_i)$ avec $x_i$ distincts, le polynôme $$L(X) = \sum_{j=0}^{n}y_j\left(\prod_{i=0,i\neq j}^{n} \frac{X-x_i}{x_j-x_i}\right)$$ quand calculé dans $x_i$  résulte le valeur $y_i$, pour $0\leq i\leq n$.
%
%	\begin{proof}
%	On pose $l_j$ le polynôme:
%		$$l_j = \prod_{i=0,i\neq j}^{n} \frac{X-x_i}{x_j-x_i}$$
%		clairement quand on calcule $l_j$ dans $x_k, k\neq i$, il vaut $0$. Quand $k=j$, toutes les termes du produit veulent $1$ et 
%		par conséquence $l_j = 1$. En effet:
%
%		$$L(x_k) = \sum_{j=0}^{n}y_jl_j = y_kl_k+\sum_{j=0,j\neq k}^{n}y_jl_j = y_k\cdot 1+\sum_{j=0,j\neq k}^ny_j\cdot0 = y_k$$
%		et le polynôme interpole les points fournis.
%	\end{proof}
%	\end{theorem}


	\begin{theorem} Interpolation Lagrangienne dans l'anneau $\Mgr{Z}{n^{s+1}}$: Soient les entiers   $x_i \in \Mgr{Z}{N_s}$ distincts et un polynôme $f$ de degré $k\leq N_s<\frac{n-1}{2}$ à coefficients entiers, l'expression:
	$$ c^{L_Z(X)} \equiv c^{N_s!\cdot f(X)}\mod{n^{s+1}}, X\in\Mgr{Z}{_\frac{(n-1)}{2}} $$
	est valide quand: $$L_Z(X) = \sum_{j=0}^{k}f(x_j)N_s!\left(\prod_{i=0,i\neq j}^{k} \frac{X-x_i}{x_j-x_i}\right).$$ 
%	quand calculé dans $x_i$  résulte le valeur $N_s!y_i$, pour $0\leq i\leq n$.
	\begin{proof}
	On pose $\lambda_{x,j}$ l'expression pour $x$ entier $\leq N_s$:
		$$\lambda_{x,j} = N_s!\prod_{i=0,i\neq j}^{k} \frac{x-x_i}{x_j-x_i}$$
		elle est entière pour $x$ entière: $\binom{N_s}{z} = \frac{N_s!}{z!(N_s-z)!},z\in\mathbb{Z}$ est entière et alors $\frac{N_s!}{\prod|x_j-x_i|}$ est clairement
		entière.
		
		Il est évident que $\lambda_{x_t,j} = 0$, quand $t\neq j$ ($x_t-x_t$ multiplie le terme). Quand $t=j$ toutes les termes du produit valent $1$ et 
		par conséquent $\lambda_{x_j,j} = N_s!$. On a donc:
		$$  L_Z(x_h) = \sum_{j=0}^{k}f(x_j)\lambda_{x_h,j} = f(x_h) N_s! + \sum_{j=0,j\neq h}^{k}f(x_j)\cdot 0 = f(x_h)N_s! $$
%		$$L_Z(X) = \sum_{j=0}^{k}f(x_j)N_s!\left(\prod_{i=0,i\neq j}^{k} \frac{X-x_i}{x_j-x_i}\right) = f(x_j)\cdot N_s! + $$
%		$$L(x_k) = \sum_{j=0}^{k}y_jl_j = y_kl_k+\sum_{j=0,j\neq k}^{n}y_jl_j = y_k\cdot 1+\sum_{j=0,j\neq k}^ny_j\cdot0 = y_k$$
		et $c^{L_Z(X)} = c^{N_s!f(X)}$.
		%le polynôme interpole les points fournis.
	\end{proof}
	\end{theorem}


	Soit $N_s$ le nombre de serveurs, $k$ le seuil (nombre de serveurs tiers non tous corruptibles nécessaires à calculer $d$), 
	$r_i\in_R\Mgr{Z}{n^sm}$, 
	$\lambda(n)$ la fonction de Carmichael et $d$ la valeur définie dans l'expression (\ref{eq:defd}), on pose $f$ la fonction:
			\begin{equation}
				f : \begin{array}[t]{ccl} \mathbb{Z} &\rightarrow &   \Mgr{Z}{\phi(n^{s+1})}  \\
					x  &\mapsto& 2N_s!(d + \sum_{i = 1}^{k-1} x^ir_i) \mod{\phi(n^{s+1})} 
				\end{array}
			\end{equation}
	Quand calculé en $0$, elle nous donne $2N_s!d$ et alors, on peut utiliser l'interpolation de Lagrange pour 
	distribuer les clés $f(i)$ et composer la puissance du mot chiffré $c^{f(0)2N_s!} =  c^{4N_s!^2d}$ à la fin. 
	Avec l'équation \ref{eq:lambdad}:
		
			\begin{equation}c^{4N_s!^2d} \equiv (1+n)^{m4N_s!^2} \mod{n^{s+1}} \end{equation}

	Le facteur $4N_s!^2$ peut être enlevé avec le pré calcul de $(1+n)^{-4N_s!^2}\bmod{n^{s+1}}$. 

	\begin{algorithm}
		\caption{Procédures de calcul de logarithme modulaire}
		\label{algo:logm}
		\begin{algorithmic}[1]
			\Function{init\_values}{$maxS,n$}
				\State $Cn_0 = 1$
				\For{$j \gets 1 \textrm{ to }maxS$}
					\State $Cn_j = Cn_{j-1}\cdot n$
					\State $FE_{0,j} = 1$ 
					\For{$k \gets 1 \textrm{ to } j$}
						\State $FE_{k,j} = FE_{k-1,j}\cdot n \cdot k^{-1}\%Cn_{j}$ 
					\EndFor
				\EndFor
			\EndFunction
			\Function{find\_next\_m}{$m,c, j$}
				\State $l\gets ((c\%Cn_{j+1})-1)/n$ 
				\State $b \gets m$
				\For {$ k \gets 2 \textrm{ to } j$}
					\State $b \gets b\cdot(m-k+1) \% Cn_j$
					\State $l \gets (l- b\cdot FE_{k,j} )\% Cn_j$
				\EndFor
				\Return{$l$}
			\EndFunction
			\Function{logarithm\_mod}{$c,s$}
				\State $m \gets 0$
				\For {$j \gets 1 \textrm{ to }s$}
					\State $m = \text{FIND\_NEXT\_M($m,c,j$)}$
				\EndFor
				\Return{m}
			\EndFunction
		\end{algorithmic}
	\end{algorithm}
		\begin{figure}
			\center
			\caption{algorithme reparti pour la preuve d'égalité du logarithme.}
			%LaTeX with PSTricks extensions
%%Creator: inkscape 0.48.3.1
%%Please note this file requires PSTricks extensions
\psset{xunit=.5pt,yunit=.5pt,runit=.5pt}
\begin{pspicture}(640.15838623,490.95605469)
{
\newrgbcolor{curcolor}{0 0 0}
\pscustom[linewidth=0.69359075,linecolor=curcolor]
{
\newpath
\moveto(390.80745369,309.78225812)
\curveto(393.38445544,208.11393152)(338.46749373,4.6589667)(96.61054568,3.05677761)
}
}
{
\newrgbcolor{curcolor}{0 0 0}
\pscustom[linestyle=none,fillstyle=solid,fillcolor=curcolor]
{
\newpath
\moveto(103.54630102,3.10272374)
\lineto(106.30222471,5.89540433)
\lineto(96.61054568,3.05677761)
\lineto(106.33898162,0.34680006)
\lineto(103.54630102,3.10272374)
\closepath
}
}
{
\newrgbcolor{curcolor}{0 0 0}
\pscustom[linewidth=0.69359075,linecolor=curcolor]
{
\newpath
\moveto(103.54630102,3.10272374)
\lineto(106.30222471,5.89540433)
\lineto(96.61054568,3.05677761)
\lineto(106.33898162,0.34680006)
\lineto(103.54630102,3.10272374)
\closepath
}
}
{
\newrgbcolor{curcolor}{0 0 0}
\pscustom[linestyle=none,fillstyle=solid,fillcolor=curcolor]
{
\newpath
\moveto(51.28402995,255.20192898)
\lineto(51.399777,252.74809148)
\lineto(51.10655114,252.74809148)
\curveto(51.04995478,253.18020537)(50.97279016,253.48886387)(50.87505704,253.6740679)
\curveto(50.71557474,253.97242884)(50.50208595,254.19106194)(50.23459002,254.32996786)
\curveto(49.9722222,254.4739989)(49.62498139,254.54601921)(49.19286655,254.54602902)
\lineto(47.71902076,254.54602902)
\lineto(47.71902076,246.55176598)
\curveto(47.71901518,245.90872563)(47.78846334,245.50746959)(47.92736545,245.34799664)
\curveto(48.12284338,245.13193509)(48.42378541,245.02390461)(48.83019245,245.0239049)
\lineto(49.19286655,245.0239049)
\lineto(49.19286655,244.7383955)
\lineto(44.75589623,244.7383955)
\lineto(44.75589623,245.0239049)
\lineto(45.1262868,245.0239049)
\curveto(45.56869432,245.02390461)(45.88249713,245.15765663)(46.06769615,245.42516134)
\curveto(46.18086701,245.58977852)(46.2374544,245.96531302)(46.2374585,246.55176598)
\lineto(46.2374585,254.54602902)
\lineto(44.97967387,254.54602902)
\curveto(44.49096174,254.54601921)(44.14372093,254.51000906)(43.9379504,254.43799844)
\curveto(43.67044457,254.34024688)(43.44152285,254.15247963)(43.25118456,253.87469612)
\curveto(43.06084404,253.59689434)(42.94766926,253.22135984)(42.91165988,252.74809148)
\lineto(42.61843401,252.74809148)
\lineto(42.74189753,255.20192898)
\lineto(51.28402995,255.20192898)
}
}
{
\newrgbcolor{curcolor}{0 0 0}
\pscustom[linestyle=none,fillstyle=solid,fillcolor=curcolor]
{
\newpath
\moveto(53.83818156,252.01502682)
\lineto(53.83818156,250.42543398)
\curveto(54.42977445,251.4851558)(55.03680283,252.01501955)(55.65926851,252.01502682)
\curveto(55.94220108,252.01501955)(56.17626711,251.92756631)(56.36146729,251.75266684)
\curveto(56.5466573,251.58289765)(56.63925485,251.38484178)(56.63926021,251.15849864)
\curveto(56.63925485,250.9578642)(56.57237884,250.78810202)(56.43863199,250.64921161)
\curveto(56.30487481,250.51030938)(56.14540126,250.44086122)(55.96021084,250.44086692)
\curveto(55.78015537,250.44086122)(55.57695519,250.52831446)(55.3506097,250.7032269)
\curveto(55.12940037,250.88327173)(54.96478251,250.97329712)(54.85675562,250.97330336)
\curveto(54.76415449,250.97329712)(54.66384048,250.92185404)(54.55581328,250.81897396)
\curveto(54.32431613,250.60805124)(54.0851058,250.26081043)(53.83818156,249.77725049)
\lineto(53.83818156,246.38972011)
\curveto(53.838179,245.99875103)(53.88704993,245.7029533)(53.98479449,245.50232604)
\curveto(54.05166779,245.36342896)(54.16998688,245.24768202)(54.33975212,245.15508489)
\curveto(54.50951123,245.06248692)(54.75386587,245.01618815)(55.07281678,245.01618843)
\lineto(55.07281678,244.7383955)
\lineto(51.45379229,244.7383955)
\lineto(51.45379229,245.01618843)
\curveto(51.8138937,245.01618815)(52.08139773,245.07277554)(52.25630519,245.18595077)
\curveto(52.38491191,245.26825925)(52.47493731,245.39943912)(52.52638164,245.57949075)
\curveto(52.55210193,245.66694314)(52.5649627,245.9164421)(52.56496399,246.32798835)
\lineto(52.56496399,249.06733524)
\curveto(52.5649627,249.89042023)(52.54695762,250.37912952)(52.5109487,250.53346456)
\curveto(52.48008162,250.69293232)(52.41834992,250.80867926)(52.32575342,250.88070572)
\curveto(52.23829913,250.95271989)(52.1276965,250.98873005)(51.9939452,250.9887363)
\curveto(51.83447093,250.98873005)(51.65442014,250.95014774)(51.45379229,250.87298925)
\lineto(51.37662759,251.15078217)
\lineto(53.51408982,252.01502682)
\lineto(53.83818156,252.01502682)
}
}
{
\newrgbcolor{curcolor}{0 0 0}
\pscustom[linestyle=none,fillstyle=solid,fillcolor=curcolor]
{
\newpath
\moveto(63.24455863,251.80668213)
\lineto(63.24455863,247.52404122)
\curveto(63.24455194,246.70609341)(63.26255702,246.20452336)(63.29857392,246.01932954)
\curveto(63.33972165,245.83927747)(63.40145335,245.71324192)(63.48376921,245.64122251)
\curveto(63.57121552,245.56920129)(63.67152953,245.53319113)(63.78471154,245.53319192)
\curveto(63.94417787,245.53319113)(64.12422866,245.57691775)(64.32486445,245.66437192)
\lineto(64.43289503,245.39429546)
\lineto(62.31858222,244.52233434)
\lineto(61.97134106,244.52233434)
\lineto(61.97134106,246.01932954)
\curveto(61.36430727,245.3608568)(60.90131953,244.94673999)(60.58237644,244.77697785)
\curveto(60.2634253,244.60721564)(59.92647311,244.52233456)(59.57151886,244.52233434)
\curveto(59.1754041,244.52233456)(58.83073545,244.63550934)(58.53751186,244.86185902)
\curveto(58.24942862,245.09335277)(58.04880059,245.3891505)(57.93562719,245.74925309)
\curveto(57.82245103,246.10935366)(57.76586364,246.61864017)(57.76586485,247.27711417)
\lineto(57.76586485,250.43315045)
\curveto(57.76586364,250.76752479)(57.72985348,250.99901866)(57.65783427,251.12763276)
\curveto(57.58581285,251.25623408)(57.47778238,251.35397593)(57.33374252,251.42085862)
\curveto(57.19484542,251.49287226)(56.94020216,251.52631026)(56.56981198,251.52117274)
\lineto(56.56981198,251.80668213)
\lineto(59.04679889,251.80668213)
\lineto(59.04679889,247.07648595)
\curveto(59.0467964,246.41801215)(59.15997118,245.98589026)(59.38632358,245.78011897)
\curveto(59.61781462,245.5743456)(59.89560726,245.47145943)(60.21970235,245.47146016)
\curveto(60.44090394,245.47145943)(60.69040289,245.54090759)(60.96819995,245.67980486)
\curveto(61.25113249,245.81870024)(61.58551253,246.08363211)(61.97134106,246.47460128)
\lineto(61.97134106,250.47944927)
\curveto(61.97133565,250.88069957)(61.89674318,251.15077576)(61.74756343,251.28967863)
\curveto(61.60351761,251.43371271)(61.30000342,251.51087734)(60.83701996,251.52117274)
\lineto(60.83701996,251.80668213)
\lineto(63.24455863,251.80668213)
}
}
{
\newrgbcolor{curcolor}{0 0 0}
\pscustom[linestyle=none,fillstyle=solid,fillcolor=curcolor]
{
\newpath
\moveto(69.51804884,252.01502682)
\lineto(69.51804884,249.60748815)
\lineto(69.26340533,249.60748815)
\curveto(69.0679168,250.36369659)(68.8158457,250.87812742)(68.50719125,251.15078217)
\curveto(68.20367301,251.4234241)(67.81527774,251.55974826)(67.34200427,251.55975509)
\curveto(66.9818998,251.55974826)(66.69124638,251.46457856)(66.47004314,251.27424569)
\curveto(66.24883587,251.08389975)(66.13823325,250.87298311)(66.13823493,250.64149514)
\curveto(66.13823325,250.35340798)(66.22054218,250.10648118)(66.38516197,249.90071401)
\curveto(66.5446336,249.68979221)(66.86872502,249.4660148)(67.35743721,249.22938111)
\lineto(68.48404184,248.68151173)
\curveto(69.52833239,248.17222127)(70.05047968,247.50088904)(70.05048528,246.66751303)
\curveto(70.05047968,246.02447257)(69.80612504,245.50489743)(69.31742062,245.10878607)
\curveto(68.83385078,244.71781827)(68.29112626,244.52233456)(67.68924542,244.52233434)
\curveto(67.25712029,244.52233456)(66.7632667,244.59949918)(66.20768316,244.75382844)
\curveto(66.03791923,244.80527151)(65.89902291,244.83099305)(65.79099377,244.83099314)
\curveto(65.67267335,244.83099305)(65.5800758,244.76411704)(65.51320085,244.63036492)
\lineto(65.25855733,244.63036492)
\lineto(65.25855733,247.15365065)
\lineto(65.51320085,247.15365065)
\curveto(65.65724042,246.43344508)(65.93246091,245.89072055)(66.33886315,245.52547545)
\curveto(66.74526162,245.16022878)(67.2005329,244.97760584)(67.70467836,244.97760608)
\curveto(68.05963238,244.97760584)(68.34771365,245.080492)(68.56892301,245.28626488)
\curveto(68.79526847,245.49718097)(68.90844325,245.74925208)(68.9084477,246.04247895)
\curveto(68.90844325,246.39743492)(68.78240769,246.6958048)(68.53034066,246.93758949)
\curveto(68.28340979,247.17936978)(67.78698405,247.48545612)(67.04106193,247.85584943)
\curveto(66.29513465,248.22623651)(65.80642536,248.56061655)(65.57493261,248.85899055)
\curveto(65.34343762,249.152212)(65.22769068,249.52260219)(65.22769145,249.97016224)
\curveto(65.22769068,250.55146384)(65.42574655,251.03760098)(65.82185965,251.42857509)
\curveto(66.22311433,251.81953583)(66.74011731,252.01501955)(67.37287015,252.01502682)
\curveto(67.65065988,252.01501955)(67.98761207,251.95586)(68.38372773,251.83754801)
\curveto(68.64608353,251.76037629)(68.82099001,251.72179398)(68.9084477,251.72180096)
\curveto(68.99075218,251.72179398)(69.05505603,251.73979905)(69.10135945,251.77581625)
\curveto(69.14765358,251.81181937)(69.20166882,251.89155615)(69.26340533,252.01502682)
\lineto(69.51804884,252.01502682)
}
}
{
\newrgbcolor{curcolor}{0 0 0}
\pscustom[linestyle=none,fillstyle=solid,fillcolor=curcolor]
{
\newpath
\moveto(73.14478979,254.12933964)
\lineto(73.14478979,251.80668213)
\lineto(74.7961144,251.80668213)
\lineto(74.7961144,251.26652922)
\lineto(73.14478979,251.26652922)
\lineto(73.14478979,246.68294597)
\curveto(73.14478725,246.22510059)(73.2090911,245.9164421)(73.33770155,245.75696956)
\curveto(73.47145082,245.59749498)(73.641213,245.51775821)(73.84698857,245.51775898)
\curveto(74.0167475,245.51775821)(74.18136536,245.56920129)(74.34084266,245.67208839)
\curveto(74.50031248,245.78011793)(74.62377587,245.93701933)(74.71123323,246.14279306)
\lineto(75.01217556,246.14279306)
\curveto(74.83212036,245.63864945)(74.5774771,245.25797064)(74.24824502,245.00075549)
\curveto(73.91900564,244.74868412)(73.5794813,244.62264857)(73.22967097,244.62264845)
\curveto(72.99303015,244.62264857)(72.76153628,244.68695242)(72.53518865,244.8155602)
\curveto(72.30883715,244.94931214)(72.14164714,245.13707939)(72.0336181,245.37886252)
\curveto(71.92558619,245.62578868)(71.87157095,246.00389534)(71.87157222,246.51318363)
\lineto(71.87157222,251.26652922)
\lineto(70.75268406,251.26652922)
\lineto(70.75268406,251.52117274)
\curveto(71.03562086,251.63434073)(71.32370212,251.82468014)(71.61692871,252.09219152)
\curveto(71.91529757,252.36483251)(72.18022945,252.68635178)(72.41172513,253.05675029)
\curveto(72.53004241,253.25222568)(72.69466027,253.60975511)(72.90557922,254.12933964)
\lineto(73.14478979,254.12933964)
}
}
{
\newrgbcolor{curcolor}{0 0 0}
\pscustom[linestyle=none,fillstyle=solid,fillcolor=curcolor]
{
\newpath
\moveto(76.66350017,249.14449994)
\curveto(76.65835418,248.09505665)(76.91299744,247.27196732)(77.42743071,246.6752295)
\curveto(77.94185909,246.07848781)(78.54631531,245.78011793)(79.24080119,245.78011897)
\curveto(79.70378467,245.78011793)(80.10504072,245.90615348)(80.44457053,246.158226)
\curveto(80.78923372,246.41544)(81.07731498,246.8527062)(81.30881518,247.47002592)
\lineto(81.54802575,247.31569652)
\curveto(81.43998871,246.61092371)(81.12618591,245.96788518)(80.6066164,245.38657899)
\curveto(80.08703564,244.81041582)(79.43628064,244.52233456)(78.65434946,244.52233434)
\curveto(77.80553492,244.52233456)(77.0776153,244.85157028)(76.47058841,245.51004251)
\curveto(75.86870286,246.17365751)(75.56776082,247.06362284)(75.56776141,248.17994117)
\curveto(75.56776082,249.38885018)(75.87641932,250.33025859)(76.49373783,251.00416924)
\curveto(77.11619761,251.68321166)(77.89556032,252.02273601)(78.83182827,252.02274329)
\curveto(79.62404789,252.02273601)(80.27480289,251.76037629)(80.78409521,251.23566334)
\curveto(81.29337593,250.71608171)(81.54801919,250.01902794)(81.54802575,249.14449994)
\lineto(76.66350017,249.14449994)
\moveto(76.66350017,249.59205521)
\lineto(79.9352835,249.59205521)
\curveto(79.909557,250.04474948)(79.85554177,250.36369659)(79.77323762,250.5488975)
\curveto(79.64462513,250.83697295)(79.45171357,251.06332252)(79.19450237,251.22794687)
\curveto(78.94242705,251.39255825)(78.67749517,251.47486718)(78.39970594,251.47487392)
\curveto(77.97272494,251.47486718)(77.58947397,251.30767716)(77.2499519,250.97330336)
\curveto(76.91556959,250.64406139)(76.72008588,250.1836458)(76.66350017,249.59205521)
}
}
{
\newrgbcolor{curcolor}{0 0 0}
\pscustom[linestyle=none,fillstyle=solid,fillcolor=curcolor]
{
\newpath
\moveto(87.4742748,245.53319192)
\curveto(87.12960066,245.17308955)(86.79264847,244.91330198)(86.46341722,244.75382844)
\curveto(86.13417701,244.59949918)(85.77921974,244.52233456)(85.39854434,244.52233434)
\curveto(84.62689469,244.52233456)(83.95299031,244.84385382)(83.37682917,245.4868931)
\curveto(82.80066525,246.1350752)(82.51258399,246.96588098)(82.51258452,247.97931295)
\curveto(82.51258399,248.99273844)(82.8315311,249.91871393)(83.46942681,250.75724219)
\curveto(84.10731955,251.60090273)(84.92783672,252.02273601)(85.93098078,252.02274329)
\curveto(86.55343814,252.02273601)(87.06786896,251.82468014)(87.4742748,251.42857509)
\lineto(87.4742748,252.73265854)
\curveto(87.47426932,253.54030695)(87.45369208,254.0367327)(87.41254304,254.22193728)
\curveto(87.37652746,254.40712289)(87.31736791,254.53315844)(87.23506423,254.60004431)
\curveto(87.15275005,254.66691046)(87.04986388,254.70034846)(86.92640542,254.70035842)
\curveto(86.79264847,254.70034846)(86.61516983,254.659194)(86.39396899,254.5768949)
\lineto(86.29365487,254.84697136)
\lineto(88.40025122,255.71121601)
\lineto(88.74749237,255.71121601)
\lineto(88.74749237,247.53947416)
\curveto(88.74748561,246.71123772)(88.76549069,246.20452336)(88.80150766,246.01932954)
\curveto(88.84265531,245.83927747)(88.90438701,245.71324192)(88.98670294,245.64122251)
\curveto(89.07414919,245.56920129)(89.1744632,245.53319113)(89.28764528,245.53319192)
\curveto(89.4265343,245.53319113)(89.6117294,245.57691775)(89.84323113,245.66437192)
\lineto(89.9281123,245.39429546)
\lineto(87.82923243,244.52233434)
\lineto(87.4742748,244.52233434)
\lineto(87.4742748,245.53319192)
\moveto(87.4742748,246.07334483)
\lineto(87.4742748,249.71551873)
\curveto(87.44340347,250.06532671)(87.35080592,250.38427383)(87.19648188,250.67236102)
\curveto(87.04214742,250.96043635)(86.83637509,251.1764973)(86.57916427,251.32054451)
\curveto(86.32708857,251.46972287)(86.08016178,251.54431534)(85.83838314,251.54432215)
\curveto(85.38568016,251.54431534)(84.98185196,251.34111516)(84.62689733,250.93472101)
\curveto(84.15876264,250.39970675)(83.92469661,249.61777189)(83.92469855,248.58891409)
\curveto(83.92469661,247.54975997)(84.15104617,246.75239219)(84.60374792,246.19680835)
\curveto(85.05644443,245.64636591)(85.56058664,245.37114542)(86.11617606,245.37114605)
\curveto(86.58430399,245.37114542)(87.03700311,245.60521145)(87.4742748,246.07334483)
}
}
{
\newrgbcolor{curcolor}{0 0 0}
\pscustom[linewidth=0.79016654,linecolor=curcolor]
{
\newpath
\moveto(130.31922411,314.94876994)
\lineto(285.60665597,316.42781157)
}
}
{
\newrgbcolor{curcolor}{0 0 0}
\pscustom[linestyle=none,fillstyle=solid,fillcolor=curcolor]
{
\newpath
\moveto(277.70534896,316.35255524)
\lineto(274.57492868,313.16192991)
\lineto(285.60665597,316.42781157)
\lineto(274.51472362,319.48297552)
\lineto(277.70534896,316.35255524)
\closepath
}
}
{
\newrgbcolor{curcolor}{0 0 0}
\pscustom[linewidth=0.79016654,linecolor=curcolor]
{
\newpath
\moveto(277.70534896,316.35255524)
\lineto(274.57492868,313.16192991)
\lineto(285.60665597,316.42781157)
\lineto(274.51472362,319.48297552)
\lineto(277.70534896,316.35255524)
\closepath
}
}
{
\newrgbcolor{curcolor}{0 0 0}
\pscustom[linewidth=0.79016654,linecolor=curcolor]
{
\newpath
\moveto(128.60913778,263.95321513)
\lineto(261.52878018,192.91883932)
}
}
{
\newrgbcolor{curcolor}{0 0 0}
\pscustom[linestyle=none,fillstyle=solid,fillcolor=curcolor]
{
\newpath
\moveto(254.55985766,196.64314261)
\lineto(250.28256734,195.34529491)
\lineto(261.52878018,192.91883932)
\lineto(253.26200997,200.92043293)
\lineto(254.55985766,196.64314261)
\closepath
}
}
{
\newrgbcolor{curcolor}{0 0 0}
\pscustom[linewidth=0.79016654,linecolor=curcolor]
{
\newpath
\moveto(254.55985766,196.64314261)
\lineto(250.28256734,195.34529491)
\lineto(261.52878018,192.91883932)
\lineto(253.26200997,200.92043293)
\lineto(254.55985766,196.64314261)
\closepath
}
}
{
\newrgbcolor{curcolor}{0 0 0}
\pscustom[linewidth=0.79016654,linecolor=curcolor]
{
\newpath
\moveto(77.97074615,230.83814141)
\lineto(77.92618075,98.22837862)
}
}
{
\newrgbcolor{curcolor}{0 0 0}
\pscustom[linestyle=none,fillstyle=solid,fillcolor=curcolor]
{
\newpath
\moveto(77.92883622,106.13004357)
\lineto(74.76923243,109.29177174)
\lineto(77.92618075,98.22837862)
\lineto(81.09056439,109.28964737)
\lineto(77.92883622,106.13004357)
\closepath
}
}
{
\newrgbcolor{curcolor}{0 0 0}
\pscustom[linewidth=0.79016654,linecolor=curcolor]
{
\newpath
\moveto(77.92883622,106.13004357)
\lineto(74.76923243,109.29177174)
\lineto(77.92618075,98.22837862)
\lineto(81.09056439,109.28964737)
\lineto(77.92883622,106.13004357)
\closepath
}
}
{
\newrgbcolor{curcolor}{0 0 0}
\pscustom[linestyle=none,fillstyle=solid,fillcolor=curcolor]
{
\newpath
\moveto(304.02693373,64.60210402)
\lineto(304.02693373,64.12376015)
\lineto(304.84515352,64.06082016)
\lineto(304.97103348,63.87200021)
\lineto(303.62411783,56.88566201)
\lineto(303.42270988,56.70943005)
\lineto(302.61707809,56.63390207)
\lineto(302.61707809,56.15555819)
\lineto(305.27314541,56.15555819)
\lineto(305.27314541,56.63390207)
\lineto(304.46751361,56.70943005)
\lineto(304.32904565,56.88566201)
\lineto(305.55008133,63.1544844)
\lineto(308.86072448,56.15555819)
\lineto(309.5908283,56.15555819)
\lineto(311.06362392,63.87200021)
\lineto(311.26503187,64.06082016)
\lineto(312.07066366,64.12376015)
\lineto(312.07066366,64.60210402)
\lineto(309.42718434,64.60210402)
\lineto(309.42718434,64.12376015)
\lineto(310.23281613,64.06082016)
\lineto(310.3586961,63.87200021)
\lineto(309.2006004,57.90528975)
\lineto(306.19206917,64.2874041)
\lineto(306.19206917,64.60210402)
\lineto(304.02693373,64.60210402)
}
}
{
\newrgbcolor{curcolor}{0 0 0}
\pscustom[linestyle=none,fillstyle=solid,fillcolor=curcolor]
{
\newpath
\moveto(313.7888771,56.65723227)
\curveto(314.09162238,56.67359688)(314.26345125,56.84542569)(314.26345125,57.07453022)
\curveto(314.26345125,57.36091089)(313.96888752,57.58183362)(313.50249616,57.58183362)
\curveto(312.71699491,57.58183362)(312.09513835,57.01725336)(312.09513835,56.34630438)
\curveto(312.09513835,55.88809532)(312.40606671,55.66717207)(312.87245807,55.31533297)
\curveto(313.21611486,55.05349922)(313.43703765,54.86530545)(313.43703765,54.55437788)
\curveto(313.43703765,54.12071573)(313.15883855,53.85888129)(312.69244719,53.85888129)
\curveto(312.43879574,53.85888129)(312.23423754,53.95706923)(312.10332067,54.14526224)
\lineto(312.2506023,54.55437788)
\curveto(312.11968542,54.65256553)(312.01331511,54.70165951)(311.89876284,54.70165951)
\curveto(311.6451114,54.70165951)(311.47328258,54.538013)(311.47328258,54.28436155)
\curveto(311.47328258,53.85888171)(311.87421659,53.56431803)(312.56153018,53.56431803)
\curveto(313.45340139,53.56431803)(314.05889343,54.1125337)(314.05889343,54.8243942)
\curveto(314.05889343,55.31533248)(313.72341811,55.52807348)(313.23247983,55.90445949)
\curveto(312.84791152,56.19902246)(312.61880637,56.37085162)(312.61880637,56.66541458)
\curveto(312.61880637,56.98452446)(312.86427608,57.21362954)(313.19156827,57.21362954)
\curveto(313.28157362,57.21362954)(313.36339692,57.18908255)(313.44521997,57.13180641)
\lineto(313.7888771,56.65723227)
}
}
{
\newrgbcolor{curcolor}{0 0 0}
\pscustom[linestyle=none,fillstyle=solid,fillcolor=curcolor]
{
}
}
{
\newrgbcolor{curcolor}{0 0 0}
\pscustom[linestyle=none,fillstyle=solid,fillcolor=curcolor]
{
\newpath
\moveto(319.6849197,61.44251684)
\lineto(319.6849197,60.72500102)
\lineto(327.0488978,60.72500102)
\lineto(327.0488978,61.44251684)
\lineto(319.6849197,61.44251684)
\moveto(319.6849197,59.08856144)
\lineto(319.6849197,58.37104563)
\lineto(327.0488978,58.37104563)
\lineto(327.0488978,59.08856144)
\lineto(319.6849197,59.08856144)
}
}
{
\newrgbcolor{curcolor}{0 0 0}
\pscustom[linestyle=none,fillstyle=solid,fillcolor=curcolor]
{
}
}
{
\newrgbcolor{curcolor}{0 0 0}
\pscustom[linestyle=none,fillstyle=solid,fillcolor=curcolor]
{
\newpath
\moveto(336.2045022,60.800529)
\curveto(335.04640765,61.29146038)(334.49253464,61.73204147)(334.49253464,62.43696858)
\curveto(334.49253464,63.46918328)(335.3107554,64.21187612)(336.28003018,64.21187612)
\curveto(337.12342512,64.21187612)(337.62694583,63.70835544)(337.62694583,62.89013646)
\curveto(337.62694583,62.02156556)(337.09824907,61.21593248)(336.2045022,60.800529)
\moveto(336.70802207,60.59912105)
\curveto(337.96682048,61.09005244)(338.62139758,61.93344974)(338.62139758,62.96566444)
\curveto(338.62139758,64.04823108)(337.80317645,64.76574798)(336.46885013,64.76574798)
\curveto(334.73170831,64.76574798)(333.43514291,63.60765083)(333.43514291,62.16003265)
\curveto(333.43514291,61.41734159)(333.76243154,60.9264086)(334.47994664,60.56135706)
\curveto(333.07009241,59.91936987)(332.33998719,58.98785641)(332.33998719,57.93046574)
\curveto(332.33998719,56.75978321)(333.2715006,55.94156225)(334.92052653,55.94156225)
\curveto(336.89684004,55.94156225)(338.30669766,57.23812741)(338.30669766,58.73609753)
\curveto(338.30669766,59.59208046)(337.80317669,60.07042572)(336.70802207,60.59912105)
\moveto(334.90793853,60.28442113)
\curveto(336.43108461,59.592082)(337.17377795,59.31514449)(337.17377795,58.42139761)
\curveto(337.17377795,57.40177089)(336.31779298,56.55837409)(335.13452247,56.55837409)
\curveto(334.08971978,56.55837409)(333.39737892,57.1877749)(333.39737892,58.15704968)
\curveto(333.39737892,59.07597253)(333.88831181,59.70537386)(334.90793853,60.28442113)
}
}
{
\newrgbcolor{curcolor}{0 0 0}
\pscustom[linestyle=none,fillstyle=solid,fillcolor=curcolor]
{
\newpath
\moveto(192.85200423,334.25757532)
\curveto(193.24223174,334.33310323)(193.46881607,334.58486357)(193.46881607,334.91215115)
\curveto(193.46881607,335.28979068)(193.02823558,335.65484296)(192.42401234,335.65484296)
\curveto(191.25332981,335.65484296)(190.1959365,334.4841571)(189.78053302,332.31902382)
\lineto(188.82384527,331.92879592)
\lineto(188.81125727,331.51339203)
\lineto(189.65465305,331.60150801)
\lineto(189.12595719,327.98875293)
\curveto(188.88678549,326.31455104)(188.67278872,325.16904104)(188.08115346,324.55222982)
\lineto(186.7720018,324.89210573)
\curveto(186.59577002,324.76622589)(186.49506587,324.58999361)(186.49506587,324.38858586)
\curveto(186.49506587,323.98577037)(186.7720022,323.72142203)(187.17481769,323.72142203)
\curveto(187.77904093,323.72142203)(188.47138203,324.19976685)(189.13854519,325.14386566)
\curveto(189.69241649,325.92432068)(189.91900115,326.86842238)(190.08264494,328.02651692)
\lineto(190.58616482,331.60150801)
\lineto(192.46177633,331.60150801)
\lineto(192.46177633,332.16796786)
\lineto(190.68686879,332.16796786)
\curveto(190.96380444,333.8044058)(191.3414451,334.7862715)(191.8197885,335.10097111)
\lineto(192.85200423,334.25757532)
}
}
{
\newrgbcolor{curcolor}{0 0 0}
\pscustom[linestyle=none,fillstyle=solid,fillcolor=curcolor]
{
\newpath
\moveto(195.19302208,329.51134963)
\lineto(193.2865432,328.97131699)
\lineto(193.2865432,328.63584216)
\lineto(194.47297856,328.7913061)
\lineto(194.55480168,328.70948298)
\lineto(193.7365704,324.46286262)
\lineto(193.6056534,324.32376331)
\lineto(192.75469287,324.27466943)
\lineto(192.75469287,323.92282998)
\lineto(195.2012044,323.92282998)
\lineto(195.2012044,324.26648712)
\lineto(194.48934318,324.30739868)
\lineto(194.41570237,324.43013337)
\lineto(195.38939759,329.51134963)
\lineto(195.19302208,329.51134963)
}
}
{
\newrgbcolor{curcolor}{0 0 0}
\pscustom[linestyle=none,fillstyle=solid,fillcolor=curcolor]
{
\newpath
\moveto(195.92083508,324.31305788)
\curveto(197.45656915,324.50187764)(198.50137441,325.74809068)(198.50137441,326.91877321)
\curveto(198.50137441,327.46005653)(198.21184997,327.77475699)(197.69574262,327.77475699)
\curveto(197.19222325,327.77475699)(196.80199485,327.40970457)(196.80199485,326.89359721)
\curveto(196.80199485,326.75512939)(196.83975892,326.64183717)(196.91528682,326.52854531)
\lineto(197.72091861,326.52854531)
\curveto(197.58245079,325.53409456)(197.00340179,324.91728153)(195.99636306,324.72846177)
\lineto(195.92083508,324.31305788)
}
}
{
\newrgbcolor{curcolor}{0 0 0}
\pscustom[linestyle=none,fillstyle=solid,fillcolor=curcolor]
{
\newpath
\moveto(202.96185253,335.70519495)
\lineto(200.98553704,335.34014304)
\lineto(200.98553704,334.98767913)
\lineto(202.14363274,334.92473915)
\lineto(200.44425317,326.44042933)
\lineto(201.48905691,326.44042933)
\lineto(201.8918728,328.85732471)
\lineto(202.60938862,329.51190054)
\lineto(203.67936834,327.3467651)
\curveto(204.00665593,326.69218992)(204.35912067,326.37748935)(204.86264004,326.37748935)
\curveto(205.10181174,326.37748935)(205.60533258,326.47819356)(206.33543566,326.71736526)
\lineto(206.33543566,327.06982917)
\curveto(206.08367598,327.01947723)(205.86967959,326.98171319)(205.68085983,326.98171319)
\curveto(204.81228892,326.98171319)(204.30876723,328.05169508)(203.35208043,330.21682836)
\curveto(203.90595173,330.72034773)(204.43464869,331.1861045)(204.97593201,331.57633201)
\lineto(205.42909989,331.08540014)
\curveto(205.89485531,331.2742199)(206.12143972,331.53856835)(206.12143972,331.86585594)
\curveto(206.12143972,332.23090748)(205.81932738,332.50784377)(205.4039239,332.50784377)
\curveto(205.10181228,332.50784377)(204.73675963,332.29384737)(204.29618019,331.84067994)
\lineto(201.96740078,329.43637256)
\lineto(203.31431644,335.70519495)
\lineto(202.96185253,335.70519495)
}
}
{
\newrgbcolor{curcolor}{0 0 0}
\pscustom[linestyle=none,fillstyle=solid,fillcolor=curcolor]
{
\newpath
\moveto(207.76633674,331.75256397)
\lineto(208.16915263,334.55968724)
\curveto(208.18174062,334.66039112)(208.19432863,334.74850726)(208.19432863,334.81144718)
\curveto(208.19432863,335.21426267)(207.99292035,335.42825902)(207.66563276,335.42825902)
\curveto(207.33834517,335.42825902)(207.14952489,335.21426266)(207.14952489,334.79885918)
\curveto(207.14952489,334.72333128)(207.14952491,334.64780313)(207.16211289,334.55968724)
\lineto(207.56492879,331.75256397)
\lineto(207.76633674,331.75256397)
}
}
{
\newrgbcolor{curcolor}{0 0 0}
\pscustom[linestyle=none,fillstyle=solid,fillcolor=curcolor]
{
\newpath
\moveto(208.01711259,324.31305788)
\curveto(209.55284666,324.50187764)(210.59765193,325.74809068)(210.59765193,326.91877321)
\curveto(210.59765193,327.46005653)(210.30812749,327.77475699)(209.79202014,327.77475699)
\curveto(209.28850077,327.77475699)(208.89827237,327.40970457)(208.89827237,326.89359721)
\curveto(208.89827237,326.75512939)(208.93603643,326.64183717)(209.01156434,326.52854531)
\lineto(209.81719613,326.52854531)
\curveto(209.67872831,325.53409456)(209.09967931,324.91728153)(208.09264057,324.72846177)
\lineto(208.01711259,324.31305788)
}
}
{
\newrgbcolor{curcolor}{0 0 0}
\pscustom[linestyle=none,fillstyle=solid,fillcolor=curcolor]
{
\newpath
\moveto(213.39651447,332.53301977)
\lineto(212.1251268,332.11761587)
\lineto(212.1251268,331.80291595)
\lineto(213.09440255,331.72738797)
\curveto(213.30839828,331.58892015)(213.40910253,331.23645552)(213.47204245,330.65740825)
\lineto(213.78674237,327.711817)
\curveto(213.79933036,327.56076119)(213.81191837,327.40970493)(213.81191837,327.25864912)
\curveto(213.81191837,326.9565375)(213.76156625,326.67960103)(213.63568641,326.44042933)
\lineto(214.23991026,326.44042933)
\curveto(216.58127531,329.08390601)(217.68902137,330.94693304)(217.68902137,331.81550395)
\curveto(217.68902137,332.26867138)(217.449849,332.53301977)(217.02185754,332.53301977)
\curveto(216.80786181,332.53301977)(216.59386545,332.44490362)(216.3924577,332.28125983)
\lineto(216.63162964,330.74552423)
\curveto(216.30434205,329.87695332)(215.66235293,328.85732349)(214.70566614,327.63628902)
\lineto(214.37837822,331.08540014)
\curveto(214.27767435,332.09243887)(214.13920556,332.53301977)(213.42169047,332.53301977)
\lineto(213.39651447,332.53301977)
}
}
{
\newrgbcolor{curcolor}{0 0 0}
\pscustom[linestyle=none,fillstyle=solid,fillcolor=curcolor]
{
\newpath
\moveto(198.87317051,91.86273153)
\curveto(199.55776236,90.91533773)(199.49832213,89.7033462)(198.74040705,89.1556727)
\curveto(197.98249197,88.6079992)(196.81310906,88.93203653)(196.12851721,89.87943033)
\curveto(195.44392536,90.82682414)(195.50336559,92.03881566)(196.26128067,92.58648916)
\curveto(197.01919575,93.13416267)(198.18857867,92.81012534)(198.87317051,91.86273153)
\closepath
}
}
{
\newrgbcolor{curcolor}{0 0 0}
\pscustom[linestyle=none,fillstyle=solid,fillcolor=curcolor]
{
\newpath
\moveto(208.13636951,98.55636662)
\curveto(208.82096135,97.60897281)(208.76152113,96.39698129)(208.00360605,95.84930779)
\curveto(207.24569097,95.30163428)(206.07630805,95.62567161)(205.39171621,96.57306542)
\curveto(204.70712436,97.52045923)(204.76656458,98.73245075)(205.52447966,99.28012425)
\curveto(206.28239475,99.82779776)(207.45177766,99.50376043)(208.13636951,98.55636662)
\closepath
}
}
{
\newrgbcolor{curcolor}{0 0 0}
\pscustom[linestyle=none,fillstyle=solid,fillcolor=curcolor]
{
\newpath
\moveto(217.39958035,105.25000961)
\curveto(218.0841722,104.3026158)(218.02473198,103.09062428)(217.2668169,102.54295078)
\curveto(216.50890181,101.99527727)(215.3395189,102.3193146)(214.65492705,103.26670841)
\curveto(213.97033521,104.21410222)(214.02977543,105.42609374)(214.78769051,105.97376724)
\curveto(215.54560559,106.52144075)(216.71498851,106.19740342)(217.39958035,105.25000961)
\closepath
}
}
{
\newrgbcolor{curcolor}{0 0 0}
\pscustom[linewidth=0.79016654,linecolor=curcolor,linestyle=dashed,dash=8 8]
{
\newpath
\moveto(425.60345265,394.48165871)
\lineto(425.60345265,385.4194362)
\lineto(288.01570387,385.4194362)
\lineto(288.01570387,243.33761523)
\lineto(425.60345265,243.33761523)
\lineto(425.60345265,221.65742079)
\lineto(639.46696524,221.65742079)
\lineto(639.46696524,394.48165871)
\lineto(425.60345265,394.48165871)
\closepath
}
}
{
\newrgbcolor{curcolor}{0 0 0}
\pscustom[linewidth=0.79016654,linecolor=curcolor,linestyle=dashed,dash=8 8]
{
\newpath
\moveto(0.39508331,490.56097143)
\lineto(0.39508331,393.22233078)
\lineto(16.81573172,393.22233078)
\lineto(16.81573172,234.89271033)
\lineto(126.96988593,234.89271033)
\lineto(126.96988593,393.22233078)
\lineto(298.65825945,393.22233078)
\lineto(298.65825945,490.56097143)
\lineto(0.39508331,490.56097143)
\closepath
}
}
{
\newrgbcolor{curcolor}{0 0 0}
\pscustom[linestyle=none,fillstyle=solid,fillcolor=curcolor]
{
\newpath
\moveto(18.60305341,476.94900486)
\curveto(19.09295618,477.04382475)(19.37741662,477.35989187)(19.37741662,477.77077806)
\curveto(19.37741662,478.24487751)(18.82429928,478.70317458)(18.06574016,478.70317458)
\curveto(16.59603187,478.70317458)(15.26855009,477.23346209)(14.7470407,474.51529191)
\lineto(13.54598755,474.02538866)
\lineto(13.53018422,473.50387874)
\lineto(14.58900739,473.61450206)
\lineto(13.92526749,469.07894612)
\curveto(13.62500451,466.97710523)(13.35634684,465.53899925)(12.61359104,464.76463681)
\lineto(10.97004463,465.19132674)
\curveto(10.74879822,465.03329359)(10.62237136,464.81204655)(10.62237136,464.55919351)
\curveto(10.62237136,464.05348743)(10.97004514,463.72161698)(11.47575122,463.72161698)
\curveto(12.23431034,463.72161698)(13.10349513,464.32214473)(13.94107082,465.50739336)
\curveto(14.63641668,466.48719889)(14.92087754,467.67245113)(15.12632063,469.12635611)
\lineto(15.75845387,473.61450206)
\lineto(18.11315016,473.61450206)
\lineto(18.11315016,474.32565195)
\lineto(15.88488051,474.32565195)
\curveto(16.23255344,476.38008289)(16.70665431,477.61274515)(17.30718028,478.00782802)
\lineto(18.60305341,476.94900486)
}
}
{
\newrgbcolor{curcolor}{0 0 0}
\pscustom[linestyle=none,fillstyle=solid,fillcolor=curcolor]
{
\newpath
\moveto(23.73864207,478.48192795)
\curveto(20.925652,478.13425502)(19.31370944,475.74794761)(19.31370944,471.63908571)
\curveto(19.31370944,467.59343707)(20.98886526,465.20712976)(23.73864207,464.89106346)
\lineto(23.73864207,465.55480335)
\curveto(21.62099786,465.9182796)(20.65699256,467.90950335)(20.65699256,471.62328238)
\curveto(20.65699256,475.41607798)(21.57358791,477.43890849)(23.73864207,477.81818805)
\lineto(23.73864207,478.48192795)
}
}
{
\newrgbcolor{curcolor}{0 0 0}
\pscustom[linestyle=none,fillstyle=solid,fillcolor=curcolor]
{
\newpath
\moveto(24.52831475,474.73653855)
\lineto(24.52831475,474.15181531)
\lineto(25.49231793,473.97797867)
\lineto(27.60996426,470.89632916)
\lineto(25.33428462,467.84628632)
\lineto(24.43349477,467.71985967)
\lineto(24.43349477,467.13513643)
\lineto(27.42032429,467.13513643)
\lineto(27.42032429,467.71985967)
\lineto(26.51953443,467.83048299)
\lineto(26.44051778,468.02012296)
\lineto(28.05245752,470.31160592)
\lineto(29.52216728,467.95690963)
\lineto(29.44315063,467.83048299)
\lineto(28.77941074,467.71985967)
\lineto(28.77941074,467.13513643)
\lineto(32.1455202,467.13513643)
\lineto(32.1455202,467.71985967)
\lineto(31.19732035,467.84628632)
\lineto(28.87423072,471.22819911)
\lineto(30.96027039,473.993782)
\lineto(31.84525691,474.15181531)
\lineto(31.84525691,474.73653855)
\lineto(29.00065737,474.73653855)
\lineto(29.00065737,474.15181531)
\lineto(29.75921725,474.05699532)
\lineto(29.88564389,473.85155202)
\lineto(28.44754079,471.82872568)
\lineto(27.120061,473.89896201)
\lineto(27.19907766,474.05699532)
\lineto(27.92603087,474.15181531)
\lineto(27.92603087,474.73653855)
\lineto(24.52831475,474.73653855)
}
}
{
\newrgbcolor{curcolor}{0 0 0}
\pscustom[linestyle=none,fillstyle=solid,fillcolor=curcolor]
{
\newpath
\moveto(32.69715521,478.48192795)
\lineto(32.69715521,477.81818805)
\curveto(34.86220937,477.43890849)(35.77880472,475.41607798)(35.77880472,471.62328238)
\curveto(35.77880472,467.90950335)(34.81479942,465.9182796)(32.69715521,465.55480335)
\lineto(32.69715521,464.89106346)
\curveto(35.44693202,465.20712976)(37.12208784,467.59343707)(37.12208784,471.63908571)
\curveto(37.12208784,475.74794761)(35.51014528,478.13425502)(32.69715521,478.48192795)
}
}
{
\newrgbcolor{curcolor}{0 0 0}
\pscustom[linestyle=none,fillstyle=solid,fillcolor=curcolor]
{
}
}
{
\newrgbcolor{curcolor}{0 0 0}
\pscustom[linestyle=none,fillstyle=solid,fillcolor=curcolor]
{
\newpath
\moveto(45.15511842,473.77253537)
\lineto(45.15511842,472.87174551)
\lineto(54.40006694,472.87174551)
\lineto(54.40006694,473.77253537)
\lineto(45.15511842,473.77253537)
\moveto(45.15511842,470.81731251)
\lineto(45.15511842,469.91652265)
\lineto(54.40006694,469.91652265)
\lineto(54.40006694,470.81731251)
\lineto(45.15511842,470.81731251)
}
}
{
\newrgbcolor{curcolor}{0 0 0}
\pscustom[linestyle=none,fillstyle=solid,fillcolor=curcolor]
{
}
}
{
\newrgbcolor{curcolor}{0 0 0}
\pscustom[linestyle=none,fillstyle=solid,fillcolor=curcolor]
{
\newpath
\moveto(62.21209783,468.38359956)
\curveto(66.57381276,471.38622941)(68.92851342,473.28263439)(68.92851342,475.55831175)
\curveto(68.92851342,477.04382336)(67.93290181,477.92881137)(66.16293053,477.92881137)
\curveto(64.96187859,477.92881137)(63.79243015,477.47051408)(63.03387103,476.77516822)
\curveto(62.57557489,476.34847871)(62.3543278,475.87437794)(62.3543278,475.44768843)
\curveto(62.3543278,475.08421219)(62.63878822,474.8155552)(63.09708435,474.8155552)
\curveto(63.34993739,474.8155552)(63.60279119,474.89457201)(63.85564423,475.05260516)
\lineto(64.17171085,476.74356156)
\curveto(64.69322024,476.9964146)(65.16732118,477.1228415)(65.65722394,477.1228415)
\curveto(66.76345599,477.1228415)(67.379787,476.49070717)(67.379787,475.40027844)
\curveto(67.379787,473.01397788)(64.96187302,471.18078607)(60.60015808,468.13074627)
\lineto(60.41051811,467.13513643)
\lineto(67.72746028,467.13513643)
\lineto(68.21736353,469.72688268)
\lineto(67.50621364,469.72688268)
\lineto(66.9689004,468.38359956)
\lineto(62.21209783,468.38359956)
}
}
{
\newrgbcolor{curcolor}{0 0 0}
\pscustom[linestyle=none,fillstyle=solid,fillcolor=curcolor]
{
\newpath
\moveto(70.83503712,477.7391714)
\lineto(70.83503712,477.13864483)
\lineto(71.86225362,477.05962817)
\lineto(72.02028693,476.82257821)
\lineto(70.32933054,468.05172962)
\lineto(70.07647724,467.83048299)
\lineto(69.06506407,467.735663)
\lineto(69.06506407,467.13513643)
\lineto(72.39956687,467.13513643)
\lineto(72.39956687,467.735663)
\lineto(71.3881537,467.83048299)
\lineto(71.21431706,468.05172962)
\lineto(72.74724015,475.92178836)
\lineto(76.90351615,467.13513643)
\lineto(77.82010933,467.13513643)
\lineto(79.66909904,476.82257821)
\lineto(79.92195233,477.05962817)
\lineto(80.9333655,477.13864483)
\lineto(80.9333655,477.7391714)
\lineto(77.61466603,477.7391714)
\lineto(77.61466603,477.13864483)
\lineto(78.62607921,477.05962817)
\lineto(78.78411251,476.82257821)
\lineto(77.33020608,469.33179941)
\lineto(73.55321002,477.34408813)
\lineto(73.55321002,477.7391714)
\lineto(70.83503712,477.7391714)
}
}
{
\newrgbcolor{curcolor}{0 0 0}
\pscustom[linestyle=none,fillstyle=solid,fillcolor=curcolor]
{
\newpath
\moveto(83.09045987,467.76495255)
\curveto(83.47053495,467.78549715)(83.68625383,468.00121596)(83.68625383,468.28884035)
\curveto(83.68625383,468.64837083)(83.3164501,468.92572355)(82.73092903,468.92572355)
\curveto(81.74478829,468.92572355)(80.96409176,468.21693334)(80.96409176,467.37460478)
\curveto(80.96409176,466.79935602)(81.35444012,466.52200264)(81.93996118,466.08029376)
\curveto(82.37139776,465.75158018)(82.64875055,465.51531634)(82.64875055,465.12496896)
\curveto(82.64875055,464.58053709)(82.29949144,464.25182263)(81.71397037,464.25182263)
\curveto(81.39552909,464.25182263)(81.13872086,464.37509059)(80.97436407,464.61135347)
\lineto(81.15926565,465.12496896)
\curveto(80.99490886,465.24823655)(80.86136852,465.30987053)(80.71755633,465.30987053)
\curveto(80.39911505,465.30987053)(80.18339622,465.10442402)(80.18339622,464.78598274)
\curveto(80.18339622,464.25182317)(80.68674026,463.88201948)(81.54961342,463.88201948)
\curveto(82.66929405,463.88201948)(83.42944609,464.57026513)(83.42944609,465.46395518)
\curveto(83.42944609,466.08029314)(83.00828078,466.34737429)(82.39194281,466.81990006)
\curveto(81.90914474,467.18970284)(81.62151958,467.40542208)(81.62151958,467.77522486)
\curveto(81.62151958,468.17584454)(81.92968929,468.46346961)(82.34058126,468.46346961)
\curveto(82.45357656,468.46346961)(82.55629987,468.43265261)(82.65902286,468.36074651)
\lineto(83.09045987,467.76495255)
}
}
{
\newrgbcolor{curcolor}{0 0 0}
\pscustom[linestyle=none,fillstyle=solid,fillcolor=curcolor]
{
\newpath
\moveto(88.30622916,475.87437836)
\curveto(88.48006563,476.52231428)(88.57488579,476.98061176)(88.57488579,477.21766148)
\curveto(88.57488579,477.6601543)(88.32203205,477.92881137)(87.87953923,477.92881137)
\curveto(87.02616022,477.92881137)(86.80491258,477.23346345)(86.64687943,475.87437836)
\lineto(85.93572954,469.837506)
\lineto(86.71009275,469.837506)
\lineto(88.30622916,475.87437836)
\moveto(86.04635286,468.79448617)
\curveto(85.57225341,468.79448617)(85.16136633,468.38359907)(85.16136633,467.89369631)
\curveto(85.16136633,467.38799023)(85.57225341,466.99290645)(86.04635286,466.99290645)
\curveto(86.55205894,466.99290645)(86.96294604,467.38799023)(86.96294604,467.89369631)
\curveto(86.96294604,468.38359907)(86.55205894,468.79448617)(86.04635286,468.79448617)
}
}
{
\newrgbcolor{curcolor}{0 0 0}
\pscustom[linestyle=none,fillstyle=solid,fillcolor=curcolor]
{
\newpath
\moveto(96.74347932,478.48192795)
\curveto(95.6688539,478.40291137)(94.61002876,477.9130072)(93.70923981,477.0754315)
\curveto(92.76104091,476.20624918)(92.09729964,475.22644062)(91.67061013,474.04119199)
\curveto(91.03847753,472.27122071)(90.73821362,470.65927784)(90.73821362,469.30019275)
\curveto(90.73821362,466.78746567)(92.12890873,465.04909661)(94.13592974,464.89106346)
\lineto(94.26235638,465.55480335)
\curveto(92.82425472,465.77604976)(92.03408674,466.97710519)(92.03408674,469.04733946)
\curveto(92.03408674,470.16937482)(92.28694065,471.79712107)(92.90326994,473.85155202)
\curveto(93.21933624,474.91037413)(93.69343717,475.81116581)(94.38878303,476.58552825)
\curveto(94.989309,477.26507079)(95.77947698,477.69176153)(96.61705267,477.81818805)
\lineto(96.74347932,478.48192795)
}
}
{
\newrgbcolor{curcolor}{0 0 0}
\pscustom[linestyle=none,fillstyle=solid,fillcolor=curcolor]
{
\newpath
\moveto(103.63249691,478.7663879)
\lineto(101.2145873,478.30809131)
\lineto(101.2145873,477.86559804)
\lineto(102.60528041,477.70756474)
\lineto(102.05216383,474.54689858)
\curveto(101.62547433,474.73653836)(101.21458687,474.8155552)(100.78789737,474.8155552)
\curveto(99.47622223,474.8155552)(98.08552695,474.02538732)(97.23214794,472.68210554)
\curveto(96.72644186,471.87613648)(96.45778473,470.89632809)(96.45778473,469.82170267)
\curveto(96.45778473,468.02012476)(97.26375587,466.99290645)(98.52802107,466.99290645)
\curveto(99.36559676,466.99290645)(100.21897824,467.48281069)(100.99334067,468.46261622)
\curveto(100.97753736,468.35199301)(100.97753734,468.24136949)(100.97753734,468.1465496)
\curveto(100.97753734,467.37218717)(101.30940801,467.02451312)(102.0363605,467.02451312)
\curveto(102.44724669,467.02451312)(103.11098787,467.16674341)(103.98017019,467.48280971)
\lineto(103.98017019,467.95690963)
\curveto(103.50607074,467.84628643)(103.15839678,467.78307299)(102.95295369,467.78307299)
\curveto(102.47885424,467.78307299)(102.25760713,468.00432007)(102.25760713,468.44681289)
\curveto(102.25760713,468.65225598)(102.28921386,468.95251982)(102.35242712,469.30019275)
\lineto(104.04338351,478.7663879)
\lineto(103.63249691,478.7663879)
\moveto(101.07235732,474.15181531)
\lineto(101.8309172,473.25102545)
\lineto(101.11976732,469.23697943)
\curveto(100.40861814,468.33619047)(99.68166363,467.89369631)(99.09694097,467.89369631)
\curveto(98.27516859,467.89369631)(97.73785453,468.58904411)(97.73785453,469.837506)
\curveto(97.73785453,472.4766596)(99.27077904,474.1992253)(100.69307739,474.1992253)
\curveto(100.80370059,474.1992253)(100.93012749,474.16761862)(101.07235732,474.15181531)
}
}
{
\newrgbcolor{curcolor}{0 0 0}
\pscustom[linestyle=none,fillstyle=solid,fillcolor=curcolor]
{
\newpath
\moveto(110.5296631,476.55392159)
\lineto(110.5296631,472.3028256)
\lineto(106.35758377,472.3028256)
\lineto(106.35758377,471.40203575)
\lineto(110.5296631,471.40203575)
\lineto(110.5296631,467.13513643)
\lineto(111.44625629,467.13513643)
\lineto(111.44625629,471.40203575)
\lineto(115.60253229,471.40203575)
\lineto(115.60253229,472.3028256)
\lineto(111.44625629,472.3028256)
\lineto(111.44625629,476.55392159)
\lineto(110.5296631,476.55392159)
}
}
{
\newrgbcolor{curcolor}{0 0 0}
\pscustom[linestyle=none,fillstyle=solid,fillcolor=curcolor]
{
\newpath
\moveto(119.61632535,474.79975187)
\lineto(118.02018894,474.24663529)
\lineto(118.02018894,473.86735535)
\lineto(119.3318654,473.69351871)
\curveto(119.7585549,473.29843584)(120.12203235,472.38184082)(120.53291854,470.94373915)
\curveto(119.88498263,469.37921097)(119.04740473,468.36779585)(118.33625556,467.98851629)
\lineto(117.60930234,468.66805952)
\curveto(117.11939958,468.57323963)(116.83493913,468.28877915)(116.83493913,467.86208965)
\curveto(116.83493913,467.35638357)(117.21421967,467.00870978)(117.81474564,467.00870978)
\curveto(118.85776443,467.00870978)(119.80596635,468.0675351)(120.83318183,470.21678594)
\curveto(121.35469122,467.97271521)(121.56013615,467.04031645)(122.6663682,467.04031645)
\curveto(123.0930577,467.04031645)(123.69358544,467.19835005)(124.43634125,467.49861304)
\lineto(124.43634125,467.98851629)
\curveto(124.05706169,467.87789309)(123.77260118,467.81467966)(123.59876472,467.81467966)
\curveto(122.65056582,467.81467966)(122.42931762,468.81029163)(121.81298834,470.94373915)
\curveto(122.5083342,472.55567728)(123.20368217,473.45646894)(123.93063466,473.64610872)
\lineto(124.61017789,473.02977882)
\curveto(125.11588397,473.23522191)(125.3845411,473.53548578)(125.3845411,473.91476534)
\curveto(125.3845411,474.40466811)(124.98945732,474.79975187)(124.48375124,474.79975187)
\curveto(123.4881424,474.79975187)(122.50833385,473.74092659)(121.46531506,471.62328238)
\curveto(120.91219903,473.77253322)(120.72255758,474.8155552)(119.79016199,474.8155552)
\curveto(119.72694873,474.8155552)(119.67953861,474.79975187)(119.61632535,474.79975187)
}
}
{
\newrgbcolor{curcolor}{0 0 0}
\pscustom[linestyle=none,fillstyle=solid,fillcolor=curcolor]
{
\newpath
\moveto(129.0124482,480.47245628)
\lineto(126.61900004,479.79448384)
\lineto(126.61900004,479.37331914)
\lineto(128.10848495,479.56849302)
\lineto(128.21120805,479.46576993)
\lineto(127.18397708,474.13444119)
\lineto(127.01962012,473.95981193)
\lineto(125.95129991,473.89817807)
\lineto(125.95129991,473.45646875)
\lineto(129.02272051,473.45646875)
\lineto(129.02272051,473.88790576)
\lineto(128.12902957,473.93926731)
\lineto(128.03657878,474.09335195)
\lineto(129.25898364,480.47245628)
\lineto(129.0124482,480.47245628)
}
}
{
\newrgbcolor{curcolor}{0 0 0}
\pscustom[linestyle=none,fillstyle=solid,fillcolor=curcolor]
{
\newpath
\moveto(133.62414882,474.75234188)
\lineto(131.28525586,474.23083196)
\lineto(131.28525586,473.83574869)
\lineto(132.58112899,473.59869873)
\lineto(131.26945253,467.13513643)
\lineto(132.59693232,467.13513643)
\lineto(133.24486888,470.84891917)
\curveto(133.49772192,472.30282415)(134.00343031,473.15620607)(135.55215518,473.75673204)
\lineto(135.96304178,472.90335217)
\lineto(136.0736651,472.90335217)
\curveto(136.72160101,472.90335217)(137.13248826,473.33004271)(137.13248826,473.93056868)
\curveto(137.13248826,474.40466813)(136.78481445,474.75234188)(136.24750174,474.75234188)
\curveto(135.45733599,474.75234188)(134.44592101,474.02538722)(133.43450885,472.58728556)
\lineto(133.94021544,474.75234188)
\lineto(133.62414882,474.75234188)
}
}
{
\newrgbcolor{curcolor}{0 0 0}
\pscustom[linestyle=none,fillstyle=solid,fillcolor=curcolor]
{
\newpath
\moveto(141.02707657,470.9904578)
\lineto(138.63362841,470.31248536)
\lineto(138.63362841,469.89132066)
\lineto(140.12311332,470.08649454)
\lineto(140.22583642,469.98377145)
\lineto(139.19860545,464.65244271)
\lineto(139.03424849,464.47781345)
\lineto(137.96592828,464.41617959)
\lineto(137.96592828,463.97447027)
\lineto(141.03734888,463.97447027)
\lineto(141.03734888,464.40590728)
\lineto(140.14365794,464.45726883)
\lineto(140.05120715,464.61135347)
\lineto(141.27361201,470.9904578)
\lineto(141.02707657,470.9904578)
}
}
{
\newrgbcolor{curcolor}{0 0 0}
\pscustom[linestyle=none,fillstyle=solid,fillcolor=curcolor]
{
\newpath
\moveto(148.97327999,476.55392159)
\lineto(148.97327999,472.3028256)
\lineto(144.80120066,472.3028256)
\lineto(144.80120066,471.40203575)
\lineto(148.97327999,471.40203575)
\lineto(148.97327999,467.13513643)
\lineto(149.88987318,467.13513643)
\lineto(149.88987318,471.40203575)
\lineto(154.04614918,471.40203575)
\lineto(154.04614918,472.3028256)
\lineto(149.88987318,472.3028256)
\lineto(149.88987318,476.55392159)
\lineto(148.97327999,476.55392159)
}
}
{
\newrgbcolor{curcolor}{0 0 0}
\pscustom[linestyle=none,fillstyle=solid,fillcolor=curcolor]
{
\newpath
\moveto(157.36460172,468.93671614)
\curveto(156.84309233,468.93671614)(156.4322052,468.51002569)(156.4322052,467.98851629)
\curveto(156.4322052,467.4670069)(156.85889564,467.05611978)(157.36460172,467.05611978)
\curveto(157.88611112,467.05611978)(158.31280157,467.4670069)(158.31280157,467.98851629)
\curveto(158.31280157,468.51002569)(157.88611112,468.93671614)(157.36460172,468.93671614)
}
}
{
\newrgbcolor{curcolor}{0 0 0}
\pscustom[linestyle=none,fillstyle=solid,fillcolor=curcolor]
{
\newpath
\moveto(161.74755675,468.93671614)
\curveto(161.22604735,468.93671614)(160.81516023,468.51002569)(160.81516023,467.98851629)
\curveto(160.81516023,467.4670069)(161.24185067,467.05611978)(161.74755675,467.05611978)
\curveto(162.26906614,467.05611978)(162.6957566,467.4670069)(162.6957566,467.98851629)
\curveto(162.6957566,468.51002569)(162.26906614,468.93671614)(161.74755675,468.93671614)
}
}
{
\newrgbcolor{curcolor}{0 0 0}
\pscustom[linestyle=none,fillstyle=solid,fillcolor=curcolor]
{
\newpath
\moveto(166.13051177,468.93671614)
\curveto(165.60900238,468.93671614)(165.19811526,468.51002569)(165.19811526,467.98851629)
\curveto(165.19811526,467.4670069)(165.62480569,467.05611978)(166.13051177,467.05611978)
\curveto(166.65202117,467.05611978)(167.07871162,467.4670069)(167.07871162,467.98851629)
\curveto(167.07871162,468.51002569)(166.65202117,468.93671614)(166.13051177,468.93671614)
}
}
{
\newrgbcolor{curcolor}{0 0 0}
\pscustom[linestyle=none,fillstyle=solid,fillcolor=curcolor]
{
\newpath
\moveto(175.30187603,476.55392159)
\lineto(175.30187603,472.3028256)
\lineto(171.1297967,472.3028256)
\lineto(171.1297967,471.40203575)
\lineto(175.30187603,471.40203575)
\lineto(175.30187603,467.13513643)
\lineto(176.21846922,467.13513643)
\lineto(176.21846922,471.40203575)
\lineto(180.37474522,471.40203575)
\lineto(180.37474522,472.3028256)
\lineto(176.21846922,472.3028256)
\lineto(176.21846922,476.55392159)
\lineto(175.30187603,476.55392159)
}
}
{
\newrgbcolor{curcolor}{0 0 0}
\pscustom[linestyle=none,fillstyle=solid,fillcolor=curcolor]
{
\newpath
\moveto(184.38854432,474.79975187)
\lineto(182.7924079,474.24663529)
\lineto(182.7924079,473.86735535)
\lineto(184.10408436,473.69351871)
\curveto(184.53077387,473.29843584)(184.89425131,472.38184082)(185.3051375,470.94373915)
\curveto(184.65720159,469.37921097)(183.8196237,468.36779585)(183.10847452,467.98851629)
\lineto(182.3815213,468.66805952)
\curveto(181.89161854,468.57323963)(181.60715809,468.28877915)(181.60715809,467.86208965)
\curveto(181.60715809,467.35638357)(181.98643863,467.00870978)(182.5869646,467.00870978)
\curveto(183.62998339,467.00870978)(184.57818531,468.0675351)(185.60540079,470.21678594)
\curveto(186.12691018,467.97271521)(186.33235511,467.04031645)(187.43858716,467.04031645)
\curveto(187.86527667,467.04031645)(188.4658044,467.19835005)(189.20856021,467.49861304)
\lineto(189.20856021,467.98851629)
\curveto(188.82928065,467.87789309)(188.54482014,467.81467966)(188.37098368,467.81467966)
\curveto(187.42278478,467.81467966)(187.20153658,468.81029163)(186.5852073,470.94373915)
\curveto(187.28055316,472.55567728)(187.97590113,473.45646894)(188.70285362,473.64610872)
\lineto(189.38239685,473.02977882)
\curveto(189.88810293,473.23522191)(190.15676006,473.53548578)(190.15676006,473.91476534)
\curveto(190.15676006,474.40466811)(189.76167628,474.79975187)(189.2559702,474.79975187)
\curveto(188.26036136,474.79975187)(187.28055281,473.74092659)(186.23753402,471.62328238)
\curveto(185.68441799,473.77253322)(185.49477654,474.8155552)(184.56238095,474.8155552)
\curveto(184.49916769,474.8155552)(184.45175758,474.79975187)(184.38854432,474.79975187)
}
}
{
\newrgbcolor{curcolor}{0 0 0}
\pscustom[linestyle=none,fillstyle=solid,fillcolor=curcolor]
{
\newpath
\moveto(192.19245916,481.01688869)
\lineto(190.57970654,480.71899171)
\lineto(190.57970654,480.43136704)
\lineto(191.52475903,480.38000549)
\lineto(190.13799722,473.45646875)
\lineto(190.99059892,473.45646875)
\lineto(191.31931284,475.42875221)
\lineto(191.90483449,475.96291232)
\lineto(192.77798081,474.19607505)
\curveto(193.0450606,473.66191548)(193.33268595,473.4051072)(193.74357793,473.4051072)
\curveto(193.93875161,473.4051072)(194.34964479,473.48728588)(194.94543816,473.68245956)
\lineto(194.94543816,473.97008424)
\curveto(194.73999217,473.92899504)(194.56536255,473.89817807)(194.41127806,473.89817807)
\curveto(193.7024894,473.89817807)(193.29159552,474.77132616)(192.51090076,476.53816166)
\curveto(192.96288194,476.94905364)(193.39431984,477.32912983)(193.83602871,477.64757111)
\lineto(194.20583186,477.24695103)
\curveto(194.58590694,477.40103552)(194.7708089,477.61675445)(194.7708089,477.88383423)
\curveto(194.7708089,478.18173092)(194.52427312,478.40772203)(194.18528724,478.40772203)
\curveto(193.93875206,478.40772203)(193.64085447,478.23309239)(193.28132399,477.86328961)
\lineto(191.38094669,475.90127846)
\lineto(192.48008383,481.01688869)
\lineto(192.19245916,481.01688869)
}
}
{
\newrgbcolor{curcolor}{0 0 0}
\pscustom[linestyle=none,fillstyle=solid,fillcolor=curcolor]
{
\newpath
\moveto(195.52758723,476.27108161)
\lineto(195.39404721,475.58283686)
\lineto(197.69504458,475.58283686)
\lineto(197.82858461,476.27108161)
\lineto(195.52758723,476.27108161)
}
}
{
\newrgbcolor{curcolor}{0 0 0}
\pscustom[linestyle=none,fillstyle=solid,fillcolor=curcolor]
{
\newpath
\moveto(202.33155004,480.47245628)
\lineto(199.93810188,479.79448384)
\lineto(199.93810188,479.37331914)
\lineto(201.42758678,479.56849302)
\lineto(201.53030988,479.46576993)
\lineto(200.50307891,474.13444119)
\lineto(200.33872195,473.95981193)
\lineto(199.27040174,473.89817807)
\lineto(199.27040174,473.45646875)
\lineto(202.34182235,473.45646875)
\lineto(202.34182235,473.88790576)
\lineto(201.4481314,473.93926731)
\lineto(201.35568061,474.09335195)
\lineto(202.57808547,480.47245628)
\lineto(202.33155004,480.47245628)
}
}
{
\newrgbcolor{curcolor}{0 0 0}
\pscustom[linestyle=none,fillstyle=solid,fillcolor=curcolor]
{
\newpath
\moveto(206.9432386,474.75234188)
\lineto(204.60434564,474.23083196)
\lineto(204.60434564,473.83574869)
\lineto(205.90021877,473.59869873)
\lineto(204.58854231,467.13513643)
\lineto(205.9160221,467.13513643)
\lineto(206.56395866,470.84891917)
\curveto(206.8168117,472.30282415)(207.32252009,473.15620607)(208.87124496,473.75673204)
\lineto(209.28213156,472.90335217)
\lineto(209.39275487,472.90335217)
\curveto(210.04069079,472.90335217)(210.45157804,473.33004271)(210.45157804,473.93056868)
\curveto(210.45157804,474.40466813)(210.10390422,474.75234188)(209.56659151,474.75234188)
\curveto(208.77642576,474.75234188)(207.76501079,474.02538722)(206.75359863,472.58728556)
\lineto(207.25930521,474.75234188)
\lineto(206.9432386,474.75234188)
}
}
{
\newrgbcolor{curcolor}{0 0 0}
\pscustom[linestyle=none,fillstyle=solid,fillcolor=curcolor]
{
\newpath
\moveto(212.75395835,471.53489021)
\lineto(211.14120572,471.23699323)
\lineto(211.14120572,470.94936856)
\lineto(212.08625822,470.89800701)
\lineto(210.69949641,463.97447027)
\lineto(211.55209811,463.97447027)
\lineto(211.88081202,465.94675373)
\lineto(212.46633367,466.48091384)
\lineto(213.33948,464.71407657)
\curveto(213.60655978,464.179917)(213.89418513,463.92310872)(214.30507711,463.92310872)
\curveto(214.5002508,463.92310872)(214.91114398,464.0052874)(215.50693735,464.20046108)
\lineto(215.50693735,464.48808576)
\curveto(215.30149136,464.44699656)(215.12686173,464.41617959)(214.97277724,464.41617959)
\curveto(214.26398858,464.41617959)(213.8530947,465.28932768)(213.07239995,467.05616318)
\curveto(213.52438112,467.46705516)(213.95581902,467.84713135)(214.3975279,468.16557263)
\lineto(214.76733105,467.76495255)
\curveto(215.14740613,467.91903704)(215.33230808,468.13475597)(215.33230808,468.40183575)
\curveto(215.33230808,468.69973244)(215.08577231,468.92572355)(214.74678643,468.92572355)
\curveto(214.50025124,468.92572355)(214.20235365,468.75109391)(213.84282317,468.38129113)
\lineto(211.94244588,466.41927998)
\lineto(213.04158302,471.53489021)
\lineto(212.75395835,471.53489021)
}
}
{
\newrgbcolor{curcolor}{0 0 0}
\pscustom[linestyle=none,fillstyle=solid,fillcolor=curcolor]
{
\newpath
\moveto(216.08908642,466.78908313)
\lineto(215.95554639,466.10083838)
\lineto(218.25654377,466.10083838)
\lineto(218.39008379,466.78908313)
\lineto(216.08908642,466.78908313)
}
}
{
\newrgbcolor{curcolor}{0 0 0}
\pscustom[linestyle=none,fillstyle=solid,fillcolor=curcolor]
{
\newpath
\moveto(222.89304922,470.9904578)
\lineto(220.49960106,470.31248536)
\lineto(220.49960106,469.89132066)
\lineto(221.98908597,470.08649454)
\lineto(222.09180907,469.98377145)
\lineto(221.06457809,464.65244271)
\lineto(220.90022114,464.47781345)
\lineto(219.83190093,464.41617959)
\lineto(219.83190093,463.97447027)
\lineto(222.90332153,463.97447027)
\lineto(222.90332153,464.40590728)
\lineto(222.00963059,464.45726883)
\lineto(221.9171798,464.61135347)
\lineto(223.13958465,470.9904578)
\lineto(222.89304922,470.9904578)
}
}
{
\newrgbcolor{curcolor}{0 0 0}
\pscustom[linestyle=none,fillstyle=solid,fillcolor=curcolor]
{
\newpath
\moveto(226.2721021,478.48192795)
\lineto(226.14567545,477.81818805)
\curveto(227.55217048,477.61274496)(228.35814176,476.41168962)(228.35814176,474.42047193)
\curveto(228.35814176,473.42486309)(228.15269791,471.86033032)(227.59958188,469.837506)
\curveto(227.26771227,468.62065074)(226.76200461,467.60923553)(226.0192488,466.78746315)
\curveto(225.41872283,466.10792061)(224.62855486,465.66542656)(223.79097916,465.55480335)
\lineto(223.66455251,464.89106346)
\curveto(224.7075713,464.95427672)(225.76639644,465.44418089)(226.69879203,466.2975599)
\curveto(228.54777988,468.00431792)(229.65401489,470.80151254)(229.65401489,474.16761864)
\curveto(229.65401489,476.58552583)(228.2791231,478.30809148)(226.2721021,478.48192795)
}
}
{
\newrgbcolor{curcolor}{0 0 0}
\pscustom[linestyle=none,fillstyle=solid,fillcolor=curcolor]
{
\newpath
\moveto(13.53018422,447.3133929)
\lineto(17.51262359,447.3133929)
\lineto(17.51262359,447.96132946)
\lineto(16.3905871,448.04034611)
\lineto(16.12193047,448.26159275)
\lineto(16.12193047,452.30724543)
\lineto(18.71367673,452.30724543)
\lineto(18.93492336,452.07019547)
\lineto(19.09295667,450.91655232)
\lineto(19.83571321,450.91655232)
\lineto(19.83571321,454.40908843)
\lineto(19.09295667,454.40908843)
\lineto(18.93492336,453.25544528)
\lineto(18.71367673,453.03419865)
\lineto(16.12193047,453.03419865)
\lineto(16.12193047,457.19047465)
\lineto(19.6776799,457.19047465)
\lineto(19.88312321,456.95342469)
\lineto(20.12017317,455.45210826)
\lineto(20.98935636,455.45210826)
\lineto(20.98935636,457.91742786)
\lineto(13.40375758,457.91742786)
\lineto(13.40375758,457.28529463)
\lineto(14.43097408,457.19047465)
\lineto(14.6996307,456.96922802)
\lineto(14.6996307,448.26159275)
\lineto(14.43097408,448.04034611)
\lineto(13.53018422,447.96132946)
\lineto(13.53018422,447.3133929)
}
}
{
\newrgbcolor{curcolor}{0 0 0}
\pscustom[linestyle=none,fillstyle=solid,fillcolor=curcolor]
{
\newpath
\moveto(23.93050438,458.5021511)
\curveto(23.4247983,458.5021511)(23.01391119,458.10706731)(23.01391119,457.58555792)
\curveto(23.01391119,457.07985184)(23.4247983,456.68476806)(23.93050438,456.68476806)
\curveto(24.43621046,456.68476806)(24.84709757,457.07985184)(24.84709757,457.58555792)
\curveto(24.84709757,458.091264)(24.43621046,458.5021511)(23.93050438,458.5021511)
\moveto(23.86729106,454.99381167)
\lineto(22.04990801,454.42489176)
\lineto(22.04990801,453.9349885)
\lineto(23.15614117,453.9349885)
\lineto(23.15614117,448.21418275)
\lineto(22.88748455,447.96132946)
\lineto(22.04990801,447.88231281)
\lineto(22.04990801,447.3133929)
\lineto(25.60565744,447.3133929)
\lineto(25.60565744,447.88231281)
\lineto(24.75227758,447.96132946)
\lineto(24.46781763,448.21418275)
\lineto(24.46781763,454.99381167)
\lineto(23.86729106,454.99381167)
}
}
{
\newrgbcolor{curcolor}{0 0 0}
\pscustom[linestyle=none,fillstyle=solid,fillcolor=curcolor]
{
\newpath
\moveto(26.1950723,454.91479501)
\lineto(26.1950723,454.33007177)
\lineto(27.15907548,454.15623513)
\lineto(29.2767218,451.07458563)
\lineto(27.00104217,448.02454278)
\lineto(26.10025231,447.89811614)
\lineto(26.10025231,447.3133929)
\lineto(29.08708183,447.3133929)
\lineto(29.08708183,447.89811614)
\lineto(28.18629198,448.00873945)
\lineto(28.10727532,448.19837942)
\lineto(29.71921507,450.48986239)
\lineto(31.18892483,448.1351661)
\lineto(31.10990818,448.00873945)
\lineto(30.44616828,447.89811614)
\lineto(30.44616828,447.3133929)
\lineto(33.81227774,447.3133929)
\lineto(33.81227774,447.89811614)
\lineto(32.8640779,448.02454278)
\lineto(30.54098827,451.40645558)
\lineto(32.62702793,454.17203847)
\lineto(33.51201446,454.33007177)
\lineto(33.51201446,454.91479501)
\lineto(30.66741491,454.91479501)
\lineto(30.66741491,454.33007177)
\lineto(31.42597479,454.23525179)
\lineto(31.55240144,454.02980849)
\lineto(30.11429834,452.00698215)
\lineto(28.78681855,454.07721848)
\lineto(28.8658352,454.23525179)
\lineto(29.59278842,454.33007177)
\lineto(29.59278842,454.91479501)
\lineto(26.1950723,454.91479501)
}
}
{
\newrgbcolor{curcolor}{0 0 0}
\pscustom[linestyle=none,fillstyle=solid,fillcolor=curcolor]
{
\newpath
\moveto(40.9539017,448.98854596)
\curveto(40.05311275,448.38801999)(39.26294452,448.07195278)(38.47277877,448.07195278)
\curveto(36.87664395,448.07195278)(35.81781919,449.27300794)(35.81781919,451.29583226)
\lineto(35.81781919,451.4696689)
\lineto(40.93809837,451.4696689)
\curveto(40.95390169,451.64350536)(40.9539017,451.801539)(40.9539017,451.95957215)
\curveto(40.9539017,453.9033799)(39.83186349,455.07282832)(38.10930216,455.07282832)
\curveto(35.99165795,455.07282832)(34.45873274,453.44508286)(34.45873274,451.0587823)
\curveto(34.45873274,448.73569499)(35.89683793,447.17116292)(37.98287551,447.17116292)
\curveto(38.97848436,447.17116292)(40.0057028,447.48723017)(40.9539017,448.11936277)
\lineto(40.9539017,448.98854596)
\moveto(39.54740526,452.13340879)
\lineto(35.89683585,452.13340879)
\curveto(35.99165574,453.47669057)(36.8924467,454.36167843)(37.90385886,454.36167843)
\curveto(38.96268096,454.36167843)(39.54740526,453.6663306)(39.54740526,452.38626208)
\lineto(39.54740526,452.13340879)
}
}
{
\newrgbcolor{curcolor}{0 0 0}
\pscustom[linestyle=none,fillstyle=solid,fillcolor=curcolor]
{
}
}
{
\newrgbcolor{curcolor}{0 0 0}
\pscustom[linestyle=none,fillstyle=solid,fillcolor=curcolor]
{
\newpath
\moveto(49.56004996,458.94464437)
\lineto(47.07892702,458.48634777)
\lineto(47.07892702,458.04385451)
\lineto(48.53283346,457.96483786)
\lineto(46.3993838,447.3133929)
\lineto(47.71106025,447.3133929)
\lineto(48.21676684,450.34763241)
\lineto(49.1175567,451.16940561)
\lineto(50.46083981,448.45123272)
\curveto(50.871726,447.62946034)(51.31422031,447.23437624)(51.94635291,447.23437624)
\curveto(52.24661589,447.23437624)(52.87875034,447.36080319)(53.79534261,447.66106618)
\lineto(53.79534261,448.10355944)
\curveto(53.47927631,448.04034618)(53.21061914,447.99293612)(52.97356941,447.99293612)
\curveto(51.88314068,447.99293612)(51.25100515,449.33622196)(50.04995321,452.05439214)
\curveto(50.74529907,452.68652474)(51.40904034,453.2712491)(52.08858289,453.76115186)
\lineto(52.65750279,453.14482196)
\curveto(53.24222545,453.38187169)(53.52668599,453.71374228)(53.52668599,454.12462847)
\curveto(53.52668599,454.58292461)(53.14740553,454.93059834)(52.62589613,454.93059834)
\curveto(52.24661657,454.93059834)(51.78831905,454.66194115)(51.23520302,454.09302181)
\lineto(48.31158682,451.07458563)
\lineto(50.00254322,458.94464437)
\lineto(49.56004996,458.94464437)
}
}
{
\newrgbcolor{curcolor}{0 0 0}
\pscustom[linestyle=none,fillstyle=solid,fillcolor=curcolor]
{
\newpath
\moveto(55.59173686,453.9823985)
\lineto(56.09744344,457.50654126)
\curveto(56.11324676,457.63296778)(56.1290501,457.74359131)(56.1290501,457.82260788)
\curveto(56.1290501,458.32831396)(55.8761964,458.59697109)(55.46531021,458.59697109)
\curveto(55.05442402,458.59697109)(54.81737365,458.32831394)(54.81737365,457.80680455)
\curveto(54.81737365,457.71198466)(54.81737366,457.61716447)(54.83317698,457.50654126)
\lineto(55.33888356,453.9823985)
\lineto(55.59173686,453.9823985)
}
}
{
\newrgbcolor{curcolor}{0 0 0}
\pscustom[linestyle=none,fillstyle=solid,fillcolor=curcolor]
{
}
}
{
\newrgbcolor{curcolor}{0 0 0}
\pscustom[linestyle=none,fillstyle=solid,fillcolor=curcolor]
{
\newpath
\moveto(68.27020592,448.98854596)
\curveto(67.36941696,448.38801999)(66.57924873,448.07195278)(65.78908298,448.07195278)
\curveto(64.19294817,448.07195278)(63.13412341,449.27300794)(63.13412341,451.29583226)
\lineto(63.13412341,451.4696689)
\lineto(68.25440259,451.4696689)
\curveto(68.2702059,451.64350536)(68.27020592,451.801539)(68.27020592,451.95957215)
\curveto(68.27020592,453.9033799)(67.14816771,455.07282832)(65.42560637,455.07282832)
\curveto(63.30796216,455.07282832)(61.77503696,453.44508286)(61.77503696,451.0587823)
\curveto(61.77503696,448.73569499)(63.21314215,447.17116292)(65.29917973,447.17116292)
\curveto(66.29478857,447.17116292)(67.32200702,447.48723017)(68.27020592,448.11936277)
\lineto(68.27020592,448.98854596)
\moveto(66.86370948,452.13340879)
\lineto(63.21314006,452.13340879)
\curveto(63.30795995,453.47669057)(64.20875091,454.36167843)(65.22016307,454.36167843)
\curveto(66.27898518,454.36167843)(66.86370948,453.6663306)(66.86370948,452.38626208)
\lineto(66.86370948,452.13340879)
}
}
{
\newrgbcolor{curcolor}{0 0 0}
\pscustom[linestyle=none,fillstyle=solid,fillcolor=curcolor]
{
\newpath
\moveto(70.32834283,453.96659516)
\lineto(70.32834283,448.86211932)
\curveto(70.32834283,447.74008395)(70.66021409,447.26598291)(71.97188923,447.26598291)
\curveto(72.58821851,447.26598291)(73.29936976,447.37660646)(74.04212556,447.61365618)
\lineto(74.04212556,448.19837942)
\curveto(73.67864932,448.11936285)(73.31517198,448.07195278)(72.95169574,448.07195278)
\curveto(71.84546369,448.07195278)(71.64001928,448.59346389)(71.64001928,449.79451583)
\lineto(71.64001928,453.96659516)
\lineto(74.04212556,453.96659516)
\lineto(74.04212556,454.70935171)
\lineto(71.64001928,454.70935171)
\lineto(71.64001928,456.38450478)
\lineto(71.05529604,456.38450478)
\lineto(70.37575282,454.91479501)
\lineto(69.33273299,454.45649842)
\lineto(69.33273299,453.96659516)
\lineto(70.32834283,453.96659516)
}
}
{
\newrgbcolor{curcolor}{0 0 0}
\pscustom[linestyle=none,fillstyle=solid,fillcolor=curcolor]
{
}
}
{
\newrgbcolor{curcolor}{0 0 0}
\pscustom[linestyle=none,fillstyle=solid,fillcolor=curcolor]
{
\newpath
\moveto(80.06838007,454.96220501)
\lineto(78.47224366,454.44069509)
\lineto(78.47224366,454.04561182)
\lineto(79.68910013,453.95079183)
\curveto(79.95775648,453.77695537)(80.08418348,453.33446121)(80.16320005,452.60750872)
\lineto(80.55828332,448.90952931)
\curveto(80.57408664,448.71988953)(80.58988998,448.53024918)(80.58988998,448.3406094)
\curveto(80.58988998,447.96132984)(80.5266765,447.61365588)(80.36864335,447.3133929)
\lineto(81.12720323,447.3133929)
\curveto(84.06661982,450.63208905)(85.45731587,452.97098642)(85.45731587,454.06141515)
\curveto(85.45731587,454.63033449)(85.15705205,454.96220501)(84.61973934,454.96220501)
\curveto(84.35108298,454.96220501)(84.08242584,454.85158148)(83.8295728,454.64613839)
\lineto(84.12983608,452.71813203)
\curveto(83.71894989,451.6277033)(82.91297841,450.34763088)(81.71192647,448.81470932)
\lineto(81.30103987,453.14482196)
\curveto(81.17461335,454.40908716)(81.00077568,454.96220501)(80.09998673,454.96220501)
\lineto(80.06838007,454.96220501)
}
}
{
\newrgbcolor{curcolor}{0 0 0}
\pscustom[linestyle=none,fillstyle=solid,fillcolor=curcolor]
{
}
}
{
\newrgbcolor{curcolor}{0 0 0}
\pscustom[linestyle=none,fillstyle=solid,fillcolor=curcolor]
{
\newpath
\moveto(91.53320271,453.96659516)
\lineto(91.53320271,448.86211932)
\curveto(91.53320271,447.74008395)(91.86507397,447.26598291)(93.17674911,447.26598291)
\curveto(93.7930784,447.26598291)(94.50422964,447.37660646)(95.24698545,447.61365618)
\lineto(95.24698545,448.19837942)
\curveto(94.8835092,448.11936285)(94.52003187,448.07195278)(94.15655562,448.07195278)
\curveto(93.05032357,448.07195278)(92.84487917,448.59346389)(92.84487917,449.79451583)
\lineto(92.84487917,453.96659516)
\lineto(95.24698545,453.96659516)
\lineto(95.24698545,454.70935171)
\lineto(92.84487917,454.70935171)
\lineto(92.84487917,456.38450478)
\lineto(92.26015593,456.38450478)
\lineto(91.5806127,454.91479501)
\lineto(90.53759287,454.45649842)
\lineto(90.53759287,453.96659516)
\lineto(91.53320271,453.96659516)
}
}
{
\newrgbcolor{curcolor}{0 0 0}
\pscustom[linestyle=none,fillstyle=solid,fillcolor=curcolor]
{
\newpath
\moveto(102.50046736,448.98854596)
\curveto(101.5996784,448.38801999)(100.80951017,448.07195278)(100.01934442,448.07195278)
\curveto(98.42320961,448.07195278)(97.36438485,449.27300794)(97.36438485,451.29583226)
\lineto(97.36438485,451.4696689)
\lineto(102.48466403,451.4696689)
\curveto(102.50046734,451.64350536)(102.50046736,451.801539)(102.50046736,451.95957215)
\curveto(102.50046736,453.9033799)(101.37842915,455.07282832)(99.65586781,455.07282832)
\curveto(97.5382236,455.07282832)(96.0052984,453.44508286)(96.0052984,451.0587823)
\curveto(96.0052984,448.73569499)(97.44340359,447.17116292)(99.52944117,447.17116292)
\curveto(100.52505001,447.17116292)(101.55226846,447.48723017)(102.50046736,448.11936277)
\lineto(102.50046736,448.98854596)
\moveto(101.09397092,452.13340879)
\lineto(97.4434015,452.13340879)
\curveto(97.53822139,453.47669057)(98.43901235,454.36167843)(99.45042451,454.36167843)
\curveto(100.50924662,454.36167843)(101.09397092,453.6663306)(101.09397092,452.38626208)
\lineto(101.09397092,452.13340879)
}
}
{
\newrgbcolor{curcolor}{0 0 0}
\pscustom[linestyle=none,fillstyle=solid,fillcolor=curcolor]
{
\newpath
\moveto(103.61040442,447.3133929)
\lineto(107.30838383,447.3133929)
\lineto(107.30838383,447.88231281)
\lineto(106.40759397,447.96132946)
\lineto(106.12313402,448.21418275)
\lineto(106.12313402,458.94464437)
\lineto(105.52260745,458.94464437)
\lineto(103.59460109,458.37572446)
\lineto(103.59460109,457.8858212)
\lineto(104.81145756,457.8858212)
\lineto(104.81145756,448.21418275)
\lineto(104.54280094,447.96132946)
\lineto(103.61040442,447.88231281)
\lineto(103.61040442,447.3133929)
}
}
{
\newrgbcolor{curcolor}{0 0 0}
\pscustom[linestyle=none,fillstyle=solid,fillcolor=curcolor]
{
}
}
{
\newrgbcolor{curcolor}{0 0 0}
\pscustom[linestyle=none,fillstyle=solid,fillcolor=curcolor]
{
\newpath
\moveto(119.24928181,454.81997503)
\curveto(118.41170611,454.99381149)(117.65314473,455.08863165)(116.9894055,455.08863165)
\curveto(114.27123532,455.08863165)(112.58027621,453.47668938)(112.58027621,450.94815898)
\curveto(112.58027621,448.70408825)(113.84454452,447.26598291)(115.69353238,447.26598291)
\curveto(116.35727161,447.26598291)(117.06842296,447.45562325)(117.87439203,447.83490281)
\lineto(117.87439203,444.76905664)
\lineto(117.6057354,444.50040002)
\lineto(116.61012556,444.43718669)
\lineto(116.61012556,443.86826678)
\lineto(120.30810497,443.86826678)
\lineto(120.30810497,444.43718669)
\lineto(119.47052844,444.50040002)
\lineto(119.18606848,444.76905664)
\lineto(119.18606848,453.66633188)
\curveto(119.18606848,453.82436503)(119.21767518,454.20364574)(119.24928181,454.81997503)
\moveto(117.87439203,454.23525179)
\lineto(117.87439203,448.3406094)
\curveto(117.24225943,448.18257625)(116.76815851,448.10355944)(116.40468226,448.10355944)
\curveto(114.91917065,448.10355944)(114.00257598,449.19399122)(114.00257598,451.15360228)
\curveto(114.00257598,453.20803323)(115.04559744,454.40908843)(116.67333889,454.40908843)
\curveto(117.0052085,454.40908843)(117.40029258,454.34587499)(117.87439203,454.23525179)
}
}
{
\newrgbcolor{curcolor}{0 0 0}
\pscustom[linestyle=none,fillstyle=solid,fillcolor=curcolor]
{
\newpath
\moveto(122.63514543,455.07282832)
\lineto(120.80195906,454.48810508)
\lineto(120.80195906,453.99820183)
\lineto(121.92399555,453.99820183)
\lineto(121.92399555,449.5732692)
\curveto(121.92399555,447.9929377)(122.68255678,447.15535959)(124.04164187,447.15535959)
\curveto(124.81600431,447.15535959)(125.65358256,447.51883688)(126.60178146,448.19837942)
\lineto(126.60178146,447.3133929)
\lineto(128.94067442,447.3133929)
\lineto(128.94067442,447.88231281)
\lineto(128.19791787,447.96132946)
\lineto(127.91345792,448.21418275)
\lineto(127.91345792,455.07282832)
\lineto(127.31293135,455.07282832)
\lineto(125.47974498,454.48810508)
\lineto(125.47974498,453.99820183)
\lineto(126.60178146,453.99820183)
\lineto(126.60178146,448.64087269)
\curveto(125.95384555,448.32480639)(125.35331774,448.16677276)(124.76859509,448.16677276)
\curveto(123.70977298,448.16677276)(123.235672,448.70408715)(123.235672,449.84192583)
\lineto(123.235672,455.07282832)
\lineto(122.63514543,455.07282832)
}
}
{
\newrgbcolor{curcolor}{0 0 0}
\pscustom[linestyle=none,fillstyle=solid,fillcolor=curcolor]
{
\newpath
\moveto(136.45293587,448.98854596)
\curveto(135.55214692,448.38801999)(134.76197869,448.07195278)(133.97181294,448.07195278)
\curveto(132.37567812,448.07195278)(131.31685336,449.27300794)(131.31685336,451.29583226)
\lineto(131.31685336,451.4696689)
\lineto(136.43713254,451.4696689)
\curveto(136.45293586,451.64350536)(136.45293587,451.801539)(136.45293587,451.95957215)
\curveto(136.45293587,453.9033799)(135.33089766,455.07282832)(133.60833633,455.07282832)
\curveto(131.49069212,455.07282832)(129.95776691,453.44508286)(129.95776691,451.0587823)
\curveto(129.95776691,448.73569499)(131.3958721,447.17116292)(133.48190968,447.17116292)
\curveto(134.47751853,447.17116292)(135.50473697,447.48723017)(136.45293587,448.11936277)
\lineto(136.45293587,448.98854596)
\moveto(135.04643943,452.13340879)
\lineto(131.39587002,452.13340879)
\curveto(131.49068991,453.47669057)(132.39148087,454.36167843)(133.40289303,454.36167843)
\curveto(134.46171513,454.36167843)(135.04643943,453.6663306)(135.04643943,452.38626208)
\lineto(135.04643943,452.13340879)
}
}
{
\newrgbcolor{curcolor}{0 0 0}
\pscustom[linestyle=none,fillstyle=solid,fillcolor=curcolor]
{
\newpath
\moveto(139.63310927,454.91479501)
\curveto(139.07999324,454.91479501)(138.63749943,454.48810451)(138.63749943,453.91918517)
\curveto(138.63749943,453.36606915)(139.07999324,452.907772)(139.63310927,452.907772)
\curveto(140.18622529,452.907772)(140.64452244,453.36606915)(140.64452244,453.91918517)
\curveto(140.64452244,454.48810451)(140.18622529,454.91479501)(139.63310927,454.91479501)
\moveto(139.63310927,449.17818593)
\curveto(139.07999324,449.17818593)(138.63749943,448.7356921)(138.63749943,448.16677276)
\curveto(138.63749943,447.59785342)(139.07999324,447.17116292)(139.63310927,447.17116292)
\curveto(140.18622529,447.17116292)(140.64452244,447.61365674)(140.64452244,448.16677276)
\curveto(140.64452244,448.7356921)(140.18622529,449.17818593)(139.63310927,449.17818593)
}
}
{
\newrgbcolor{curcolor}{0 0 0}
\pscustom[linestyle=none,fillstyle=solid,fillcolor=curcolor]
{
}
}
{
\newrgbcolor{curcolor}{0 0 0}
\pscustom[linestyle=none,fillstyle=solid,fillcolor=curcolor]
{
}
}
{
\newrgbcolor{curcolor}{0 0 0}
\pscustom[linestyle=none,fillstyle=solid,fillcolor=curcolor]
{
}
}
{
\newrgbcolor{curcolor}{0 0 0}
\pscustom[linestyle=none,fillstyle=solid,fillcolor=curcolor]
{
}
}
{
\newrgbcolor{curcolor}{0 0 0}
\pscustom[linestyle=none,fillstyle=solid,fillcolor=curcolor]
{
}
}
{
\newrgbcolor{curcolor}{0 0 0}
\pscustom[linestyle=none,fillstyle=solid,fillcolor=curcolor]
{
\newpath
\moveto(41.6660393,436.32571128)
\lineto(32.57912409,432.59612521)
\lineto(32.57912409,431.80595867)
\lineto(41.6660393,428.0763726)
\lineto(41.6660393,429.07198245)
\lineto(33.95401387,432.20104194)
\lineto(41.6660393,435.34590477)
\lineto(41.6660393,436.32571128)
}
}
{
\newrgbcolor{curcolor}{0 0 0}
\pscustom[linestyle=none,fillstyle=solid,fillcolor=curcolor]
{
\newpath
\moveto(45.18993514,435.14046147)
\lineto(43.59379873,434.61895156)
\lineto(43.59379873,434.22386829)
\lineto(44.8106552,434.1290483)
\curveto(45.07931155,433.95521184)(45.20573855,433.51271767)(45.28475512,432.78576518)
\lineto(45.67983839,429.08778578)
\curveto(45.69564171,428.898146)(45.71144505,428.70850565)(45.71144505,428.51886587)
\curveto(45.71144505,428.13958631)(45.64823157,427.79191235)(45.49019842,427.49164937)
\lineto(46.2487583,427.49164937)
\curveto(49.18817489,430.81034551)(50.57887094,433.14924288)(50.57887094,434.23967162)
\curveto(50.57887094,434.80859096)(50.27860712,435.14046147)(49.74129441,435.14046147)
\curveto(49.47263805,435.14046147)(49.20398091,435.02983795)(48.95112787,434.82439486)
\lineto(49.25139115,432.8963885)
\curveto(48.84050496,431.80595976)(48.03453348,430.52588735)(46.83348154,428.99296579)
\lineto(46.42259494,433.32307843)
\curveto(46.29616842,434.58734363)(46.12233075,435.14046147)(45.2215418,435.14046147)
\lineto(45.18993514,435.14046147)
}
}
{
\newrgbcolor{curcolor}{0 0 0}
\pscustom[linestyle=none,fillstyle=solid,fillcolor=curcolor]
{
\newpath
\moveto(52.91973932,436.32571128)
\lineto(52.91973932,435.34590477)
\lineto(60.63176475,432.20104194)
\lineto(52.91973932,429.07198245)
\lineto(52.91973932,428.0763726)
\lineto(62.00665453,431.80595867)
\lineto(62.00665453,432.59612521)
\lineto(52.91973932,436.32571128)
}
}
{
\newrgbcolor{curcolor}{0 0 0}
\pscustom[linestyle=none,fillstyle=solid,fillcolor=curcolor]
{
}
}
{
\newrgbcolor{curcolor}{0 0 0}
\pscustom[linestyle=none,fillstyle=solid,fillcolor=curcolor]
{
\newpath
\moveto(70.40340865,434.1290483)
\lineto(70.40340865,433.22825845)
\lineto(79.64835717,433.22825845)
\lineto(79.64835717,434.1290483)
\lineto(70.40340865,434.1290483)
\moveto(70.40340865,431.17382544)
\lineto(70.40340865,430.27303559)
\lineto(79.64835717,430.27303559)
\lineto(79.64835717,431.17382544)
\lineto(70.40340865,431.17382544)
}
}
{
\newrgbcolor{curcolor}{0 0 0}
\pscustom[linestyle=none,fillstyle=solid,fillcolor=curcolor]
{
\newpath
\moveto(88.22931886,438.60139092)
\lineto(88.02387556,438.60139092)
\curveto(85.85882141,438.60139092)(84.94222606,437.85863257)(84.94222606,436.05705466)
\lineto(84.94222606,434.44511492)
\curveto(84.94222606,432.97540662)(84.48392804,432.40648524)(83.06162969,432.40648524)
\curveto(82.93520317,432.40648524)(82.79297293,432.40648526)(82.65074309,432.42228857)
\lineto(82.65074309,431.72694202)
\lineto(83.0142197,431.72694202)
\curveto(84.46812468,431.72694202)(84.95802939,431.17382388)(84.95802939,429.60929569)
\lineto(84.95802939,428.26601257)
\curveto(84.95802939,426.06935179)(85.59016486,425.37400304)(87.83423559,425.37400304)
\lineto(88.22931886,425.37400304)
\lineto(88.22931886,425.99033294)
\curveto(86.44354427,426.02193957)(86.22229585,426.51184444)(86.22229585,428.09217594)
\lineto(86.22229585,429.40385239)
\curveto(86.22229585,430.98418389)(85.63757113,431.74274568)(84.15205952,432.0746153)
\curveto(85.73239102,432.40648491)(86.20649252,433.18085011)(86.20649252,434.84019819)
\lineto(86.20649252,436.12026798)
\curveto(86.20649252,437.5109597)(86.64898736,437.96925773)(88.22931886,438.01666768)
\lineto(88.22931886,438.60139092)
}
}
{
\newrgbcolor{curcolor}{0 0 0}
\pscustom[linestyle=none,fillstyle=solid,fillcolor=curcolor]
{
\newpath
\moveto(91.3477605,435.1562648)
\lineto(89.75162409,434.60314823)
\lineto(89.75162409,434.22386829)
\lineto(91.06330054,434.05003165)
\curveto(91.48999005,433.65494877)(91.8534675,432.73835375)(92.26435369,431.30025209)
\curveto(91.61641777,429.7357239)(90.77883988,428.72430879)(90.0676907,428.34502923)
\lineto(89.34073749,429.02457245)
\curveto(88.85083472,428.92975256)(88.56637428,428.64529209)(88.56637428,428.21860258)
\curveto(88.56637428,427.7128965)(88.94565482,427.36522272)(89.54618079,427.36522272)
\curveto(90.58919958,427.36522272)(91.5374015,428.42404803)(92.56461697,430.57329887)
\curveto(93.08612637,428.32922814)(93.29157129,427.39682938)(94.39780334,427.39682938)
\curveto(94.82449285,427.39682938)(95.42502059,427.55486299)(96.16777639,427.85512597)
\lineto(96.16777639,428.34502923)
\curveto(95.78849683,428.23440602)(95.50403633,428.17119259)(95.33019986,428.17119259)
\curveto(94.38200096,428.17119259)(94.16075277,429.16680456)(93.54442348,431.30025209)
\curveto(94.23976934,432.91219022)(94.93511732,433.81298187)(95.66206981,434.00262165)
\lineto(96.34161303,433.38629175)
\curveto(96.84731911,433.59173485)(97.11597624,433.89199872)(97.11597624,434.27127828)
\curveto(97.11597624,434.76118104)(96.72089247,435.1562648)(96.21518639,435.1562648)
\curveto(95.21957754,435.1562648)(94.23976899,434.09743952)(93.1967502,431.97979531)
\curveto(92.64363418,434.12904615)(92.45399272,435.17206813)(91.52159714,435.17206813)
\curveto(91.45838388,435.17206813)(91.41097376,435.1562648)(91.3477605,435.1562648)
}
}
{
\newrgbcolor{curcolor}{0 0 0}
\pscustom[linestyle=none,fillstyle=solid,fillcolor=curcolor]
{
\newpath
\moveto(97.80440884,432.46969857)
\curveto(100.68061217,434.27127648)(102.22934146,435.40911947)(102.22934146,436.78400788)
\curveto(102.22934146,437.66899352)(101.58140373,438.20630765)(100.41195842,438.20630765)
\curveto(99.62179267,438.20630765)(98.84742817,437.92184728)(98.34172209,437.51096109)
\curveto(98.0414591,437.25810805)(97.89922882,436.97364759)(97.89922882,436.72079455)
\curveto(97.89922882,436.49954814)(98.08886909,436.34151461)(98.38913208,436.34151461)
\curveto(98.54716523,436.34151461)(98.72100218,436.3889247)(98.87903533,436.48374459)
\lineto(99.10028196,437.49515776)
\curveto(99.43215158,437.6373876)(99.74821886,437.71640439)(100.08008847,437.71640439)
\curveto(100.80704096,437.71640439)(101.21792829,437.33712381)(101.21792829,436.68918789)
\curveto(101.21792829,435.25108623)(99.62178899,434.16065313)(96.72978234,432.32746859)
\lineto(96.6033557,431.72694202)
\lineto(101.43917492,431.72694202)
\lineto(101.77104487,433.27566844)
\lineto(101.29694495,433.27566844)
\lineto(100.93346834,432.46969857)
\lineto(97.80440884,432.46969857)
}
}
{
\newrgbcolor{curcolor}{0 0 0}
\pscustom[linestyle=none,fillstyle=solid,fillcolor=curcolor]
{
}
}
{
\newrgbcolor{curcolor}{0 0 0}
\pscustom[linestyle=none,fillstyle=solid,fillcolor=curcolor]
{
\newpath
\moveto(109.88654909,435.10885481)
\lineto(107.54765613,434.60314823)
\lineto(107.54765613,434.19226162)
\lineto(108.84352926,433.93940833)
\lineto(107.5318528,427.49164937)
\lineto(108.85933259,427.49164937)
\lineto(109.85494243,432.84897851)
\curveto(110.85055127,433.74976746)(111.68812921,434.14485163)(112.0990154,434.14485163)
\curveto(112.65213143,434.14485163)(112.9681986,433.79717777)(112.9681986,433.21245511)
\curveto(112.9681986,432.9912087)(112.92078851,432.62773137)(112.82596862,432.12202529)
\lineto(111.87776877,427.49164937)
\lineto(113.18944523,427.49164937)
\lineto(114.21666173,432.86478184)
\curveto(114.94361422,433.71816085)(115.76538875,434.14485163)(116.36591472,434.14485163)
\curveto(116.88742411,434.14485163)(117.14027793,433.79717769)(117.14027793,433.13343846)
\curveto(117.14027793,432.91219205)(117.10867115,432.58032142)(116.99804795,432.12202529)
\lineto(116.31850473,429.19840909)
\curveto(116.23948815,428.83493285)(116.19207808,428.53466901)(116.19207808,428.34502923)
\curveto(116.19207808,427.69709331)(116.55555548,427.39682938)(117.34572123,427.39682938)
\curveto(117.77241073,427.39682938)(118.40454523,427.57066634)(119.24212092,427.88673264)
\lineto(119.24212092,428.31342257)
\curveto(118.87864468,428.21860268)(118.57838081,428.17119259)(118.3571344,428.17119259)
\curveto(117.85142832,428.17119259)(117.55116453,428.45565305)(117.55116453,428.96135913)
\curveto(117.55116453,429.21421217)(117.63018128,429.56188613)(117.72500117,429.98857563)
\lineto(118.26231441,432.43809191)
\curveto(118.38874093,432.97540462)(118.45195438,433.41789876)(118.45195438,433.76557169)
\curveto(118.45195438,434.60314739)(117.91464038,435.10885481)(117.15608126,435.10885481)
\curveto(116.39752214,435.10885481)(115.41771372,434.63475392)(114.26407172,433.67075171)
\curveto(114.24826841,434.60314729)(113.77416768,435.10885481)(112.98400193,435.10885481)
\curveto(112.24124612,435.10885481)(111.22983098,434.61895054)(109.99717241,433.60753838)
\lineto(110.28163236,435.10885481)
\lineto(109.88654909,435.10885481)
}
}
{
\newrgbcolor{curcolor}{0 0 0}
\pscustom[linestyle=none,fillstyle=solid,fillcolor=curcolor]
{
\newpath
\moveto(124.30634766,435.12465814)
\curveto(123.23172224,435.12465814)(122.26771723,434.66636069)(121.50915811,433.81298169)
\curveto(120.71899236,432.91219273)(120.24489165,431.63212075)(120.24489165,430.35205224)
\curveto(120.24489165,428.5188677)(121.30371644,427.34941939)(122.93145788,427.34941939)
\curveto(124.02188662,427.34941939)(124.9700883,427.77611017)(125.7128441,428.62948918)
\curveto(126.53461648,429.57768808)(126.9929139,430.77874342)(126.9929139,432.0272053)
\curveto(126.9929139,433.93940642)(125.94989242,435.12465814)(124.30634766,435.12465814)
\moveto(124.10090436,434.50832824)
\curveto(125.06490658,434.50832824)(125.64963078,433.68655347)(125.64963078,432.12202529)
\curveto(125.64963078,429.71992141)(124.46437963,427.99735595)(123.12109785,427.99735595)
\curveto(122.17289895,427.99735595)(121.58817477,428.81913067)(121.58817477,430.33624891)
\curveto(121.58817477,432.78576273)(122.74181927,434.50832824)(124.10090436,434.50832824)
}
}
{
\newrgbcolor{curcolor}{0 0 0}
\pscustom[linestyle=none,fillstyle=solid,fillcolor=curcolor]
{
\newpath
\moveto(135.33633178,439.12290083)
\lineto(132.91842217,438.66460424)
\lineto(132.91842217,438.22211098)
\lineto(134.30911528,438.06407767)
\lineto(133.7559987,434.90341151)
\curveto(133.3293092,435.09305129)(132.91842174,435.17206813)(132.49173224,435.17206813)
\curveto(131.18005709,435.17206813)(129.78936182,434.38190025)(128.93598281,433.03861848)
\curveto(128.43027673,432.23264941)(128.1616196,431.25284102)(128.1616196,430.1782156)
\curveto(128.1616196,428.37663769)(128.96759073,427.34941939)(130.23185593,427.34941939)
\curveto(131.06943163,427.34941939)(131.9228131,427.83932362)(132.69717554,428.81912915)
\curveto(132.68137222,428.70850595)(132.68137221,428.59788243)(132.68137221,428.50306254)
\curveto(132.68137221,427.7287001)(133.01324288,427.38102605)(133.74019537,427.38102605)
\curveto(134.15108156,427.38102605)(134.81482273,427.52325634)(135.68400506,427.83932264)
\lineto(135.68400506,428.31342257)
\curveto(135.20990561,428.20279936)(134.86223165,428.13958593)(134.65678856,428.13958593)
\curveto(134.18268911,428.13958593)(133.961442,428.360833)(133.961442,428.80332582)
\curveto(133.961442,429.00876892)(133.99304873,429.30903275)(134.05626199,429.65670568)
\lineto(135.74721838,439.12290083)
\lineto(135.33633178,439.12290083)
\moveto(132.77619219,434.50832824)
\lineto(133.53475207,433.60753838)
\lineto(132.82360218,429.59349236)
\curveto(132.11245301,428.69270341)(131.3854985,428.25020924)(130.80077584,428.25020924)
\curveto(129.97900346,428.25020924)(129.44168939,428.94555705)(129.44168939,430.19401893)
\curveto(129.44168939,432.83317254)(130.9746139,434.55573823)(132.39691225,434.55573823)
\curveto(132.50753546,434.55573823)(132.63396236,434.52413156)(132.77619219,434.50832824)
}
}
{
\newrgbcolor{curcolor}{0 0 0}
\pscustom[linestyle=none,fillstyle=solid,fillcolor=curcolor]
{
}
}
{
\newrgbcolor{curcolor}{0 0 0}
\pscustom[linestyle=none,fillstyle=solid,fillcolor=curcolor]
{
\newpath
\moveto(143.34516352,435.10885481)
\lineto(141.00627056,434.58734489)
\lineto(141.00627056,434.19226162)
\lineto(142.30214369,433.95521166)
\lineto(140.99046723,427.49164937)
\lineto(142.33375035,427.49164937)
\lineto(143.31355686,432.83317518)
\curveto(144.27755907,433.71816082)(145.1151371,434.1290483)(145.65244981,434.1290483)
\curveto(146.25297578,434.1290483)(146.50582968,433.82878432)(146.50582968,433.13343846)
\curveto(146.50582968,432.92799537)(146.47422291,432.58032142)(146.3635997,432.12202529)
\lineto(145.68405648,429.19840909)
\curveto(145.6050399,428.85073616)(145.55762983,428.55047232)(145.55762983,428.34502923)
\curveto(145.55762983,427.71289663)(145.93691054,427.39682938)(146.71127298,427.39682938)
\curveto(147.1537658,427.39682938)(147.78590029,427.554863)(148.60767267,427.8709293)
\lineto(148.60767267,428.31342257)
\curveto(148.24419643,428.21860268)(147.95973589,428.17119259)(147.73848948,428.17119259)
\curveto(147.21698009,428.17119259)(146.91671628,428.43984972)(146.91671628,428.9455558)
\curveto(146.91671628,429.18260552)(146.99573303,429.5302795)(147.09055292,429.98857563)
\lineto(147.62786616,432.43809191)
\curveto(147.75429268,433.00701125)(147.8491128,433.44950547)(147.8491128,433.84458835)
\curveto(147.8491128,434.61895078)(147.34340539,435.10885481)(146.52163301,435.10885481)
\curveto(145.68405731,435.10885481)(144.70424872,434.5873439)(143.45578683,433.59173505)
\lineto(143.74024679,435.10885481)
\lineto(143.34516352,435.10885481)
}
}
{
\newrgbcolor{curcolor}{0 0 0}
\pscustom[linestyle=none,fillstyle=solid,fillcolor=curcolor]
{
\newpath
\moveto(152.01397096,437.60346397)
\curveto(152.39404604,437.62400856)(152.60976492,437.83972738)(152.60976492,438.12735176)
\curveto(152.60976492,438.48688224)(152.23996119,438.76423496)(151.65444012,438.76423496)
\curveto(150.66829937,438.76423496)(149.88760285,438.05544475)(149.88760285,437.2131162)
\curveto(149.88760285,436.63786743)(150.2779512,436.36051405)(150.86347227,435.91880517)
\curveto(151.29490885,435.59009159)(151.57226164,435.35382775)(151.57226164,434.96348037)
\curveto(151.57226164,434.4190485)(151.22300253,434.09033405)(150.63748146,434.09033405)
\curveto(150.31904018,434.09033405)(150.06223195,434.213602)(149.89787516,434.44986489)
\lineto(150.08277673,434.96348037)
\curveto(149.91841994,435.08674797)(149.78487961,435.14838195)(149.64106742,435.14838195)
\curveto(149.32262613,435.14838195)(149.10690731,434.94293543)(149.10690731,434.62449415)
\curveto(149.10690731,434.09033458)(149.61025135,433.7205309)(150.4731245,433.7205309)
\curveto(151.59280514,433.7205309)(152.35295718,434.40877654)(152.35295718,435.30246659)
\curveto(152.35295718,435.91880456)(151.93179186,436.1858857)(151.3154539,436.65841147)
\curveto(150.83265582,437.02821425)(150.54503067,437.2439335)(150.54503067,437.61373628)
\curveto(150.54503067,438.01435595)(150.85320037,438.30198103)(151.26409235,438.30198103)
\curveto(151.37708764,438.30198103)(151.47981096,438.27116402)(151.58253395,438.19925793)
\lineto(152.01397096,437.60346397)
}
}
{
\newrgbcolor{curcolor}{0 0 0}
\pscustom[linestyle=none,fillstyle=solid,fillcolor=curcolor]
{
\newpath
\moveto(157.05911439,439.93527827)
\lineto(157.05911439,437.17202696)
\lineto(154.34722462,437.17202696)
\lineto(154.34722462,436.58650531)
\lineto(157.05911439,436.58650531)
\lineto(157.05911439,433.81298169)
\lineto(157.65490835,433.81298169)
\lineto(157.65490835,436.58650531)
\lineto(160.3565258,436.58650531)
\lineto(160.3565258,437.17202696)
\lineto(157.65490835,437.17202696)
\lineto(157.65490835,439.93527827)
\lineto(157.05911439,439.93527827)
}
}
{
\newrgbcolor{curcolor}{0 0 0}
\pscustom[linestyle=none,fillstyle=solid,fillcolor=curcolor]
{
\newpath
\moveto(165.5438816,440.82896921)
\lineto(163.15043344,440.15099677)
\lineto(163.15043344,439.72983207)
\lineto(164.63991835,439.92500596)
\lineto(164.74264144,439.82228286)
\lineto(163.71541047,434.49095413)
\lineto(163.55105352,434.31632486)
\lineto(162.48273331,434.254691)
\lineto(162.48273331,433.81298169)
\lineto(165.55415391,433.81298169)
\lineto(165.55415391,434.24441869)
\lineto(164.66046297,434.29578024)
\lineto(164.56801218,434.44986489)
\lineto(165.79041703,440.82896921)
\lineto(165.5438816,440.82896921)
}
}
{
\newrgbcolor{curcolor}{0 0 0}
\pscustom[linestyle=none,fillstyle=solid,fillcolor=curcolor]
{
\newpath
\moveto(166.45760281,424.82088646)
\curveto(168.38560724,425.05793618)(169.69728563,426.62246764)(169.69728563,428.09217594)
\curveto(169.69728563,428.77171848)(169.33380837,429.16680243)(168.68587246,429.16680243)
\curveto(168.05373986,429.16680243)(167.56383597,428.70850519)(167.56383597,428.06056927)
\curveto(167.56383597,427.88673281)(167.61124606,427.74450252)(167.70606595,427.60227268)
\lineto(168.71747912,427.60227268)
\curveto(168.54364265,426.3538108)(167.816688,425.5794461)(166.5524228,425.34239638)
\lineto(166.45760281,424.82088646)
}
}
{
\newrgbcolor{curcolor}{0 0 0}
\pscustom[linestyle=none,fillstyle=solid,fillcolor=curcolor]
{
}
}
{
\newrgbcolor{curcolor}{0 0 0}
\pscustom[linestyle=none,fillstyle=solid,fillcolor=curcolor]
{
\newpath
\moveto(178.1629324,435.1562648)
\lineto(176.56679599,434.60314823)
\lineto(176.56679599,434.22386829)
\lineto(177.87847244,434.05003165)
\curveto(178.30516195,433.65494877)(178.66863939,432.73835375)(179.07952558,431.30025209)
\curveto(178.43158967,429.7357239)(177.59401178,428.72430879)(176.8828626,428.34502923)
\lineto(176.15590938,429.02457245)
\curveto(175.66600662,428.92975256)(175.38154618,428.64529209)(175.38154618,428.21860258)
\curveto(175.38154618,427.7128965)(175.76082672,427.36522272)(176.36135268,427.36522272)
\curveto(177.40437147,427.36522272)(178.35257339,428.42404803)(179.37978887,430.57329887)
\curveto(179.90129826,428.32922814)(180.10674319,427.39682938)(181.21297524,427.39682938)
\curveto(181.63966475,427.39682938)(182.24019249,427.55486299)(182.98294829,427.85512597)
\lineto(182.98294829,428.34502923)
\curveto(182.60366873,428.23440602)(182.31920822,428.17119259)(182.14537176,428.17119259)
\curveto(181.19717286,428.17119259)(180.97592466,429.16680456)(180.35959538,431.30025209)
\curveto(181.05494124,432.91219022)(181.75028921,433.81298187)(182.4772417,434.00262165)
\lineto(183.15678493,433.38629175)
\curveto(183.66249101,433.59173485)(183.93114814,433.89199872)(183.93114814,434.27127828)
\curveto(183.93114814,434.76118104)(183.53606436,435.1562648)(183.03035828,435.1562648)
\curveto(182.03474944,435.1562648)(181.05494089,434.09743952)(180.0119221,431.97979531)
\curveto(179.45880607,434.12904615)(179.26916462,435.17206813)(178.33676903,435.17206813)
\curveto(178.27355577,435.17206813)(178.22614566,435.1562648)(178.1629324,435.1562648)
}
}
{
\newrgbcolor{curcolor}{0 0 0}
\pscustom[linestyle=none,fillstyle=solid,fillcolor=curcolor]
{
}
}
{
\newrgbcolor{curcolor}{0 0 0}
\pscustom[linestyle=none,fillstyle=solid,fillcolor=curcolor]
{
\newpath
\moveto(198.88788957,427.49164937)
\lineto(194.31973926,427.49164937)
\curveto(192.99250145,427.49164937)(191.97392841,427.95978142)(191.26401709,428.89604693)
\curveto(190.61583103,429.77057793)(190.415203,430.82516112)(190.66213242,432.05979967)
\curveto(190.91934521,433.35616079)(191.60353821,434.45189845)(192.71471347,435.34701594)
\curveto(193.76414769,436.20096326)(194.90618412,436.62794085)(196.14082621,436.62794998)
\lineto(200.70897652,436.62794998)
\lineto(200.57008005,435.87173591)
\lineto(196.00192974,435.87173591)
\curveto(194.91132843,435.87172753)(193.90304401,435.4756158)(192.97707346,434.68339951)
\curveto(192.25686536,434.03520948)(191.77330039,433.28928478)(191.52637708,432.44562318)
\lineto(199.87559774,432.44562318)
\lineto(199.72126834,431.67397617)
\lineto(191.37204767,431.67397617)
\curveto(191.23829233,430.60395587)(191.54180651,429.72942346)(192.28259115,429.05037633)
\curveto(192.87932666,428.51536671)(193.60981844,428.24786268)(194.47406866,428.24786344)
\lineto(199.04221897,428.24786344)
\lineto(198.88788957,427.49164937)
}
}
{
\newrgbcolor{curcolor}{0 0 0}
\pscustom[linestyle=none,fillstyle=solid,fillcolor=curcolor]
{
}
}
{
\newrgbcolor{curcolor}{0 0 0}
\pscustom[linestyle=none,fillstyle=solid,fillcolor=curcolor]
{
\newpath
\moveto(207.37711787,438.09568433)
\lineto(206.90301795,435.64616806)
\lineto(207.72479115,435.64616806)
\lineto(208.26210439,437.13168115)
\lineto(208.60977767,437.33712445)
\lineto(212.52900371,437.33712445)
\lineto(205.06983157,428.13958593)
\lineto(204.95920826,427.49164937)
\lineto(212.38677373,427.49164937)
\lineto(212.92408698,430.35205224)
\lineto(212.14972377,430.35205224)
\lineto(211.51759054,428.47145587)
\lineto(211.29634391,428.26601257)
\lineto(206.7765913,428.26601257)
\lineto(214.25156677,437.43194444)
\lineto(214.39379674,438.09568433)
\lineto(207.37711787,438.09568433)
}
}
{
\newrgbcolor{curcolor}{0 0 0}
\pscustom[linestyle=none,fillstyle=solid,fillcolor=curcolor]
{
\newpath
\moveto(221.48381295,438.09568433)
\lineto(212.39689774,426.0219396)
\lineto(213.36090092,426.0219396)
\lineto(222.44781613,438.09568433)
\lineto(221.48381295,438.09568433)
}
}
{
\newrgbcolor{curcolor}{0 0 0}
\pscustom[linestyle=none,fillstyle=solid,fillcolor=curcolor]
{
\newpath
\moveto(224.60448899,435.10885481)
\lineto(222.26559603,434.58734489)
\lineto(222.26559603,434.19226162)
\lineto(223.56146915,433.95521166)
\lineto(222.2497927,427.49164937)
\lineto(223.59307582,427.49164937)
\lineto(224.57288233,432.83317518)
\curveto(225.53688454,433.71816082)(226.37446257,434.1290483)(226.91177528,434.1290483)
\curveto(227.51230125,434.1290483)(227.76515515,433.82878432)(227.76515515,433.13343846)
\curveto(227.76515515,432.92799537)(227.73354838,432.58032142)(227.62292517,432.12202529)
\lineto(226.94338195,429.19840909)
\curveto(226.86436537,428.85073616)(226.8169553,428.55047232)(226.8169553,428.34502923)
\curveto(226.8169553,427.71289663)(227.19623601,427.39682938)(227.97059845,427.39682938)
\curveto(228.41309127,427.39682938)(229.04522576,427.554863)(229.86699814,427.8709293)
\lineto(229.86699814,428.31342257)
\curveto(229.5035219,428.21860268)(229.21906136,428.17119259)(228.99781495,428.17119259)
\curveto(228.47630556,428.17119259)(228.17604175,428.43984972)(228.17604175,428.9455558)
\curveto(228.17604175,429.18260552)(228.2550585,429.5302795)(228.34987839,429.98857563)
\lineto(228.88719163,432.43809191)
\curveto(229.01361815,433.00701125)(229.10843827,433.44950547)(229.10843827,433.84458835)
\curveto(229.10843827,434.61895078)(228.60273086,435.10885481)(227.78095848,435.10885481)
\curveto(226.94338278,435.10885481)(225.96357419,434.5873439)(224.7151123,433.59173505)
\lineto(224.99957226,435.10885481)
\lineto(224.60448899,435.10885481)
}
}
{
\newrgbcolor{curcolor}{0 0 0}
\pscustom[linestyle=none,fillstyle=solid,fillcolor=curcolor]
{
\newpath
\moveto(233.27329643,437.60346397)
\curveto(233.65337151,437.62400856)(233.86909039,437.83972738)(233.86909039,438.12735176)
\curveto(233.86909039,438.48688224)(233.49928666,438.76423496)(232.91376559,438.76423496)
\curveto(231.92762484,438.76423496)(231.14692832,438.05544475)(231.14692832,437.2131162)
\curveto(231.14692832,436.63786743)(231.53727667,436.36051405)(232.12279774,435.91880517)
\curveto(232.55423432,435.59009159)(232.83158711,435.35382775)(232.83158711,434.96348037)
\curveto(232.83158711,434.4190485)(232.48232799,434.09033405)(231.89680693,434.09033405)
\curveto(231.57836564,434.09033405)(231.32155742,434.213602)(231.15720063,434.44986489)
\lineto(231.3421022,434.96348037)
\curveto(231.17774541,435.08674797)(231.04420508,435.14838195)(230.90039289,435.14838195)
\curveto(230.5819516,435.14838195)(230.36623278,434.94293543)(230.36623278,434.62449415)
\curveto(230.36623278,434.09033458)(230.86957682,433.7205309)(231.73244997,433.7205309)
\curveto(232.85213061,433.7205309)(233.61228265,434.40877654)(233.61228265,435.30246659)
\curveto(233.61228265,435.91880456)(233.19111733,436.1858857)(232.57477937,436.65841147)
\curveto(232.09198129,437.02821425)(231.80435614,437.2439335)(231.80435614,437.61373628)
\curveto(231.80435614,438.01435595)(232.11252584,438.30198103)(232.52341782,438.30198103)
\curveto(232.63641311,438.30198103)(232.73913643,438.27116402)(232.84185942,438.19925793)
\lineto(233.27329643,437.60346397)
}
}
{
\newrgbcolor{curcolor}{0 0 0}
\pscustom[linestyle=none,fillstyle=solid,fillcolor=curcolor]
{
\newpath
\moveto(238.31843985,439.93527827)
\lineto(238.31843985,437.17202696)
\lineto(235.60655009,437.17202696)
\lineto(235.60655009,436.58650531)
\lineto(238.31843985,436.58650531)
\lineto(238.31843985,433.81298169)
\lineto(238.91423382,433.81298169)
\lineto(238.91423382,436.58650531)
\lineto(241.61585127,436.58650531)
\lineto(241.61585127,437.17202696)
\lineto(238.91423382,437.17202696)
\lineto(238.91423382,439.93527827)
\lineto(238.31843985,439.93527827)
}
}
{
\newrgbcolor{curcolor}{0 0 0}
\pscustom[linestyle=none,fillstyle=solid,fillcolor=curcolor]
{
\newpath
\moveto(246.80320707,440.82896921)
\lineto(244.40975891,440.15099677)
\lineto(244.40975891,439.72983207)
\lineto(245.89924382,439.92500596)
\lineto(246.00196691,439.82228286)
\lineto(244.97473594,434.49095413)
\lineto(244.81037899,434.31632486)
\lineto(243.74205878,434.254691)
\lineto(243.74205878,433.81298169)
\lineto(246.81347938,433.81298169)
\lineto(246.81347938,434.24441869)
\lineto(245.91978844,434.29578024)
\lineto(245.82733765,434.44986489)
\lineto(247.0497425,440.82896921)
\lineto(246.80320707,440.82896921)
}
}
{
\newrgbcolor{curcolor}{0 0 0}
\pscustom[linestyle=none,fillstyle=solid,fillcolor=curcolor]
{
\newpath
\moveto(250.64053242,438.09568433)
\lineto(250.1664325,435.64616806)
\lineto(250.9882057,435.64616806)
\lineto(251.52551895,437.13168115)
\lineto(251.87319223,437.33712445)
\lineto(255.79241827,437.33712445)
\lineto(248.33324613,428.13958593)
\lineto(248.22262281,427.49164937)
\lineto(255.65018829,427.49164937)
\lineto(256.18750154,430.35205224)
\lineto(255.41313833,430.35205224)
\lineto(254.78100509,428.47145587)
\lineto(254.55975846,428.26601257)
\lineto(250.04000585,428.26601257)
\lineto(257.51498132,437.43194444)
\lineto(257.6572113,438.09568433)
\lineto(250.64053242,438.09568433)
}
}
{
\newrgbcolor{curcolor}{0 0 0}
\pscustom[linestyle=none,fillstyle=solid,fillcolor=curcolor]
{
\newpath
\moveto(258.42589519,438.60139092)
\lineto(258.42589519,438.01666768)
\curveto(260.00622669,437.96925773)(260.44872153,437.5109597)(260.44872153,436.12026798)
\lineto(260.44872153,434.84019819)
\curveto(260.44872153,433.18085011)(260.92282303,432.40648491)(262.50315453,432.0746153)
\curveto(261.01764292,431.74274568)(260.4329182,430.98418389)(260.4329182,429.40385239)
\lineto(260.4329182,428.09217594)
\curveto(260.4329182,426.51184444)(260.21166978,426.02193957)(258.42589519,425.99033294)
\lineto(258.42589519,425.37400304)
\lineto(258.80517513,425.37400304)
\curveto(261.06504917,425.37400304)(261.69718466,426.0851551)(261.69718466,428.26601257)
\lineto(261.69718466,429.60929569)
\curveto(261.69718466,431.17382388)(262.18708937,431.72694202)(263.64099435,431.72694202)
\lineto(264.00447096,431.72694202)
\lineto(264.00447096,432.42228857)
\curveto(263.86224112,432.40648526)(263.72001088,432.40648524)(263.59358436,432.40648524)
\curveto(262.17128601,432.40648524)(261.71298799,432.97540662)(261.71298799,434.44511492)
\lineto(261.71298799,436.05705466)
\curveto(261.71298799,437.85863257)(260.79639264,438.60139092)(258.63133849,438.60139092)
\lineto(258.42589519,438.60139092)
}
}
{
\newrgbcolor{curcolor}{0 0 0}
\pscustom[linestyle=none,fillstyle=solid,fillcolor=curcolor]
{
\newpath
\moveto(13.53018422,406.37512976)
\lineto(21.66889959,406.37512976)
\lineto(21.66889959,408.95107268)
\lineto(20.84712638,408.95107268)
\lineto(20.57846976,407.35493627)
\lineto(20.35722313,407.10208298)
\lineto(16.12193047,407.10208298)
\lineto(16.12193047,411.5270156)
\lineto(18.71367673,411.5270156)
\lineto(18.93492336,411.28996564)
\lineto(19.09295667,410.15212582)
\lineto(19.83571321,410.15212582)
\lineto(19.83571321,413.6288586)
\lineto(19.09295667,413.6288586)
\lineto(18.93492336,412.47521545)
\lineto(18.71367673,412.25396882)
\lineto(16.12193047,412.25396882)
\lineto(16.12193047,416.25221151)
\lineto(19.77249989,416.25221151)
\lineto(19.99374652,416.01516155)
\lineto(20.21499315,414.67187843)
\lineto(21.09997968,414.67187843)
\lineto(21.09997968,416.97916473)
\lineto(13.53018422,416.97916473)
\lineto(13.53018422,416.3470315)
\lineto(14.43097408,416.25221151)
\lineto(14.6996307,416.03096488)
\lineto(14.6996307,407.32332961)
\lineto(14.43097408,407.10208298)
\lineto(13.53018422,407.02306632)
\lineto(13.53018422,406.37512976)
}
}
{
\newrgbcolor{curcolor}{0 0 0}
\pscustom[linestyle=none,fillstyle=solid,fillcolor=curcolor]
{
\newpath
\moveto(22.58820897,406.37512976)
\lineto(26.04913841,406.37512976)
\lineto(26.04913841,406.94404967)
\lineto(25.29057854,407.02306632)
\lineto(25.02192191,407.27591962)
\lineto(25.02192191,412.57003544)
\curveto(25.70146446,412.91770837)(26.28618888,413.09154535)(26.79189496,413.09154535)
\curveto(27.96134027,413.09154535)(28.38803137,412.61744427)(28.38803137,411.46380228)
\lineto(28.38803137,407.27591962)
\lineto(28.11937475,407.02306632)
\lineto(27.36081487,406.94404967)
\lineto(27.36081487,406.37512976)
\lineto(30.82174432,406.37512976)
\lineto(30.82174432,406.94404967)
\lineto(29.96836445,407.02306632)
\lineto(29.69970783,407.27591962)
\lineto(29.69970783,411.7324589)
\curveto(29.69970783,413.3127904)(28.94114658,414.13456518)(27.56625817,414.13456518)
\curveto(26.88671563,414.13456518)(26.08074402,413.77108799)(25.02192191,413.18636534)
\lineto(25.02192191,414.05554853)
\lineto(24.42139534,414.05554853)
\lineto(22.58820897,413.48662862)
\lineto(22.58820897,412.99672537)
\lineto(23.71024546,412.99672537)
\lineto(23.71024546,407.27591962)
\lineto(23.44158883,407.02306632)
\lineto(22.58820897,406.94404967)
\lineto(22.58820897,406.37512976)
}
}
{
\newrgbcolor{curcolor}{0 0 0}
\pscustom[linestyle=none,fillstyle=solid,fillcolor=curcolor]
{
\newpath
\moveto(31.01706361,413.97653188)
\lineto(31.01706361,413.39180864)
\lineto(31.98106679,413.29698865)
\lineto(34.52540305,406.37512976)
\lineto(35.41038957,406.37512976)
\lineto(38.17597246,413.26538199)
\lineto(39.10836898,413.39180864)
\lineto(39.10836898,413.97653188)
\lineto(36.31117943,413.97653188)
\lineto(36.31117943,413.39180864)
\lineto(37.16455929,413.32859531)
\lineto(37.29098594,413.07574202)
\lineto(35.26815959,407.90805285)
\lineto(33.41916989,413.07574202)
\lineto(33.60880986,413.32859531)
\lineto(34.44638639,413.39180864)
\lineto(34.44638639,413.97653188)
\lineto(31.01706361,413.97653188)
}
}
{
\newrgbcolor{curcolor}{0 0 0}
\pscustom[linestyle=none,fillstyle=solid,fillcolor=curcolor]
{
\newpath
\moveto(43.15624401,414.13456518)
\curveto(41.02279648,414.13456518)(39.48987126,412.53842642)(39.48987126,410.18373248)
\curveto(39.48987126,407.84484187)(41.0385998,406.23289978)(43.15624401,406.23289978)
\curveto(45.27388822,406.23289978)(46.83842008,407.84484187)(46.83842008,410.18373248)
\curveto(46.83842008,412.5226231)(45.28969153,414.13456518)(43.15624401,414.13456518)
\moveto(43.15624401,413.45502196)
\curveto(44.43631252,413.45502196)(45.35290699,412.22236012)(45.35290699,410.18373248)
\curveto(45.35290699,408.12930153)(44.43631252,406.91244301)(43.15624401,406.91244301)
\curveto(41.89197881,406.91244301)(40.97538436,408.14510485)(40.97538436,410.18373248)
\curveto(40.97538436,412.23816343)(41.87617549,413.45502196)(43.15624401,413.45502196)
}
}
{
\newrgbcolor{curcolor}{0 0 0}
\pscustom[linestyle=none,fillstyle=solid,fillcolor=curcolor]
{
\newpath
\moveto(49.85784397,417.56388797)
\curveto(49.35213789,417.56388797)(48.94125079,417.16880418)(48.94125079,416.64729478)
\curveto(48.94125079,416.1415887)(49.35213789,415.74650493)(49.85784397,415.74650493)
\curveto(50.36355005,415.74650493)(50.77443716,416.1415887)(50.77443716,416.64729478)
\curveto(50.77443716,417.15300086)(50.36355005,417.56388797)(49.85784397,417.56388797)
\moveto(49.79463065,414.05554853)
\lineto(47.97724761,413.48662862)
\lineto(47.97724761,412.99672537)
\lineto(49.08348076,412.99672537)
\lineto(49.08348076,407.27591962)
\lineto(48.81482414,407.02306632)
\lineto(47.97724761,406.94404967)
\lineto(47.97724761,406.37512976)
\lineto(51.53299704,406.37512976)
\lineto(51.53299704,406.94404967)
\lineto(50.67961717,407.02306632)
\lineto(50.39515722,407.27591962)
\lineto(50.39515722,414.05554853)
\lineto(49.79463065,414.05554853)
}
}
{
\newrgbcolor{curcolor}{0 0 0}
\pscustom[linestyle=none,fillstyle=solid,fillcolor=curcolor]
{
}
}
{
\newrgbcolor{curcolor}{0 0 0}
\pscustom[linestyle=none,fillstyle=solid,fillcolor=curcolor]
{
\newpath
\moveto(56.66340023,406.37512976)
\lineto(60.36137963,406.37512976)
\lineto(60.36137963,406.94404967)
\lineto(59.46058978,407.02306632)
\lineto(59.17612982,407.27591962)
\lineto(59.17612982,418.00638123)
\lineto(58.57560325,418.00638123)
\lineto(56.64759689,417.43746132)
\lineto(56.64759689,416.94755807)
\lineto(57.86445337,416.94755807)
\lineto(57.86445337,407.27591962)
\lineto(57.59579674,407.02306632)
\lineto(56.66340023,406.94404967)
\lineto(56.66340023,406.37512976)
}
}
{
\newrgbcolor{curcolor}{0 0 0}
\pscustom[linestyle=none,fillstyle=solid,fillcolor=curcolor]
{
\newpath
\moveto(67.74548595,408.05028283)
\curveto(66.84469699,407.44975686)(66.05452876,407.13368964)(65.26436301,407.13368964)
\curveto(63.6682282,407.13368964)(62.60940344,408.3347448)(62.60940344,410.35756912)
\lineto(62.60940344,410.53140576)
\lineto(67.72968262,410.53140576)
\curveto(67.74548593,410.70524223)(67.74548595,410.86327587)(67.74548595,411.02130902)
\curveto(67.74548595,412.96511676)(66.62344774,414.13456518)(64.90088641,414.13456518)
\curveto(62.7832422,414.13456518)(61.25031699,412.50681973)(61.25031699,410.12051916)
\curveto(61.25031699,407.79743186)(62.68842218,406.23289978)(64.77445976,406.23289978)
\curveto(65.7700686,406.23289978)(66.79728705,406.54896703)(67.74548595,407.18109963)
\lineto(67.74548595,408.05028283)
\moveto(66.33898951,411.19514566)
\lineto(62.68842009,411.19514566)
\curveto(62.78323998,412.53842743)(63.68403095,413.4234153)(64.6954431,413.4234153)
\curveto(65.75426521,413.4234153)(66.33898951,412.72806746)(66.33898951,411.44799895)
\lineto(66.33898951,411.19514566)
}
}
{
\newrgbcolor{curcolor}{0 0 0}
\pscustom[linestyle=none,fillstyle=solid,fillcolor=curcolor]
{
}
}
{
\newrgbcolor{curcolor}{0 0 0}
\pscustom[linestyle=none,fillstyle=solid,fillcolor=curcolor]
{
\newpath
\moveto(75.03995775,414.13456518)
\lineto(73.20677138,413.54984194)
\lineto(73.20677138,413.05993869)
\lineto(74.32880786,413.05993869)
\lineto(74.32880786,408.63500607)
\curveto(74.32880786,407.05467457)(75.0873691,406.21709645)(76.44645419,406.21709645)
\curveto(77.22081662,406.21709645)(78.05839488,406.58057374)(79.00659378,407.26011629)
\lineto(79.00659378,406.37512976)
\lineto(81.34548674,406.37512976)
\lineto(81.34548674,406.94404967)
\lineto(80.60273019,407.02306632)
\lineto(80.31827024,407.27591962)
\lineto(80.31827024,414.13456518)
\lineto(79.71774367,414.13456518)
\lineto(77.88455729,413.54984194)
\lineto(77.88455729,413.05993869)
\lineto(79.00659378,413.05993869)
\lineto(79.00659378,407.70260955)
\curveto(78.35865786,407.38654325)(77.75813006,407.22850962)(77.17340741,407.22850962)
\curveto(76.1145853,407.22850962)(75.64048432,407.76582401)(75.64048432,408.90366269)
\lineto(75.64048432,414.13456518)
\lineto(75.03995775,414.13456518)
}
}
{
\newrgbcolor{curcolor}{0 0 0}
\pscustom[linestyle=none,fillstyle=solid,fillcolor=curcolor]
{
\newpath
\moveto(82.1413326,402.93000365)
\lineto(85.85511534,402.93000365)
\lineto(85.85511534,403.49892356)
\lineto(84.84370217,403.57794021)
\lineto(84.57504554,403.8307935)
\lineto(84.57504554,406.46994975)
\curveto(85.1597682,406.37512986)(85.64967243,406.32771977)(86.04475531,406.32771977)
\curveto(88.38364593,406.32771977)(89.93237468,407.93966201)(89.93237468,410.43658578)
\curveto(89.93237468,412.64904988)(88.63649982,414.13456518)(86.89813517,414.13456518)
\curveto(86.15537937,414.13456518)(85.41262124,413.81849794)(84.57504554,413.18636534)
\lineto(84.57504554,414.05554853)
\lineto(83.97451897,414.05554853)
\lineto(82.1413326,413.48662862)
\lineto(82.1413326,412.99672537)
\lineto(83.26336909,412.99672537)
\lineto(83.26336909,403.8307935)
\lineto(82.99471246,403.57794021)
\lineto(82.1413326,403.49892356)
\lineto(82.1413326,402.93000365)
\moveto(84.57504554,407.29172295)
\lineto(84.57504554,412.6016421)
\curveto(85.17557151,412.94931503)(85.69708257,413.13895534)(86.23439528,413.13895534)
\curveto(87.60928368,413.13895534)(88.54168157,412.0485237)(88.54168157,410.23114248)
\curveto(88.54168157,408.28733473)(87.46705353,407.03886966)(85.91832866,407.03886966)
\curveto(85.5390491,407.03886966)(85.09655494,407.11788648)(84.57504554,407.29172295)
}
}
{
\newrgbcolor{curcolor}{0 0 0}
\pscustom[linestyle=none,fillstyle=solid,fillcolor=curcolor]
{
\newpath
\moveto(90.92452755,406.37512976)
\lineto(94.62250695,406.37512976)
\lineto(94.62250695,406.94404967)
\lineto(93.7217171,407.02306632)
\lineto(93.43725714,407.27591962)
\lineto(93.43725714,418.00638123)
\lineto(92.83673057,418.00638123)
\lineto(90.90872422,417.43746132)
\lineto(90.90872422,416.94755807)
\lineto(92.12558069,416.94755807)
\lineto(92.12558069,407.27591962)
\lineto(91.85692406,407.02306632)
\lineto(90.92452755,406.94404967)
\lineto(90.92452755,406.37512976)
}
}
{
\newrgbcolor{curcolor}{0 0 0}
\pscustom[linestyle=none,fillstyle=solid,fillcolor=curcolor]
{
\newpath
\moveto(102.00661327,408.05028283)
\curveto(101.10582431,407.44975686)(100.31565608,407.13368964)(99.52549033,407.13368964)
\curveto(97.92935552,407.13368964)(96.87053076,408.3347448)(96.87053076,410.35756912)
\lineto(96.87053076,410.53140576)
\lineto(101.99080994,410.53140576)
\curveto(102.00661325,410.70524223)(102.00661327,410.86327587)(102.00661327,411.02130902)
\curveto(102.00661327,412.96511676)(100.88457506,414.13456518)(99.16201373,414.13456518)
\curveto(97.04436952,414.13456518)(95.51144431,412.50681973)(95.51144431,410.12051916)
\curveto(95.51144431,407.79743186)(96.9495495,406.23289978)(99.03558708,406.23289978)
\curveto(100.03119592,406.23289978)(101.05841437,406.54896703)(102.00661327,407.18109963)
\lineto(102.00661327,408.05028283)
\moveto(100.60011683,411.19514566)
\lineto(96.94954741,411.19514566)
\curveto(97.0443673,412.53842743)(97.94515827,413.4234153)(98.95657043,413.4234153)
\curveto(100.01539253,413.4234153)(100.60011683,412.72806746)(100.60011683,411.44799895)
\lineto(100.60011683,411.19514566)
}
}
{
\newrgbcolor{curcolor}{0 0 0}
\pscustom[linestyle=none,fillstyle=solid,fillcolor=curcolor]
{
\newpath
\moveto(104.06475018,413.02833203)
\lineto(104.06475018,407.92385618)
\curveto(104.06475018,406.80182082)(104.39662144,406.32771977)(105.70829658,406.32771977)
\curveto(106.32462587,406.32771977)(107.03577711,406.43834332)(107.77853292,406.67539305)
\lineto(107.77853292,407.26011629)
\curveto(107.41505667,407.18109971)(107.05157934,407.13368964)(106.68810309,407.13368964)
\curveto(105.58187104,407.13368964)(105.37642664,407.65520076)(105.37642664,408.8562527)
\lineto(105.37642664,413.02833203)
\lineto(107.77853292,413.02833203)
\lineto(107.77853292,413.77108858)
\lineto(105.37642664,413.77108858)
\lineto(105.37642664,415.44624164)
\lineto(104.7917034,415.44624164)
\lineto(104.11216017,413.97653188)
\lineto(103.06914034,413.51823528)
\lineto(103.06914034,413.02833203)
\lineto(104.06475018,413.02833203)
}
}
{
\newrgbcolor{curcolor}{0 0 0}
\pscustom[linestyle=none,fillstyle=solid,fillcolor=curcolor]
{
}
}
{
\newrgbcolor{curcolor}{0 0 0}
\pscustom[linestyle=none,fillstyle=solid,fillcolor=curcolor]
{
\newpath
\moveto(119.58880649,417.72192128)
\curveto(118.51418107,417.6429047)(117.45535593,417.15300053)(116.55456698,416.31542483)
\curveto(115.60636808,415.44624251)(114.94262681,414.46643395)(114.51593731,413.28118532)
\curveto(113.88380471,411.51121404)(113.58354079,409.89927117)(113.58354079,408.54018608)
\curveto(113.58354079,406.027459)(114.97423591,404.28908994)(116.98125691,404.13105679)
\lineto(117.10768356,404.79479668)
\curveto(115.66958189,405.01604309)(114.87941391,406.21709852)(114.87941391,408.28733279)
\curveto(114.87941391,409.40936815)(115.13226782,411.0371144)(115.74859711,413.09154535)
\curveto(116.06466341,414.15036746)(116.53876434,415.05115915)(117.2341102,415.82552158)
\curveto(117.83463617,416.50506412)(118.62480415,416.93175486)(119.46237985,417.05818138)
\lineto(119.58880649,417.72192128)
}
}
{
\newrgbcolor{curcolor}{0 0 0}
\pscustom[linestyle=none,fillstyle=solid,fillcolor=curcolor]
{
\newpath
\moveto(124.56562106,416.18899819)
\curveto(125.05552382,416.28381808)(125.33998427,416.5998852)(125.33998427,417.01077139)
\curveto(125.33998427,417.48487084)(124.78686693,417.94316791)(124.02830781,417.94316791)
\curveto(122.55859952,417.94316791)(121.23111774,416.47345542)(120.70960834,413.75528525)
\lineto(119.5085552,413.26538199)
\lineto(119.49275187,412.74387207)
\lineto(120.55157504,412.85449539)
\lineto(119.88783514,408.31893945)
\curveto(119.58757216,406.21709856)(119.31891449,404.77899258)(118.57615869,404.00463014)
\lineto(116.93261228,404.43132007)
\curveto(116.71136587,404.27328692)(116.58493901,404.05203988)(116.58493901,403.79918684)
\curveto(116.58493901,403.29348076)(116.93261279,402.96161031)(117.43831887,402.96161031)
\curveto(118.19687799,402.96161031)(119.06606278,403.56213806)(119.90363847,404.74738669)
\curveto(120.59898433,405.72719222)(120.88344519,406.91244446)(121.08888828,408.36634944)
\lineto(121.72102152,412.85449539)
\lineto(124.0757178,412.85449539)
\lineto(124.0757178,413.56564528)
\lineto(121.84744816,413.56564528)
\curveto(122.19512109,415.62007623)(122.66922196,416.85273848)(123.26974793,417.24782135)
\lineto(124.56562106,416.18899819)
}
}
{
\newrgbcolor{curcolor}{0 0 0}
\pscustom[linestyle=none,fillstyle=solid,fillcolor=curcolor]
{
\newpath
\moveto(125.66585828,410.44616963)
\curveto(125.35768929,410.44616963)(125.11115355,410.19963389)(125.11115355,409.89146491)
\curveto(125.11115355,409.57302363)(125.35768929,409.33676019)(125.66585828,409.33676019)
\curveto(125.96375496,409.33676019)(126.220563,409.58329593)(126.220563,409.89146491)
\curveto(126.220563,410.18936159)(125.96375496,410.44616963)(125.66585828,410.44616963)
\moveto(125.46041208,408.16571688)
\lineto(123.94011025,407.83700297)
\lineto(123.94011025,407.56992292)
\lineto(124.77216733,407.41583827)
\lineto(124.16610106,404.35468998)
\curveto(124.12501186,404.12869939)(124.1044672,403.94379747)(124.1044672,403.82052987)
\curveto(124.1044672,403.3685487)(124.32018615,403.14255743)(124.76189502,403.14255743)
\curveto(125.1522424,403.14255743)(125.64531422,403.26582539)(126.21029069,403.50208827)
\lineto(126.21029069,403.82052987)
\curveto(125.89184941,403.71780688)(125.64531346,403.66644523)(125.45013977,403.66644523)
\curveto(125.13169849,403.66644523)(124.95706891,403.79998552)(124.95706891,404.06706531)
\curveto(124.95706891,404.1286991)(124.96734124,404.21087776)(124.98788584,404.32387305)
\lineto(125.72749214,408.16571688)
\lineto(125.46041208,408.16571688)
}
}
{
\newrgbcolor{curcolor}{0 0 0}
\pscustom[linestyle=none,fillstyle=solid,fillcolor=curcolor]
{
\newpath
\moveto(125.64960522,403.70436686)
\curveto(127.57760965,403.94141658)(128.88928803,405.50594804)(128.88928803,406.97565633)
\curveto(128.88928803,407.65519888)(128.52581078,408.05028283)(127.87787486,408.05028283)
\curveto(127.24574226,408.05028283)(126.75583838,407.59198559)(126.75583838,406.94404967)
\curveto(126.75583838,406.77021321)(126.80324846,406.62798291)(126.89806835,406.48575308)
\lineto(127.90948153,406.48575308)
\curveto(127.73564506,405.23729119)(127.00869041,404.4629265)(125.74442521,404.22587677)
\lineto(125.64960522,403.70436686)
}
}
{
\newrgbcolor{curcolor}{0 0 0}
\pscustom[linestyle=none,fillstyle=solid,fillcolor=curcolor]
{
\newpath
\moveto(134.48909953,418.00638123)
\lineto(132.0079766,417.54808464)
\lineto(132.0079766,417.10559137)
\lineto(133.46188303,417.02657472)
\lineto(131.32843337,406.37512976)
\lineto(132.64010983,406.37512976)
\lineto(133.14581641,409.40936928)
\lineto(134.04660627,410.23114248)
\lineto(135.38988939,407.51296958)
\curveto(135.80077558,406.6911972)(136.24326988,406.29611311)(136.87540248,406.29611311)
\curveto(137.17566547,406.29611311)(137.80779992,406.42254005)(138.72439219,406.72280304)
\lineto(138.72439219,407.1652963)
\curveto(138.40832589,407.10208304)(138.13966871,407.05467299)(137.90261899,407.05467299)
\curveto(136.81219025,407.05467299)(136.18005473,408.39795882)(134.97900279,411.116129)
\curveto(135.67434865,411.7482616)(136.33808992,412.33298596)(137.01763246,412.82288873)
\lineto(137.58655237,412.20655883)
\curveto(138.17127502,412.44360855)(138.45573556,412.77547915)(138.45573556,413.18636534)
\curveto(138.45573556,413.64466147)(138.0764551,413.99233521)(137.55494571,413.99233521)
\curveto(137.17566615,413.99233521)(136.71736862,413.72367801)(136.1642526,413.15475867)
\lineto(133.2406364,410.13632249)
\lineto(134.9315928,418.00638123)
\lineto(134.48909953,418.00638123)
}
}
{
\newrgbcolor{curcolor}{0 0 0}
\pscustom[linestyle=none,fillstyle=solid,fillcolor=curcolor]
{
\newpath
\moveto(140.52078643,413.04413536)
\lineto(141.02649302,416.56827813)
\curveto(141.04229633,416.69470465)(141.05809968,416.80532817)(141.05809968,416.88434474)
\curveto(141.05809968,417.39005082)(140.80524597,417.65870795)(140.39435978,417.65870795)
\curveto(139.98347359,417.65870795)(139.74642322,417.39005081)(139.74642322,416.86854141)
\curveto(139.74642322,416.77372152)(139.74642324,416.67890133)(139.76222655,416.56827813)
\lineto(140.26793314,413.04413536)
\lineto(140.52078643,413.04413536)
}
}
{
\newrgbcolor{curcolor}{0 0 0}
\pscustom[linestyle=none,fillstyle=solid,fillcolor=curcolor]
{
\newpath
\moveto(140.83561841,403.70436686)
\curveto(142.76362284,403.94141658)(144.07530123,405.50594804)(144.07530123,406.97565633)
\curveto(144.07530123,407.65519888)(143.71182397,408.05028283)(143.06388805,408.05028283)
\curveto(142.43175545,408.05028283)(141.94185157,407.59198559)(141.94185157,406.94404967)
\curveto(141.94185157,406.77021321)(141.98926165,406.62798291)(142.08408154,406.48575308)
\lineto(143.09549472,406.48575308)
\curveto(142.92165825,405.23729119)(142.1947036,404.4629265)(140.9304384,404.22587677)
\lineto(140.83561841,403.70436686)
}
}
{
\newrgbcolor{curcolor}{0 0 0}
\pscustom[linestyle=none,fillstyle=solid,fillcolor=curcolor]
{
\newpath
\moveto(147.58907306,414.02394187)
\lineto(145.99293665,413.50243195)
\lineto(145.99293665,413.10734868)
\lineto(147.20979312,413.0125287)
\curveto(147.47844947,412.83869223)(147.60487647,412.39619807)(147.68389304,411.66924558)
\lineto(148.07897631,407.97126617)
\curveto(148.09477963,407.78162639)(148.11058297,407.59198604)(148.11058297,407.40234626)
\curveto(148.11058297,407.0230667)(148.04736949,406.67539275)(147.88933634,406.37512976)
\lineto(148.64789622,406.37512976)
\curveto(151.58731281,409.69382591)(152.97800886,412.03272328)(152.97800886,413.12315201)
\curveto(152.97800886,413.69207135)(152.67774504,414.02394187)(152.14043233,414.02394187)
\curveto(151.87177597,414.02394187)(151.60311883,413.91331835)(151.35026579,413.70787525)
\lineto(151.65052907,411.7798689)
\curveto(151.23964288,410.68944016)(150.4336714,409.40936774)(149.23261946,407.87644619)
\lineto(148.82173286,412.20655883)
\curveto(148.69530634,413.47082403)(148.52146867,414.02394187)(147.62067972,414.02394187)
\lineto(147.58907306,414.02394187)
}
}
{
\newrgbcolor{curcolor}{0 0 0}
\pscustom[linestyle=none,fillstyle=solid,fillcolor=curcolor]
{
\newpath
\moveto(154.84477731,417.72192128)
\lineto(154.71835067,417.05818138)
\curveto(156.1248457,416.85273829)(156.93081698,415.65168295)(156.93081698,413.66046526)
\curveto(156.93081698,412.66485642)(156.72537312,411.10032365)(156.1722571,409.07749933)
\curveto(155.84038748,407.86064407)(155.33467982,406.84922886)(154.59192402,406.02745648)
\curveto(153.99139805,405.34791394)(153.20123007,404.90541989)(152.36365438,404.79479668)
\lineto(152.23722773,404.13105679)
\curveto(153.28024652,404.19427005)(154.33907166,404.68417422)(155.27146724,405.53755323)
\curveto(157.1204551,407.24431125)(158.2266901,410.04150587)(158.2266901,413.40761197)
\curveto(158.2266901,415.82551916)(156.85179832,417.54808481)(154.84477731,417.72192128)
}
}
{
\newrgbcolor{curcolor}{0 0 0}
\pscustom[linestyle=none,fillstyle=solid,fillcolor=curcolor]
{
}
}
{
\newrgbcolor{curcolor}{0 0 0}
\pscustom[linestyle=none,fillstyle=solid,fillcolor=curcolor]
{
\newpath
\moveto(169.57767938,407.29172295)
\lineto(169.78312268,406.37512976)
\lineto(172.01139232,406.37512976)
\lineto(172.01139232,406.94404967)
\lineto(171.17381579,407.02306632)
\lineto(170.88935583,407.27591962)
\lineto(170.88935583,411.85888555)
\curveto(170.88935583,413.45502036)(170.19400762,414.13456518)(168.53465954,414.13456518)
\curveto(166.54344185,414.13456518)(165.34238672,413.31279102)(165.34238672,412.3487888)
\curveto(165.34238672,411.96950924)(165.46881373,411.85888555)(165.83228998,411.85888555)
\lineto(166.9227198,411.85888555)
\lineto(166.9227198,413.18636534)
\curveto(167.34940931,413.3602018)(167.72869007,413.43921863)(168.12377294,413.43921863)
\curveto(169.26161162,413.43921863)(169.57767938,412.91770746)(169.57767938,411.66924558)
\lineto(169.57767938,411.16353899)
\curveto(166.41701638,410.40497987)(164.96310678,409.93087755)(164.96310678,408.28733279)
\curveto(164.96310678,407.10208416)(165.73747124,406.26450645)(166.98593313,406.26450645)
\curveto(167.74449225,406.26450645)(168.58207053,406.6121804)(169.57767938,407.29172295)
\moveto(169.57767938,407.8290362)
\curveto(168.78751363,407.37074006)(168.13957575,407.11788631)(167.61806636,407.11788631)
\curveto(166.82790061,407.11788631)(166.35379989,407.639397)(166.35379989,408.41375943)
\curveto(166.35379989,409.56740143)(167.34941196,409.99409313)(169.57767938,410.61042242)
\lineto(169.57767938,407.8290362)
}
}
{
\newrgbcolor{curcolor}{0 0 0}
\pscustom[linestyle=none,fillstyle=solid,fillcolor=curcolor]
{
\newpath
\moveto(174.56856879,414.13456518)
\lineto(172.73538241,413.54984194)
\lineto(172.73538241,413.05993869)
\lineto(173.8574189,413.05993869)
\lineto(173.8574189,408.63500607)
\curveto(173.8574189,407.05467457)(174.61598014,406.21709645)(175.97506523,406.21709645)
\curveto(176.74942766,406.21709645)(177.58700592,406.58057374)(178.53520482,407.26011629)
\lineto(178.53520482,406.37512976)
\lineto(180.87409777,406.37512976)
\lineto(180.87409777,406.94404967)
\lineto(180.13134123,407.02306632)
\lineto(179.84688127,407.27591962)
\lineto(179.84688127,414.13456518)
\lineto(179.2463547,414.13456518)
\lineto(177.41316833,413.54984194)
\lineto(177.41316833,413.05993869)
\lineto(178.53520482,413.05993869)
\lineto(178.53520482,407.70260955)
\curveto(177.8872689,407.38654325)(177.2867411,407.22850962)(176.70201844,407.22850962)
\curveto(175.64319634,407.22850962)(175.16909536,407.76582401)(175.16909536,408.90366269)
\lineto(175.16909536,414.13456518)
\lineto(174.56856879,414.13456518)
}
}
{
\newrgbcolor{curcolor}{0 0 0}
\pscustom[linestyle=none,fillstyle=solid,fillcolor=curcolor]
{
}
}
{
\newrgbcolor{curcolor}{0 0 0}
\pscustom[linestyle=none,fillstyle=solid,fillcolor=curcolor]
{
\newpath
\moveto(191.12576785,413.78689191)
\curveto(190.58845514,414.02394163)(189.97212401,414.13456518)(189.27677815,414.13456518)
\curveto(187.47520024,414.13456518)(186.47958859,413.31279059)(186.47958859,411.92209887)
\curveto(186.47958859,410.86327677)(187.17493649,410.35756852)(188.51821827,409.75704255)
\curveto(189.56123706,409.2829431)(190.13015801,408.99848185)(190.13015801,408.17670947)
\curveto(190.13015801,407.43395367)(189.56123718,406.89663968)(188.64464491,406.89663968)
\curveto(188.18634878,406.89663968)(187.74385462,407.02306658)(187.3013618,407.27591962)
\lineto(187.11172183,408.61920274)
\lineto(186.35316195,408.61920274)
\lineto(186.35316195,406.56476973)
\curveto(186.90627797,406.35932664)(187.60162589,406.23289978)(188.40759495,406.23289978)
\curveto(190.27238612,406.23289978)(191.36281781,407.13369111)(191.36281781,408.6033994)
\curveto(191.36281781,409.18812206)(191.1415708,409.66222287)(190.76229124,409.96248585)
\curveto(190.24078185,410.38917536)(189.6402542,410.61042268)(189.08713818,410.87907904)
\curveto(188.20215254,411.30576854)(187.74385506,411.6376396)(187.74385506,412.31718214)
\curveto(187.74385506,413.02833132)(188.21795582,413.45502196)(189.05553151,413.45502196)
\curveto(189.43481107,413.45502196)(189.8140918,413.34439842)(190.22497799,413.12315201)
\lineto(190.35140464,412.00111553)
\lineto(191.12576785,412.00111553)
\lineto(191.12576785,413.78689191)
}
}
{
\newrgbcolor{curcolor}{0 0 0}
\pscustom[linestyle=none,fillstyle=solid,fillcolor=curcolor]
{
\newpath
\moveto(199.09681975,408.05028283)
\curveto(198.19603079,407.44975686)(197.40586256,407.13368964)(196.61569681,407.13368964)
\curveto(195.019562,407.13368964)(193.96073724,408.3347448)(193.96073724,410.35756912)
\lineto(193.96073724,410.53140576)
\lineto(199.08101642,410.53140576)
\curveto(199.09681973,410.70524223)(199.09681975,410.86327587)(199.09681975,411.02130902)
\curveto(199.09681975,412.96511676)(197.97478154,414.13456518)(196.2522202,414.13456518)
\curveto(194.134576,414.13456518)(192.60165079,412.50681973)(192.60165079,410.12051916)
\curveto(192.60165079,407.79743186)(194.03975598,406.23289978)(196.12579356,406.23289978)
\curveto(197.1214024,406.23289978)(198.14862085,406.54896703)(199.09681975,407.18109963)
\lineto(199.09681975,408.05028283)
\moveto(197.69032331,411.19514566)
\lineto(194.03975389,411.19514566)
\curveto(194.13457378,412.53842743)(195.03536474,413.4234153)(196.0467769,413.4234153)
\curveto(197.10559901,413.4234153)(197.69032331,412.72806746)(197.69032331,411.44799895)
\lineto(197.69032331,411.19514566)
}
}
{
\newrgbcolor{curcolor}{0 0 0}
\pscustom[linestyle=none,fillstyle=solid,fillcolor=curcolor]
{
\newpath
\moveto(200.23836347,406.37512976)
\lineto(204.0469662,406.37512976)
\lineto(204.0469662,406.94404967)
\lineto(202.94073304,407.02306632)
\lineto(202.67207642,407.27591962)
\lineto(202.67207642,411.79567223)
\curveto(203.32001233,412.55423135)(203.87313008,412.94931537)(204.39463947,412.94931537)
\curveto(204.72650909,412.94931537)(204.99516625,412.82288848)(205.20060934,412.57003544)
\lineto(205.70631593,412.57003544)
\lineto(205.70631593,413.92912188)
\curveto(205.37444631,414.07135172)(205.07418238,414.13456518)(204.75811608,414.13456518)
\curveto(204.63168956,414.13456518)(204.48945933,414.13456515)(204.36303281,414.10295852)
\curveto(203.79411347,413.64466239)(203.22519244,413.13895478)(202.67207642,412.57003544)
\lineto(202.67207642,414.05554853)
\lineto(202.07154985,414.05554853)
\lineto(200.23836347,413.48662862)
\lineto(200.23836347,412.99672537)
\lineto(201.36039996,412.99672537)
\lineto(201.36039996,407.27591962)
\lineto(201.07594,407.02306632)
\lineto(200.23836347,406.94404967)
\lineto(200.23836347,406.37512976)
}
}
{
\newrgbcolor{curcolor}{0 0 0}
\pscustom[linestyle=none,fillstyle=solid,fillcolor=curcolor]
{
\newpath
\moveto(205.73495947,413.97653188)
\lineto(205.73495947,413.39180864)
\lineto(206.69896264,413.29698865)
\lineto(209.2432989,406.37512976)
\lineto(210.12828543,406.37512976)
\lineto(212.89386832,413.26538199)
\lineto(213.82626484,413.39180864)
\lineto(213.82626484,413.97653188)
\lineto(211.02907528,413.97653188)
\lineto(211.02907528,413.39180864)
\lineto(211.88245515,413.32859531)
\lineto(212.00888179,413.07574202)
\lineto(209.98605545,407.90805285)
\lineto(208.13706575,413.07574202)
\lineto(208.32670572,413.32859531)
\lineto(209.16428225,413.39180864)
\lineto(209.16428225,413.97653188)
\lineto(205.73495947,413.97653188)
}
}
{
\newrgbcolor{curcolor}{0 0 0}
\pscustom[linestyle=none,fillstyle=solid,fillcolor=curcolor]
{
\newpath
\moveto(220.70293608,408.05028283)
\curveto(219.80214712,407.44975686)(219.01197889,407.13368964)(218.22181314,407.13368964)
\curveto(216.62567833,407.13368964)(215.56685357,408.3347448)(215.56685357,410.35756912)
\lineto(215.56685357,410.53140576)
\lineto(220.68713275,410.53140576)
\curveto(220.70293606,410.70524223)(220.70293608,410.86327587)(220.70293608,411.02130902)
\curveto(220.70293608,412.96511676)(219.58089787,414.13456518)(217.85833653,414.13456518)
\curveto(215.74069232,414.13456518)(214.20776712,412.50681973)(214.20776712,410.12051916)
\curveto(214.20776712,407.79743186)(215.64587231,406.23289978)(217.73190989,406.23289978)
\curveto(218.72751873,406.23289978)(219.75473718,406.54896703)(220.70293608,407.18109963)
\lineto(220.70293608,408.05028283)
\moveto(219.29643964,411.19514566)
\lineto(215.64587022,411.19514566)
\curveto(215.74069011,412.53842743)(216.64148107,413.4234153)(217.65289323,413.4234153)
\curveto(218.71171534,413.4234153)(219.29643964,412.72806746)(219.29643964,411.44799895)
\lineto(219.29643964,411.19514566)
}
}
{
\newrgbcolor{curcolor}{0 0 0}
\pscustom[linestyle=none,fillstyle=solid,fillcolor=curcolor]
{
\newpath
\moveto(223.61446491,414.13456518)
\lineto(221.78127853,413.54984194)
\lineto(221.78127853,413.05993869)
\lineto(222.90331502,413.05993869)
\lineto(222.90331502,408.63500607)
\curveto(222.90331502,407.05467457)(223.66187626,406.21709645)(225.02096135,406.21709645)
\curveto(225.79532378,406.21709645)(226.63290204,406.58057374)(227.58110094,407.26011629)
\lineto(227.58110094,406.37512976)
\lineto(229.9199939,406.37512976)
\lineto(229.9199939,406.94404967)
\lineto(229.17723735,407.02306632)
\lineto(228.89277739,407.27591962)
\lineto(228.89277739,414.13456518)
\lineto(228.29225082,414.13456518)
\lineto(226.45906445,413.54984194)
\lineto(226.45906445,413.05993869)
\lineto(227.58110094,413.05993869)
\lineto(227.58110094,407.70260955)
\curveto(226.93316502,407.38654325)(226.33263722,407.22850962)(225.74791456,407.22850962)
\curveto(224.68909246,407.22850962)(224.21499148,407.76582401)(224.21499148,408.90366269)
\lineto(224.21499148,414.13456518)
\lineto(223.61446491,414.13456518)
}
}
{
\newrgbcolor{curcolor}{0 0 0}
\pscustom[linestyle=none,fillstyle=solid,fillcolor=curcolor]
{
\newpath
\moveto(230.82646307,406.37512976)
\lineto(234.6350658,406.37512976)
\lineto(234.6350658,406.94404967)
\lineto(233.52883264,407.02306632)
\lineto(233.26017602,407.27591962)
\lineto(233.26017602,411.79567223)
\curveto(233.90811193,412.55423135)(234.46122968,412.94931537)(234.98273907,412.94931537)
\curveto(235.31460869,412.94931537)(235.58326585,412.82288848)(235.78870894,412.57003544)
\lineto(236.29441553,412.57003544)
\lineto(236.29441553,413.92912188)
\curveto(235.96254592,414.07135172)(235.66228198,414.13456518)(235.34621568,414.13456518)
\curveto(235.21978916,414.13456518)(235.07755893,414.13456515)(234.95113241,414.10295852)
\curveto(234.38221307,413.64466239)(233.81329204,413.13895478)(233.26017602,412.57003544)
\lineto(233.26017602,414.05554853)
\lineto(232.65964945,414.05554853)
\lineto(230.82646307,413.48662862)
\lineto(230.82646307,412.99672537)
\lineto(231.94849956,412.99672537)
\lineto(231.94849956,407.27591962)
\lineto(231.66403961,407.02306632)
\lineto(230.82646307,406.94404967)
\lineto(230.82646307,406.37512976)
}
}
{
\newrgbcolor{curcolor}{0 0 0}
\pscustom[linestyle=none,fillstyle=solid,fillcolor=curcolor]
{
}
}
{
\newrgbcolor{curcolor}{0 0 0}
\pscustom[linestyle=none,fillstyle=solid,fillcolor=curcolor]
{
\newpath
\moveto(243.96150024,417.50067464)
\curveto(243.48740079,417.50067464)(243.10812038,417.12139423)(243.10812038,416.64729478)
\curveto(243.10812038,416.15739202)(243.48740079,415.79391492)(243.96150024,415.79391492)
\curveto(244.41979637,415.79391492)(244.8148801,416.17319533)(244.8148801,416.64729478)
\curveto(244.8148801,417.10559092)(244.41979637,417.50067464)(243.96150024,417.50067464)
\moveto(243.64543362,413.99233521)
\lineto(241.30654066,413.48662862)
\lineto(241.30654066,413.07574202)
\lineto(242.58661046,412.83869206)
\lineto(241.65421394,408.12929948)
\curveto(241.59100068,407.78162655)(241.55939396,407.49716606)(241.55939396,407.30752628)
\curveto(241.55939396,406.61218042)(241.89126458,406.26450645)(242.57080713,406.26450645)
\curveto(243.1713331,406.26450645)(243.92989445,406.45414678)(244.79907677,406.81762302)
\lineto(244.79907677,407.30752628)
\curveto(244.30917401,407.14949313)(243.92989328,407.07047632)(243.62963029,407.07047632)
\curveto(243.13972753,407.07047632)(242.87107041,407.27592003)(242.87107041,407.68680622)
\curveto(242.87107041,407.78162611)(242.88687378,407.90805302)(242.91848041,408.08188949)
\lineto(244.05632022,413.99233521)
\lineto(243.64543362,413.99233521)
}
}
{
\newrgbcolor{curcolor}{0 0 0}
\pscustom[linewidth=0.66161243,linecolor=curcolor]
{
\newpath
\moveto(60.41509951,332.54052891)
\lineto(99.16668494,332.54052891)
\lineto(99.16668494,272.75912388)
\lineto(60.41509951,272.75912388)
\closepath
}
}
{
\newrgbcolor{curcolor}{0 0 0}
\pscustom[linewidth=0.66686956,linecolor=curcolor]
{
\newpath
\moveto(35.92648391,350.98139421)
\lineto(59.80023568,332.81840995)
\lineto(59.80023568,273.14448691)
\lineto(35.92648391,291.30747117)
\closepath
}
}
{
\newrgbcolor{curcolor}{0 0 0}
\pscustom[linewidth=0.65791488,linecolor=curcolor]
{
\newpath
\moveto(36.55280013,351.32924757)
\lineto(75.01099391,351.32924757)
\lineto(98.53665561,333.18346726)
\lineto(60.07846183,333.18346726)
\closepath
}
}
{
\newrgbcolor{curcolor}{0 0 0}
\pscustom[linewidth=0.66161243,linecolor=curcolor]
{
\newpath
\moveto(341.43909491,333.39339525)
\lineto(380.19068034,333.39339525)
\lineto(380.19068034,273.61199022)
\lineto(341.43909491,273.61199022)
\closepath
}
}
{
\newrgbcolor{curcolor}{0 0 0}
\pscustom[linewidth=0.66686956,linecolor=curcolor]
{
\newpath
\moveto(316.95047932,351.83426054)
\lineto(340.82423109,333.67127629)
\lineto(340.82423109,273.99735325)
\lineto(316.95047932,292.1603375)
\closepath
}
}
{
\newrgbcolor{curcolor}{0 0 0}
\pscustom[linewidth=0.65791488,linecolor=curcolor]
{
\newpath
\moveto(317.57679553,352.18211391)
\lineto(356.03498932,352.18211391)
\lineto(379.56065102,334.0363336)
\lineto(341.10245723,334.0363336)
\closepath
}
}
{
\newrgbcolor{curcolor}{0 0 0}
\pscustom[linewidth=0.66161243,linecolor=curcolor]
{
\newpath
\moveto(73.91128505,73.76724969)
\lineto(112.66287049,73.76724969)
\lineto(112.66287049,13.98584466)
\lineto(73.91128505,13.98584466)
\closepath
}
}
{
\newrgbcolor{curcolor}{0 0 0}
\pscustom[linewidth=0.66686956,linecolor=curcolor]
{
\newpath
\moveto(49.42266946,92.20811498)
\lineto(73.29642123,74.04513073)
\lineto(73.29642123,14.37120769)
\lineto(49.42266946,32.53419194)
\closepath
}
}
{
\newrgbcolor{curcolor}{0 0 0}
\pscustom[linewidth=0.65791488,linecolor=curcolor]
{
\newpath
\moveto(50.04898567,92.55596835)
\lineto(88.50717946,92.55596835)
\lineto(112.03284116,74.41018804)
\lineto(73.57464737,74.41018804)
\closepath
}
}
{
\newrgbcolor{curcolor}{0 0 0}
\pscustom[linewidth=0.66161243,linecolor=curcolor]
{
\newpath
\moveto(295.53474115,194.82643701)
\lineto(334.28632658,194.82643701)
\lineto(334.28632658,135.04503198)
\lineto(295.53474115,135.04503198)
\closepath
}
}
{
\newrgbcolor{curcolor}{0 0 0}
\pscustom[linewidth=0.66686956,linecolor=curcolor]
{
\newpath
\moveto(271.04612555,213.26730231)
\lineto(294.91987732,195.10431805)
\lineto(294.91987732,135.43039501)
\lineto(271.04612555,153.59337927)
\closepath
}
}
{
\newrgbcolor{curcolor}{0 0 0}
\pscustom[linewidth=0.65791488,linecolor=curcolor]
{
\newpath
\moveto(271.67244177,213.61515567)
\lineto(310.13063555,213.61515567)
\lineto(333.65629725,195.46937536)
\lineto(295.19810347,195.46937536)
\closepath
}
}
{
\newrgbcolor{curcolor}{0 0 0}
\pscustom[linewidth=0.79016654,linecolor=curcolor,linestyle=dashed,dash=8 8]
{
\newpath
\moveto(384.29255823,181.77870322)
\lineto(384.29255823,85.47715616)
\lineto(519.97896877,85.47715616)
\lineto(519.97896877,12.73244907)
\lineto(639.76327769,12.73244907)
\lineto(639.76327769,129.55363347)
\lineto(549.09166723,129.55363347)
\lineto(549.09166723,181.77870322)
\lineto(384.29255823,181.77870322)
\closepath
}
}
{
\newrgbcolor{curcolor}{0 0 0}
\pscustom[linestyle=none,fillstyle=solid,fillcolor=curcolor]
{
\newpath
\moveto(395.12658123,171.90306715)
\lineto(392.64545829,171.38155724)
\lineto(392.64545829,170.98647397)
\lineto(394.13097139,170.86004732)
\lineto(391.96591507,160.27181568)
\lineto(393.30919819,160.27181568)
\lineto(394.28900469,165.62914483)
\curveto(395.26881022,166.51413047)(396.10638826,166.92501795)(396.62789765,166.92501795)
\curveto(397.22842362,166.92501795)(397.48127752,166.59314733)(397.48127752,165.91360478)
\curveto(397.48127752,165.70816169)(397.43386741,165.36048774)(397.32324421,164.90219161)
\lineto(396.64370098,161.97857541)
\curveto(396.56468441,161.63090248)(396.53307767,161.34644197)(396.53307767,161.14099888)
\curveto(396.53307767,160.47725965)(396.89655507,160.1769957)(397.68672082,160.1769957)
\curveto(398.09760701,160.1769957)(398.74554482,160.33502934)(399.58312051,160.66689895)
\lineto(399.58312051,161.09358889)
\curveto(399.21964427,160.998769)(398.9193804,160.95135891)(398.69813399,160.95135891)
\curveto(398.17662459,160.95135891)(397.89216412,161.23581937)(397.89216412,161.74152545)
\curveto(397.89216412,162.0101818)(397.95537754,162.34205245)(398.05019743,162.76874195)
\lineto(398.603314,165.21825823)
\curveto(398.72974052,165.78717756)(398.82456063,166.24547512)(398.82456063,166.640558)
\curveto(398.82456063,167.41492043)(398.31885323,167.90482446)(397.49708085,167.90482446)
\curveto(396.67530847,167.90482446)(395.6954999,167.39911688)(394.46284133,166.40350804)
\lineto(394.98435125,169.1058776)
\curveto(395.20559766,170.21210965)(395.41104139,171.14450803)(395.61648448,171.90306715)
\lineto(395.12658123,171.90306715)
}
}
{
\newrgbcolor{curcolor}{0 0 0}
\pscustom[linestyle=none,fillstyle=solid,fillcolor=curcolor]
{
}
}
{
\newrgbcolor{curcolor}{0 0 0}
\pscustom[linestyle=none,fillstyle=solid,fillcolor=curcolor]
{
\newpath
\moveto(406.6491848,166.90921462)
\lineto(406.6491848,166.00842477)
\lineto(415.89413331,166.00842477)
\lineto(415.89413331,166.90921462)
\lineto(406.6491848,166.90921462)
\moveto(406.6491848,163.95399176)
\lineto(406.6491848,163.05320191)
\lineto(415.89413331,163.05320191)
\lineto(415.89413331,163.95399176)
\lineto(406.6491848,163.95399176)
}
}
{
\newrgbcolor{curcolor}{0 0 0}
\pscustom[linestyle=none,fillstyle=solid,fillcolor=curcolor]
{
}
}
{
\newrgbcolor{curcolor}{0 0 0}
\pscustom[linestyle=none,fillstyle=solid,fillcolor=curcolor]
{
\newpath
\moveto(423.57973755,170.87585065)
\lineto(423.57973755,170.27532408)
\lineto(424.60695405,170.19630743)
\lineto(424.76498736,169.95925747)
\lineto(423.07403097,161.18840887)
\lineto(422.82117767,160.96716224)
\lineto(421.8097645,160.87234226)
\lineto(421.8097645,160.27181568)
\lineto(425.64997389,160.27181568)
\lineto(425.64997389,160.87234226)
\lineto(424.63856072,160.96716224)
\lineto(424.44892075,161.18840887)
\lineto(425.27069395,165.45530819)
\lineto(430.40677646,165.45530819)
\lineto(429.60080659,161.18840887)
\lineto(429.36375662,160.96716224)
\lineto(428.33654012,160.87234226)
\lineto(428.33654012,160.27181568)
\lineto(432.16094618,160.27181568)
\lineto(432.16094618,160.87234226)
\lineto(431.149533,160.96716224)
\lineto(430.9914997,161.18840887)
\lineto(432.66665276,169.95925747)
\lineto(432.91950605,170.19630743)
\lineto(433.96252589,170.27532408)
\lineto(433.96252589,170.87585065)
\lineto(430.10651317,170.87585065)
\lineto(430.10651317,170.27532408)
\lineto(431.11792634,170.19630743)
\lineto(431.30756631,169.95925747)
\lineto(430.56480977,166.1822614)
\lineto(425.41292392,166.1822614)
\lineto(426.15568047,169.95925747)
\lineto(426.39273043,170.19630743)
\lineto(427.43575027,170.27532408)
\lineto(427.43575027,170.87585065)
\lineto(423.57973755,170.87585065)
}
}
{
\newrgbcolor{curcolor}{0 0 0}
\pscustom[linestyle=none,fillstyle=solid,fillcolor=curcolor]
{
\newpath
\moveto(439.99766976,167.88902113)
\lineto(439.68160315,167.54134785)
\curveto(439.19170038,167.77839758)(438.7017962,167.88902113)(438.25930338,167.88902113)
\curveto(436.69477519,167.88902113)(435.41470303,167.14626326)(434.60873396,165.8187848)
\curveto(434.07142125,164.93379916)(433.78696076,163.9539907)(433.78696076,162.8951686)
\curveto(433.78696076,161.15680395)(434.62453863,160.12958571)(435.96782041,160.12958571)
\curveto(436.72637953,160.12958571)(437.484941,160.57207985)(438.3225167,161.45706549)
\lineto(438.3225167,161.33063885)
\curveto(438.3225167,160.52466978)(438.62278065,160.12958571)(439.28651988,160.12958571)
\curveto(439.71320938,160.12958571)(440.36114724,160.31922604)(441.23032957,160.68270229)
\lineto(441.23032957,161.15680221)
\curveto(440.80364006,161.06198232)(440.51917951,160.9987689)(440.34534304,160.9987689)
\curveto(439.76062039,160.9987689)(439.49196318,161.22001604)(439.49196318,161.72572212)
\curveto(439.49196318,161.94696853)(439.53937328,162.31044588)(439.64999649,162.83195527)
\lineto(440.21891639,165.62914483)
\curveto(440.45596612,166.79859014)(440.66140982,167.55715152)(440.81944297,167.88902113)
\lineto(439.99766976,167.88902113)
\moveto(438.52796,167.27269123)
\lineto(439.16009323,166.30868805)
\lineto(438.33832003,162.13660872)
\curveto(437.69038411,161.44126286)(437.07405296,161.09358889)(436.473527,161.09358889)
\curveto(435.55693473,161.09358889)(435.09863722,161.72572338)(435.09863722,162.98998858)
\curveto(435.09863722,165.50271567)(436.45772551,167.27269123)(438.30671337,167.27269123)
\lineto(438.52796,167.27269123)
}
}
{
\newrgbcolor{curcolor}{0 0 0}
\pscustom[linestyle=none,fillstyle=solid,fillcolor=curcolor]
{
\newpath
\moveto(446.02491197,166.10324475)
\curveto(446.60963463,166.13485138)(446.94150516,166.4667218)(446.94150516,166.90921462)
\curveto(446.94150516,167.46233065)(446.37258435,167.88902113)(445.4717954,167.88902113)
\curveto(443.95467716,167.88902113)(442.7536225,166.79859001)(442.7536225,165.50271818)
\curveto(442.7536225,164.61773254)(443.35414997,164.19104104)(444.25493892,163.5114985)
\curveto(444.91867815,163.00579242)(445.34536875,162.6423147)(445.34536875,162.04178873)
\curveto(445.34536875,161.20421304)(444.8080546,160.69850562)(443.90726565,160.69850562)
\curveto(443.41736288,160.69850562)(443.02227887,160.88814595)(442.76942583,161.25162219)
\lineto(443.05388578,162.04178873)
\curveto(442.80103274,162.23142851)(442.59558897,162.32624869)(442.37434256,162.32624869)
\curveto(441.88443979,162.32624869)(441.55256936,162.01018158)(441.55256936,161.52027882)
\curveto(441.55256936,160.69850644)(442.32693389,160.12958571)(443.65441235,160.12958571)
\curveto(445.37697369,160.12958571)(446.54642189,161.18841025)(446.54642189,162.56329865)
\curveto(446.54642189,163.51149755)(445.89848438,163.92238583)(444.95028548,164.64933832)
\curveto(444.20752968,165.21825766)(443.76503567,165.55012874)(443.76503567,166.11904808)
\curveto(443.76503567,166.73537737)(444.23913623,167.17787124)(444.87126883,167.17787124)
\curveto(445.04510529,167.17787124)(445.20313893,167.13046114)(445.36117208,167.01983794)
\lineto(446.02491197,166.10324475)
}
}
{
\newrgbcolor{curcolor}{0 0 0}
\pscustom[linestyle=none,fillstyle=solid,fillcolor=curcolor]
{
\newpath
\moveto(451.17902016,171.90306715)
\lineto(448.69789722,171.38155724)
\lineto(448.69789722,170.98647397)
\lineto(450.18341032,170.86004732)
\lineto(448.018354,160.27181568)
\lineto(449.36163712,160.27181568)
\lineto(450.34144363,165.62914483)
\curveto(451.32124916,166.51413047)(452.15882719,166.92501795)(452.68033658,166.92501795)
\curveto(453.28086255,166.92501795)(453.53371645,166.59314733)(453.53371645,165.91360478)
\curveto(453.53371645,165.70816169)(453.48630634,165.36048774)(453.37568314,164.90219161)
\lineto(452.69613992,161.97857541)
\curveto(452.61712334,161.63090248)(452.5855166,161.34644197)(452.5855166,161.14099888)
\curveto(452.5855166,160.47725965)(452.948994,160.1769957)(453.73915975,160.1769957)
\curveto(454.15004594,160.1769957)(454.79798375,160.33502934)(455.63555944,160.66689895)
\lineto(455.63555944,161.09358889)
\curveto(455.2720832,160.998769)(454.97181933,160.95135891)(454.75057292,160.95135891)
\curveto(454.22906352,160.95135891)(453.94460305,161.23581937)(453.94460305,161.74152545)
\curveto(453.94460305,162.0101818)(454.00781647,162.34205245)(454.10263636,162.76874195)
\lineto(454.65575293,165.21825823)
\curveto(454.78217945,165.78717756)(454.87699957,166.24547512)(454.87699957,166.640558)
\curveto(454.87699957,167.41492043)(454.37129216,167.90482446)(453.54951978,167.90482446)
\curveto(452.7277474,167.90482446)(451.74793883,167.39911688)(450.51528026,166.40350804)
\lineto(451.03679018,169.1058776)
\curveto(451.25803659,170.21210965)(451.46348032,171.14450803)(451.66892341,171.90306715)
\lineto(451.17902016,171.90306715)
}
}
{
\newrgbcolor{curcolor}{0 0 0}
\pscustom[linestyle=none,fillstyle=solid,fillcolor=curcolor]
{
\newpath
\moveto(463.61278452,171.6186072)
\curveto(462.5381591,171.53959062)(461.47933396,171.04968645)(460.57854501,170.21211076)
\curveto(459.63034611,169.34292843)(458.96660484,168.36311987)(458.53991533,167.17787124)
\curveto(457.90778273,165.40789996)(457.60751882,163.79595709)(457.60751882,162.436872)
\curveto(457.60751882,159.92414492)(458.99821393,158.18577586)(461.00523494,158.02774271)
\lineto(461.13166158,158.6914826)
\curveto(459.69355992,158.91272901)(458.90339194,160.11378445)(458.90339194,162.18401871)
\curveto(458.90339194,163.30605408)(459.15624585,164.93380033)(459.77257514,166.98823127)
\curveto(460.08864144,168.04705338)(460.56274237,168.94784507)(461.25808823,169.7222075)
\curveto(461.8586142,170.40175005)(462.64878218,170.82844079)(463.48635787,170.95486731)
\lineto(463.61278452,171.6186072)
}
}
{
\newrgbcolor{curcolor}{0 0 0}
\pscustom[linestyle=none,fillstyle=solid,fillcolor=curcolor]
{
\newpath
\moveto(467.84684254,165.96101477)
\curveto(468.44736851,165.97681809)(468.84245238,166.3402952)(468.84245238,166.83019797)
\curveto(468.84245238,167.43072394)(468.24192494,167.93643112)(467.37274261,167.93643112)
\curveto(466.39293708,167.93643112)(465.34991547,167.44652698)(464.5439464,166.56154134)
\curveto(463.70637071,165.64494907)(463.23226995,164.39648362)(463.23226995,162.98998858)
\curveto(463.23226995,161.20421399)(464.18047118,160.11378238)(465.5711629,160.11378238)
\curveto(466.50355849,160.11378238)(467.54658023,160.61948999)(468.52638576,161.64670546)
\lineto(468.52638576,162.37365868)
\curveto(467.67300675,161.50447636)(466.78801865,161.06198222)(466.06106616,161.06198222)
\curveto(465.08126063,161.06198222)(464.48073308,161.83634681)(464.48073308,163.21123521)
\curveto(464.48073308,165.56592915)(465.57116418,167.16206791)(466.8512327,167.16206791)
\lineto(466.93024935,167.16206791)
\lineto(467.84684254,165.96101477)
}
}
{
\newrgbcolor{curcolor}{0 0 0}
\pscustom[linestyle=none,fillstyle=solid,fillcolor=curcolor]
{
\newpath
\moveto(468.32933798,157.60105278)
\curveto(470.25734241,157.8381025)(471.5690208,159.40263396)(471.5690208,160.87234226)
\curveto(471.5690208,161.5518848)(471.20554354,161.94696875)(470.55760763,161.94696875)
\curveto(469.92547503,161.94696875)(469.43557114,161.48867151)(469.43557114,160.84073559)
\curveto(469.43557114,160.66689913)(469.48298123,160.52466884)(469.57780112,160.382439)
\lineto(470.58921429,160.382439)
\curveto(470.41537782,159.13397712)(469.68842317,158.35961242)(468.42415797,158.1225627)
\lineto(468.32933798,157.60105278)
}
}
{
\newrgbcolor{curcolor}{0 0 0}
\pscustom[linestyle=none,fillstyle=solid,fillcolor=curcolor]
{
\newpath
\moveto(478.46470542,165.96101477)
\curveto(479.06523139,165.97681809)(479.46031526,166.3402952)(479.46031526,166.83019797)
\curveto(479.46031526,167.43072394)(478.85978782,167.93643112)(477.9906055,167.93643112)
\curveto(477.01079997,167.93643112)(475.96777835,167.44652698)(475.16180928,166.56154134)
\curveto(474.32423359,165.64494907)(473.85013283,164.39648362)(473.85013283,162.98998858)
\curveto(473.85013283,161.20421399)(474.79833407,160.11378238)(476.18902578,160.11378238)
\curveto(477.12142137,160.11378238)(478.16444311,160.61948999)(479.14424864,161.64670546)
\lineto(479.14424864,162.37365868)
\curveto(478.29086963,161.50447636)(477.40588153,161.06198222)(476.67892904,161.06198222)
\curveto(475.69912351,161.06198222)(475.09859596,161.83634681)(475.09859596,163.21123521)
\curveto(475.09859596,165.56592915)(476.18902706,167.16206791)(477.46909558,167.16206791)
\lineto(477.54811223,167.16206791)
\lineto(478.46470542,165.96101477)
}
}
{
\newrgbcolor{curcolor}{0 0 0}
\pscustom[linestyle=none,fillstyle=solid,fillcolor=curcolor]
{
\newpath
\moveto(481.91272497,164.34285556)
\curveto(481.60455599,164.34285556)(481.35802025,164.09631982)(481.35802025,163.78815083)
\curveto(481.35802025,163.46970955)(481.60455599,163.23344611)(481.91272497,163.23344611)
\curveto(482.21062165,163.23344611)(482.46742969,163.47998185)(482.46742969,163.78815083)
\curveto(482.46742969,164.08604752)(482.21062165,164.34285556)(481.91272497,164.34285556)
\moveto(481.70727878,162.0624028)
\lineto(480.18697694,161.73368889)
\lineto(480.18697694,161.46660884)
\lineto(481.01903403,161.31252419)
\lineto(480.41296775,158.2513759)
\curveto(480.37187856,158.02538531)(480.35133389,157.84048339)(480.35133389,157.7172158)
\curveto(480.35133389,157.26523462)(480.56705284,157.03924336)(481.00876172,157.03924336)
\curveto(481.39910909,157.03924336)(481.89218092,157.16251131)(482.45715738,157.3987742)
\lineto(482.45715738,157.7172158)
\curveto(482.1387161,157.6144928)(481.89218015,157.56313115)(481.69700647,157.56313115)
\curveto(481.37856518,157.56313115)(481.2039356,157.69667144)(481.2039356,157.96375123)
\curveto(481.2039356,158.02538503)(481.21420793,158.10756368)(481.23475253,158.22055897)
\lineto(481.97435883,162.0624028)
\lineto(481.70727878,162.0624028)
}
}
{
\newrgbcolor{curcolor}{0 0 0}
\pscustom[linestyle=none,fillstyle=solid,fillcolor=curcolor]
{
\newpath
\moveto(481.89647794,157.60105278)
\curveto(483.82448237,157.8381025)(485.13616076,159.40263396)(485.13616076,160.87234226)
\curveto(485.13616076,161.5518848)(484.7726835,161.94696875)(484.12474758,161.94696875)
\curveto(483.49261498,161.94696875)(483.0027111,161.48867151)(483.0027111,160.84073559)
\curveto(483.0027111,160.66689913)(483.05012118,160.52466884)(483.14494107,160.382439)
\lineto(484.15635425,160.382439)
\curveto(483.98251778,159.13397712)(483.25556313,158.35961242)(481.99129793,158.1225627)
\lineto(481.89647794,157.60105278)
}
}
{
\newrgbcolor{curcolor}{0 0 0}
\pscustom[linestyle=none,fillstyle=solid,fillcolor=curcolor]
{
\newpath
\moveto(488.64993259,167.92062779)
\lineto(487.05379618,167.39911788)
\lineto(487.05379618,167.00403461)
\lineto(488.27065265,166.90921462)
\curveto(488.539309,166.73537816)(488.665736,166.29288399)(488.74475257,165.5659315)
\lineto(489.13983584,161.8679521)
\curveto(489.15563916,161.67831232)(489.1714425,161.48867197)(489.1714425,161.29903219)
\curveto(489.1714425,160.91975263)(489.10822902,160.57207867)(488.95019587,160.27181568)
\lineto(489.70875575,160.27181568)
\curveto(492.64817234,163.59051183)(494.03886839,165.9294092)(494.03886839,167.01983794)
\curveto(494.03886839,167.58875728)(493.73860457,167.92062779)(493.20129186,167.92062779)
\curveto(492.9326355,167.92062779)(492.66397836,167.81000427)(492.41112532,167.60456118)
\lineto(492.7113886,165.67655482)
\curveto(492.30050241,164.58612608)(491.49453093,163.30605367)(490.29347899,161.77313211)
\lineto(489.88259239,166.10324475)
\curveto(489.75616587,167.36750995)(489.5823282,167.92062779)(488.68153925,167.92062779)
\lineto(488.64993259,167.92062779)
}
}
{
\newrgbcolor{curcolor}{0 0 0}
\pscustom[linestyle=none,fillstyle=solid,fillcolor=curcolor]
{
\newpath
\moveto(493.44031121,157.60105278)
\curveto(495.36831564,157.8381025)(496.67999402,159.40263396)(496.67999402,160.87234226)
\curveto(496.67999402,161.5518848)(496.31651677,161.94696875)(495.66858085,161.94696875)
\curveto(495.03644825,161.94696875)(494.54654436,161.48867151)(494.54654436,160.84073559)
\curveto(494.54654436,160.66689913)(494.59395445,160.52466884)(494.68877434,160.382439)
\lineto(495.70018751,160.382439)
\curveto(495.52635105,159.13397712)(494.79939639,158.35961242)(493.53513119,158.1225627)
\lineto(493.44031121,157.60105278)
}
}
{
\newrgbcolor{curcolor}{0 0 0}
\pscustom[linestyle=none,fillstyle=solid,fillcolor=curcolor]
{
\newpath
\moveto(500.19376586,167.92062779)
\lineto(498.59762944,167.39911788)
\lineto(498.59762944,167.00403461)
\lineto(499.81448592,166.90921462)
\curveto(500.08314227,166.73537816)(500.20956926,166.29288399)(500.28858584,165.5659315)
\lineto(500.68366911,161.8679521)
\curveto(500.69947242,161.67831232)(500.71527577,161.48867197)(500.71527577,161.29903219)
\curveto(500.71527577,160.91975263)(500.65206229,160.57207867)(500.49402914,160.27181568)
\lineto(501.25258902,160.27181568)
\curveto(504.19200561,163.59051183)(505.58270166,165.9294092)(505.58270166,167.01983794)
\curveto(505.58270166,167.58875728)(505.28243784,167.92062779)(504.74512513,167.92062779)
\curveto(504.47646877,167.92062779)(504.20781163,167.81000427)(503.95495859,167.60456118)
\lineto(504.25522187,165.67655482)
\curveto(503.84433568,164.58612608)(503.0383642,163.30605367)(501.83731226,161.77313211)
\lineto(501.42642566,166.10324475)
\curveto(501.29999914,167.36750995)(501.12616147,167.92062779)(500.22537252,167.92062779)
\lineto(500.19376586,167.92062779)
}
}
{
\newrgbcolor{curcolor}{0 0 0}
\pscustom[linestyle=none,fillstyle=solid,fillcolor=curcolor]
{
\newpath
\moveto(507.94967461,164.34285556)
\curveto(507.64150563,164.34285556)(507.39496989,164.09631982)(507.39496989,163.78815083)
\curveto(507.39496989,163.46970955)(507.64150563,163.23344611)(507.94967461,163.23344611)
\curveto(508.24757129,163.23344611)(508.50437933,163.47998185)(508.50437933,163.78815083)
\curveto(508.50437933,164.08604752)(508.24757129,164.34285556)(507.94967461,164.34285556)
\moveto(507.74422842,162.0624028)
\lineto(506.22392658,161.73368889)
\lineto(506.22392658,161.46660884)
\lineto(507.05598367,161.31252419)
\lineto(506.44991739,158.2513759)
\curveto(506.4088282,158.02538531)(506.38828353,157.84048339)(506.38828353,157.7172158)
\curveto(506.38828353,157.26523462)(506.60400248,157.03924336)(507.04571136,157.03924336)
\curveto(507.43605873,157.03924336)(507.92913056,157.16251131)(508.49410702,157.3987742)
\lineto(508.49410702,157.7172158)
\curveto(508.17566574,157.6144928)(507.9291298,157.56313115)(507.73395611,157.56313115)
\curveto(507.41551482,157.56313115)(507.24088524,157.69667144)(507.24088524,157.96375123)
\curveto(507.24088524,158.02538503)(507.25115757,158.10756368)(507.27170217,158.22055897)
\lineto(508.01130847,162.0624028)
\lineto(507.74422842,162.0624028)
}
}
{
\newrgbcolor{curcolor}{0 0 0}
\pscustom[linestyle=none,fillstyle=solid,fillcolor=curcolor]
{
\newpath
\moveto(507.93343361,157.60105278)
\curveto(509.86143804,157.8381025)(511.17311642,159.40263396)(511.17311642,160.87234226)
\curveto(511.17311642,161.5518848)(510.80963917,161.94696875)(510.16170325,161.94696875)
\curveto(509.52957065,161.94696875)(509.03966677,161.48867151)(509.03966677,160.84073559)
\curveto(509.03966677,160.66689913)(509.08707685,160.52466884)(509.18189674,160.382439)
\lineto(510.19330991,160.382439)
\curveto(510.01947345,159.13397712)(509.2925188,158.35961242)(508.0282536,158.1225627)
\lineto(507.93343361,157.60105278)
}
}
{
\newrgbcolor{curcolor}{0 0 0}
\pscustom[linestyle=none,fillstyle=solid,fillcolor=curcolor]
{
\newpath
\moveto(519.69654412,167.88902113)
\lineto(519.3804775,167.54134785)
\curveto(518.89057474,167.77839758)(518.40067055,167.88902113)(517.95817773,167.88902113)
\curveto(516.39364955,167.88902113)(515.11357738,167.14626326)(514.30760832,165.8187848)
\curveto(513.77029561,164.93379916)(513.48583512,163.9539907)(513.48583512,162.8951686)
\curveto(513.48583512,161.15680395)(514.32341299,160.12958571)(515.66669477,160.12958571)
\curveto(516.42525389,160.12958571)(517.18381536,160.57207985)(518.02139106,161.45706549)
\lineto(518.02139106,161.33063885)
\curveto(518.02139106,160.52466978)(518.321655,160.12958571)(518.98539423,160.12958571)
\curveto(519.41208374,160.12958571)(520.0600216,160.31922604)(520.92920392,160.68270229)
\lineto(520.92920392,161.15680221)
\curveto(520.50251442,161.06198232)(520.21805386,160.9987689)(520.0442174,160.9987689)
\curveto(519.45949474,160.9987689)(519.19083754,161.22001604)(519.19083754,161.72572212)
\curveto(519.19083754,161.94696853)(519.23824764,162.31044588)(519.34887084,162.83195527)
\lineto(519.91779075,165.62914483)
\curveto(520.15484048,166.79859014)(520.36028417,167.55715152)(520.51831732,167.88902113)
\lineto(519.69654412,167.88902113)
\moveto(518.22683436,167.27269123)
\lineto(518.85896759,166.30868805)
\lineto(518.03719439,162.13660872)
\curveto(517.38925847,161.44126286)(516.77292732,161.09358889)(516.17240135,161.09358889)
\curveto(515.25580908,161.09358889)(514.79751157,161.72572338)(514.79751157,162.98998858)
\curveto(514.79751157,165.50271567)(516.15659987,167.27269123)(518.00558773,167.27269123)
\lineto(518.22683436,167.27269123)
}
}
{
\newrgbcolor{curcolor}{0 0 0}
\pscustom[linestyle=none,fillstyle=solid,fillcolor=curcolor]
{
\newpath
\moveto(520.61931048,157.60105278)
\curveto(522.54731491,157.8381025)(523.8589933,159.40263396)(523.8589933,160.87234226)
\curveto(523.8589933,161.5518848)(523.49551604,161.94696875)(522.84758013,161.94696875)
\curveto(522.21544753,161.94696875)(521.72554364,161.48867151)(521.72554364,160.84073559)
\curveto(521.72554364,160.66689913)(521.77295373,160.52466884)(521.86777362,160.382439)
\lineto(522.87918679,160.382439)
\curveto(522.70535032,159.13397712)(521.97839567,158.35961242)(520.71413047,158.1225627)
\lineto(520.61931048,157.60105278)
}
}
{
\newrgbcolor{curcolor}{0 0 0}
\pscustom[linestyle=none,fillstyle=solid,fillcolor=curcolor]
{
\newpath
\moveto(529.17434484,171.90306715)
\lineto(526.77223856,171.42896723)
\lineto(526.77223856,171.03388396)
\lineto(528.14712834,170.78103067)
\lineto(526.29813864,160.82493226)
\curveto(527.05669776,160.35083281)(527.86266922,160.12958571)(528.70024492,160.12958571)
\curveto(529.90129686,160.12958571)(531.00753202,160.6510966)(531.81350108,161.63090213)
\curveto(532.61947015,162.61070766)(533.04616089,163.87497647)(533.04616089,165.23406156)
\curveto(533.04616089,166.861803)(532.17697651,167.92062779)(530.99172788,167.92062779)
\curveto(530.35959528,167.92062779)(529.58523051,167.52554373)(528.65283492,166.73537798)
\lineto(529.33237815,170.44916072)
\curveto(529.42719804,170.98647343)(529.55362492,171.46057433)(529.69585476,171.90306715)
\lineto(529.17434484,171.90306715)
\moveto(527.64142175,161.4096555)
\lineto(528.54221161,166.11904808)
\curveto(529.30077073,166.64055748)(529.86969197,166.95662461)(530.4386113,166.95662461)
\curveto(531.21297374,166.95662461)(531.75028776,166.24547351)(531.75028776,165.02861826)
\curveto(531.75028776,163.93818952)(531.37100732,162.75293786)(530.86530124,161.99437874)
\curveto(530.32798853,161.18840968)(529.71165742,160.77752227)(529.04791819,160.77752227)
\curveto(528.5264088,160.77752227)(528.06811126,160.982966)(527.64142175,161.4096555)
}
}
{
\newrgbcolor{curcolor}{0 0 0}
\pscustom[linestyle=none,fillstyle=solid,fillcolor=curcolor]
{
\newpath
\moveto(535.49271404,171.6186072)
\lineto(535.36628739,170.95486731)
\curveto(536.77278242,170.74942421)(537.5787537,169.54836887)(537.5787537,167.55715118)
\curveto(537.5787537,166.56154234)(537.37330985,164.99700957)(536.82019382,162.97418525)
\curveto(536.48832421,161.75733)(535.98261655,160.74591479)(535.23986074,159.92414241)
\curveto(534.63933477,159.24459986)(533.8491668,158.80210581)(533.0115911,158.6914826)
\lineto(532.88516445,158.02774271)
\curveto(533.92818324,158.09095597)(534.98700838,158.58086014)(535.91940397,159.43423915)
\curveto(537.76839182,161.14099717)(538.87462683,163.9381918)(538.87462683,167.30429789)
\curveto(538.87462683,169.72220509)(537.49973504,171.44477073)(535.49271404,171.6186072)
}
}
{
\newrgbcolor{curcolor}{0 0 0}
\pscustom[linestyle=none,fillstyle=solid,fillcolor=curcolor]
{
\newpath
\moveto(391.23896185,151.18926718)
\lineto(391.23896185,150.55713395)
\lineto(392.25037502,150.36749398)
\lineto(395.96415776,140.50621556)
\lineto(397.05458758,140.50621556)
\lineto(400.68935367,150.36749398)
\lineto(401.66916018,150.55713395)
\lineto(401.66916018,151.18926718)
\lineto(398.52429735,151.18926718)
\lineto(398.52429735,150.55713395)
\lineto(399.50410386,150.46231397)
\lineto(399.66213717,150.24106734)
\lineto(396.80173429,142.27618861)
\lineto(393.8307081,150.24106734)
\lineto(394.00454474,150.46231397)
\lineto(394.92113793,150.55713395)
\lineto(394.92113793,151.18926718)
\lineto(391.23896185,151.18926718)
}
}
{
\newrgbcolor{curcolor}{0 0 0}
\pscustom[linestyle=none,fillstyle=solid,fillcolor=curcolor]
{
\newpath
\moveto(408.43594889,142.26038528)
\curveto(407.53515993,141.65985931)(406.7449917,141.3437921)(405.95482595,141.3437921)
\curveto(404.35869114,141.3437921)(403.29986638,142.54484726)(403.29986638,144.56767158)
\lineto(403.29986638,144.74150822)
\lineto(408.42014555,144.74150822)
\curveto(408.43594887,144.91534468)(408.43594889,145.07337832)(408.43594889,145.23141147)
\curveto(408.43594889,147.17521922)(407.31391068,148.34466764)(405.59134934,148.34466764)
\curveto(403.47370513,148.34466764)(401.94077993,146.71692218)(401.94077993,144.33062162)
\curveto(401.94077993,142.00753431)(403.37888512,140.44300224)(405.4649227,140.44300224)
\curveto(406.46053154,140.44300224)(407.48774999,140.75906949)(408.43594889,141.39120209)
\lineto(408.43594889,142.26038528)
\moveto(407.02945244,145.40524811)
\lineto(403.37888303,145.40524811)
\curveto(403.47370292,146.74852989)(404.37449388,147.63351775)(405.38590604,147.63351775)
\curveto(406.44472815,147.63351775)(407.02945244,146.93816992)(407.02945244,145.6581014)
\lineto(407.02945244,145.40524811)
}
}
{
\newrgbcolor{curcolor}{0 0 0}
\pscustom[linestyle=none,fillstyle=solid,fillcolor=curcolor]
{
\newpath
\moveto(409.57749261,140.58523222)
\lineto(413.38609533,140.58523222)
\lineto(413.38609533,141.15415213)
\lineto(412.27986218,141.23316878)
\lineto(412.01120555,141.48602207)
\lineto(412.01120555,146.00577468)
\curveto(412.65914147,146.7643338)(413.21225921,147.15941783)(413.73376861,147.15941783)
\curveto(414.06563822,147.15941783)(414.33429538,147.03299093)(414.53973848,146.78013789)
\lineto(415.04544507,146.78013789)
\lineto(415.04544507,148.13922434)
\curveto(414.71357545,148.28145418)(414.41331152,148.34466764)(414.09724522,148.34466764)
\curveto(413.9708187,148.34466764)(413.82858847,148.34466761)(413.70216195,148.31306098)
\curveto(413.13324261,147.85476484)(412.56432158,147.34905723)(412.01120555,146.78013789)
\lineto(412.01120555,148.26565099)
\lineto(411.41067898,148.26565099)
\lineto(409.57749261,147.69673108)
\lineto(409.57749261,147.20682782)
\lineto(410.6995291,147.20682782)
\lineto(410.6995291,141.48602207)
\lineto(410.41506914,141.23316878)
\lineto(409.57749261,141.15415213)
\lineto(409.57749261,140.58523222)
}
}
{
\newrgbcolor{curcolor}{0 0 0}
\pscustom[linestyle=none,fillstyle=solid,fillcolor=curcolor]
{
\newpath
\moveto(417.5868182,151.77399042)
\curveto(417.08111212,151.77399042)(416.67022501,151.37890663)(416.67022501,150.85739724)
\curveto(416.67022501,150.35169116)(417.08111212,149.95660738)(417.5868182,149.95660738)
\curveto(418.09252428,149.95660738)(418.50341139,150.35169116)(418.50341139,150.85739724)
\curveto(418.50341139,151.36310332)(418.09252428,151.77399042)(417.5868182,151.77399042)
\moveto(417.52360488,148.26565099)
\lineto(415.70622183,147.69673108)
\lineto(415.70622183,147.20682782)
\lineto(416.81245499,147.20682782)
\lineto(416.81245499,141.48602207)
\lineto(416.54379837,141.23316878)
\lineto(415.70622183,141.15415213)
\lineto(415.70622183,140.58523222)
\lineto(419.26197126,140.58523222)
\lineto(419.26197126,141.15415213)
\lineto(418.4085914,141.23316878)
\lineto(418.12413145,141.48602207)
\lineto(418.12413145,148.26565099)
\lineto(417.52360488,148.26565099)
}
}
{
\newrgbcolor{curcolor}{0 0 0}
\pscustom[linestyle=none,fillstyle=solid,fillcolor=curcolor]
{
\newpath
\moveto(420.02522276,140.58523222)
\lineto(423.70739883,140.58523222)
\lineto(423.70739883,141.15415213)
\lineto(422.72759232,141.23316878)
\lineto(422.4589357,141.48602207)
\lineto(422.4589357,147.23843448)
\lineto(424.67140201,147.23843448)
\lineto(424.67140201,147.98119103)
\lineto(422.47473903,147.98119103)
\lineto(422.47473903,149.13483418)
\curveto(422.47473903,150.46231264)(422.82241299,151.28408749)(423.50195553,151.60015379)
\lineto(424.94005864,150.65195394)
\curveto(425.35094483,150.76257714)(425.5879952,151.06284092)(425.5879952,151.44212048)
\curveto(425.5879952,151.94782656)(425.08228789,152.34291033)(424.3553354,152.34291033)
\curveto(422.49054423,152.34291033)(421.14725924,150.77837796)(421.14725924,148.15502767)
\lineto(421.14725924,147.98119103)
\lineto(420.13584607,147.74414107)
\lineto(420.13584607,147.23843448)
\lineto(421.14725924,147.23843448)
\lineto(421.14725924,141.48602207)
\lineto(420.86279929,141.23316878)
\lineto(420.02522276,141.15415213)
\lineto(420.02522276,140.58523222)
}
}
{
\newrgbcolor{curcolor}{0 0 0}
\pscustom[linestyle=none,fillstyle=solid,fillcolor=curcolor]
{
\newpath
\moveto(427.0626435,151.77399042)
\curveto(426.55693742,151.77399042)(426.14605032,151.37890663)(426.14605032,150.85739724)
\curveto(426.14605032,150.35169116)(426.55693742,149.95660738)(427.0626435,149.95660738)
\curveto(427.56834958,149.95660738)(427.97923669,150.35169116)(427.97923669,150.85739724)
\curveto(427.97923669,151.36310332)(427.56834958,151.77399042)(427.0626435,151.77399042)
\moveto(426.99943018,148.26565099)
\lineto(425.18204714,147.69673108)
\lineto(425.18204714,147.20682782)
\lineto(426.28828029,147.20682782)
\lineto(426.28828029,141.48602207)
\lineto(426.01962367,141.23316878)
\lineto(425.18204714,141.15415213)
\lineto(425.18204714,140.58523222)
\lineto(428.73779657,140.58523222)
\lineto(428.73779657,141.15415213)
\lineto(427.88441671,141.23316878)
\lineto(427.59995675,141.48602207)
\lineto(427.59995675,148.26565099)
\lineto(426.99943018,148.26565099)
}
}
{
\newrgbcolor{curcolor}{0 0 0}
\pscustom[linestyle=none,fillstyle=solid,fillcolor=curcolor]
{
\newpath
\moveto(436.12264367,142.26038528)
\curveto(435.22185471,141.65985931)(434.43168648,141.3437921)(433.64152073,141.3437921)
\curveto(432.04538592,141.3437921)(430.98656116,142.54484726)(430.98656116,144.56767158)
\lineto(430.98656116,144.74150822)
\lineto(436.10684034,144.74150822)
\curveto(436.12264365,144.91534468)(436.12264367,145.07337832)(436.12264367,145.23141147)
\curveto(436.12264367,147.17521922)(435.00060546,148.34466764)(433.27804412,148.34466764)
\curveto(431.16039991,148.34466764)(429.62747471,146.71692218)(429.62747471,144.33062162)
\curveto(429.62747471,142.00753431)(431.0655799,140.44300224)(433.15161748,140.44300224)
\curveto(434.14722632,140.44300224)(435.17444477,140.75906949)(436.12264367,141.39120209)
\lineto(436.12264367,142.26038528)
\moveto(434.71614722,145.40524811)
\lineto(431.06557781,145.40524811)
\curveto(431.1603977,146.74852989)(432.06118866,147.63351775)(433.07260082,147.63351775)
\curveto(434.13142293,147.63351775)(434.71614722,146.93816992)(434.71614722,145.6581014)
\lineto(434.71614722,145.40524811)
}
}
{
\newrgbcolor{curcolor}{0 0 0}
\pscustom[linestyle=none,fillstyle=solid,fillcolor=curcolor]
{
\newpath
\moveto(439.30281706,148.18663433)
\curveto(438.74970104,148.18663433)(438.30720722,147.75994383)(438.30720722,147.19102449)
\curveto(438.30720722,146.63790847)(438.74970104,146.17961132)(439.30281706,146.17961132)
\curveto(439.85593309,146.17961132)(440.31423023,146.63790847)(440.31423023,147.19102449)
\curveto(440.31423023,147.75994383)(439.85593309,148.18663433)(439.30281706,148.18663433)
\moveto(439.30281706,142.45002525)
\curveto(438.74970104,142.45002525)(438.30720722,142.00753142)(438.30720722,141.43861208)
\curveto(438.30720722,140.86969274)(438.74970104,140.44300224)(439.30281706,140.44300224)
\curveto(439.85593309,140.44300224)(440.31423023,140.88549606)(440.31423023,141.43861208)
\curveto(440.31423023,142.00753142)(439.85593309,142.45002525)(439.30281706,142.45002525)
}
}
{
\newrgbcolor{curcolor}{0 0 0}
\pscustom[linestyle=none,fillstyle=solid,fillcolor=curcolor]
{
}
}
{
\newrgbcolor{curcolor}{0 0 0}
\pscustom[linestyle=none,fillstyle=solid,fillcolor=curcolor]
{
}
}
{
\newrgbcolor{curcolor}{0 0 0}
\pscustom[linestyle=none,fillstyle=solid,fillcolor=curcolor]
{
}
}
{
\newrgbcolor{curcolor}{0 0 0}
\pscustom[linestyle=none,fillstyle=solid,fillcolor=curcolor]
{
}
}
{
\newrgbcolor{curcolor}{0 0 0}
\pscustom[linestyle=none,fillstyle=solid,fillcolor=curcolor]
{
}
}
{
\newrgbcolor{curcolor}{0 0 0}
\pscustom[linestyle=none,fillstyle=solid,fillcolor=curcolor]
{
}
}
{
\newrgbcolor{curcolor}{0 0 0}
\pscustom[linestyle=none,fillstyle=solid,fillcolor=curcolor]
{
}
}
{
\newrgbcolor{curcolor}{0 0 0}
\pscustom[linestyle=none,fillstyle=solid,fillcolor=curcolor]
{
}
}
{
\newrgbcolor{curcolor}{0 0 0}
\pscustom[linestyle=none,fillstyle=solid,fillcolor=curcolor]
{
}
}
{
\newrgbcolor{curcolor}{0 0 0}
\pscustom[linestyle=none,fillstyle=solid,fillcolor=curcolor]
{
\newpath
\moveto(433.38249426,121.49480861)
\lineto(433.38249426,122.31658181)
\curveto(432.71875503,121.87408899)(432.07081711,121.65284192)(431.37547125,121.65284192)
\curveto(429.73192649,121.65284192)(428.68890501,122.80648701)(428.68890501,124.75029476)
\curveto(428.68890501,126.74151245)(429.65290955,127.95837091)(431.01199464,127.95837091)
\curveto(431.280651,127.95837091)(431.53350479,127.87935411)(431.77055452,127.73712428)
\lineto(431.77055452,126.44125115)
\curveto(432.00760424,126.283218)(432.24465468,126.20420119)(432.48170441,126.20420119)
\curveto(433.01901712,126.20420119)(433.36669093,126.53607164)(433.36669093,127.04177772)
\curveto(433.36669093,127.95836999)(432.51330955,128.59050414)(430.99619131,128.59050414)
\curveto(428.78372721,128.59050414)(427.25080191,126.99436537)(427.25080191,124.63967144)
\curveto(427.25080191,122.30078082)(428.6573005,120.7204454)(430.80655134,120.7204454)
\curveto(431.65993035,120.7204454)(432.51331194,120.98910253)(433.38249426,121.49480861)
}
}
{
\newrgbcolor{curcolor}{0 0 0}
\pscustom[linestyle=none,fillstyle=solid,fillcolor=curcolor]
{
\newpath
\moveto(434.16575335,127.15240104)
\lineto(438.1308649,127.15240104)
\lineto(438.1308649,128.66243056)
\lineto(437.60697711,128.66243056)
\lineto(437.4939817,127.78928424)
\lineto(437.33989705,127.62492728)
\lineto(435.13135047,127.62492728)
\lineto(437.97678026,131.70303424)
\lineto(437.97678026,132.09338201)
\lineto(434.26847645,132.09338201)
\lineto(434.26847645,130.66553096)
\lineto(434.75127501,130.66553096)
\lineto(434.86427041,131.4564988)
\lineto(435.00808275,131.62085576)
\lineto(436.90846005,131.62085576)
\lineto(434.16575335,127.66601652)
\lineto(434.16575335,127.15240104)
}
}
{
\newrgbcolor{curcolor}{0 0 0}
\pscustom[linestyle=none,fillstyle=solid,fillcolor=curcolor]
{
}
}
{
\newrgbcolor{curcolor}{0 0 0}
\pscustom[linestyle=none,fillstyle=solid,fillcolor=curcolor]
{
\newpath
\moveto(444.86121113,127.46846765)
\lineto(444.86121113,126.5676778)
\lineto(454.10615965,126.5676778)
\lineto(454.10615965,127.46846765)
\lineto(444.86121113,127.46846765)
\moveto(444.86121113,124.51324479)
\lineto(444.86121113,123.61245494)
\lineto(454.10615965,123.61245494)
\lineto(454.10615965,124.51324479)
\lineto(444.86121113,124.51324479)
}
}
{
\newrgbcolor{curcolor}{0 0 0}
\pscustom[linestyle=none,fillstyle=solid,fillcolor=curcolor]
{
}
}
{
\newrgbcolor{curcolor}{0 0 0}
\pscustom[linestyle=none,fillstyle=solid,fillcolor=curcolor]
{
\newpath
\moveto(465.74259659,121.7476619)
\lineto(465.94803989,120.83106872)
\lineto(468.17630953,120.83106872)
\lineto(468.17630953,121.39998863)
\lineto(467.338733,121.47900528)
\lineto(467.05427304,121.73185857)
\lineto(467.05427304,126.31482451)
\curveto(467.05427304,127.91095932)(466.35892483,128.59050414)(464.69957675,128.59050414)
\curveto(462.70835906,128.59050414)(461.50730393,127.76872997)(461.50730393,126.80472776)
\curveto(461.50730393,126.4254482)(461.63373094,126.31482451)(461.99720719,126.31482451)
\lineto(463.08763701,126.31482451)
\lineto(463.08763701,127.64230429)
\curveto(463.51432652,127.81614076)(463.89360728,127.89515759)(464.28869015,127.89515759)
\curveto(465.42652883,127.89515759)(465.74259659,127.37364642)(465.74259659,126.12518454)
\lineto(465.74259659,125.61947795)
\curveto(462.58193359,124.86091883)(461.12802399,124.3868165)(461.12802399,122.74327174)
\curveto(461.12802399,121.55802312)(461.90238845,120.7204454)(463.15085033,120.7204454)
\curveto(463.90940945,120.7204454)(464.74698774,121.06811936)(465.74259659,121.7476619)
\moveto(465.74259659,122.28497515)
\curveto(464.95243084,121.82667902)(464.30449296,121.57382527)(463.78298357,121.57382527)
\curveto(462.99281782,121.57382527)(462.5187171,122.09533596)(462.5187171,122.86969839)
\curveto(462.5187171,124.02334039)(463.51432917,124.45003209)(465.74259659,125.06636137)
\lineto(465.74259659,122.28497515)
}
}
{
\newrgbcolor{curcolor}{0 0 0}
\pscustom[linestyle=none,fillstyle=solid,fillcolor=curcolor]
{
\newpath
\moveto(474.58949871,132.17786023)
\curveto(471.77650864,131.8301873)(470.16456608,129.44387989)(470.16456608,125.335018)
\curveto(470.16456608,121.28936936)(471.8397219,118.90306204)(474.58949871,118.58699574)
\lineto(474.58949871,119.25073564)
\curveto(472.4718545,119.61421188)(471.5078492,121.60543564)(471.5078492,125.31921466)
\curveto(471.5078492,129.11201026)(472.42444455,131.13484078)(474.58949871,131.51412034)
\lineto(474.58949871,132.17786023)
}
}
{
\newrgbcolor{curcolor}{0 0 0}
\pscustom[linestyle=none,fillstyle=solid,fillcolor=curcolor]
{
\newpath
\moveto(481.81112703,121.49480861)
\lineto(481.81112703,122.31658181)
\curveto(481.1473878,121.87408899)(480.49944988,121.65284192)(479.80410402,121.65284192)
\curveto(478.16055926,121.65284192)(477.11753778,122.80648701)(477.11753778,124.75029476)
\curveto(477.11753778,126.74151245)(478.08154232,127.95837091)(479.44062741,127.95837091)
\curveto(479.70928377,127.95837091)(479.96213756,127.87935411)(480.19918729,127.73712428)
\lineto(480.19918729,126.44125115)
\curveto(480.43623701,126.283218)(480.67328745,126.20420119)(480.91033717,126.20420119)
\curveto(481.44764988,126.20420119)(481.7953237,126.53607164)(481.7953237,127.04177772)
\curveto(481.7953237,127.95836999)(480.94194232,128.59050414)(479.42482408,128.59050414)
\curveto(477.21235998,128.59050414)(475.67943468,126.99436537)(475.67943468,124.63967144)
\curveto(475.67943468,122.30078082)(477.08593327,120.7204454)(479.23518411,120.7204454)
\curveto(480.08856312,120.7204454)(480.94194471,120.98910253)(481.81112703,121.49480861)
}
}
{
\newrgbcolor{curcolor}{0 0 0}
\pscustom[linestyle=none,fillstyle=solid,fillcolor=curcolor]
{
\newpath
\moveto(483.71406788,124.94319783)
\curveto(483.3853543,124.94319783)(483.11827392,124.68638975)(483.11827392,124.34740387)
\curveto(483.11827392,124.01869028)(483.3853543,123.76188221)(483.71406788,123.76188221)
\curveto(484.04278146,123.76188221)(484.30986184,124.01869028)(484.30986184,124.34740387)
\curveto(484.30986184,124.67611745)(484.04278146,124.94319783)(483.71406788,124.94319783)
\moveto(483.67297864,122.66274507)
\lineto(482.49166303,122.29294192)
\lineto(482.49166303,121.97450032)
\lineto(483.21072471,121.97450032)
\lineto(483.21072471,118.25592421)
\lineto(483.03609544,118.09156726)
\lineto(482.49166303,118.04020571)
\lineto(482.49166303,117.67040256)
\lineto(484.80293271,117.67040256)
\lineto(484.80293271,118.04020571)
\lineto(484.24822799,118.09156726)
\lineto(484.06332641,118.25592421)
\lineto(484.06332641,122.66274507)
\lineto(483.67297864,122.66274507)
}
}
{
\newrgbcolor{curcolor}{0 0 0}
\pscustom[linestyle=none,fillstyle=solid,fillcolor=curcolor]
{
\newpath
\moveto(485.49106123,132.17786023)
\lineto(485.49106123,131.51412034)
\curveto(487.65611539,131.13484078)(488.57271074,129.11201026)(488.57271074,125.31921466)
\curveto(488.57271074,121.60543564)(487.60870544,119.61421188)(485.49106123,119.25073564)
\lineto(485.49106123,118.58699574)
\curveto(488.24083804,118.90306204)(489.91599385,121.28936936)(489.91599385,125.335018)
\curveto(489.91599385,129.44387989)(488.3040513,131.8301873)(485.49106123,132.17786023)
}
}
{
\newrgbcolor{curcolor}{0 0 0}
\pscustom[linestyle=none,fillstyle=solid,fillcolor=curcolor]
{
\newpath
\moveto(493.89718146,134.71282098)
\lineto(492.28442884,134.37383476)
\lineto(492.28442884,134.11702702)
\lineto(493.25002595,134.03484854)
\lineto(491.84271952,127.15240104)
\lineto(492.71586584,127.15240104)
\lineto(493.35274905,130.63471403)
\curveto(493.98963161,131.2099628)(494.534065,131.47704342)(494.87305088,131.47704342)
\curveto(495.26339826,131.47704342)(495.42775561,131.26132448)(495.42775561,130.8196156)
\curveto(495.42775561,130.68607571)(495.39693861,130.46008446)(495.32503251,130.16218778)
\lineto(494.88332319,128.26181049)
\curveto(494.83196169,128.0358199)(494.81141702,127.85091796)(494.81141702,127.71737807)
\curveto(494.81141702,127.2859415)(495.04768066,127.09076718)(495.56129563,127.09076718)
\curveto(495.82837542,127.09076718)(496.24954093,127.19349049)(496.7939728,127.40920878)
\lineto(496.7939728,127.68656114)
\curveto(496.55770991,127.62492735)(496.36253565,127.59411035)(496.21872345,127.59411035)
\curveto(495.87973757,127.59411035)(495.69483566,127.77901226)(495.69483566,128.10772584)
\curveto(495.69483566,128.28235493)(495.73592496,128.49807389)(495.79755876,128.77542597)
\lineto(496.1570896,130.36763397)
\curveto(496.23926799,130.73743675)(496.30090193,131.03533436)(496.30090193,131.29214185)
\curveto(496.30090193,131.79548452)(495.97218749,132.11392662)(495.43802792,132.11392662)
\curveto(494.90386835,132.11392662)(494.26698381,131.78521207)(493.46574445,131.1380572)
\lineto(493.80473067,132.89462216)
\curveto(493.94854287,133.61368312)(494.08208317,134.21975061)(494.21562306,134.71282098)
\lineto(493.89718146,134.71282098)
}
}
{
\newrgbcolor{curcolor}{0 0 0}
\pscustom[linestyle=none,fillstyle=solid,fillcolor=curcolor]
{
}
}
{
\newrgbcolor{curcolor}{0 0 0}
\pscustom[linestyle=none,fillstyle=solid,fillcolor=curcolor]
{
}
}
{
\newrgbcolor{curcolor}{0 0 0}
\pscustom[linestyle=none,fillstyle=solid,fillcolor=curcolor]
{
}
}
{
\newrgbcolor{curcolor}{0 0 0}
\pscustom[linestyle=none,fillstyle=solid,fillcolor=curcolor]
{
}
}
{
\newrgbcolor{curcolor}{0 0 0}
\pscustom[linestyle=none,fillstyle=solid,fillcolor=curcolor]
{
}
}
{
\newrgbcolor{curcolor}{0 0 0}
\pscustom[linestyle=none,fillstyle=solid,fillcolor=curcolor]
{
}
}
{
\newrgbcolor{curcolor}{0 0 0}
\pscustom[linestyle=none,fillstyle=solid,fillcolor=curcolor]
{
}
}
{
\newrgbcolor{curcolor}{0 0 0}
\pscustom[linestyle=none,fillstyle=solid,fillcolor=curcolor]
{
}
}
{
\newrgbcolor{curcolor}{0 0 0}
\pscustom[linestyle=none,fillstyle=solid,fillcolor=curcolor]
{
\newpath
\moveto(428.13578844,108.65813729)
\lineto(426.53965203,108.13662738)
\lineto(426.53965203,107.74154411)
\lineto(427.7565085,107.64672412)
\curveto(428.02516485,107.47288766)(428.15159185,107.03039349)(428.23060842,106.303441)
\lineto(428.62569169,102.6054616)
\curveto(428.64149501,102.41582182)(428.65729835,102.22618147)(428.65729835,102.03654169)
\curveto(428.65729835,101.65726213)(428.59408487,101.30958817)(428.43605172,101.00932518)
\lineto(429.1946116,101.00932518)
\curveto(432.13402819,104.32802133)(433.52472424,106.6669187)(433.52472424,107.75734744)
\curveto(433.52472424,108.32626678)(433.22446042,108.65813729)(432.68714771,108.65813729)
\curveto(432.41849135,108.65813729)(432.14983421,108.54751377)(431.89698117,108.34207068)
\lineto(432.19724445,106.41406432)
\curveto(431.78635826,105.32363558)(430.98038678,104.04356317)(429.77933484,102.51064161)
\lineto(429.36844824,106.84075425)
\curveto(429.24202172,108.10501945)(429.06818405,108.65813729)(428.1673951,108.65813729)
\lineto(428.13578844,108.65813729)
}
}
{
\newrgbcolor{curcolor}{0 0 0}
\pscustom[linestyle=none,fillstyle=solid,fillcolor=curcolor]
{
\newpath
\moveto(437.65853446,112.36408926)
\lineto(437.46336058,111.86074608)
\curveto(437.35036528,111.80938459)(437.24764198,111.77856761)(437.15519128,111.77856761)
\curveto(436.939473,111.77856761)(436.72375401,111.88129084)(436.45667422,112.01483073)
\curveto(436.17932214,112.15864292)(435.92251381,112.29218309)(435.61434483,112.29218309)
\curveto(434.90555617,112.29218309)(434.22758302,111.57312069)(434.22758302,110.85405973)
\curveto(434.22758302,110.47398465)(434.42275719,110.24799346)(434.71038157,110.24799346)
\curveto(434.91582756,110.24799346)(435.06991254,110.36098908)(435.19318013,110.57670737)
\lineto(434.64874772,111.23413519)
\curveto(434.80283221,111.44985348)(435.02882344,111.5628491)(435.29590323,111.5628491)
\curveto(435.51162152,111.5628491)(435.8197914,111.46012581)(436.19986648,111.26495212)
\lineto(437.21682514,111.49094293)
\lineto(433.6831506,107.52583139)
\lineto(434.24812764,107.2176621)
\lineto(434.49466307,107.65937142)
\curveto(434.60765836,107.72100521)(434.71038169,107.7518222)(434.82337698,107.7518222)
\curveto(435.31644735,107.7518222)(435.84033623,107.31011289)(436.4258573,107.31011289)
\curveto(437.17573515,107.31011289)(437.85370834,108.03944758)(437.85370834,108.74823624)
\curveto(437.85370834,109.11803902)(437.63798955,109.36457483)(437.35036517,109.36457483)
\curveto(437.17573608,109.36457483)(437.00110649,109.24130686)(436.8572943,108.99477168)
\lineto(437.42227134,108.36816078)
\curveto(437.25791455,108.1216256)(437.04219562,107.99835764)(436.78538813,107.99835764)
\curveto(436.59021445,107.99835764)(436.24095522,108.11135327)(435.73761255,108.33734386)
\lineto(434.7411985,108.08053611)
\lineto(438.28514535,112.03537535)
\lineto(437.65853446,112.36408926)
}
}
{
\newrgbcolor{curcolor}{0 0 0}
\pscustom[linestyle=none,fillstyle=solid,fillcolor=curcolor]
{
}
}
{
\newrgbcolor{curcolor}{0 0 0}
\pscustom[linestyle=none,fillstyle=solid,fillcolor=curcolor]
{
\newpath
\moveto(444.68373232,107.64672412)
\lineto(444.68373232,106.74593427)
\lineto(453.92868084,106.74593427)
\lineto(453.92868084,107.64672412)
\lineto(444.68373232,107.64672412)
\moveto(444.68373232,104.69150126)
\lineto(444.68373232,103.79071141)
\lineto(453.92868084,103.79071141)
\lineto(453.92868084,104.69150126)
\lineto(444.68373232,104.69150126)
}
}
{
\newrgbcolor{curcolor}{0 0 0}
\pscustom[linestyle=none,fillstyle=solid,fillcolor=curcolor]
{
}
}
{
\newrgbcolor{curcolor}{0 0 0}
\pscustom[linestyle=none,fillstyle=solid,fillcolor=curcolor]
{
\newpath
\moveto(463.55809476,112.64057665)
\lineto(461.15598848,112.16647673)
\lineto(461.15598848,111.77139346)
\lineto(462.53087826,111.51854017)
\lineto(460.68188856,101.56244176)
\curveto(461.44044768,101.08834231)(462.24641914,100.86709521)(463.08399484,100.86709521)
\curveto(464.28504678,100.86709521)(465.39128194,101.3886061)(466.197251,102.36841163)
\curveto(467.00322007,103.34821716)(467.42991081,104.61248597)(467.42991081,105.97157106)
\curveto(467.42991081,107.5993125)(466.56072643,108.65813729)(465.3754778,108.65813729)
\curveto(464.7433452,108.65813729)(463.96898043,108.26305323)(463.03658484,107.47288748)
\lineto(463.71612807,111.18667022)
\curveto(463.81094796,111.72398293)(463.93737484,112.19808383)(464.07960468,112.64057665)
\lineto(463.55809476,112.64057665)
\moveto(462.02517167,102.147165)
\lineto(462.92596153,106.85655758)
\curveto(463.68452065,107.37806698)(464.25344189,107.69413411)(464.82236123,107.69413411)
\curveto(465.59672366,107.69413411)(466.13403768,106.98298301)(466.13403768,105.76612776)
\curveto(466.13403768,104.67569902)(465.75475724,103.49044736)(465.24905116,102.73188824)
\curveto(464.71173845,101.92591918)(464.09540734,101.51503177)(463.43166811,101.51503177)
\curveto(462.91015872,101.51503177)(462.45186118,101.7204755)(462.02517167,102.147165)
}
}
{
\newrgbcolor{curcolor}{0 0 0}
\pscustom[linestyle=none,fillstyle=solid,fillcolor=curcolor]
{
\newpath
\moveto(475.56566304,112.3561167)
\curveto(474.49103762,112.27710012)(473.43221249,111.78719595)(472.53142353,110.94962026)
\curveto(471.58322463,110.08043793)(470.91948336,109.10062937)(470.49279386,107.91538074)
\curveto(469.86066126,106.14540946)(469.56039734,104.53346659)(469.56039734,103.1743815)
\curveto(469.56039734,100.66165442)(470.95109246,98.92328536)(472.95811346,98.76525221)
\lineto(473.08454011,99.4289921)
\curveto(471.64643844,99.65023851)(470.85627047,100.85129395)(470.85627047,102.92152821)
\curveto(470.85627047,104.04356358)(471.10912438,105.67130983)(471.72545366,107.72574077)
\curveto(472.04151996,108.78456288)(472.5156209,109.68535457)(473.21096676,110.459717)
\curveto(473.81149273,111.13925955)(474.6016607,111.56595029)(475.4392364,111.69237681)
\lineto(475.56566304,112.3561167)
}
}
{
\newrgbcolor{curcolor}{0 0 0}
\pscustom[linestyle=none,fillstyle=solid,fillcolor=curcolor]
{
\newpath
\moveto(476.41780827,108.65813729)
\lineto(474.82167186,108.13662738)
\lineto(474.82167186,107.74154411)
\lineto(476.03852833,107.64672412)
\curveto(476.30718469,107.47288766)(476.43361168,107.03039349)(476.51262826,106.303441)
\lineto(476.90771153,102.6054616)
\curveto(476.92351484,102.41582182)(476.93931819,102.22618147)(476.93931819,102.03654169)
\curveto(476.93931819,101.65726213)(476.87610471,101.30958817)(476.71807156,101.00932518)
\lineto(477.47663144,101.00932518)
\curveto(480.41604803,104.32802133)(481.80674408,106.6669187)(481.80674408,107.75734744)
\curveto(481.80674408,108.32626678)(481.50648025,108.65813729)(480.96916754,108.65813729)
\curveto(480.70051119,108.65813729)(480.43185404,108.54751377)(480.179001,108.34207068)
\lineto(480.47926429,106.41406432)
\curveto(480.0683781,105.32363558)(479.26240662,104.04356317)(478.06135468,102.51064161)
\lineto(477.65046807,106.84075425)
\curveto(477.52404155,108.10501945)(477.35020389,108.65813729)(476.44941493,108.65813729)
\lineto(476.41780827,108.65813729)
}
}
{
\newrgbcolor{curcolor}{0 0 0}
\pscustom[linestyle=none,fillstyle=solid,fillcolor=curcolor]
{
\newpath
\moveto(484.17371703,105.08036506)
\curveto(483.86554804,105.08036506)(483.6190123,104.83382932)(483.6190123,104.52566033)
\curveto(483.6190123,104.20721905)(483.86554804,103.97095561)(484.17371703,103.97095561)
\curveto(484.47161371,103.97095561)(484.72842175,104.21749135)(484.72842175,104.52566033)
\curveto(484.72842175,104.82355702)(484.47161371,105.08036506)(484.17371703,105.08036506)
\moveto(483.96827083,102.7999123)
\lineto(482.447969,102.47119839)
\lineto(482.447969,102.20411834)
\lineto(483.28002608,102.05003369)
\lineto(482.67395981,98.9888854)
\curveto(482.63287061,98.76289481)(482.61232595,98.57799289)(482.61232595,98.4547253)
\curveto(482.61232595,98.00274412)(482.8280449,97.77675286)(483.26975377,97.77675286)
\curveto(483.66010115,97.77675286)(484.15317297,97.90002081)(484.71814944,98.1362837)
\lineto(484.71814944,98.4547253)
\curveto(484.39970816,98.3520023)(484.15317221,98.30064065)(483.95799852,98.30064065)
\curveto(483.63955724,98.30064065)(483.46492766,98.43418094)(483.46492766,98.70126073)
\curveto(483.46492766,98.76289453)(483.47519999,98.84507318)(483.49574459,98.95806847)
\lineto(484.23535089,102.7999123)
\lineto(483.96827083,102.7999123)
}
}
{
\newrgbcolor{curcolor}{0 0 0}
\pscustom[linestyle=none,fillstyle=solid,fillcolor=curcolor]
{
\newpath
\moveto(486.62278358,112.3561167)
\lineto(486.49635693,111.69237681)
\curveto(487.90285196,111.48693371)(488.70882324,110.28587837)(488.70882324,108.29466068)
\curveto(488.70882324,107.29905184)(488.50337939,105.73451907)(487.95026336,103.71169475)
\curveto(487.61839375,102.4948395)(487.11268609,101.48342429)(486.36993028,100.66165191)
\curveto(485.76940431,99.98210936)(484.97923633,99.53961531)(484.14166064,99.4289921)
\lineto(484.01523399,98.76525221)
\curveto(485.05825278,98.82846547)(486.11707792,99.31836964)(487.04947351,100.17174865)
\curveto(488.89846136,101.87850667)(490.00469637,104.6757013)(490.00469637,108.04180739)
\curveto(490.00469637,110.45971459)(488.62980458,112.18228023)(486.62278358,112.3561167)
}
}
{
\newrgbcolor{curcolor}{0 0 0}
\pscustom[linestyle=none,fillstyle=solid,fillcolor=curcolor]
{
\newpath
\moveto(493.98514319,114.89107745)
\lineto(492.37239057,114.55209123)
\lineto(492.37239057,114.29528348)
\lineto(493.33798768,114.21310501)
\lineto(491.93068125,107.3306575)
\lineto(492.80382758,107.3306575)
\lineto(493.44071078,110.81297049)
\curveto(494.07759334,111.38821926)(494.62202673,111.65529989)(494.96101261,111.65529989)
\curveto(495.35135999,111.65529989)(495.51571734,111.43958094)(495.51571734,110.99787207)
\curveto(495.51571734,110.86433218)(495.48490034,110.63834093)(495.41299424,110.34044425)
\lineto(494.97128492,108.44006695)
\curveto(494.91992343,108.21407637)(494.89937875,108.02917443)(494.89937875,107.89563454)
\curveto(494.89937875,107.46419796)(495.13564239,107.26902365)(495.64925736,107.26902365)
\curveto(495.91633715,107.26902365)(496.33750266,107.37174696)(496.88193453,107.58746525)
\lineto(496.88193453,107.86481761)
\curveto(496.64567164,107.80318381)(496.45049738,107.77236682)(496.30668518,107.77236682)
\curveto(495.9676993,107.77236682)(495.78279739,107.95726873)(495.78279739,108.28598231)
\curveto(495.78279739,108.4606114)(495.82388669,108.67633035)(495.88552049,108.95368244)
\lineto(496.24505133,110.54589044)
\curveto(496.32722972,110.91569322)(496.38886366,111.21359083)(496.38886366,111.47039832)
\curveto(496.38886366,111.97374099)(496.06014922,112.29218309)(495.52598965,112.29218309)
\curveto(494.99183008,112.29218309)(494.35494554,111.96346853)(493.55370618,111.31631367)
\lineto(493.8926924,113.07287863)
\curveto(494.0365046,113.79193959)(494.1700449,114.39800707)(494.30358479,114.89107745)
\lineto(493.98514319,114.89107745)
}
}
{
\newrgbcolor{curcolor}{0 0 0}
\pscustom[linestyle=none,fillstyle=solid,fillcolor=curcolor]
{
\newpath
\moveto(197.04658093,255.99383178)
\curveto(197.43680844,256.06935969)(197.66339277,256.32112002)(197.66339277,256.64840761)
\curveto(197.66339277,257.02604714)(197.22281228,257.39109942)(196.61858904,257.39109942)
\curveto(195.44790651,257.39109942)(194.3905132,256.22041356)(193.97510972,254.05528028)
\lineto(193.01842197,253.66505238)
\lineto(193.00583397,253.24964849)
\lineto(193.84922975,253.33776446)
\lineto(193.32053389,249.72500939)
\curveto(193.08136219,248.0508075)(192.86736542,246.9052975)(192.27573016,246.28848627)
\lineto(190.9665785,246.62836219)
\curveto(190.79034672,246.50248235)(190.68964257,246.32625006)(190.68964257,246.12484232)
\curveto(190.68964257,245.72202682)(190.9665789,245.45767849)(191.36939439,245.45767849)
\curveto(191.97361763,245.45767849)(192.66595873,245.93602331)(193.33312189,246.88012212)
\curveto(193.88699319,247.66057714)(194.11357785,248.60467884)(194.27722164,249.76277338)
\lineto(194.78074152,253.33776446)
\lineto(196.65635303,253.33776446)
\lineto(196.65635303,253.90422432)
\lineto(194.88144549,253.90422432)
\curveto(195.15838114,255.54066226)(195.5360218,256.52252796)(196.0143652,256.83722756)
\lineto(197.04658093,255.99383178)
}
}
{
\newrgbcolor{curcolor}{0 0 0}
\pscustom[linestyle=none,fillstyle=solid,fillcolor=curcolor]
{
\newpath
\moveto(197.03109269,246.30548915)
\curveto(199.28940877,247.86012703)(200.50857564,248.8420073)(200.50857564,250.02025917)
\curveto(200.50857564,250.7893958)(199.99308902,251.24760609)(199.0766709,251.24760609)
\curveto(198.45481574,251.24760609)(197.84932358,251.01031866)(197.45657296,250.65029725)
\curveto(197.21928613,250.42937503)(197.10473351,250.1839052)(197.10473351,249.96298298)
\curveto(197.10473351,249.77478997)(197.25201538,249.63569046)(197.48930221,249.63569046)
\curveto(197.62021909,249.63569046)(197.75113635,249.67660211)(197.88205323,249.75842516)
\lineto(198.04569948,250.63393263)
\curveto(198.31571554,250.7648495)(198.56118544,250.83030814)(198.81483689,250.83030814)
\curveto(199.38759821,250.83030814)(199.70670898,250.50301506)(199.70670898,249.93843604)
\curveto(199.70670898,248.70290804)(198.45481286,247.75375694)(196.19649679,246.17457214)
\lineto(196.09830903,245.65908644)
\lineto(199.88671987,245.65908644)
\lineto(200.14037156,247.00098574)
\lineto(199.77216749,247.00098574)
\lineto(199.49396885,246.30548915)
\lineto(197.03109269,246.30548915)
}
}
{
\newrgbcolor{curcolor}{0 0 0}
\pscustom[linestyle=none,fillstyle=solid,fillcolor=curcolor]
{
\newpath
\moveto(200.11541178,246.04931434)
\curveto(201.65114585,246.2381341)(202.69595111,247.48434714)(202.69595111,248.65502967)
\curveto(202.69595111,249.19631299)(202.40642667,249.51101345)(201.89031932,249.51101345)
\curveto(201.38679995,249.51101345)(200.99657155,249.14596102)(200.99657155,248.62985367)
\curveto(200.99657155,248.49138585)(201.03433562,248.37809362)(201.10986352,248.26480177)
\lineto(201.91549531,248.26480177)
\curveto(201.77702749,247.27035102)(201.19797849,246.65353799)(200.19093976,246.46471823)
\lineto(200.11541178,246.04931434)
}
}
{
\newrgbcolor{curcolor}{0 0 0}
\pscustom[linestyle=none,fillstyle=solid,fillcolor=curcolor]
{
\newpath
\moveto(207.15642923,257.44145141)
\lineto(205.18011374,257.0763995)
\lineto(205.18011374,256.72393559)
\lineto(206.33820944,256.66099561)
\lineto(204.63882987,248.17668579)
\lineto(205.68363361,248.17668579)
\lineto(206.0864495,250.59358117)
\lineto(206.80396532,251.248157)
\lineto(207.87394504,249.08302156)
\curveto(208.20123263,248.42844638)(208.55369737,248.11374581)(209.05721674,248.11374581)
\curveto(209.29638844,248.11374581)(209.79990928,248.21445002)(210.53001236,248.45362172)
\lineto(210.53001236,248.80608563)
\curveto(210.27825268,248.75573369)(210.06425629,248.71796965)(209.87543653,248.71796965)
\curveto(209.00686562,248.71796965)(208.50334392,249.78795154)(207.54665713,251.95308482)
\curveto(208.10052843,252.45660419)(208.62922539,252.92236096)(209.17050871,253.31258847)
\lineto(209.62367659,252.8216566)
\curveto(210.08943201,253.01047636)(210.31601642,253.27482481)(210.31601642,253.60211239)
\curveto(210.31601642,253.96716394)(210.01390408,254.24410023)(209.5985006,254.24410023)
\curveto(209.29638898,254.24410023)(208.93133633,254.03010383)(208.49075688,253.5769364)
\lineto(206.16197748,251.17262902)
\lineto(207.50889314,257.44145141)
\lineto(207.15642923,257.44145141)
}
}
{
\newrgbcolor{curcolor}{0 0 0}
\pscustom[linestyle=none,fillstyle=solid,fillcolor=curcolor]
{
\newpath
\moveto(211.96091343,253.48882042)
\lineto(212.36372933,256.2959437)
\curveto(212.37631732,256.39664758)(212.38890532,256.48476372)(212.38890532,256.54770364)
\curveto(212.38890532,256.95051913)(212.18749705,257.16451548)(211.86020946,257.16451548)
\curveto(211.53292187,257.16451548)(211.34410159,256.95051912)(211.34410159,256.53511564)
\curveto(211.34410159,256.45958774)(211.34410161,256.38405959)(211.35668959,256.2959437)
\lineto(211.75950549,253.48882042)
\lineto(211.96091343,253.48882042)
}
}
{
\newrgbcolor{curcolor}{0 0 0}
\pscustom[linestyle=none,fillstyle=solid,fillcolor=curcolor]
{
\newpath
\moveto(212.21168929,246.04931434)
\curveto(213.74742336,246.2381341)(214.79222863,247.48434714)(214.79222863,248.65502967)
\curveto(214.79222863,249.19631299)(214.50270419,249.51101345)(213.98659684,249.51101345)
\curveto(213.48307747,249.51101345)(213.09284907,249.14596102)(213.09284907,248.62985367)
\curveto(213.09284907,248.49138585)(213.13061313,248.37809362)(213.20614104,248.26480177)
\lineto(214.01177283,248.26480177)
\curveto(213.87330501,247.27035102)(213.29425601,246.65353799)(212.28721727,246.46471823)
\lineto(212.21168929,246.04931434)
}
}
{
\newrgbcolor{curcolor}{0 0 0}
\pscustom[linestyle=none,fillstyle=solid,fillcolor=curcolor]
{
\newpath
\moveto(217.59109117,254.26927622)
\lineto(216.3197035,253.85387233)
\lineto(216.3197035,253.53917241)
\lineto(217.28897925,253.46364443)
\curveto(217.50297498,253.3251766)(217.60367923,252.97271198)(217.66661915,252.39366471)
\lineto(217.98131907,249.44807346)
\curveto(217.99390706,249.29701765)(218.00649507,249.14596139)(218.00649507,248.99490558)
\curveto(218.00649507,248.69279396)(217.95614295,248.41585749)(217.83026311,248.17668579)
\lineto(218.43448696,248.17668579)
\curveto(220.77585201,250.82016247)(221.88359807,252.6831895)(221.88359807,253.55176041)
\curveto(221.88359807,254.00492784)(221.6444257,254.26927622)(221.21643424,254.26927622)
\curveto(221.00243851,254.26927622)(220.78844215,254.18116008)(220.5870344,254.01751629)
\lineto(220.82620634,252.48178068)
\curveto(220.49891875,251.61320977)(219.85692963,250.59357995)(218.90024284,249.37254548)
\lineto(218.57295492,252.8216566)
\curveto(218.47225105,253.82869533)(218.33378226,254.26927622)(217.61626717,254.26927622)
\lineto(217.59109117,254.26927622)
}
}
{
\newrgbcolor{curcolor}{0 0 0}
\pscustom[linestyle=none,fillstyle=solid,fillcolor=curcolor]
{
\newpath
\moveto(101.72439861,163.63195265)
\curveto(102.11462612,163.70748055)(102.34121045,163.95924089)(102.34121045,164.28652848)
\curveto(102.34121045,164.66416801)(101.90062996,165.02922029)(101.29640672,165.02922029)
\curveto(100.12572419,165.02922029)(99.06833088,163.85853443)(98.6529274,161.69340115)
\lineto(97.69623965,161.30317325)
\lineto(97.68365165,160.88776936)
\lineto(98.52704743,160.97588533)
\lineto(97.99835157,157.36313026)
\curveto(97.75917987,155.68892837)(97.54518309,154.54341837)(96.95354784,153.92660714)
\lineto(95.64439617,154.26648306)
\curveto(95.4681644,154.14060322)(95.36746025,153.96437093)(95.36746025,153.76296319)
\curveto(95.36746025,153.36014769)(95.64439658,153.09579936)(96.04721207,153.09579936)
\curveto(96.65143531,153.09579936)(97.34377641,153.57414418)(98.01093957,154.51824299)
\curveto(98.56481087,155.29869801)(98.79139553,156.24279971)(98.95503932,157.40089425)
\lineto(99.45855919,160.97588533)
\lineto(101.33417071,160.97588533)
\lineto(101.33417071,161.54234519)
\lineto(99.55926317,161.54234519)
\curveto(99.83619882,163.17878313)(100.21383948,164.16064883)(100.69218288,164.47534843)
\lineto(101.72439861,163.63195265)
}
}
{
\newrgbcolor{curcolor}{0 0 0}
\pscustom[linestyle=none,fillstyle=solid,fillcolor=curcolor]
{
\newpath
\moveto(103.61538926,156.31648073)
\curveto(102.86261723,156.63559062)(102.50259472,156.92197234)(102.50259472,157.3801814)
\curveto(102.50259472,158.05113038)(103.03444568,158.53388751)(103.66448314,158.53388751)
\curveto(104.21269755,158.53388751)(104.53999061,158.20659446)(104.53999061,157.67474466)
\curveto(104.53999061,157.11016564)(104.19633289,156.58649679)(103.61538926,156.31648073)
\moveto(103.94268177,156.18556373)
\curveto(104.76091223,156.50467361)(105.18639332,157.05288956)(105.18639332,157.72383854)
\curveto(105.18639332,158.42751674)(104.65454212,158.89390927)(103.78721783,158.89390927)
\curveto(102.65805979,158.89390927)(101.81528044,158.14113555)(101.81528044,157.20017052)
\curveto(101.81528044,156.71741455)(102.02802104,156.39830363)(102.4944124,156.16101679)
\curveto(101.57799428,155.74371926)(101.10341922,155.138227)(101.10341922,154.45091341)
\curveto(101.10341922,153.68995908)(101.70891144,153.15810799)(102.78079335,153.15810799)
\curveto(104.06541518,153.15810799)(104.9818355,154.00088718)(104.9818355,154.97458143)
\curveto(104.9818355,155.53097815)(104.65454227,155.84190694)(103.94268177,156.18556373)
\moveto(102.77261104,155.98100591)
\curveto(103.7626699,155.53097915)(104.24542735,155.35096724)(104.24542735,154.77002361)
\curveto(104.24542735,154.10725694)(103.6890293,153.55904132)(102.91989267,153.55904132)
\curveto(102.24076138,153.55904132)(101.7907335,153.96815759)(101.7907335,154.59819504)
\curveto(101.7907335,155.19550328)(102.10984436,155.6046199)(102.77261104,155.98100591)
}
}
{
\newrgbcolor{curcolor}{0 0 0}
\pscustom[linestyle=none,fillstyle=solid,fillcolor=curcolor]
{
\newpath
\moveto(104.79322946,153.68743521)
\curveto(106.32896353,153.87625497)(107.37376879,155.12246801)(107.37376879,156.29315054)
\curveto(107.37376879,156.83443386)(107.08424435,157.14913432)(106.568137,157.14913432)
\curveto(106.06461763,157.14913432)(105.67438923,156.78408189)(105.67438923,156.26797454)
\curveto(105.67438923,156.12950672)(105.7121533,156.01621449)(105.7876812,155.90292264)
\lineto(106.59331299,155.90292264)
\curveto(106.45484517,154.90847189)(105.87579617,154.29165886)(104.86875744,154.1028391)
\lineto(104.79322946,153.68743521)
}
}
{
\newrgbcolor{curcolor}{0 0 0}
\pscustom[linestyle=none,fillstyle=solid,fillcolor=curcolor]
{
\newpath
\moveto(111.83424691,165.07957228)
\lineto(109.85793141,164.71452037)
\lineto(109.85793141,164.36205646)
\lineto(111.01602712,164.29911648)
\lineto(109.31664755,155.81480666)
\lineto(110.36145129,155.81480666)
\lineto(110.76426718,158.23170204)
\lineto(111.481783,158.88627787)
\lineto(112.55176272,156.72114243)
\curveto(112.87905031,156.06656725)(113.23151505,155.75186668)(113.73503442,155.75186668)
\curveto(113.97420612,155.75186668)(114.47772696,155.85257089)(115.20783004,156.09174259)
\lineto(115.20783004,156.4442065)
\curveto(114.95607036,156.39385456)(114.74207397,156.35609052)(114.55325421,156.35609052)
\curveto(113.6846833,156.35609052)(113.1811616,157.42607241)(112.22447481,159.59120569)
\curveto(112.77834611,160.09472506)(113.30704307,160.56048183)(113.84832639,160.95070934)
\lineto(114.30149427,160.45977747)
\curveto(114.76724969,160.64859723)(114.99383409,160.91294568)(114.99383409,161.24023326)
\curveto(114.99383409,161.60528481)(114.69172176,161.8822211)(114.27631828,161.8822211)
\curveto(113.97420666,161.8822211)(113.60915401,161.6682247)(113.16857456,161.21505727)
\lineto(110.83979516,158.81074989)
\lineto(112.18671082,165.07957228)
\lineto(111.83424691,165.07957228)
}
}
{
\newrgbcolor{curcolor}{0 0 0}
\pscustom[linestyle=none,fillstyle=solid,fillcolor=curcolor]
{
\newpath
\moveto(116.63873111,161.12694129)
\lineto(117.04154701,163.93406457)
\curveto(117.05413499,164.03476845)(117.066723,164.12288459)(117.066723,164.18582451)
\curveto(117.066723,164.58864)(116.86531473,164.80263635)(116.53802714,164.80263635)
\curveto(116.21073955,164.80263635)(116.02191927,164.58863999)(116.02191927,164.17323651)
\curveto(116.02191927,164.09770861)(116.02191929,164.02218046)(116.03450727,163.93406457)
\lineto(116.43732317,161.12694129)
\lineto(116.63873111,161.12694129)
}
}
{
\newrgbcolor{curcolor}{0 0 0}
\pscustom[linestyle=none,fillstyle=solid,fillcolor=curcolor]
{
\newpath
\moveto(116.88950697,153.68743521)
\curveto(118.42524104,153.87625497)(119.47004631,155.12246801)(119.47004631,156.29315054)
\curveto(119.47004631,156.83443386)(119.18052187,157.14913432)(118.66441452,157.14913432)
\curveto(118.16089515,157.14913432)(117.77066675,156.78408189)(117.77066675,156.26797454)
\curveto(117.77066675,156.12950672)(117.80843081,156.01621449)(117.88395872,155.90292264)
\lineto(118.68959051,155.90292264)
\curveto(118.55112268,154.90847189)(117.97207369,154.29165886)(116.96503495,154.1028391)
\lineto(116.88950697,153.68743521)
}
}
{
\newrgbcolor{curcolor}{0 0 0}
\pscustom[linestyle=none,fillstyle=solid,fillcolor=curcolor]
{
\newpath
\moveto(122.26890885,161.90739709)
\lineto(120.99752118,161.4919932)
\lineto(120.99752118,161.17729328)
\lineto(121.96679693,161.1017653)
\curveto(122.18079266,160.96329747)(122.28149691,160.61083285)(122.34443683,160.03178558)
\lineto(122.65913675,157.08619433)
\curveto(122.67172474,156.93513852)(122.68431274,156.78408226)(122.68431274,156.63302645)
\curveto(122.68431274,156.33091483)(122.63396063,156.05397836)(122.50808079,155.81480666)
\lineto(123.11230463,155.81480666)
\curveto(125.45366969,158.45828334)(126.56141575,160.32131037)(126.56141575,161.18988128)
\curveto(126.56141575,161.64304871)(126.32224338,161.90739709)(125.89425192,161.90739709)
\curveto(125.68025619,161.90739709)(125.46625983,161.81928095)(125.26485208,161.65563716)
\lineto(125.50402402,160.11990155)
\curveto(125.17673643,159.25133064)(124.53474731,158.23170082)(123.57806051,157.01066635)
\lineto(123.2507726,160.45977747)
\curveto(123.15006873,161.4668162)(123.01159994,161.90739709)(122.29408484,161.90739709)
\lineto(122.26890885,161.90739709)
}
}
{
\newrgbcolor{curcolor}{0 0 0}
\pscustom[linestyle=none,fillstyle=solid,fillcolor=curcolor]
{
\newpath
\moveto(360.36143916,314.22889256)
\lineto(353.50279359,312.17445955)
\lineto(353.50279359,310.84697977)
\lineto(358.46503946,311.73196629)
\lineto(358.46503946,294.72758235)
\lineto(358.08575952,294.1902691)
\lineto(354.45099344,294.00062913)
\lineto(354.45099344,292.64154269)
\lineto(365.03922507,292.64154269)
\lineto(365.03922507,293.96902247)
\lineto(361.62570562,294.12705578)
\lineto(361.24642568,294.6011557)
\lineto(361.24642568,314.22889256)
\lineto(360.36143916,314.22889256)
}
}
{
\newrgbcolor{curcolor}{0 0 0}
\pscustom[linestyle=none,fillstyle=solid,fillcolor=curcolor]
{
\newpath
\moveto(310.19253379,154.89008128)
\curveto(317.17759902,160.45284816)(321.00201206,164.02441046)(321.00201206,168.00684584)
\curveto(321.00201206,171.57839503)(318.50508171,173.98050488)(314.42782645,173.98050488)
\curveto(310.50860433,173.98050488)(307.72721419,171.7364297)(307.72721419,169.5239656)
\curveto(307.72721419,168.44934018)(308.39095519,167.75399255)(309.49718724,167.75399255)
\curveto(310.00289332,167.75399255)(310.47699419,167.91202617)(310.91948701,168.22809247)
\lineto(310.91948701,171.61000526)
\curveto(311.86768591,172.14731797)(312.78428099,172.36856514)(313.73247989,172.36856514)
\curveto(316.35583018,172.36856514)(317.93616588,170.59858909)(317.93616588,167.59595924)
\curveto(317.93616588,163.45549071)(313.98532626,160.01035483)(307.06347429,154.38437469)
\lineto(307.06347429,152.39315501)
\lineto(321.69735861,152.39315501)
\lineto(321.69735861,157.57664751)
\lineto(320.27505884,157.57664751)
\lineto(319.70613893,154.89008128)
\lineto(310.19253379,154.89008128)
}
}
{
\newrgbcolor{curcolor}{0 0 0}
\pscustom[linestyle=none,fillstyle=solid,fillcolor=curcolor]
{
\newpath
\moveto(94.03777455,42.40566883)
\curveto(90.59265188,43.6383274)(88.91749537,45.06063049)(88.91749537,47.14666807)
\curveto(88.91749537,49.23270565)(90.56104396,50.87625414)(92.74190142,50.87625414)
\curveto(95.04918541,50.87625414)(96.56630748,49.29591853)(96.56630748,46.76738813)
\curveto(96.56630748,45.09223674)(95.68131931,43.48029425)(94.03777455,42.40566883)
\moveto(95.36525434,41.83674892)
\curveto(97.70414496,43.06940749)(99.00002042,44.99741717)(99.00002042,47.08345475)
\curveto(99.00002042,50.14929786)(96.62951701,52.3933739)(92.83672141,52.3933739)
\curveto(88.91749929,52.3933739)(86.29414246,50.02287093)(86.29414246,46.67256815)
\curveto(86.29414246,44.39689079)(87.43198446,42.72173453)(89.61284193,41.80514226)
\curveto(86.7366386,40.69891021)(85.28272929,38.83411357)(85.28272929,36.33718981)
\curveto(85.28272929,32.76564062)(88.0325134,30.23710412)(92.58386812,30.23710412)
\curveto(96.97718969,30.23710412)(99.88500695,33.05010035)(99.88500695,36.40040313)
\curveto(99.88500695,38.86572027)(98.33627756,40.79373013)(95.36525434,41.83674892)
\moveto(90.81389507,41.14140237)
\curveto(95.04918349,39.56107087)(97.16683405,38.58126028)(97.16683405,36.08433651)
\curveto(97.16683405,33.58741274)(95.39685828,31.88065052)(92.6786881,31.88065052)
\curveto(89.64445162,31.88065052)(87.71644223,33.74544634)(87.71644223,36.52682977)
\curveto(87.71644223,38.64447398)(88.66464423,40.00356369)(90.81389507,41.14140237)
}
}
{
\newrgbcolor{curcolor}{0 0 0}
\pscustom[linestyle=none,fillstyle=solid,fillcolor=curcolor]
{
\newpath
\moveto(439.98848523,382.80799009)
\lineto(437.64959228,382.28648018)
\lineto(437.64959228,381.89139691)
\lineto(438.9454654,381.65434694)
\lineto(437.63378895,375.19078465)
\lineto(438.96126873,375.19078465)
\lineto(439.6092053,378.90456739)
\curveto(439.86205834,380.35847237)(440.36776672,381.21185428)(441.91649159,381.81238025)
\lineto(442.32737819,380.95900039)
\lineto(442.43800151,380.95900039)
\curveto(443.08593742,380.95900039)(443.49682467,381.38569092)(443.49682467,381.98621689)
\curveto(443.49682467,382.46031634)(443.14915086,382.80799009)(442.61183815,382.80799009)
\curveto(441.8216724,382.80799009)(440.81025743,382.08103544)(439.79884527,380.64293377)
\lineto(440.30455185,382.80799009)
\lineto(439.98848523,382.80799009)
}
}
{
\newrgbcolor{curcolor}{0 0 0}
\pscustom[linestyle=none,fillstyle=solid,fillcolor=curcolor]
{
}
}
{
\newrgbcolor{curcolor}{0 0 0}
\pscustom[linestyle=none,fillstyle=solid,fillcolor=curcolor]
{
\newpath
\moveto(449.26158344,381.84398691)
\lineto(449.26158344,376.73951107)
\curveto(449.26158344,375.6174757)(449.59345469,375.14337466)(450.90512984,375.14337466)
\curveto(451.52145912,375.14337466)(452.23261037,375.25399821)(452.97536617,375.49104793)
\lineto(452.97536617,376.07577117)
\curveto(452.61188993,375.9967546)(452.24841259,375.94934453)(451.88493635,375.94934453)
\curveto(450.7787043,375.94934453)(450.57325989,376.47085564)(450.57325989,377.67190758)
\lineto(450.57325989,381.84398691)
\lineto(452.97536617,381.84398691)
\lineto(452.97536617,382.58674346)
\lineto(450.57325989,382.58674346)
\lineto(450.57325989,384.26189653)
\lineto(449.98853665,384.26189653)
\lineto(449.30899343,382.79218676)
\lineto(448.2659736,382.33389017)
\lineto(448.2659736,381.84398691)
\lineto(449.26158344,381.84398691)
}
}
{
\newrgbcolor{curcolor}{0 0 0}
\pscustom[linestyle=none,fillstyle=solid,fillcolor=curcolor]
{
\newpath
\moveto(460.22884808,376.86593771)
\curveto(459.32805913,376.26541174)(458.5378909,375.94934453)(457.74772515,375.94934453)
\curveto(456.15159033,375.94934453)(455.09276557,377.15039969)(455.09276557,379.17322401)
\lineto(455.09276557,379.34706065)
\lineto(460.21304475,379.34706065)
\curveto(460.22884807,379.52089711)(460.22884808,379.67893075)(460.22884808,379.8369639)
\curveto(460.22884808,381.78077165)(459.10680987,382.95022007)(457.38424854,382.95022007)
\curveto(455.26660433,382.95022007)(453.73367913,381.32247461)(453.73367913,378.93617405)
\curveto(453.73367913,376.61308674)(455.17178431,375.04855467)(457.25782189,375.04855467)
\curveto(458.25343074,375.04855467)(459.28064918,375.36462192)(460.22884808,375.99675452)
\lineto(460.22884808,376.86593771)
\moveto(458.82235164,380.01080054)
\lineto(455.17178223,380.01080054)
\curveto(455.26660212,381.35408232)(456.16739308,382.23907018)(457.17880524,382.23907018)
\curveto(458.23762734,382.23907018)(458.82235164,381.54372235)(458.82235164,380.26365383)
\lineto(458.82235164,380.01080054)
}
}
{
\newrgbcolor{curcolor}{0 0 0}
\pscustom[linestyle=none,fillstyle=solid,fillcolor=curcolor]
{
\newpath
\moveto(461.33878515,375.19078465)
\lineto(465.03676455,375.19078465)
\lineto(465.03676455,375.75970456)
\lineto(464.1359747,375.83872121)
\lineto(463.85151474,376.0915745)
\lineto(463.85151474,386.82203612)
\lineto(463.25098817,386.82203612)
\lineto(461.32298181,386.25311621)
\lineto(461.32298181,385.76321295)
\lineto(462.53983829,385.76321295)
\lineto(462.53983829,376.0915745)
\lineto(462.27118166,375.83872121)
\lineto(461.33878515,375.75970456)
\lineto(461.33878515,375.19078465)
}
}
{
\newrgbcolor{curcolor}{0 0 0}
\pscustom[linestyle=none,fillstyle=solid,fillcolor=curcolor]
{
}
}
{
\newrgbcolor{curcolor}{0 0 0}
\pscustom[linestyle=none,fillstyle=solid,fillcolor=curcolor]
{
\newpath
\moveto(476.97766253,382.69736678)
\curveto(476.14008684,382.87120324)(475.38152546,382.9660234)(474.71778623,382.9660234)
\curveto(471.99961605,382.9660234)(470.30865694,381.35408113)(470.30865694,378.82555073)
\curveto(470.30865694,376.58148)(471.57292525,375.14337466)(473.4219131,375.14337466)
\curveto(474.08565233,375.14337466)(474.79680369,375.333015)(475.60277275,375.71229456)
\lineto(475.60277275,372.64644839)
\lineto(475.33411613,372.37779177)
\lineto(474.33850629,372.31457844)
\lineto(474.33850629,371.74565853)
\lineto(478.0364857,371.74565853)
\lineto(478.0364857,372.31457844)
\lineto(477.19890917,372.37779177)
\lineto(476.91444921,372.64644839)
\lineto(476.91444921,381.54372363)
\curveto(476.91444921,381.70175678)(476.9460559,382.08103749)(476.97766253,382.69736678)
\moveto(475.60277275,382.11264354)
\lineto(475.60277275,376.21800115)
\curveto(474.97064016,376.059968)(474.49653924,375.98095119)(474.13306299,375.98095119)
\curveto(472.64755138,375.98095119)(471.73095671,377.07138297)(471.73095671,379.03099403)
\curveto(471.73095671,381.08542498)(472.77397817,382.28648018)(474.40171961,382.28648018)
\curveto(474.73358923,382.28648018)(475.12867331,382.22326674)(475.60277275,382.11264354)
}
}
{
\newrgbcolor{curcolor}{0 0 0}
\pscustom[linestyle=none,fillstyle=solid,fillcolor=curcolor]
{
\newpath
\moveto(480.36352616,382.95022007)
\lineto(478.53033979,382.36549683)
\lineto(478.53033979,381.87559358)
\lineto(479.65237627,381.87559358)
\lineto(479.65237627,377.45066095)
\curveto(479.65237627,375.87032945)(480.41093751,375.03275134)(481.7700226,375.03275134)
\curveto(482.54438503,375.03275134)(483.38196329,375.39622863)(484.33016219,376.07577117)
\lineto(484.33016219,375.19078465)
\lineto(486.66905515,375.19078465)
\lineto(486.66905515,375.75970456)
\lineto(485.9262986,375.83872121)
\lineto(485.64183865,376.0915745)
\lineto(485.64183865,382.95022007)
\lineto(485.04131208,382.95022007)
\lineto(483.2081257,382.36549683)
\lineto(483.2081257,381.87559358)
\lineto(484.33016219,381.87559358)
\lineto(484.33016219,376.51826443)
\curveto(483.68222627,376.20219814)(483.08169847,376.04416451)(482.49697582,376.04416451)
\curveto(481.43815371,376.04416451)(480.96405273,376.5814789)(480.96405273,377.71931758)
\lineto(480.96405273,382.95022007)
\lineto(480.36352616,382.95022007)
}
}
{
\newrgbcolor{curcolor}{0 0 0}
\pscustom[linestyle=none,fillstyle=solid,fillcolor=curcolor]
{
\newpath
\moveto(494.1813166,376.86593771)
\curveto(493.28052764,376.26541174)(492.49035941,375.94934453)(491.70019366,375.94934453)
\curveto(490.10405885,375.94934453)(489.04523409,377.15039969)(489.04523409,379.17322401)
\lineto(489.04523409,379.34706065)
\lineto(494.16551327,379.34706065)
\curveto(494.18131658,379.52089711)(494.1813166,379.67893075)(494.1813166,379.8369639)
\curveto(494.1813166,381.78077165)(493.05927839,382.95022007)(491.33671706,382.95022007)
\curveto(489.21907285,382.95022007)(487.68614764,381.32247461)(487.68614764,378.93617405)
\curveto(487.68614764,376.61308674)(489.12425283,375.04855467)(491.21029041,375.04855467)
\curveto(492.20589925,375.04855467)(493.2331177,375.36462192)(494.1813166,375.99675452)
\lineto(494.1813166,376.86593771)
\moveto(492.77482016,380.01080054)
\lineto(489.12425074,380.01080054)
\curveto(489.21907063,381.35408232)(490.1198616,382.23907018)(491.13127376,382.23907018)
\curveto(492.19009586,382.23907018)(492.77482016,381.54372235)(492.77482016,380.26365383)
\lineto(492.77482016,380.01080054)
}
}
{
\newrgbcolor{curcolor}{0 0 0}
\pscustom[linestyle=none,fillstyle=solid,fillcolor=curcolor]
{
}
}
{
\newrgbcolor{curcolor}{0 0 0}
\pscustom[linestyle=none,fillstyle=solid,fillcolor=curcolor]
{
\newpath
\moveto(502.69264487,387.26452938)
\lineto(502.69264487,371.46119858)
\lineto(503.67245138,371.46119858)
\lineto(503.67245138,387.26452938)
\lineto(502.69264487,387.26452938)
}
}
{
\newrgbcolor{curcolor}{0 0 0}
\pscustom[linestyle=none,fillstyle=solid,fillcolor=curcolor]
{
\newpath
\moveto(509.93057038,382.80799009)
\lineto(507.59167742,382.28648018)
\lineto(507.59167742,381.89139691)
\lineto(508.88755054,381.65434694)
\lineto(507.57587409,375.19078465)
\lineto(508.90335388,375.19078465)
\lineto(509.55129044,378.90456739)
\curveto(509.80414348,380.35847237)(510.30985187,381.21185428)(511.85857673,381.81238025)
\lineto(512.26946334,380.95900039)
\lineto(512.38008665,380.95900039)
\curveto(513.02802257,380.95900039)(513.43890981,381.38569092)(513.43890981,381.98621689)
\curveto(513.43890981,382.46031634)(513.091236,382.80799009)(512.55392329,382.80799009)
\curveto(511.76375754,382.80799009)(510.75234257,382.08103544)(509.74093041,380.64293377)
\lineto(510.24663699,382.80799009)
\lineto(509.93057038,382.80799009)
}
}
{
\newrgbcolor{curcolor}{0 0 0}
\pscustom[linestyle=none,fillstyle=solid,fillcolor=curcolor]
{
\newpath
\moveto(516.89094989,387.26452938)
\lineto(516.89094989,371.46119858)
\lineto(517.8707564,371.46119858)
\lineto(517.8707564,387.26452938)
\lineto(516.89094989,387.26452938)
}
}
{
\newrgbcolor{curcolor}{0 0 0}
\pscustom[linestyle=none,fillstyle=solid,fillcolor=curcolor]
{
}
}
{
\newrgbcolor{curcolor}{0 0 0}
\pscustom[linestyle=none,fillstyle=solid,fillcolor=curcolor]
{
\newpath
\moveto(527.67425389,381.82818358)
\lineto(527.67425389,380.92739373)
\lineto(536.9192024,380.92739373)
\lineto(536.9192024,381.82818358)
\lineto(527.67425389,381.82818358)
\moveto(527.67425389,378.87296072)
\lineto(527.67425389,377.97217087)
\lineto(536.9192024,377.97217087)
\lineto(536.9192024,378.87296072)
\lineto(527.67425389,378.87296072)
}
}
{
\newrgbcolor{curcolor}{0 0 0}
\pscustom[linestyle=none,fillstyle=solid,fillcolor=curcolor]
{
}
}
{
\newrgbcolor{curcolor}{0 0 0}
\pscustom[linestyle=none,fillstyle=solid,fillcolor=curcolor]
{
\newpath
\moveto(550.53105569,386.53757616)
\curveto(549.45643027,386.45855959)(548.39760513,385.96865542)(547.49681618,385.13107972)
\curveto(546.54861728,384.2618974)(545.88487601,383.28208883)(545.4581865,382.09684021)
\curveto(544.8260539,380.32686893)(544.52578999,378.71492606)(544.52578999,377.35584097)
\curveto(544.52578999,374.84311388)(545.91648511,373.10474482)(547.92350611,372.94671167)
\lineto(548.04993276,373.61045157)
\curveto(546.61183109,373.83169798)(545.82166311,375.03275341)(545.82166311,377.10298767)
\curveto(545.82166311,378.22502304)(546.07451702,379.85276929)(546.69084631,381.90720024)
\curveto(547.00691261,382.96602234)(547.48101354,383.86681403)(548.1763594,384.64117647)
\curveto(548.77688537,385.32071901)(549.56705335,385.74740975)(550.40462905,385.87383627)
\lineto(550.53105569,386.53757616)
}
}
{
\newrgbcolor{curcolor}{0 0 0}
\pscustom[linestyle=none,fillstyle=solid,fillcolor=curcolor]
{
\newpath
\moveto(554.11717715,381.02221371)
\curveto(554.7018998,381.05382034)(555.03377033,381.38569076)(555.03377033,381.82818358)
\curveto(555.03377033,382.38129961)(554.46484952,382.80799009)(553.56406057,382.80799009)
\curveto(552.04694233,382.80799009)(550.84588767,381.71755897)(550.84588767,380.42168714)
\curveto(550.84588767,379.5367015)(551.44641514,379.11001001)(552.3472041,378.43046746)
\curveto(553.01094333,377.92476138)(553.43763392,377.56128367)(553.43763392,376.9607577)
\curveto(553.43763392,376.123182)(552.90031978,375.61747458)(551.99953082,375.61747458)
\curveto(551.50962806,375.61747458)(551.11454404,375.80711491)(550.861691,376.17059116)
\lineto(551.14615096,376.9607577)
\curveto(550.89329792,377.15039748)(550.68785414,377.24521765)(550.46660773,377.24521765)
\curveto(549.97670497,377.24521765)(549.64483453,376.92915055)(549.64483453,376.43924778)
\curveto(549.64483453,375.6174754)(550.41919907,375.04855467)(551.74667753,375.04855467)
\curveto(553.46923886,375.04855467)(554.63868706,376.10737921)(554.63868706,377.48226761)
\curveto(554.63868706,378.43046651)(553.99074955,378.84135479)(553.04255065,379.56830728)
\curveto(552.29979485,380.13722662)(551.85730084,380.4690977)(551.85730084,381.03801704)
\curveto(551.85730084,381.65434633)(552.3314014,382.09684021)(552.963534,382.09684021)
\curveto(553.13737046,382.09684021)(553.2954041,382.0494301)(553.45343725,381.9388069)
\lineto(554.11717715,381.02221371)
}
}
{
\newrgbcolor{curcolor}{0 0 0}
\pscustom[linestyle=none,fillstyle=solid,fillcolor=curcolor]
{
\newpath
\moveto(561.87883491,384.6095698)
\lineto(561.87883491,380.35847382)
\lineto(557.70675558,380.35847382)
\lineto(557.70675558,379.45768396)
\lineto(561.87883491,379.45768396)
\lineto(561.87883491,375.19078465)
\lineto(562.7954281,375.19078465)
\lineto(562.7954281,379.45768396)
\lineto(566.9517041,379.45768396)
\lineto(566.9517041,380.35847382)
\lineto(562.7954281,380.35847382)
\lineto(562.7954281,384.6095698)
\lineto(561.87883491,384.6095698)
}
}
{
\newrgbcolor{curcolor}{0 0 0}
\pscustom[linestyle=none,fillstyle=solid,fillcolor=curcolor]
{
\newpath
\moveto(574.93213923,385.98445958)
\lineto(571.24996315,384.94143975)
\lineto(571.24996315,384.29350319)
\lineto(573.54144612,384.59376647)
\lineto(573.69947943,384.43573317)
\lineto(572.11914635,376.23380448)
\lineto(571.86629305,375.96514786)
\lineto(570.22274665,375.87032787)
\lineto(570.22274665,375.19078465)
\lineto(574.94794256,375.19078465)
\lineto(574.94794256,375.85452454)
\lineto(573.57305278,375.9335412)
\lineto(573.4308228,376.17059116)
\lineto(575.31141917,385.98445958)
\lineto(574.93213923,385.98445958)
}
}
{
\newrgbcolor{curcolor}{0 0 0}
\pscustom[linestyle=none,fillstyle=solid,fillcolor=curcolor]
{
\newpath
\moveto(579.28817921,386.53757616)
\lineto(579.16175256,385.87383627)
\curveto(580.5682476,385.66839317)(581.37421887,384.46733784)(581.37421887,382.47612015)
\curveto(581.37421887,381.4805113)(581.16877502,379.91597853)(580.61565899,377.89315421)
\curveto(580.28378938,376.67629896)(579.77808172,375.66488375)(579.03532591,374.84311137)
\curveto(578.43479994,374.16356883)(577.64463197,373.72107477)(576.80705627,373.61045157)
\lineto(576.68062962,372.94671167)
\curveto(577.72364841,373.00992493)(578.78247355,373.49982911)(579.71486914,374.35320812)
\curveto(581.56385699,376.05996614)(582.670092,378.85716076)(582.670092,382.22326685)
\curveto(582.670092,384.64117405)(581.29520021,386.3637397)(579.28817921,386.53757616)
}
}
{
\newrgbcolor{curcolor}{0 0 0}
\pscustom[linestyle=none,fillstyle=solid,fillcolor=curcolor]
{
\newpath
\moveto(587.91556319,386.82203612)
\lineto(585.43444025,386.36373952)
\lineto(585.43444025,385.92124626)
\lineto(586.88834669,385.84222961)
\lineto(584.75489703,375.19078465)
\lineto(586.06657348,375.19078465)
\lineto(586.57228007,378.22502416)
\lineto(587.47306993,379.04679736)
\lineto(588.81635304,376.32862447)
\curveto(589.22723923,375.50685209)(589.66973354,375.11176799)(590.30186614,375.11176799)
\curveto(590.60212912,375.11176799)(591.23426357,375.23819494)(592.15085584,375.53845793)
\lineto(592.15085584,375.98095119)
\curveto(591.83478954,375.91773793)(591.56613237,375.87032787)(591.32908264,375.87032787)
\curveto(590.23865391,375.87032787)(589.60651838,377.21361371)(588.40546644,379.93178389)
\curveto(589.1008123,380.56391649)(589.76455357,381.14864085)(590.44409612,381.63854361)
\lineto(591.01301603,381.02221371)
\curveto(591.59773868,381.25926344)(591.88219922,381.59113403)(591.88219922,382.00202022)
\curveto(591.88219922,382.46031636)(591.50291876,382.80799009)(590.98140936,382.80799009)
\curveto(590.6021298,382.80799009)(590.14383228,382.5393329)(589.59071625,381.97041356)
\lineto(586.66710006,378.95197738)
\lineto(588.35805645,386.82203612)
\lineto(587.91556319,386.82203612)
}
}
{
\newrgbcolor{curcolor}{0 0 0}
\pscustom[linestyle=none,fillstyle=solid,fillcolor=curcolor]
{
\newpath
\moveto(598.60923267,384.6095698)
\lineto(598.60923267,380.35847382)
\lineto(594.43715334,380.35847382)
\lineto(594.43715334,379.45768396)
\lineto(598.60923267,379.45768396)
\lineto(598.60923267,375.19078465)
\lineto(599.52582586,375.19078465)
\lineto(599.52582586,379.45768396)
\lineto(603.68210186,379.45768396)
\lineto(603.68210186,380.35847382)
\lineto(599.52582586,380.35847382)
\lineto(599.52582586,384.6095698)
\lineto(598.60923267,384.6095698)
}
}
{
\newrgbcolor{curcolor}{0 0 0}
\pscustom[linestyle=none,fillstyle=solid,fillcolor=curcolor]
{
\newpath
\moveto(607.11117772,376.43924778)
\curveto(611.47289265,379.44187763)(613.82759331,381.3382826)(613.82759331,383.61395996)
\curveto(613.82759331,385.09947157)(612.83198169,385.98445958)(611.06201042,385.98445958)
\curveto(609.86095848,385.98445958)(608.69151004,385.5261623)(607.93295092,384.83081644)
\curveto(607.47465478,384.40412693)(607.25340769,383.93002615)(607.25340769,383.50333665)
\curveto(607.25340769,383.1398604)(607.53786811,382.87120342)(607.99616424,382.87120342)
\curveto(608.24901728,382.87120342)(608.50187108,382.95022023)(608.75472412,383.10825338)
\lineto(609.07079073,384.79920977)
\curveto(609.59230013,385.05206281)(610.06640106,385.17848971)(610.55630383,385.17848971)
\curveto(611.66253588,385.17848971)(612.27886689,384.54635539)(612.27886689,383.45592666)
\curveto(612.27886689,381.06962609)(609.86095291,379.23643428)(605.49923797,376.18639449)
\lineto(605.309598,375.19078465)
\lineto(612.62654016,375.19078465)
\lineto(613.11644342,377.7825309)
\lineto(612.40529353,377.7825309)
\lineto(611.86798029,376.43924778)
\lineto(607.11117772,376.43924778)
}
}
{
\newrgbcolor{curcolor}{0 0 0}
\pscustom[linestyle=none,fillstyle=solid,fillcolor=curcolor]
{
\newpath
\moveto(618.00979665,386.82203612)
\lineto(615.52867371,386.36373952)
\lineto(615.52867371,385.92124626)
\lineto(616.98258014,385.84222961)
\lineto(614.84913049,375.19078465)
\lineto(616.16080694,375.19078465)
\lineto(616.66651353,378.22502416)
\lineto(617.56730338,379.04679736)
\lineto(618.9105865,376.32862447)
\curveto(619.32147269,375.50685209)(619.763967,375.11176799)(620.3960996,375.11176799)
\curveto(620.69636258,375.11176799)(621.32849703,375.23819494)(622.2450893,375.53845793)
\lineto(622.2450893,375.98095119)
\curveto(621.929023,375.91773793)(621.66036582,375.87032787)(621.4233161,375.87032787)
\curveto(620.33288736,375.87032787)(619.70075184,377.21361371)(618.4996999,379.93178389)
\curveto(619.19504576,380.56391649)(619.85878703,381.14864085)(620.53832957,381.63854361)
\lineto(621.10724948,381.02221371)
\curveto(621.69197214,381.25926344)(621.97643268,381.59113403)(621.97643268,382.00202022)
\curveto(621.97643268,382.46031636)(621.59715222,382.80799009)(621.07564282,382.80799009)
\curveto(620.69636326,382.80799009)(620.23806574,382.5393329)(619.68494971,381.97041356)
\lineto(616.76133351,378.95197738)
\lineto(618.45228991,386.82203612)
\lineto(618.00979665,386.82203612)
}
}
{
\newrgbcolor{curcolor}{0 0 0}
\pscustom[linestyle=none,fillstyle=solid,fillcolor=curcolor]
{
\newpath
\moveto(624.04148354,381.85979025)
\lineto(624.54719013,385.38393301)
\curveto(624.56299344,385.51035953)(624.57879679,385.62098305)(624.57879679,385.69999963)
\curveto(624.57879679,386.20570571)(624.32594309,386.47436284)(623.9150569,386.47436284)
\curveto(623.50417071,386.47436284)(623.26712033,386.20570569)(623.26712033,385.6841963)
\curveto(623.26712033,385.58937641)(623.26712035,385.49455622)(623.28292366,385.38393301)
\lineto(623.78863025,381.85979025)
\lineto(624.04148354,381.85979025)
}
}
{
\newrgbcolor{curcolor}{0 0 0}
\pscustom[linestyle=none,fillstyle=solid,fillcolor=curcolor]
{
\newpath
\moveto(442.04291824,361.0582402)
\curveto(442.64344421,361.07404352)(443.03852808,361.43752063)(443.03852808,361.9274234)
\curveto(443.03852808,362.52794937)(442.43800064,363.03365655)(441.56881831,363.03365655)
\curveto(440.58901279,363.03365655)(439.54599117,362.54375241)(438.7400221,361.65876677)
\curveto(437.90244641,360.7421745)(437.42834565,359.49370905)(437.42834565,358.08721401)
\curveto(437.42834565,356.30143942)(438.37654688,355.21100781)(439.7672386,355.21100781)
\curveto(440.69963419,355.21100781)(441.74265593,355.71671542)(442.72246146,356.74393089)
\lineto(442.72246146,357.47088411)
\curveto(441.86908245,356.60170179)(440.98409435,356.15920765)(440.25714186,356.15920765)
\curveto(439.27733633,356.15920765)(438.67680878,356.93357224)(438.67680878,358.30846064)
\curveto(438.67680878,360.66315458)(439.76723988,362.25929334)(441.0473084,362.25929334)
\lineto(441.12632505,362.25929334)
\lineto(442.04291824,361.0582402)
}
}
{
\newrgbcolor{curcolor}{0 0 0}
\pscustom[linestyle=none,fillstyle=solid,fillcolor=curcolor]
{
\newpath
\moveto(447.32968123,359.22436248)
\lineto(444.93623306,358.54639004)
\lineto(444.93623306,358.12522534)
\lineto(446.42571797,358.32039923)
\lineto(446.52844107,358.21767613)
\lineto(445.5012101,352.8863474)
\lineto(445.33685314,352.71171813)
\lineto(444.26853293,352.65008427)
\lineto(444.26853293,352.20837495)
\lineto(447.33995354,352.20837495)
\lineto(447.33995354,352.63981196)
\lineto(446.44626259,352.69117351)
\lineto(446.3538118,352.84525816)
\lineto(447.57621666,359.22436248)
\lineto(447.32968123,359.22436248)
}
}
{
\newrgbcolor{curcolor}{0 0 0}
\pscustom[linestyle=none,fillstyle=solid,fillcolor=curcolor]
{
}
}
{
\newrgbcolor{curcolor}{0 0 0}
\pscustom[linestyle=none,fillstyle=solid,fillcolor=curcolor]
{
\newpath
\moveto(455.48675733,362.00644005)
\lineto(455.48675733,361.1056502)
\lineto(464.73170584,361.1056502)
\lineto(464.73170584,362.00644005)
\lineto(455.48675733,362.00644005)
\moveto(455.48675733,359.05121719)
\lineto(455.48675733,358.15042734)
\lineto(464.73170584,358.15042734)
\lineto(464.73170584,359.05121719)
\lineto(455.48675733,359.05121719)
}
}
{
\newrgbcolor{curcolor}{0 0 0}
\pscustom[linestyle=none,fillstyle=solid,fillcolor=curcolor]
{
}
}
{
\newrgbcolor{curcolor}{0 0 0}
\pscustom[linestyle=none,fillstyle=solid,fillcolor=curcolor]
{
\newpath
\moveto(475.94145285,361.0582402)
\curveto(476.54197882,361.07404352)(476.93706269,361.43752063)(476.93706269,361.9274234)
\curveto(476.93706269,362.52794937)(476.33653525,363.03365655)(475.46735292,363.03365655)
\curveto(474.48754739,363.03365655)(473.44452578,362.54375241)(472.63855671,361.65876677)
\curveto(471.80098102,360.7421745)(471.32688025,359.49370905)(471.32688025,358.08721401)
\curveto(471.32688025,356.30143942)(472.27508149,355.21100781)(473.66577321,355.21100781)
\curveto(474.5981688,355.21100781)(475.64119054,355.71671542)(476.62099607,356.74393089)
\lineto(476.62099607,357.47088411)
\curveto(475.76761706,356.60170179)(474.88262896,356.15920765)(474.15567647,356.15920765)
\curveto(473.17587094,356.15920765)(472.57534339,356.93357224)(472.57534339,358.30846064)
\curveto(472.57534339,360.66315458)(473.66577449,362.25929334)(474.94584301,362.25929334)
\lineto(475.02485966,362.25929334)
\lineto(475.94145285,361.0582402)
}
}
{
\newrgbcolor{curcolor}{0 0 0}
\pscustom[linestyle=none,fillstyle=solid,fillcolor=curcolor]
{
\newpath
\moveto(480.96113578,368.06947776)
\curveto(481.27957706,368.13111156)(481.46447896,368.33655808)(481.46447896,368.60363787)
\curveto(481.46447896,368.91180685)(481.10494763,369.20970414)(480.61187725,369.20970414)
\curveto(479.65655341,369.20970414)(478.7936781,368.25437757)(478.45469221,366.48754207)
\lineto(477.67399668,366.16910047)
\lineto(477.66372437,365.83011425)
\lineto(478.35196912,365.90202041)
\lineto(477.92053211,362.95386753)
\curveto(477.72535842,361.5876517)(477.55072848,360.65286965)(477.0679304,360.14952698)
\lineto(475.9996102,360.42687934)
\curveto(475.855798,360.32415635)(475.77361938,360.18034374)(475.77361938,360.01598695)
\curveto(475.77361938,359.68727337)(475.99961052,359.47155454)(476.32832411,359.47155454)
\curveto(476.82139448,359.47155454)(477.38637255,359.86190308)(477.93080442,360.63232554)
\curveto(478.38278559,361.2692081)(478.56768776,362.03963291)(478.70122765,362.98468446)
\lineto(479.11212004,365.90202041)
\lineto(480.64269418,365.90202041)
\lineto(480.64269418,366.36427435)
\lineto(479.19429851,366.36427435)
\curveto(479.4202891,367.69967328)(479.72845901,368.50091503)(480.11880639,368.75772251)
\lineto(480.96113578,368.06947776)
}
}
{
\newrgbcolor{curcolor}{0 0 0}
\pscustom[linestyle=none,fillstyle=solid,fillcolor=curcolor]
{
\newpath
\moveto(484.23768138,368.70636096)
\lineto(481.84423322,368.02838852)
\lineto(481.84423322,367.60722382)
\lineto(483.33371813,367.80239771)
\lineto(483.43644122,367.69967461)
\lineto(482.40921025,362.36834588)
\lineto(482.2448533,362.19371661)
\lineto(481.17653309,362.13208275)
\lineto(481.17653309,361.69037343)
\lineto(484.24795369,361.69037343)
\lineto(484.24795369,362.12181044)
\lineto(483.35426275,362.17317199)
\lineto(483.26181196,362.32725664)
\lineto(484.48421681,368.70636096)
\lineto(484.23768138,368.70636096)
}
}
{
\newrgbcolor{curcolor}{0 0 0}
\pscustom[linestyle=none,fillstyle=solid,fillcolor=curcolor]
{
}
}
{
\newrgbcolor{curcolor}{0 0 0}
\pscustom[linestyle=none,fillstyle=solid,fillcolor=curcolor]
{
\newpath
\moveto(496.07585856,362.06965337)
\curveto(495.71238231,362.76499923)(495.1118546,363.12847654)(494.33749217,363.12847654)
\curveto(493.67375294,363.12847654)(492.85197803,362.79660597)(491.80895924,362.18027669)
\lineto(491.80895924,363.04945988)
\lineto(491.20843267,363.04945988)
\lineto(489.3752463,362.48053997)
\lineto(489.3752463,361.99063672)
\lineto(490.49728278,361.99063672)
\lineto(490.49728278,356.26983097)
\lineto(490.22862616,356.01697768)
\lineto(489.3752463,355.93796102)
\lineto(489.3752463,355.36904111)
\lineto(492.83617574,355.36904111)
\lineto(492.83617574,355.93796102)
\lineto(492.07761586,356.01697768)
\lineto(491.80895924,356.26983097)
\lineto(491.80895924,361.56394679)
\curveto(492.5043051,361.91161972)(493.08902956,362.0854567)(493.61053895,362.0854567)
\curveto(494.55873785,362.0854567)(495.00123206,361.56394568)(495.00123206,360.45771363)
\lineto(495.00123206,356.26983097)
\lineto(494.73257544,356.01697768)
\lineto(493.97401556,355.93796102)
\lineto(493.97401556,355.36904111)
\lineto(497.34012502,355.36904111)
\lineto(497.34012502,355.93796102)
\lineto(496.59736847,356.01697768)
\lineto(496.31290852,356.26983097)
\lineto(496.31290852,361.56394679)
\curveto(496.99245106,361.8958164)(497.59297882,362.0854567)(498.0986849,362.0854567)
\curveto(499.07849043,362.0854567)(499.52098467,361.56394568)(499.52098467,360.45771363)
\lineto(499.52098467,356.26983097)
\lineto(499.23652472,356.01697768)
\lineto(498.49376817,355.93796102)
\lineto(498.49376817,355.36904111)
\lineto(501.93889428,355.36904111)
\lineto(501.93889428,355.93796102)
\lineto(501.10131775,356.01697768)
\lineto(500.83266113,356.26983097)
\lineto(500.83266113,360.72637026)
\curveto(500.83266113,362.29089844)(500.04249334,363.12847654)(498.79403145,363.12847654)
\curveto(498.16189885,363.12847654)(497.26110718,362.76499923)(496.07585856,362.06965337)
}
}
{
\newrgbcolor{curcolor}{0 0 0}
\pscustom[linestyle=none,fillstyle=solid,fillcolor=curcolor]
{
\newpath
\moveto(506.48630272,363.12847654)
\curveto(504.3528552,363.12847654)(502.81992997,361.53233777)(502.81992997,359.17764384)
\curveto(502.81992997,356.83875322)(504.36865851,355.22681114)(506.48630272,355.22681114)
\curveto(508.60394693,355.22681114)(510.1684788,356.83875322)(510.1684788,359.17764384)
\curveto(510.1684788,361.51653446)(508.61975024,363.12847654)(506.48630272,363.12847654)
\moveto(506.48630272,362.44893331)
\curveto(507.76637123,362.44893331)(508.6829657,361.21627147)(508.6829657,359.17764384)
\curveto(508.6829657,357.12321289)(507.76637123,355.90635436)(506.48630272,355.90635436)
\curveto(505.22203752,355.90635436)(504.30544307,357.1390162)(504.30544307,359.17764384)
\curveto(504.30544307,361.23207479)(505.20623421,362.44893331)(506.48630272,362.44893331)
}
}
{
\newrgbcolor{curcolor}{0 0 0}
\pscustom[linestyle=none,fillstyle=solid,fillcolor=curcolor]
{
\newpath
\moveto(516.80686544,365.94146942)
\lineto(516.80686544,362.97044323)
\curveto(516.31696268,363.08106643)(515.84286182,363.14427987)(515.400369,363.14427987)
\curveto(512.99826512,363.14427987)(511.32310965,361.45332094)(511.32310965,358.92479054)
\curveto(511.32310965,356.71232645)(512.65059116,355.22681114)(514.3731525,355.22681114)
\curveto(515.24233482,355.22681114)(516.00089638,355.5428784)(516.80686544,356.19081432)
\lineto(516.80686544,355.36904111)
\lineto(519.1457584,355.36904111)
\lineto(519.1457584,355.93796102)
\lineto(518.38719852,356.01697768)
\lineto(518.1185419,356.26983097)
\lineto(518.1185419,367.00029258)
\lineto(517.51801533,367.00029258)
\lineto(515.59000897,366.43137267)
\lineto(515.59000897,365.94146942)
\lineto(516.80686544,365.94146942)
\moveto(516.80686544,356.77553756)
\curveto(516.22214279,356.41206131)(515.65322172,356.22242098)(515.08430238,356.22242098)
\curveto(513.66200403,356.22242098)(512.71380276,357.34445931)(512.71380276,359.19344717)
\curveto(512.71380276,361.16886154)(513.82003752,362.44893331)(515.41617233,362.44893331)
\curveto(515.79545189,362.44893331)(516.26955273,362.35411317)(516.80686544,362.19608002)
\lineto(516.80686544,356.77553756)
}
}
{
\newrgbcolor{curcolor}{0 0 0}
\pscustom[linestyle=none,fillstyle=solid,fillcolor=curcolor]
{
}
}
{
\newrgbcolor{curcolor}{0 0 0}
\pscustom[linestyle=none,fillstyle=solid,fillcolor=curcolor]
{
\newpath
\moveto(526.9177871,362.98624656)
\lineto(524.57889414,362.46473664)
\lineto(524.57889414,362.06965337)
\lineto(525.87476727,361.83260341)
\lineto(524.56309081,355.36904111)
\lineto(525.90637393,355.36904111)
\lineto(526.88618044,360.71056693)
\curveto(527.85018265,361.59555256)(528.68776069,362.00644005)(529.2250734,362.00644005)
\curveto(529.82559937,362.00644005)(530.07845326,361.70617607)(530.07845326,361.01083021)
\curveto(530.07845326,360.80538712)(530.04684649,360.45771317)(529.93622328,359.99941704)
\lineto(529.25668006,357.07580084)
\curveto(529.17766348,356.72812791)(529.13025341,356.42786407)(529.13025341,356.22242098)
\curveto(529.13025341,355.59028838)(529.50953413,355.27422113)(530.28389656,355.27422113)
\curveto(530.72638938,355.27422113)(531.35852388,355.43225475)(532.18029626,355.74832105)
\lineto(532.18029626,356.19081432)
\curveto(531.81682001,356.09599443)(531.53235947,356.04858434)(531.31111306,356.04858434)
\curveto(530.78960367,356.04858434)(530.48933986,356.31724147)(530.48933986,356.82294755)
\curveto(530.48933986,357.05999727)(530.56835661,357.40767125)(530.6631765,357.86596738)
\lineto(531.20048975,360.31548366)
\curveto(531.32691627,360.884403)(531.42173638,361.32689722)(531.42173638,361.7219801)
\curveto(531.42173638,362.49634253)(530.91602897,362.98624656)(530.09425659,362.98624656)
\curveto(529.2566809,362.98624656)(528.2768723,362.46473565)(527.02841042,361.4691268)
\lineto(527.31287037,362.98624656)
\lineto(526.9177871,362.98624656)
}
}
{
\newrgbcolor{curcolor}{0 0 0}
\pscustom[linestyle=none,fillstyle=solid,fillcolor=curcolor]
{
\newpath
\moveto(535.58659454,365.48085572)
\curveto(535.96666962,365.50140031)(536.1823885,365.71711913)(536.1823885,366.00474351)
\curveto(536.1823885,366.36427399)(535.81258477,366.64162671)(535.2270637,366.64162671)
\curveto(534.24092296,366.64162671)(533.46022643,365.9328365)(533.46022643,365.09050795)
\curveto(533.46022643,364.51525918)(533.85057479,364.2379058)(534.43609585,363.79619692)
\curveto(534.86753243,363.46748334)(535.14488522,363.2312195)(535.14488522,362.84087212)
\curveto(535.14488522,362.29644025)(534.79562611,361.9677258)(534.21010504,361.9677258)
\curveto(533.89166376,361.9677258)(533.63485553,362.09099375)(533.47049874,362.32725664)
\lineto(533.65540032,362.84087212)
\curveto(533.49104353,362.96413971)(533.35750319,363.0257737)(533.213691,363.0257737)
\curveto(532.89524972,363.0257737)(532.67953089,362.82032718)(532.67953089,362.5018859)
\curveto(532.67953089,361.96772633)(533.18287493,361.59792265)(534.04574809,361.59792265)
\curveto(535.16542872,361.59792265)(535.92558076,362.28616829)(535.92558076,363.17985834)
\curveto(535.92558076,363.79619631)(535.50441545,364.06327745)(534.88807748,364.53580322)
\curveto(534.40527941,364.905606)(534.11765425,365.12132525)(534.11765425,365.49112803)
\curveto(534.11765425,365.8917477)(534.42582396,366.17937278)(534.83671593,366.17937278)
\curveto(534.94971123,366.17937278)(535.05243454,366.14855577)(535.15515753,366.07664968)
\lineto(535.58659454,365.48085572)
}
}
{
\newrgbcolor{curcolor}{0 0 0}
\pscustom[linestyle=none,fillstyle=solid,fillcolor=curcolor]
{
\newpath
\moveto(540.631744,367.81267002)
\lineto(540.631744,365.04941871)
\lineto(537.91985424,365.04941871)
\lineto(537.91985424,364.46389705)
\lineto(540.631744,364.46389705)
\lineto(540.631744,361.69037343)
\lineto(541.22753796,361.69037343)
\lineto(541.22753796,364.46389705)
\lineto(543.92915541,364.46389705)
\lineto(543.92915541,365.04941871)
\lineto(541.22753796,365.04941871)
\lineto(541.22753796,367.81267002)
\lineto(540.631744,367.81267002)
}
}
{
\newrgbcolor{curcolor}{0 0 0}
\pscustom[linestyle=none,fillstyle=solid,fillcolor=curcolor]
{
\newpath
\moveto(549.11651724,368.70636096)
\lineto(546.72306908,368.02838852)
\lineto(546.72306908,367.60722382)
\lineto(548.21255399,367.80239771)
\lineto(548.31527708,367.69967461)
\lineto(547.28804611,362.36834588)
\lineto(547.12368916,362.19371661)
\lineto(546.05536895,362.13208275)
\lineto(546.05536895,361.69037343)
\lineto(549.12678955,361.69037343)
\lineto(549.12678955,362.12181044)
\lineto(548.23309861,362.17317199)
\lineto(548.14064782,362.32725664)
\lineto(549.36305267,368.70636096)
\lineto(549.11651724,368.70636096)
}
}
{
\newrgbcolor{curcolor}{0 0 0}
\pscustom[linestyle=none,fillstyle=solid,fillcolor=curcolor]
{
\newpath
\moveto(438.66100545,343.19610969)
\lineto(437.06486904,342.67459977)
\lineto(437.06486904,342.2795165)
\lineto(438.28172551,342.18469652)
\curveto(438.55038186,342.01086005)(438.67680886,341.56836589)(438.75582543,340.8414134)
\lineto(439.1509087,337.14343399)
\curveto(439.16671202,336.95379421)(439.18251536,336.76415386)(439.18251536,336.57451408)
\curveto(439.18251536,336.19523452)(439.11930188,335.84756057)(438.96126873,335.54729758)
\lineto(439.71982861,335.54729758)
\curveto(442.6592452,338.86599373)(444.04994125,341.2048911)(444.04994125,342.29531983)
\curveto(444.04994125,342.86423917)(443.74967743,343.19610969)(443.21236472,343.19610969)
\curveto(442.94370836,343.19610969)(442.67505122,343.08548617)(442.42219818,342.88004307)
\lineto(442.72246146,340.95203672)
\curveto(442.31157527,339.86160798)(441.50560379,338.58153556)(440.30455185,337.04861401)
\lineto(439.89366525,341.37872665)
\curveto(439.76723873,342.64299185)(439.59340106,343.19610969)(438.69261211,343.19610969)
\lineto(438.66100545,343.19610969)
}
}
{
\newrgbcolor{curcolor}{0 0 0}
\pscustom[linestyle=none,fillstyle=solid,fillcolor=curcolor]
{
\newpath
\moveto(448.25565764,339.40261895)
\lineto(445.86220948,338.72464651)
\lineto(445.86220948,338.30348181)
\lineto(447.35169439,338.4986557)
\lineto(447.45441748,338.3959326)
\lineto(446.42718651,333.06460386)
\lineto(446.26282956,332.8899746)
\lineto(445.19450935,332.82834074)
\lineto(445.19450935,332.38663142)
\lineto(448.26592995,332.38663142)
\lineto(448.26592995,332.81806843)
\lineto(447.372239,332.86942998)
\lineto(447.27978822,333.02351462)
\lineto(448.50219307,339.40261895)
\lineto(448.25565764,339.40261895)
}
}
{
\newrgbcolor{curcolor}{0 0 0}
\pscustom[linestyle=none,fillstyle=solid,fillcolor=curcolor]
{
}
}
{
\newrgbcolor{curcolor}{0 0 0}
\pscustom[linestyle=none,fillstyle=solid,fillcolor=curcolor]
{
\newpath
\moveto(456.41273374,342.18469652)
\lineto(456.41273374,341.28390666)
\lineto(465.65768226,341.28390666)
\lineto(465.65768226,342.18469652)
\lineto(456.41273374,342.18469652)
\moveto(456.41273374,339.22947366)
\lineto(456.41273374,338.3286838)
\lineto(465.65768226,338.3286838)
\lineto(465.65768226,339.22947366)
\lineto(456.41273374,339.22947366)
}
}
{
\newrgbcolor{curcolor}{0 0 0}
\pscustom[linestyle=none,fillstyle=solid,fillcolor=curcolor]
{
}
}
{
\newrgbcolor{curcolor}{0 0 0}
\pscustom[linestyle=none,fillstyle=solid,fillcolor=curcolor]
{
\newpath
\moveto(473.48551647,343.19610969)
\lineto(471.88938006,342.67459977)
\lineto(471.88938006,342.2795165)
\lineto(473.10623653,342.18469652)
\curveto(473.37489289,342.01086005)(473.50131988,341.56836589)(473.58033646,340.8414134)
\lineto(473.97541973,337.14343399)
\curveto(473.99122304,336.95379421)(474.00702639,336.76415386)(474.00702639,336.57451408)
\curveto(474.00702639,336.19523452)(473.94381291,335.84756057)(473.78577976,335.54729758)
\lineto(474.54433963,335.54729758)
\curveto(477.48375622,338.86599373)(478.87445227,341.2048911)(478.87445227,342.29531983)
\curveto(478.87445227,342.86423917)(478.57418845,343.19610969)(478.03687574,343.19610969)
\curveto(477.76821939,343.19610969)(477.49956224,343.08548617)(477.2467092,342.88004307)
\lineto(477.54697249,340.95203672)
\curveto(477.1360863,339.86160798)(476.33011481,338.58153556)(475.12906287,337.04861401)
\lineto(474.71817627,341.37872665)
\curveto(474.59174975,342.64299185)(474.41791209,343.19610969)(473.51712313,343.19610969)
\lineto(473.48551647,343.19610969)
}
}
{
\newrgbcolor{curcolor}{0 0 0}
\pscustom[linestyle=none,fillstyle=solid,fillcolor=curcolor]
{
\newpath
\moveto(482.81308861,348.24773423)
\curveto(483.13152989,348.30936802)(483.31643179,348.51481455)(483.31643179,348.78189433)
\curveto(483.31643179,349.09006332)(482.95690045,349.3879606)(482.46383008,349.3879606)
\curveto(481.50850623,349.3879606)(480.64563092,348.43263404)(480.30664504,346.66579853)
\lineto(479.52594951,346.34735693)
\lineto(479.5156772,346.00837071)
\lineto(480.20392195,346.08027688)
\lineto(479.77248494,343.132124)
\curveto(479.57731125,341.76590817)(479.40268131,340.83112612)(478.91988323,340.32778345)
\lineto(477.85156302,340.60513581)
\curveto(477.70775083,340.50241281)(477.62557221,340.35860021)(477.62557221,340.19424342)
\curveto(477.62557221,339.86552984)(477.85156335,339.64981101)(478.18027693,339.64981101)
\curveto(478.67334731,339.64981101)(479.23832538,340.04015954)(479.78275725,340.810582)
\curveto(480.23473842,341.44746457)(480.41964058,342.21788938)(480.55318048,343.16294092)
\lineto(480.96407286,346.08027688)
\lineto(482.49464701,346.08027688)
\lineto(482.49464701,346.54253082)
\lineto(481.04625134,346.54253082)
\curveto(481.27224193,347.87792974)(481.58041184,348.67917149)(481.97075922,348.93597898)
\lineto(482.81308861,348.24773423)
}
}
{
\newrgbcolor{curcolor}{0 0 0}
\pscustom[linestyle=none,fillstyle=solid,fillcolor=curcolor]
{
\newpath
\moveto(486.08963421,348.88461743)
\lineto(483.69618605,348.20664499)
\lineto(483.69618605,347.78548029)
\lineto(485.18567095,347.98065418)
\lineto(485.28839405,347.87793108)
\lineto(484.26116308,342.54660234)
\lineto(484.09680613,342.37197308)
\lineto(483.02848592,342.31033922)
\lineto(483.02848592,341.8686299)
\lineto(486.09990652,341.8686299)
\lineto(486.09990652,342.30006691)
\lineto(485.20621557,342.35142846)
\lineto(485.11376479,342.5055131)
\lineto(486.33616964,348.88461743)
\lineto(486.08963421,348.88461743)
}
}
{
\newrgbcolor{curcolor}{0 0 0}
\pscustom[linestyle=none,fillstyle=solid,fillcolor=curcolor]
{
}
}
{
\newrgbcolor{curcolor}{0 0 0}
\pscustom[linestyle=none,fillstyle=solid,fillcolor=curcolor]
{
}
}
{
\newrgbcolor{curcolor}{0 0 0}
\pscustom[linestyle=none,fillstyle=solid,fillcolor=curcolor]
{
\newpath
\moveto(500.77677435,342.24790984)
\curveto(500.41329811,342.9432557)(499.8127704,343.306733)(499.03840796,343.306733)
\curveto(498.37466873,343.306733)(497.55289382,342.97486244)(496.50987503,342.35853316)
\lineto(496.50987503,343.22771635)
\lineto(495.90934846,343.22771635)
\lineto(494.07616209,342.65879644)
\lineto(494.07616209,342.16889319)
\lineto(495.19819858,342.16889319)
\lineto(495.19819858,336.44808744)
\lineto(494.92954195,336.19523414)
\lineto(494.07616209,336.11621749)
\lineto(494.07616209,335.54729758)
\lineto(497.53709154,335.54729758)
\lineto(497.53709154,336.11621749)
\lineto(496.77853166,336.19523414)
\lineto(496.50987503,336.44808744)
\lineto(496.50987503,341.74220326)
\curveto(497.20522089,342.08987619)(497.78994535,342.26371317)(498.31145475,342.26371317)
\curveto(499.25965365,342.26371317)(499.70214786,341.74220215)(499.70214786,340.6359701)
\lineto(499.70214786,336.44808744)
\lineto(499.43349123,336.19523414)
\lineto(498.67493135,336.11621749)
\lineto(498.67493135,335.54729758)
\lineto(502.04104081,335.54729758)
\lineto(502.04104081,336.11621749)
\lineto(501.29828427,336.19523414)
\lineto(501.01382431,336.44808744)
\lineto(501.01382431,341.74220326)
\curveto(501.69336686,342.07407287)(502.29389461,342.26371317)(502.79960069,342.26371317)
\curveto(503.77940622,342.26371317)(504.22190046,341.74220215)(504.22190046,340.6359701)
\lineto(504.22190046,336.44808744)
\lineto(503.93744051,336.19523414)
\lineto(503.19468396,336.11621749)
\lineto(503.19468396,335.54729758)
\lineto(506.63981008,335.54729758)
\lineto(506.63981008,336.11621749)
\lineto(505.80223354,336.19523414)
\lineto(505.53357692,336.44808744)
\lineto(505.53357692,340.90462672)
\curveto(505.53357692,342.46915491)(504.74340913,343.306733)(503.49494725,343.306733)
\curveto(502.86281465,343.306733)(501.96202298,342.9432557)(500.77677435,342.24790984)
}
}
{
\newrgbcolor{curcolor}{0 0 0}
\pscustom[linestyle=none,fillstyle=solid,fillcolor=curcolor]
{
\newpath
\moveto(511.18721851,343.306733)
\curveto(509.05377099,343.306733)(507.52084577,341.71059424)(507.52084577,339.3559003)
\curveto(507.52084577,337.01700969)(509.06957431,335.4050676)(511.18721851,335.4050676)
\curveto(513.30486272,335.4050676)(514.86939459,337.01700969)(514.86939459,339.3559003)
\curveto(514.86939459,341.69479092)(513.32066604,343.306733)(511.18721851,343.306733)
\moveto(511.18721851,342.62718978)
\curveto(512.46728703,342.62718978)(513.3838815,341.39452794)(513.3838815,339.3559003)
\curveto(513.3838815,337.30146935)(512.46728703,336.08461083)(511.18721851,336.08461083)
\curveto(509.92295332,336.08461083)(509.00635886,337.31727267)(509.00635886,339.3559003)
\curveto(509.00635886,341.41033125)(509.90715,342.62718978)(511.18721851,342.62718978)
}
}
{
\newrgbcolor{curcolor}{0 0 0}
\pscustom[linestyle=none,fillstyle=solid,fillcolor=curcolor]
{
\newpath
\moveto(521.50778124,346.11972589)
\lineto(521.50778124,343.1486997)
\curveto(521.01787847,343.2593229)(520.54377761,343.32253634)(520.10128479,343.32253634)
\curveto(517.69918091,343.32253634)(516.02402545,341.63157741)(516.02402545,339.10304701)
\curveto(516.02402545,336.89058291)(517.35150696,335.4050676)(519.07406829,335.4050676)
\curveto(519.94325062,335.4050676)(520.70181217,335.72113487)(521.50778124,336.36907078)
\lineto(521.50778124,335.54729758)
\lineto(523.84667419,335.54729758)
\lineto(523.84667419,336.11621749)
\lineto(523.08811432,336.19523414)
\lineto(522.81945769,336.44808744)
\lineto(522.81945769,347.17854905)
\lineto(522.21893112,347.17854905)
\lineto(520.29092476,346.60962914)
\lineto(520.29092476,346.11972589)
\lineto(521.50778124,346.11972589)
\moveto(521.50778124,336.95379402)
\curveto(520.92305858,336.59031778)(520.35413752,336.40067744)(519.78521818,336.40067744)
\curveto(518.36291983,336.40067744)(517.41471856,337.52271578)(517.41471856,339.37170364)
\curveto(517.41471856,341.34711801)(518.52095331,342.62718978)(520.11708813,342.62718978)
\curveto(520.49636768,342.62718978)(520.97046853,342.53236964)(521.50778124,342.37433649)
\lineto(521.50778124,336.95379402)
}
}
{
\newrgbcolor{curcolor}{0 0 0}
\pscustom[linestyle=none,fillstyle=solid,fillcolor=curcolor]
{
}
}
{
\newrgbcolor{curcolor}{0 0 0}
\pscustom[linestyle=none,fillstyle=solid,fillcolor=curcolor]
{
\newpath
\moveto(531.6187029,343.16450303)
\lineto(529.27980994,342.64299311)
\lineto(529.27980994,342.24790984)
\lineto(530.57568306,342.01085988)
\lineto(529.26400661,335.54729758)
\lineto(530.60728972,335.54729758)
\lineto(531.58709623,340.88882339)
\curveto(532.55109845,341.77380903)(533.38867648,342.18469652)(533.92598919,342.18469652)
\curveto(534.52651516,342.18469652)(534.77936906,341.88443254)(534.77936906,341.18908668)
\curveto(534.77936906,340.98364358)(534.74776228,340.63596964)(534.63713908,340.17767351)
\lineto(533.95759585,337.25405731)
\curveto(533.87857928,336.90638438)(533.83116921,336.60612054)(533.83116921,336.40067744)
\curveto(533.83116921,335.76854485)(534.21044992,335.4524776)(534.98481236,335.4524776)
\curveto(535.42730518,335.4524776)(536.05943967,335.61051122)(536.88121205,335.92657752)
\lineto(536.88121205,336.36907078)
\curveto(536.51773581,336.27425089)(536.23327527,336.22684081)(536.01202886,336.22684081)
\curveto(535.49051946,336.22684081)(535.19025566,336.49549794)(535.19025566,337.00120402)
\curveto(535.19025566,337.23825374)(535.26927241,337.58592771)(535.3640923,338.04422385)
\lineto(535.90140554,340.49374012)
\curveto(536.02783206,341.06265946)(536.12265217,341.50515369)(536.12265217,341.90023656)
\curveto(536.12265217,342.674599)(535.61694477,343.16450303)(534.79517239,343.16450303)
\curveto(533.95759669,343.16450303)(532.9777881,342.64299212)(531.72932621,341.64738327)
\lineto(532.01378617,343.16450303)
\lineto(531.6187029,343.16450303)
}
}
{
\newrgbcolor{curcolor}{0 0 0}
\pscustom[linestyle=none,fillstyle=solid,fillcolor=curcolor]
{
\newpath
\moveto(540.28751034,345.65911218)
\curveto(540.66758541,345.67965678)(540.8833043,345.89537559)(540.8833043,346.18299998)
\curveto(540.8833043,346.54253046)(540.51350056,346.81988318)(539.9279795,346.81988318)
\curveto(538.94183875,346.81988318)(538.16114223,346.11109297)(538.16114223,345.26876441)
\curveto(538.16114223,344.69351565)(538.55149058,344.41616227)(539.13701165,343.97445339)
\curveto(539.56844823,343.64573981)(539.84580102,343.40947597)(539.84580102,343.01912859)
\curveto(539.84580102,342.47469672)(539.4965419,342.14598226)(538.91102084,342.14598226)
\curveto(538.59257955,342.14598226)(538.33577133,342.26925022)(538.17141454,342.5055131)
\lineto(538.35631611,343.01912859)
\curveto(538.19195932,343.14239618)(538.05841899,343.20403016)(537.91460679,343.20403016)
\curveto(537.59616551,343.20403016)(537.38044669,342.99858365)(537.38044669,342.68014237)
\curveto(537.38044669,342.1459828)(537.88379073,341.77617911)(538.74666388,341.77617911)
\curveto(539.86634452,341.77617911)(540.62649656,342.46442476)(540.62649656,343.35811481)
\curveto(540.62649656,343.97445277)(540.20533124,344.24153392)(539.58899328,344.71405969)
\curveto(539.1061952,345.08386247)(538.81857005,345.29958171)(538.81857005,345.66938449)
\curveto(538.81857005,346.07000417)(539.12673975,346.35762924)(539.53763173,346.35762924)
\curveto(539.65062702,346.35762924)(539.75335033,346.32681224)(539.85607333,346.25490615)
\lineto(540.28751034,345.65911218)
}
}
{
\newrgbcolor{curcolor}{0 0 0}
\pscustom[linestyle=none,fillstyle=solid,fillcolor=curcolor]
{
\newpath
\moveto(545.33265979,347.99092649)
\lineto(545.33265979,345.22767517)
\lineto(542.62077003,345.22767517)
\lineto(542.62077003,344.64215352)
\lineto(545.33265979,344.64215352)
\lineto(545.33265979,341.8686299)
\lineto(545.92845375,341.8686299)
\lineto(545.92845375,344.64215352)
\lineto(548.63007121,344.64215352)
\lineto(548.63007121,345.22767517)
\lineto(545.92845375,345.22767517)
\lineto(545.92845375,347.99092649)
\lineto(545.33265979,347.99092649)
}
}
{
\newrgbcolor{curcolor}{0 0 0}
\pscustom[linestyle=none,fillstyle=solid,fillcolor=curcolor]
{
\newpath
\moveto(553.81742701,348.88461743)
\lineto(551.42397885,348.20664499)
\lineto(551.42397885,347.78548029)
\lineto(552.91346375,347.98065418)
\lineto(553.01618685,347.87793108)
\lineto(551.98895588,342.54660234)
\lineto(551.82459893,342.37197308)
\lineto(550.75627872,342.31033922)
\lineto(550.75627872,341.8686299)
\lineto(553.82769932,341.8686299)
\lineto(553.82769932,342.30006691)
\lineto(552.93400837,342.35142846)
\lineto(552.84155759,342.5055131)
\lineto(554.06396244,348.88461743)
\lineto(553.81742701,348.88461743)
}
}
{
\newrgbcolor{curcolor}{0 0 0}
\pscustom[linestyle=none,fillstyle=solid,fillcolor=curcolor]
{
\newpath
\moveto(443.67066131,323.34275949)
\lineto(443.3545947,322.99508622)
\curveto(442.86469193,323.23213594)(442.37478774,323.34275949)(441.93229492,323.34275949)
\curveto(440.36776674,323.34275949)(439.08769457,322.60000162)(438.28172551,321.27252316)
\curveto(437.7444128,320.38753752)(437.45995231,319.40772907)(437.45995231,318.34890696)
\curveto(437.45995231,316.61054231)(438.29753018,315.58332407)(439.64081196,315.58332407)
\curveto(440.39937108,315.58332407)(441.15793255,316.02581822)(441.99550825,316.91080386)
\lineto(441.99550825,316.78437721)
\curveto(441.99550825,315.97840815)(442.2957722,315.58332407)(442.95951143,315.58332407)
\curveto(443.38620093,315.58332407)(444.03413879,315.7729644)(444.90332111,316.13644065)
\lineto(444.90332111,316.61054057)
\curveto(444.47663161,316.51572068)(444.19217105,316.45250727)(444.01833459,316.45250727)
\curveto(443.43361193,316.45250727)(443.16495473,316.6737544)(443.16495473,317.17946048)
\curveto(443.16495473,317.40070689)(443.21236483,317.76418424)(443.32298803,318.28569364)
\lineto(443.89190794,321.08288319)
\curveto(444.12895767,322.2523285)(444.33440136,323.01088988)(444.49243451,323.34275949)
\lineto(443.67066131,323.34275949)
\moveto(442.20095155,322.72642959)
\lineto(442.83308478,321.76242641)
\lineto(442.01131158,317.59034708)
\curveto(441.36337566,316.89500122)(440.74704451,316.54732725)(440.14651854,316.54732725)
\curveto(439.22992627,316.54732725)(438.77162876,317.17946175)(438.77162876,318.44372695)
\curveto(438.77162876,320.95645403)(440.13071706,322.72642959)(441.97970492,322.72642959)
\lineto(442.20095155,322.72642959)
}
}
{
\newrgbcolor{curcolor}{0 0 0}
\pscustom[linestyle=none,fillstyle=solid,fillcolor=curcolor]
{
}
}
{
\newrgbcolor{curcolor}{0 0 0}
\pscustom[linestyle=none,fillstyle=solid,fillcolor=curcolor]
{
\newpath
\moveto(451.83678558,322.36295298)
\lineto(451.83678558,321.46216313)
\lineto(461.08173409,321.46216313)
\lineto(461.08173409,322.36295298)
\lineto(451.83678558,322.36295298)
\moveto(451.83678558,319.40773013)
\lineto(451.83678558,318.50694027)
\lineto(461.08173409,318.50694027)
\lineto(461.08173409,319.40773013)
\lineto(451.83678558,319.40773013)
}
}
{
\newrgbcolor{curcolor}{0 0 0}
\pscustom[linestyle=none,fillstyle=solid,fillcolor=curcolor]
{
}
}
{
\newrgbcolor{curcolor}{0 0 0}
\pscustom[linestyle=none,fillstyle=solid,fillcolor=curcolor]
{
\newpath
\moveto(472.2914811,321.41475314)
\curveto(472.89200707,321.43055645)(473.28709094,321.79403357)(473.28709094,322.28393633)
\curveto(473.28709094,322.8844623)(472.6865635,323.39016949)(471.81738117,323.39016949)
\curveto(470.83757564,323.39016949)(469.79455403,322.90026535)(468.98858496,322.01527971)
\curveto(468.15100927,321.09868744)(467.6769085,319.85022198)(467.6769085,318.44372695)
\curveto(467.6769085,316.65795235)(468.62510974,315.56752074)(470.01580146,315.56752074)
\curveto(470.94819705,315.56752074)(471.99121879,316.07322835)(472.97102432,317.10044383)
\lineto(472.97102432,317.82739705)
\curveto(472.11764531,316.95821472)(471.23265721,316.51572059)(470.50570472,316.51572059)
\curveto(469.52589919,316.51572059)(468.92537164,317.29008517)(468.92537164,318.66497358)
\curveto(468.92537164,321.01966751)(470.01580274,322.61580628)(471.29587126,322.61580628)
\lineto(471.37488791,322.61580628)
\lineto(472.2914811,321.41475314)
}
}
{
\newrgbcolor{curcolor}{0 0 0}
\pscustom[linestyle=none,fillstyle=solid,fillcolor=curcolor]
{
\newpath
\moveto(475.49296521,326.99813965)
\lineto(473.97266338,326.65915343)
\lineto(473.97266338,326.40234568)
\lineto(474.81499277,326.24826104)
\lineto(473.96239107,322.04688637)
\lineto(474.82526508,322.04688637)
\lineto(475.24642978,324.46087915)
\curveto(475.41078657,325.4059307)(475.73950165,325.96063676)(476.746187,326.35098413)
\lineto(477.01326705,325.79627941)
\lineto(477.08517322,325.79627941)
\curveto(477.5063375,325.79627941)(477.77341797,326.07363216)(477.77341797,326.46397954)
\curveto(477.77341797,326.77214852)(477.54742681,326.99813965)(477.19816863,326.99813965)
\curveto(476.68455365,326.99813965)(476.02712466,326.52561246)(475.3696975,325.59083322)
\lineto(475.69841141,326.99813965)
\lineto(475.49296521,326.99813965)
}
}
{
\newrgbcolor{curcolor}{0 0 0}
\pscustom[linestyle=none,fillstyle=solid,fillcolor=curcolor]
{
}
}
{
\newrgbcolor{curcolor}{0 0 0}
\pscustom[linestyle=none,fillstyle=solid,fillcolor=curcolor]
{
\newpath
\moveto(487.79131155,322.42616631)
\curveto(487.42783531,323.12151217)(486.8273076,323.48498947)(486.05294517,323.48498947)
\curveto(485.38920594,323.48498947)(484.56743103,323.15311891)(483.52441224,322.53678962)
\lineto(483.52441224,323.40597282)
\lineto(482.92388567,323.40597282)
\lineto(481.09069929,322.83705291)
\lineto(481.09069929,322.34714965)
\lineto(482.21273578,322.34714965)
\lineto(482.21273578,316.6263439)
\lineto(481.94407916,316.37349061)
\lineto(481.09069929,316.29447396)
\lineto(481.09069929,315.72555405)
\lineto(484.55162874,315.72555405)
\lineto(484.55162874,316.29447396)
\lineto(483.79306886,316.37349061)
\lineto(483.52441224,316.6263439)
\lineto(483.52441224,321.92045972)
\curveto(484.2197581,322.26813265)(484.80448255,322.44196964)(485.32599195,322.44196964)
\curveto(486.27419085,322.44196964)(486.71668506,321.92045862)(486.71668506,320.81422657)
\lineto(486.71668506,316.6263439)
\lineto(486.44802844,316.37349061)
\lineto(485.68946856,316.29447396)
\lineto(485.68946856,315.72555405)
\lineto(489.05557802,315.72555405)
\lineto(489.05557802,316.29447396)
\lineto(488.31282147,316.37349061)
\lineto(488.02836152,316.6263439)
\lineto(488.02836152,321.92045972)
\curveto(488.70790406,322.25232934)(489.30843182,322.44196964)(489.8141379,322.44196964)
\curveto(490.79394343,322.44196964)(491.23643767,321.92045862)(491.23643767,320.81422657)
\lineto(491.23643767,316.6263439)
\lineto(490.95197771,316.37349061)
\lineto(490.20922117,316.29447396)
\lineto(490.20922117,315.72555405)
\lineto(493.65434728,315.72555405)
\lineto(493.65434728,316.29447396)
\lineto(492.81677075,316.37349061)
\lineto(492.54811412,316.6263439)
\lineto(492.54811412,321.08288319)
\curveto(492.54811412,322.64741137)(491.75794634,323.48498947)(490.50948445,323.48498947)
\curveto(489.87735185,323.48498947)(488.97656018,323.12151217)(487.79131155,322.42616631)
}
}
{
\newrgbcolor{curcolor}{0 0 0}
\pscustom[linestyle=none,fillstyle=solid,fillcolor=curcolor]
{
\newpath
\moveto(498.20175572,323.48498947)
\curveto(496.06830819,323.48498947)(494.53538297,321.88885071)(494.53538297,319.53415677)
\curveto(494.53538297,317.19526615)(496.08411151,315.58332407)(498.20175572,315.58332407)
\curveto(500.31939993,315.58332407)(501.88393179,317.19526615)(501.88393179,319.53415677)
\curveto(501.88393179,321.87304739)(500.33520324,323.48498947)(498.20175572,323.48498947)
\moveto(498.20175572,322.80544625)
\curveto(499.48182423,322.80544625)(500.3984187,321.57278441)(500.3984187,319.53415677)
\curveto(500.3984187,317.47972582)(499.48182423,316.2628673)(498.20175572,316.2628673)
\curveto(496.93749052,316.2628673)(496.02089607,317.49552914)(496.02089607,319.53415677)
\curveto(496.02089607,321.58858772)(496.9216872,322.80544625)(498.20175572,322.80544625)
}
}
{
\newrgbcolor{curcolor}{0 0 0}
\pscustom[linestyle=none,fillstyle=solid,fillcolor=curcolor]
{
\newpath
\moveto(508.52231844,326.29798235)
\lineto(508.52231844,323.32695616)
\curveto(508.03241567,323.43757937)(507.55831482,323.5007928)(507.115822,323.5007928)
\curveto(504.71371812,323.5007928)(503.03856265,321.80983388)(503.03856265,319.28130348)
\curveto(503.03856265,317.06883938)(504.36604416,315.58332407)(506.0886055,315.58332407)
\curveto(506.95778782,315.58332407)(507.71634937,315.89939134)(508.52231844,316.54732725)
\lineto(508.52231844,315.72555405)
\lineto(510.8612114,315.72555405)
\lineto(510.8612114,316.29447396)
\lineto(510.10265152,316.37349061)
\lineto(509.8339949,316.6263439)
\lineto(509.8339949,327.35680552)
\lineto(509.23346832,327.35680552)
\lineto(507.30546197,326.78788561)
\lineto(507.30546197,326.29798235)
\lineto(508.52231844,326.29798235)
\moveto(508.52231844,317.13205049)
\curveto(507.93759578,316.76857424)(507.36867472,316.57893391)(506.79975538,316.57893391)
\curveto(505.37745703,316.57893391)(504.42925576,317.70097225)(504.42925576,319.5499601)
\curveto(504.42925576,321.52537448)(505.53549051,322.80544625)(507.13162533,322.80544625)
\curveto(507.51090489,322.80544625)(507.98500573,322.7106261)(508.52231844,322.55259295)
\lineto(508.52231844,317.13205049)
}
}
{
\newrgbcolor{curcolor}{0 0 0}
\pscustom[linestyle=none,fillstyle=solid,fillcolor=curcolor]
{
}
}
{
\newrgbcolor{curcolor}{0 0 0}
\pscustom[linestyle=none,fillstyle=solid,fillcolor=curcolor]
{
\newpath
\moveto(518.6332401,323.34275949)
\lineto(516.29434714,322.82124958)
\lineto(516.29434714,322.42616631)
\lineto(517.59022027,322.18911635)
\lineto(516.27854381,315.72555405)
\lineto(517.62182693,315.72555405)
\lineto(518.60163344,321.06707986)
\curveto(519.56563565,321.9520655)(520.40321369,322.36295298)(520.9405264,322.36295298)
\curveto(521.54105237,322.36295298)(521.79390626,322.062689)(521.79390626,321.36734314)
\curveto(521.79390626,321.16190005)(521.76229949,320.81422611)(521.65167628,320.35592997)
\lineto(520.97213306,317.43231378)
\curveto(520.89311648,317.08464085)(520.84570641,316.78437701)(520.84570641,316.57893391)
\curveto(520.84570641,315.94680131)(521.22498712,315.63073406)(521.99934956,315.63073406)
\curveto(522.44184238,315.63073406)(523.07397688,315.78876769)(523.89574926,316.10483399)
\lineto(523.89574926,316.54732725)
\curveto(523.53227301,316.45250736)(523.24781247,316.40509727)(523.02656606,316.40509727)
\curveto(522.50505667,316.40509727)(522.20479286,316.6737544)(522.20479286,317.17946048)
\curveto(522.20479286,317.41651021)(522.28380961,317.76418418)(522.3786295,318.22248032)
\lineto(522.91594275,320.67199659)
\curveto(523.04236927,321.24091593)(523.13718938,321.68341016)(523.13718938,322.07849303)
\curveto(523.13718938,322.85285547)(522.63148197,323.34275949)(521.80970959,323.34275949)
\curveto(520.9721339,323.34275949)(519.9923253,322.82124858)(518.74386341,321.82563974)
\lineto(519.02832337,323.34275949)
\lineto(518.6332401,323.34275949)
}
}
{
\newrgbcolor{curcolor}{0 0 0}
\pscustom[linestyle=none,fillstyle=solid,fillcolor=curcolor]
{
\newpath
\moveto(527.30204754,325.83736865)
\curveto(527.68212262,325.85791325)(527.8978415,326.07363206)(527.8978415,326.36125644)
\curveto(527.8978415,326.72078692)(527.52803777,326.99813965)(526.9425167,326.99813965)
\curveto(525.95637595,326.99813965)(525.17567943,326.28934943)(525.17567943,325.44702088)
\curveto(525.17567943,324.87177211)(525.56602778,324.59441873)(526.15154885,324.15270986)
\curveto(526.58298543,323.82399628)(526.86033822,323.58773243)(526.86033822,323.19738506)
\curveto(526.86033822,322.65295319)(526.51107911,322.32423873)(525.92555804,322.32423873)
\curveto(525.60711676,322.32423873)(525.35030853,322.44750668)(525.18595174,322.68376957)
\lineto(525.37085332,323.19738506)
\curveto(525.20649652,323.32065265)(525.07295619,323.38228663)(524.929144,323.38228663)
\curveto(524.61070272,323.38228663)(524.39498389,323.17684012)(524.39498389,322.85839884)
\curveto(524.39498389,322.32423926)(524.89832793,321.95443558)(525.76120108,321.95443558)
\curveto(526.88088172,321.95443558)(527.64103376,322.64268123)(527.64103376,323.53637128)
\curveto(527.64103376,324.15270924)(527.21986845,324.41979038)(526.60353048,324.89231616)
\curveto(526.12073241,325.26211894)(525.83310725,325.47783818)(525.83310725,325.84764096)
\curveto(525.83310725,326.24826064)(526.14127695,326.53588571)(526.55216893,326.53588571)
\curveto(526.66516422,326.53588571)(526.76788754,326.50506871)(526.87061053,326.43316261)
\lineto(527.30204754,325.83736865)
}
}
{
\newrgbcolor{curcolor}{0 0 0}
\pscustom[linestyle=none,fillstyle=solid,fillcolor=curcolor]
{
\newpath
\moveto(532.347197,328.16918295)
\lineto(532.347197,325.40593164)
\lineto(529.63530723,325.40593164)
\lineto(529.63530723,324.82040999)
\lineto(532.347197,324.82040999)
\lineto(532.347197,322.04688637)
\lineto(532.94299096,322.04688637)
\lineto(532.94299096,324.82040999)
\lineto(535.64460841,324.82040999)
\lineto(535.64460841,325.40593164)
\lineto(532.94299096,325.40593164)
\lineto(532.94299096,328.16918295)
\lineto(532.347197,328.16918295)
}
}
{
\newrgbcolor{curcolor}{0 0 0}
\pscustom[linestyle=none,fillstyle=solid,fillcolor=curcolor]
{
\newpath
\moveto(540.83196421,329.0628739)
\lineto(538.43851605,328.38490146)
\lineto(538.43851605,327.96373676)
\lineto(539.92800096,328.15891064)
\lineto(540.03072405,328.05618755)
\lineto(539.00349308,322.72485881)
\lineto(538.83913613,322.55022954)
\lineto(537.77081592,322.48859569)
\lineto(537.77081592,322.04688637)
\lineto(540.84223652,322.04688637)
\lineto(540.84223652,322.47832338)
\lineto(539.94854558,322.52968492)
\lineto(539.85609479,322.68376957)
\lineto(541.07849964,329.0628739)
\lineto(540.83196421,329.0628739)
}
}
{
\newrgbcolor{curcolor}{0 0 0}
\pscustom[linestyle=none,fillstyle=solid,fillcolor=curcolor]
{
\newpath
\moveto(440.46258516,307.53506198)
\lineto(438.06047888,307.06096206)
\lineto(438.06047888,306.66587879)
\lineto(439.43536866,306.4130255)
\lineto(437.58637895,296.45692709)
\curveto(438.34493807,295.98282764)(439.15090954,295.76158054)(439.98848523,295.76158054)
\curveto(441.18953717,295.76158054)(442.29577234,296.28309143)(443.1017414,297.26289696)
\curveto(443.90771047,298.24270249)(444.3344012,299.5069713)(444.3344012,300.86605639)
\curveto(444.3344012,302.49379783)(443.46521683,303.55262262)(442.2799682,303.55262262)
\curveto(441.6478356,303.55262262)(440.87347083,303.15753856)(439.94107524,302.36737281)
\lineto(440.62061847,306.08115555)
\curveto(440.71543836,306.61846826)(440.84186524,307.09256916)(440.98409508,307.53506198)
\lineto(440.46258516,307.53506198)
\moveto(438.92966207,297.04165033)
\lineto(439.83045193,301.75104291)
\curveto(440.58901105,302.27255231)(441.15793228,302.58861944)(441.72685162,302.58861944)
\curveto(442.50121406,302.58861944)(443.03852808,301.87746834)(443.03852808,300.66061309)
\curveto(443.03852808,299.57018435)(442.65924763,298.38493269)(442.15354155,297.62637357)
\curveto(441.61622884,296.82040451)(440.99989774,296.4095171)(440.33615851,296.4095171)
\curveto(439.81464912,296.4095171)(439.35635158,296.61496083)(438.92966207,297.04165033)
}
}
{
\newrgbcolor{curcolor}{0 0 0}
\pscustom[linestyle=none,fillstyle=solid,fillcolor=curcolor]
{
\newpath
\moveto(447.17603762,302.54120945)
\lineto(447.17603762,301.6404196)
\lineto(456.42098614,301.6404196)
\lineto(456.42098614,302.54120945)
\lineto(447.17603762,302.54120945)
\moveto(447.17603762,299.58598659)
\lineto(447.17603762,298.68519674)
\lineto(456.42098614,298.68519674)
\lineto(456.42098614,299.58598659)
\lineto(447.17603762,299.58598659)
}
}
{
\newrgbcolor{curcolor}{0 0 0}
\pscustom[linestyle=none,fillstyle=solid,fillcolor=curcolor]
{
\newpath
\moveto(459.86586533,303.55262262)
\lineto(458.26972892,303.03111271)
\lineto(458.26972892,302.63602944)
\lineto(459.48658539,302.54120945)
\curveto(459.75524175,302.36737299)(459.88166874,301.92487882)(459.96068531,301.19792633)
\lineto(460.35576858,297.49994693)
\curveto(460.3715719,297.31030715)(460.38737525,297.1206668)(460.38737525,296.93102702)
\curveto(460.38737525,296.55174746)(460.32416176,296.2040735)(460.16612861,295.90381052)
\lineto(460.92468849,295.90381052)
\curveto(463.86410508,299.22250666)(465.25480113,301.56140403)(465.25480113,302.65183277)
\curveto(465.25480113,303.22075211)(464.95453731,303.55262262)(464.4172246,303.55262262)
\curveto(464.14856825,303.55262262)(463.8799111,303.4419991)(463.62705806,303.23655601)
\lineto(463.92732135,301.30854965)
\curveto(463.51643516,300.21812091)(462.71046367,298.9380485)(461.50941173,297.40512694)
\lineto(461.09852513,301.73523958)
\curveto(460.97209861,302.99950478)(460.79826095,303.55262262)(459.89747199,303.55262262)
\lineto(459.86586533,303.55262262)
}
}
{
\newrgbcolor{curcolor}{0 0 0}
\pscustom[linestyle=none,fillstyle=solid,fillcolor=curcolor]
{
\newpath
\moveto(467.37523865,307.17639611)
\lineto(465.85493682,306.83740989)
\lineto(465.85493682,306.58060215)
\lineto(466.69726621,306.4265175)
\lineto(465.84466451,302.22514284)
\lineto(466.70753852,302.22514284)
\lineto(467.12870322,304.63913562)
\curveto(467.29306001,305.58418716)(467.62177509,306.13889322)(468.62846044,306.5292406)
\lineto(468.89554049,305.97453588)
\lineto(468.96744666,305.97453588)
\curveto(469.38861093,305.97453588)(469.65569141,306.25188863)(469.65569141,306.64223601)
\curveto(469.65569141,306.95040499)(469.42970024,307.17639611)(469.08044206,307.17639611)
\curveto(468.56682709,307.17639611)(467.9093981,306.70386893)(467.25197094,305.76908968)
\lineto(467.58068485,307.17639611)
\lineto(467.37523865,307.17639611)
}
}
{
\newrgbcolor{curcolor}{0 0 0}
\pscustom[linestyle=none,fillstyle=solid,fillcolor=curcolor]
{
}
}
{
\newrgbcolor{curcolor}{0 0 0}
\pscustom[linestyle=none,fillstyle=solid,fillcolor=curcolor]
{
\newpath
\moveto(479.67358499,302.60442277)
\curveto(479.31010875,303.29976863)(478.70958104,303.66324594)(477.9352186,303.66324594)
\curveto(477.27147937,303.66324594)(476.44970446,303.33137538)(475.40668567,302.71504609)
\lineto(475.40668567,303.58422928)
\lineto(474.8061591,303.58422928)
\lineto(472.97297273,303.01530938)
\lineto(472.97297273,302.52540612)
\lineto(474.09500922,302.52540612)
\lineto(474.09500922,296.80460037)
\lineto(473.82635259,296.55174708)
\lineto(472.97297273,296.47273042)
\lineto(472.97297273,295.90381052)
\lineto(476.43390218,295.90381052)
\lineto(476.43390218,296.47273042)
\lineto(475.6753423,296.55174708)
\lineto(475.40668567,296.80460037)
\lineto(475.40668567,302.09871619)
\curveto(476.10203153,302.44638912)(476.68675599,302.62022611)(477.20826539,302.62022611)
\curveto(478.15646429,302.62022611)(478.5989585,302.09871508)(478.5989585,300.99248303)
\lineto(478.5989585,296.80460037)
\lineto(478.33030187,296.55174708)
\lineto(477.57174199,296.47273042)
\lineto(477.57174199,295.90381052)
\lineto(480.93785145,295.90381052)
\lineto(480.93785145,296.47273042)
\lineto(480.19509491,296.55174708)
\lineto(479.91063495,296.80460037)
\lineto(479.91063495,302.09871619)
\curveto(480.5901775,302.4305858)(481.19070525,302.62022611)(481.69641133,302.62022611)
\curveto(482.67621686,302.62022611)(483.1187111,302.09871508)(483.1187111,300.99248303)
\lineto(483.1187111,296.80460037)
\lineto(482.83425115,296.55174708)
\lineto(482.0914946,296.47273042)
\lineto(482.0914946,295.90381052)
\lineto(485.53662072,295.90381052)
\lineto(485.53662072,296.47273042)
\lineto(484.69904418,296.55174708)
\lineto(484.43038756,296.80460037)
\lineto(484.43038756,301.26113966)
\curveto(484.43038756,302.82566784)(483.64021977,303.66324594)(482.39175789,303.66324594)
\curveto(481.75962529,303.66324594)(480.85883362,303.29976863)(479.67358499,302.60442277)
}
}
{
\newrgbcolor{curcolor}{0 0 0}
\pscustom[linestyle=none,fillstyle=solid,fillcolor=curcolor]
{
\newpath
\moveto(490.08402916,303.66324594)
\curveto(487.95058163,303.66324594)(486.41765641,302.06710717)(486.41765641,299.71241324)
\curveto(486.41765641,297.37352262)(487.96638495,295.76158054)(490.08402916,295.76158054)
\curveto(492.20167336,295.76158054)(493.76620523,297.37352262)(493.76620523,299.71241324)
\curveto(493.76620523,302.05130386)(492.21747668,303.66324594)(490.08402916,303.66324594)
\moveto(490.08402916,302.98370271)
\curveto(491.36409767,302.98370271)(492.28069214,301.75104087)(492.28069214,299.71241324)
\curveto(492.28069214,297.65798229)(491.36409767,296.44112376)(490.08402916,296.44112376)
\curveto(488.81976396,296.44112376)(487.9031695,297.6737856)(487.9031695,299.71241324)
\curveto(487.9031695,301.76684419)(488.80396064,302.98370271)(490.08402916,302.98370271)
}
}
{
\newrgbcolor{curcolor}{0 0 0}
\pscustom[linestyle=none,fillstyle=solid,fillcolor=curcolor]
{
\newpath
\moveto(500.40459188,306.47623882)
\lineto(500.40459188,303.50521263)
\curveto(499.91468911,303.61583584)(499.44058825,303.67904927)(498.99809543,303.67904927)
\curveto(496.59599155,303.67904927)(494.92083609,301.98809035)(494.92083609,299.45955995)
\curveto(494.92083609,297.24709585)(496.2483176,295.76158054)(497.97087893,295.76158054)
\curveto(498.84006126,295.76158054)(499.59862281,296.0776478)(500.40459188,296.72558372)
\lineto(500.40459188,295.90381052)
\lineto(502.74348483,295.90381052)
\lineto(502.74348483,296.47273042)
\lineto(501.98492496,296.55174708)
\lineto(501.71626833,296.80460037)
\lineto(501.71626833,307.53506198)
\lineto(501.11574176,307.53506198)
\lineto(499.1877354,306.96614208)
\lineto(499.1877354,306.47623882)
\lineto(500.40459188,306.47623882)
\moveto(500.40459188,297.31030696)
\curveto(499.81986922,296.94683071)(499.25094816,296.75719038)(498.68202882,296.75719038)
\curveto(497.25973047,296.75719038)(496.3115292,297.87922871)(496.3115292,299.72821657)
\curveto(496.3115292,301.70363094)(497.41776395,302.98370271)(499.01389877,302.98370271)
\curveto(499.39317833,302.98370271)(499.86727917,302.88888257)(500.40459188,302.73084942)
\lineto(500.40459188,297.31030696)
}
}
{
\newrgbcolor{curcolor}{0 0 0}
\pscustom[linestyle=none,fillstyle=solid,fillcolor=curcolor]
{
}
}
{
\newrgbcolor{curcolor}{0 0 0}
\pscustom[linestyle=none,fillstyle=solid,fillcolor=curcolor]
{
\newpath
\moveto(510.51551354,303.52101596)
\lineto(508.17662058,302.99950604)
\lineto(508.17662058,302.60442277)
\lineto(509.4724937,302.36737281)
\lineto(508.16081725,295.90381052)
\lineto(509.50410036,295.90381052)
\lineto(510.48390687,301.24533633)
\curveto(511.44790909,302.13032197)(512.28548712,302.54120945)(512.82279983,302.54120945)
\curveto(513.4233258,302.54120945)(513.6761797,302.24094547)(513.6761797,301.54559961)
\curveto(513.6761797,301.34015652)(513.64457292,300.99248257)(513.53394972,300.53418644)
\lineto(512.85440649,297.61057024)
\curveto(512.77538992,297.26289731)(512.72797985,296.96263347)(512.72797985,296.75719038)
\curveto(512.72797985,296.12505778)(513.10726056,295.80899053)(513.881623,295.80899053)
\curveto(514.32411582,295.80899053)(514.95625031,295.96702415)(515.77802269,296.28309045)
\lineto(515.77802269,296.72558372)
\curveto(515.41454645,296.63076383)(515.13008591,296.58335374)(514.9088395,296.58335374)
\curveto(514.3873301,296.58335374)(514.0870663,296.85201087)(514.0870663,297.35771695)
\curveto(514.0870663,297.59476667)(514.16608305,297.94244065)(514.26090294,298.40073678)
\lineto(514.79821618,300.85025306)
\curveto(514.9246427,301.4191724)(515.01946281,301.86166662)(515.01946281,302.2567495)
\curveto(515.01946281,303.03111193)(514.51375541,303.52101596)(513.69198303,303.52101596)
\curveto(512.85440733,303.52101596)(511.87459874,302.99950505)(510.62613685,302.0038962)
\lineto(510.91059681,303.52101596)
\lineto(510.51551354,303.52101596)
}
}
{
\newrgbcolor{curcolor}{0 0 0}
\pscustom[linestyle=none,fillstyle=solid,fillcolor=curcolor]
{
\newpath
\moveto(519.18432098,306.01562512)
\curveto(519.56439606,306.03616972)(519.78011494,306.25188853)(519.78011494,306.53951291)
\curveto(519.78011494,306.89904339)(519.4103112,307.17639611)(518.82479014,307.17639611)
\curveto(517.83864939,307.17639611)(517.05795287,306.4676059)(517.05795287,305.62527735)
\curveto(517.05795287,305.05002858)(517.44830122,304.7726752)(518.03382229,304.33096632)
\curveto(518.46525887,304.00225274)(518.74261166,303.7659889)(518.74261166,303.37564152)
\curveto(518.74261166,302.83120965)(518.39335254,302.5024952)(517.80783148,302.5024952)
\curveto(517.48939019,302.5024952)(517.23258197,302.62576315)(517.06822518,302.86202604)
\lineto(517.25312675,303.37564152)
\curveto(517.08876996,303.49890912)(516.95522963,303.5605431)(516.81141743,303.5605431)
\curveto(516.49297615,303.5605431)(516.27725733,303.35509658)(516.27725733,303.0366553)
\curveto(516.27725733,302.50249573)(516.78060137,302.13269205)(517.64347452,302.13269205)
\curveto(518.76315516,302.13269205)(519.5233072,302.82093769)(519.5233072,303.71462774)
\curveto(519.5233072,304.33096571)(519.10214188,304.59804685)(518.48580392,305.07057262)
\curveto(518.00300584,305.4403754)(517.71538069,305.65609465)(517.71538069,306.02589743)
\curveto(517.71538069,306.4265171)(518.02355039,306.71414218)(518.43444237,306.71414218)
\curveto(518.54743766,306.71414218)(518.65016097,306.68332518)(518.75288397,306.61141908)
\lineto(519.18432098,306.01562512)
}
}
{
\newrgbcolor{curcolor}{0 0 0}
\pscustom[linestyle=none,fillstyle=solid,fillcolor=curcolor]
{
\newpath
\moveto(524.22947043,308.34743942)
\lineto(524.22947043,305.58418811)
\lineto(521.51758067,305.58418811)
\lineto(521.51758067,304.99866646)
\lineto(524.22947043,304.99866646)
\lineto(524.22947043,302.22514284)
\lineto(524.82526439,302.22514284)
\lineto(524.82526439,304.99866646)
\lineto(527.52688185,304.99866646)
\lineto(527.52688185,305.58418811)
\lineto(524.82526439,305.58418811)
\lineto(524.82526439,308.34743942)
\lineto(524.22947043,308.34743942)
}
}
{
\newrgbcolor{curcolor}{0 0 0}
\pscustom[linestyle=none,fillstyle=solid,fillcolor=curcolor]
{
\newpath
\moveto(532.71423765,309.24113036)
\lineto(530.32078949,308.56315792)
\lineto(530.32078949,308.14199323)
\lineto(531.81027439,308.33716711)
\lineto(531.91299749,308.23444401)
\lineto(530.88576652,302.90311528)
\lineto(530.72140957,302.72848601)
\lineto(529.65308936,302.66685215)
\lineto(529.65308936,302.22514284)
\lineto(532.72450996,302.22514284)
\lineto(532.72450996,302.65657984)
\lineto(531.83081901,302.70794139)
\lineto(531.73836823,302.86202604)
\lineto(532.96077308,309.24113036)
\lineto(532.71423765,309.24113036)
}
}
{
\newrgbcolor{curcolor}{0 0 0}
\pscustom[linestyle=none,fillstyle=solid,fillcolor=curcolor]
{
\newpath
\moveto(438.51877547,286.68610195)
\lineto(438.51877547,286.08557538)
\lineto(439.54599197,286.00655873)
\lineto(439.70402528,285.76950876)
\lineto(438.01306888,276.99866017)
\lineto(437.76021559,276.77741354)
\lineto(436.74880242,276.68259355)
\lineto(436.74880242,276.08206698)
\lineto(440.58901181,276.08206698)
\lineto(440.58901181,276.68259355)
\lineto(439.57759863,276.77741354)
\lineto(439.38795866,276.99866017)
\lineto(440.20973187,281.26555949)
\lineto(445.34581438,281.26555949)
\lineto(444.53984451,276.99866017)
\lineto(444.30279454,276.77741354)
\lineto(443.27557804,276.68259355)
\lineto(443.27557804,276.08206698)
\lineto(447.09998409,276.08206698)
\lineto(447.09998409,276.68259355)
\lineto(446.08857092,276.77741354)
\lineto(445.93053762,276.99866017)
\lineto(447.60569068,285.76950876)
\lineto(447.85854397,286.00655873)
\lineto(448.90156381,286.08557538)
\lineto(448.90156381,286.68610195)
\lineto(445.04555109,286.68610195)
\lineto(445.04555109,286.08557538)
\lineto(446.05696426,286.00655873)
\lineto(446.24660423,285.76950876)
\lineto(445.50384768,281.9925127)
\lineto(440.35196184,281.9925127)
\lineto(441.09471839,285.76950876)
\lineto(441.33176835,286.00655873)
\lineto(442.37478819,286.08557538)
\lineto(442.37478819,286.68610195)
\lineto(438.51877547,286.68610195)
}
}
{
\newrgbcolor{curcolor}{0 0 0}
\pscustom[linestyle=none,fillstyle=solid,fillcolor=curcolor]
{
}
}
{
\newrgbcolor{curcolor}{0 0 0}
\pscustom[linestyle=none,fillstyle=solid,fillcolor=curcolor]
{
\newpath
\moveto(454.7999101,282.71946592)
\lineto(454.7999101,281.81867606)
\lineto(464.04485862,281.81867606)
\lineto(464.04485862,282.71946592)
\lineto(454.7999101,282.71946592)
\moveto(454.7999101,279.76424306)
\lineto(454.7999101,278.8634532)
\lineto(464.04485862,278.8634532)
\lineto(464.04485862,279.76424306)
\lineto(454.7999101,279.76424306)
}
}
{
\newrgbcolor{curcolor}{0 0 0}
\pscustom[linestyle=none,fillstyle=solid,fillcolor=curcolor]
{
}
}
{
\newrgbcolor{curcolor}{0 0 0}
\pscustom[linestyle=none,fillstyle=solid,fillcolor=curcolor]
{
\newpath
\moveto(471.73046285,286.68610195)
\lineto(471.73046285,286.08557538)
\lineto(472.75767936,286.00655873)
\lineto(472.91571266,285.76950876)
\lineto(471.22475627,276.99866017)
\lineto(470.97190298,276.77741354)
\lineto(469.96048981,276.68259355)
\lineto(469.96048981,276.08206698)
\lineto(473.80069919,276.08206698)
\lineto(473.80069919,276.68259355)
\lineto(472.78928602,276.77741354)
\lineto(472.59964605,276.99866017)
\lineto(473.42141925,281.26555949)
\lineto(478.55750176,281.26555949)
\lineto(477.75153189,276.99866017)
\lineto(477.51448193,276.77741354)
\lineto(476.48726543,276.68259355)
\lineto(476.48726543,276.08206698)
\lineto(480.31167148,276.08206698)
\lineto(480.31167148,276.68259355)
\lineto(479.30025831,276.77741354)
\lineto(479.142225,276.99866017)
\lineto(480.81737806,285.76950876)
\lineto(481.07023136,286.00655873)
\lineto(482.11325119,286.08557538)
\lineto(482.11325119,286.68610195)
\lineto(478.25723848,286.68610195)
\lineto(478.25723848,286.08557538)
\lineto(479.26865165,286.00655873)
\lineto(479.45829162,285.76950876)
\lineto(478.71553507,281.9925127)
\lineto(473.56364923,281.9925127)
\lineto(474.30640578,285.76950876)
\lineto(474.54345574,286.00655873)
\lineto(475.58647557,286.08557538)
\lineto(475.58647557,286.68610195)
\lineto(471.73046285,286.68610195)
}
}
{
\newrgbcolor{curcolor}{0 0 0}
\pscustom[linestyle=none,fillstyle=solid,fillcolor=curcolor]
{
\newpath
\moveto(488.14839507,283.69927243)
\lineto(487.83232845,283.35159915)
\curveto(487.34242569,283.58864888)(486.8525215,283.69927243)(486.41002868,283.69927243)
\curveto(484.84550049,283.69927243)(483.56542833,282.95651455)(482.75945926,281.62903609)
\curveto(482.22214655,280.74405045)(481.93768606,279.764242)(481.93768606,278.7054199)
\curveto(481.93768606,276.96705525)(482.77526394,275.93983701)(484.11854571,275.93983701)
\curveto(484.87710483,275.93983701)(485.63566631,276.38233115)(486.473242,277.26731679)
\lineto(486.473242,277.14089015)
\curveto(486.473242,276.33492108)(486.77350595,275.93983701)(487.43724518,275.93983701)
\curveto(487.86393469,275.93983701)(488.51187254,276.12947734)(489.38105487,276.49295358)
\lineto(489.38105487,276.96705351)
\curveto(488.95436536,276.87223362)(488.66990481,276.8090202)(488.49606834,276.8090202)
\curveto(487.91134569,276.8090202)(487.64268848,277.03026734)(487.64268848,277.53597342)
\curveto(487.64268848,277.75721983)(487.69009858,278.12069718)(487.80072179,278.64220657)
\lineto(488.3696417,281.43939612)
\curveto(488.60669142,282.60884143)(488.81213512,283.36740281)(488.97016827,283.69927243)
\lineto(488.14839507,283.69927243)
\moveto(486.6786853,283.08294253)
\lineto(487.31081853,282.11893935)
\lineto(486.48904533,277.94686002)
\curveto(485.84110942,277.25151416)(485.22477827,276.90384018)(484.6242523,276.90384018)
\curveto(483.70766003,276.90384018)(483.24936252,277.53597468)(483.24936252,278.80023988)
\curveto(483.24936252,281.31296696)(484.60845082,283.08294253)(486.45743867,283.08294253)
\lineto(486.6786853,283.08294253)
}
}
{
\newrgbcolor{curcolor}{0 0 0}
\pscustom[linestyle=none,fillstyle=solid,fillcolor=curcolor]
{
\newpath
\moveto(494.17563728,281.91349605)
\curveto(494.76035993,281.94510268)(495.09223046,282.2769731)(495.09223046,282.71946592)
\curveto(495.09223046,283.27258194)(494.52330965,283.69927243)(493.6225207,283.69927243)
\curveto(492.10540246,283.69927243)(490.9043478,282.60884131)(490.9043478,281.31296948)
\curveto(490.9043478,280.42798384)(491.50487527,280.00129234)(492.40566423,279.3217498)
\curveto(493.06940346,278.81604372)(493.49609405,278.452566)(493.49609405,277.85204003)
\curveto(493.49609405,277.01446434)(492.95877991,276.50875691)(492.05799095,276.50875691)
\curveto(491.56808819,276.50875691)(491.17300417,276.69839725)(490.92015113,277.06187349)
\lineto(491.20461109,277.85204003)
\curveto(490.95175805,278.04167981)(490.74631427,278.13649999)(490.52506786,278.13649999)
\curveto(490.0351651,278.13649999)(489.70329466,277.82043288)(489.70329466,277.33053012)
\curveto(489.70329466,276.50875774)(490.4776592,275.93983701)(491.80513766,275.93983701)
\curveto(493.52769899,275.93983701)(494.69714719,276.99866154)(494.69714719,278.37354995)
\curveto(494.69714719,279.32174885)(494.04920968,279.73263712)(493.10101078,280.45958961)
\curveto(492.35825498,281.02850895)(491.91576097,281.36038004)(491.91576097,281.92929938)
\curveto(491.91576097,282.54562866)(492.38986153,282.98812254)(493.02199413,282.98812254)
\curveto(493.19583059,282.98812254)(493.35386423,282.94071244)(493.51189738,282.83008923)
\lineto(494.17563728,281.91349605)
}
}
{
\newrgbcolor{curcolor}{0 0 0}
\pscustom[linestyle=none,fillstyle=solid,fillcolor=curcolor]
{
\newpath
\moveto(499.32974546,287.71331845)
\lineto(496.84862253,287.19180854)
\lineto(496.84862253,286.79672527)
\lineto(498.33413562,286.67029862)
\lineto(496.1690793,276.08206698)
\lineto(497.51236242,276.08206698)
\lineto(498.49216893,281.43939612)
\curveto(499.47197446,282.32438176)(500.30955249,282.73526925)(500.83106189,282.73526925)
\curveto(501.43158786,282.73526925)(501.68444175,282.40339862)(501.68444175,281.72385608)
\curveto(501.68444175,281.51841298)(501.63703165,281.17073904)(501.52640844,280.71244291)
\lineto(500.84686522,277.78882671)
\curveto(500.76784864,277.44115378)(500.7362419,277.15669327)(500.7362419,276.95125018)
\curveto(500.7362419,276.28751095)(501.0997193,275.987247)(501.88988505,275.987247)
\curveto(502.30077124,275.987247)(502.94870905,276.14528064)(503.78628475,276.47715025)
\lineto(503.78628475,276.90384018)
\curveto(503.4228085,276.80902029)(503.12254463,276.76161021)(502.90129822,276.76161021)
\curveto(502.37978883,276.76161021)(502.09532835,277.04607067)(502.09532835,277.55177675)
\curveto(502.09532835,277.8204331)(502.15854177,278.15230374)(502.25336166,278.57899325)
\lineto(502.80647824,281.02850952)
\curveto(502.93290476,281.59742886)(503.02772487,282.05572642)(503.02772487,282.4508093)
\curveto(503.02772487,283.22517173)(502.52201746,283.71507576)(501.70024508,283.71507576)
\curveto(500.8784727,283.71507576)(499.89866414,283.20936818)(498.66600557,282.21375933)
\lineto(499.18751548,284.9161289)
\curveto(499.40876189,286.02236095)(499.61420562,286.95475933)(499.81964872,287.71331845)
\lineto(499.32974546,287.71331845)
}
}
{
\newrgbcolor{curcolor}{0 0 0}
\pscustom[linestyle=none,fillstyle=solid,fillcolor=curcolor]
{
\newpath
\moveto(511.76350982,287.4288585)
\curveto(510.6888844,287.34984192)(509.63005926,286.85993775)(508.72927031,286.02236206)
\curveto(507.78107141,285.15317973)(507.11733014,284.17337117)(506.69064064,282.98812254)
\curveto(506.05850804,281.21815126)(505.75824412,279.60620839)(505.75824412,278.2471233)
\curveto(505.75824412,275.73439622)(507.14893924,273.99602716)(509.15596024,273.83799401)
\lineto(509.28238689,274.5017339)
\curveto(507.84428522,274.72298031)(507.05411724,275.92403574)(507.05411724,277.99427001)
\curveto(507.05411724,279.11630537)(507.30697115,280.74405162)(507.92330044,282.79848257)
\curveto(508.23936674,283.85730468)(508.71346767,284.75809637)(509.40881353,285.5324588)
\curveto(510.0093395,286.21200135)(510.79950748,286.63869208)(511.63708318,286.7651186)
\lineto(511.76350982,287.4288585)
}
}
{
\newrgbcolor{curcolor}{0 0 0}
\pscustom[linestyle=none,fillstyle=solid,fillcolor=curcolor]
{
\newpath
\moveto(515.99756784,281.77126607)
\curveto(516.59809381,281.78706939)(516.99317768,282.1505465)(516.99317768,282.64044926)
\curveto(516.99317768,283.24097523)(516.39265024,283.74668242)(515.52346792,283.74668242)
\curveto(514.54366239,283.74668242)(513.50064077,283.25677828)(512.6946717,282.37179264)
\curveto(511.85709601,281.45520037)(511.38299525,280.20673491)(511.38299525,278.80023988)
\curveto(511.38299525,277.01446529)(512.33119649,275.92403367)(513.72188821,275.92403367)
\curveto(514.65428379,275.92403367)(515.69730554,276.42974129)(516.67711107,277.45695676)
\lineto(516.67711107,278.18390998)
\curveto(515.82373206,277.31472765)(514.93874395,276.87223352)(514.21179146,276.87223352)
\curveto(513.23198593,276.87223352)(512.63145838,277.64659811)(512.63145838,279.02148651)
\curveto(512.63145838,281.37618045)(513.72188949,282.97231921)(515.001958,282.97231921)
\lineto(515.08097466,282.97231921)
\lineto(515.99756784,281.77126607)
}
}
{
\newrgbcolor{curcolor}{0 0 0}
\pscustom[linestyle=none,fillstyle=solid,fillcolor=curcolor]
{
\newpath
\moveto(516.48006329,273.41130408)
\curveto(518.40806772,273.6483538)(519.7197461,275.21288526)(519.7197461,276.68259355)
\curveto(519.7197461,277.3621361)(519.35626884,277.75722005)(518.70833293,277.75722005)
\curveto(518.07620033,277.75722005)(517.58629644,277.29892281)(517.58629644,276.65098689)
\curveto(517.58629644,276.47715043)(517.63370653,276.33492013)(517.72852642,276.1926903)
\lineto(518.73993959,276.1926903)
\curveto(518.56610312,274.94422841)(517.83914847,274.16986372)(516.57488327,273.93281399)
\lineto(516.48006329,273.41130408)
}
}
{
\newrgbcolor{curcolor}{0 0 0}
\pscustom[linestyle=none,fillstyle=solid,fillcolor=curcolor]
{
\newpath
\moveto(526.61543072,281.77126607)
\curveto(527.21595669,281.78706939)(527.61104056,282.1505465)(527.61104056,282.64044926)
\curveto(527.61104056,283.24097523)(527.01051312,283.74668242)(526.1413308,283.74668242)
\curveto(525.16152527,283.74668242)(524.11850365,283.25677828)(523.31253459,282.37179264)
\curveto(522.47495889,281.45520037)(522.00085813,280.20673491)(522.00085813,278.80023988)
\curveto(522.00085813,277.01446529)(522.94905937,275.92403367)(524.33975109,275.92403367)
\curveto(525.27214667,275.92403367)(526.31516842,276.42974129)(527.29497395,277.45695676)
\lineto(527.29497395,278.18390998)
\curveto(526.44159494,277.31472765)(525.55660683,276.87223352)(524.82965434,276.87223352)
\curveto(523.84984881,276.87223352)(523.24932126,277.64659811)(523.24932126,279.02148651)
\curveto(523.24932126,281.37618045)(524.33975237,282.97231921)(525.61982088,282.97231921)
\lineto(525.69883754,282.97231921)
\lineto(526.61543072,281.77126607)
}
}
{
\newrgbcolor{curcolor}{0 0 0}
\pscustom[linestyle=none,fillstyle=solid,fillcolor=curcolor]
{
\newpath
\moveto(531.90219371,279.93738835)
\lineto(529.50874555,279.25941591)
\lineto(529.50874555,278.83825121)
\lineto(530.99823046,279.0334251)
\lineto(531.10095355,278.930702)
\lineto(530.07372258,273.59937326)
\lineto(529.90936563,273.424744)
\lineto(528.84104542,273.36311014)
\lineto(528.84104542,272.92140082)
\lineto(531.91246602,272.92140082)
\lineto(531.91246602,273.35283783)
\lineto(531.01877508,273.40419938)
\lineto(530.92632429,273.55828402)
\lineto(532.14872914,279.93738835)
\lineto(531.90219371,279.93738835)
}
}
{
\newrgbcolor{curcolor}{0 0 0}
\pscustom[linestyle=none,fillstyle=solid,fillcolor=curcolor]
{
\newpath
\moveto(532.81590889,273.41130408)
\curveto(534.74391332,273.6483538)(536.05559171,275.21288526)(536.05559171,276.68259355)
\curveto(536.05559171,277.3621361)(535.69211445,277.75722005)(535.04417854,277.75722005)
\curveto(534.41204594,277.75722005)(533.92214205,277.29892281)(533.92214205,276.65098689)
\curveto(533.92214205,276.47715043)(533.96955214,276.33492013)(534.06437203,276.1926903)
\lineto(535.0757852,276.1926903)
\curveto(534.90194873,274.94422841)(534.17499408,274.16986372)(532.91072888,273.93281399)
\lineto(532.81590889,273.41130408)
}
}
{
\newrgbcolor{curcolor}{0 0 0}
\pscustom[linestyle=none,fillstyle=solid,fillcolor=curcolor]
{
\newpath
\moveto(539.56936354,283.73087909)
\lineto(537.97322713,283.20936917)
\lineto(537.97322713,282.8142859)
\lineto(539.1900836,282.71946592)
\curveto(539.45873996,282.54562945)(539.58516695,282.10313529)(539.66418353,281.3761828)
\lineto(540.0592668,277.67820339)
\curveto(540.07507011,277.48856361)(540.09087346,277.29892326)(540.09087346,277.10928348)
\curveto(540.09087346,276.73000392)(540.02765998,276.38232997)(539.86962683,276.08206698)
\lineto(540.6281867,276.08206698)
\curveto(543.56760329,279.40076313)(544.95829934,281.7396605)(544.95829934,282.83008923)
\curveto(544.95829934,283.39900857)(544.65803552,283.73087909)(544.12072281,283.73087909)
\curveto(543.85206646,283.73087909)(543.58340931,283.62025557)(543.33055627,283.41481247)
\lineto(543.63081956,281.48680612)
\curveto(543.21993337,280.39637738)(542.41396188,279.11630496)(541.21290994,277.58338341)
\lineto(540.80202334,281.91349605)
\curveto(540.67559682,283.17776125)(540.50175916,283.73087909)(539.6009702,283.73087909)
\lineto(539.56936354,283.73087909)
}
}
{
\newrgbcolor{curcolor}{0 0 0}
\pscustom[linestyle=none,fillstyle=solid,fillcolor=curcolor]
{
\newpath
\moveto(544.35975422,273.41130408)
\curveto(546.28775865,273.6483538)(547.59943703,275.21288526)(547.59943703,276.68259355)
\curveto(547.59943703,277.3621361)(547.23595978,277.75722005)(546.58802386,277.75722005)
\curveto(545.95589126,277.75722005)(545.46598737,277.29892281)(545.46598737,276.65098689)
\curveto(545.46598737,276.47715043)(545.51339746,276.33492013)(545.60821735,276.1926903)
\lineto(546.61963052,276.1926903)
\curveto(546.44579406,274.94422841)(545.7188394,274.16986372)(544.4545742,273.93281399)
\lineto(544.35975422,273.41130408)
}
}
{
\newrgbcolor{curcolor}{0 0 0}
\pscustom[linestyle=none,fillstyle=solid,fillcolor=curcolor]
{
\newpath
\moveto(551.11320886,283.73087909)
\lineto(549.51707245,283.20936917)
\lineto(549.51707245,282.8142859)
\lineto(550.73392893,282.71946592)
\curveto(551.00258528,282.54562945)(551.12901227,282.10313529)(551.20802885,281.3761828)
\lineto(551.60311212,277.67820339)
\curveto(551.61891543,277.48856361)(551.63471878,277.29892326)(551.63471878,277.10928348)
\curveto(551.63471878,276.73000392)(551.5715053,276.38232997)(551.41347215,276.08206698)
\lineto(552.17203203,276.08206698)
\curveto(555.11144862,279.40076313)(556.50214467,281.7396605)(556.50214467,282.83008923)
\curveto(556.50214467,283.39900857)(556.20188084,283.73087909)(555.66456813,283.73087909)
\curveto(555.39591178,283.73087909)(555.12725463,283.62025557)(554.87440159,283.41481247)
\lineto(555.17466488,281.48680612)
\curveto(554.76377869,280.39637738)(553.95780721,279.11630496)(552.75675527,277.58338341)
\lineto(552.34586867,281.91349605)
\curveto(552.21944215,283.17776125)(552.04560448,283.73087909)(551.14481553,283.73087909)
\lineto(551.11320886,283.73087909)
}
}
{
\newrgbcolor{curcolor}{0 0 0}
\pscustom[linestyle=none,fillstyle=solid,fillcolor=curcolor]
{
\newpath
\moveto(560.70786106,279.93738835)
\lineto(558.3144129,279.25941591)
\lineto(558.3144129,278.83825121)
\lineto(559.8038978,279.0334251)
\lineto(559.9066209,278.930702)
\lineto(558.87938993,273.59937326)
\lineto(558.71503297,273.424744)
\lineto(557.64671276,273.36311014)
\lineto(557.64671276,272.92140082)
\lineto(560.71813337,272.92140082)
\lineto(560.71813337,273.35283783)
\lineto(559.82444242,273.40419938)
\lineto(559.73199163,273.55828402)
\lineto(560.95439649,279.93738835)
\lineto(560.70786106,279.93738835)
}
}
{
\newrgbcolor{curcolor}{0 0 0}
\pscustom[linestyle=none,fillstyle=solid,fillcolor=curcolor]
{
\newpath
\moveto(561.62157021,273.41130408)
\curveto(563.54957464,273.6483538)(564.86125303,275.21288526)(564.86125303,276.68259355)
\curveto(564.86125303,277.3621361)(564.49777577,277.75722005)(563.84983985,277.75722005)
\curveto(563.21770725,277.75722005)(562.72780337,277.29892281)(562.72780337,276.65098689)
\curveto(562.72780337,276.47715043)(562.77521346,276.33492013)(562.87003335,276.1926903)
\lineto(563.88144652,276.1926903)
\curveto(563.70761005,274.94422841)(562.9806554,274.16986372)(561.7163902,273.93281399)
\lineto(561.62157021,273.41130408)
}
}
{
\newrgbcolor{curcolor}{0 0 0}
\pscustom[linestyle=none,fillstyle=solid,fillcolor=curcolor]
{
\newpath
\moveto(573.38468072,283.69927243)
\lineto(573.06861411,283.35159915)
\curveto(572.57871134,283.58864888)(572.08880715,283.69927243)(571.64631433,283.69927243)
\curveto(570.08178615,283.69927243)(568.80171398,282.95651455)(567.99574492,281.62903609)
\curveto(567.45843221,280.74405045)(567.17397172,279.764242)(567.17397172,278.7054199)
\curveto(567.17397172,276.96705525)(568.01154959,275.93983701)(569.35483137,275.93983701)
\curveto(570.11339049,275.93983701)(570.87195196,276.38233115)(571.70952766,277.26731679)
\lineto(571.70952766,277.14089015)
\curveto(571.70952766,276.33492108)(572.00979161,275.93983701)(572.67353084,275.93983701)
\curveto(573.10022034,275.93983701)(573.7481582,276.12947734)(574.61734052,276.49295358)
\lineto(574.61734052,276.96705351)
\curveto(574.19065102,276.87223362)(573.90619046,276.8090202)(573.732354,276.8090202)
\curveto(573.14763134,276.8090202)(572.87897414,277.03026734)(572.87897414,277.53597342)
\curveto(572.87897414,277.75721983)(572.92638424,278.12069718)(573.03700744,278.64220657)
\lineto(573.60592735,281.43939612)
\curveto(573.84297708,282.60884143)(574.04842077,283.36740281)(574.20645392,283.69927243)
\lineto(573.38468072,283.69927243)
\moveto(571.91497096,283.08294253)
\lineto(572.54710419,282.11893935)
\lineto(571.72533099,277.94686002)
\curveto(571.07739507,277.25151416)(570.46106392,276.90384018)(569.86053795,276.90384018)
\curveto(568.94394568,276.90384018)(568.48564817,277.53597468)(568.48564817,278.80023988)
\curveto(568.48564817,281.31296696)(569.84473647,283.08294253)(571.69372433,283.08294253)
\lineto(571.91497096,283.08294253)
}
}
{
\newrgbcolor{curcolor}{0 0 0}
\pscustom[linestyle=none,fillstyle=solid,fillcolor=curcolor]
{
\newpath
\moveto(574.30744708,273.41130408)
\curveto(576.23545151,273.6483538)(577.5471299,275.21288526)(577.5471299,276.68259355)
\curveto(577.5471299,277.3621361)(577.18365264,277.75722005)(576.53571673,277.75722005)
\curveto(575.90358413,277.75722005)(575.41368024,277.29892281)(575.41368024,276.65098689)
\curveto(575.41368024,276.47715043)(575.46109033,276.33492013)(575.55591022,276.1926903)
\lineto(576.56732339,276.1926903)
\curveto(576.39348692,274.94422841)(575.66653227,274.16986372)(574.40226707,273.93281399)
\lineto(574.30744708,273.41130408)
}
}
{
\newrgbcolor{curcolor}{0 0 0}
\pscustom[linestyle=none,fillstyle=solid,fillcolor=curcolor]
{
\newpath
\moveto(582.86248144,287.71331845)
\lineto(580.46037516,287.23921853)
\lineto(580.46037516,286.84413526)
\lineto(581.83526494,286.59128196)
\lineto(579.98627524,276.63518356)
\curveto(580.74483436,276.16108411)(581.55080582,275.93983701)(582.38838152,275.93983701)
\curveto(583.58943346,275.93983701)(584.69566862,276.4613479)(585.50163769,277.44115343)
\curveto(586.30760675,278.42095896)(586.73429749,279.68522776)(586.73429749,281.04431285)
\curveto(586.73429749,282.6720543)(585.86511311,283.73087909)(584.67986448,283.73087909)
\curveto(584.04773188,283.73087909)(583.27336711,283.33579503)(582.34097153,282.54562928)
\lineto(583.02051475,286.25941202)
\curveto(583.11533464,286.79672473)(583.24176152,287.27082563)(583.38399136,287.71331845)
\lineto(582.86248144,287.71331845)
\moveto(581.32955835,277.2199068)
\lineto(582.23034821,281.92929938)
\curveto(582.98890733,282.45080877)(583.55782857,282.76687591)(584.12674791,282.76687591)
\curveto(584.90111034,282.76687591)(585.43842436,282.05572481)(585.43842436,280.83886955)
\curveto(585.43842436,279.74844082)(585.05914392,278.56318916)(584.55343784,277.80463004)
\curveto(584.01612513,276.99866098)(583.39979403,276.58777357)(582.7360548,276.58777357)
\curveto(582.2145454,276.58777357)(581.75624786,276.7932173)(581.32955835,277.2199068)
}
}
{
\newrgbcolor{curcolor}{0 0 0}
\pscustom[linestyle=none,fillstyle=solid,fillcolor=curcolor]
{
\newpath
\moveto(589.18085064,287.4288585)
\lineto(589.05442399,286.7651186)
\curveto(590.46091903,286.55967551)(591.2668903,285.35862017)(591.2668903,283.36740248)
\curveto(591.2668903,282.37179364)(591.06144645,280.80726087)(590.50833042,278.78443655)
\curveto(590.17646081,277.56758129)(589.67075315,276.55616608)(588.92799734,275.73439371)
\curveto(588.32747138,275.05485116)(587.5373034,274.61235711)(586.6997277,274.5017339)
\lineto(586.57330106,273.83799401)
\curveto(587.61631985,273.90120727)(588.67514498,274.39111144)(589.60754057,275.24449045)
\curveto(591.45652842,276.95124847)(592.56276343,279.74844309)(592.56276343,283.11454919)
\curveto(592.56276343,285.53245638)(591.18787164,287.25502203)(589.18085064,287.4288585)
}
}
{
\newrgbcolor{curcolor}{0 0 0}
\pscustom[linestyle=none,fillstyle=solid,fillcolor=curcolor]
{
\newpath
\moveto(443.08593807,264.07153557)
\lineto(442.78567479,263.29717237)
\curveto(442.61183832,263.21815579)(442.4538047,263.17074572)(442.31157486,263.17074572)
\curveto(441.97970525,263.17074572)(441.64783456,263.32877923)(441.23694837,263.53422233)
\curveto(440.81025886,263.75546874)(440.41517469,263.96091226)(439.94107524,263.96091226)
\curveto(438.85064651,263.96091226)(437.80762558,262.854678)(437.80762558,261.74844595)
\curveto(437.80762558,261.16372329)(438.10788931,260.81604943)(438.55038213,260.81604943)
\curveto(438.86644843,260.81604943)(439.1034989,260.9898864)(439.29313868,261.32175602)
\lineto(438.45556215,262.33316919)
\curveto(438.69261187,262.6650388)(439.0402858,262.83887577)(439.45117199,262.83887577)
\curveto(439.7830416,262.83887577)(440.25714244,262.68084216)(440.8418651,262.38057918)
\lineto(442.40639485,262.72825246)
\lineto(436.97004905,256.62816677)
\lineto(437.83923225,256.15406684)
\lineto(438.21851219,256.83361007)
\curveto(438.39234865,256.92842996)(438.55038231,256.97584005)(438.72421877,256.97584005)
\curveto(439.48277789,256.97584005)(440.28874942,256.29629682)(441.18953838,256.29629682)
\curveto(442.34318037,256.29629682)(443.38620136,257.4183344)(443.38620136,258.50876313)
\curveto(443.38620136,259.07768247)(443.05433097,259.45696298)(442.61183815,259.45696298)
\curveto(442.34318179,259.45696298)(442.07452468,259.26732263)(441.85327827,258.88804307)
\lineto(442.72246146,257.92403989)
\curveto(442.46960842,257.54476033)(442.13773783,257.35511998)(441.74265495,257.35511998)
\curveto(441.44239197,257.35511998)(440.90507765,257.52895697)(440.13071521,257.8766299)
\lineto(438.59779212,257.48154663)
\lineto(444.04994125,263.56582899)
\lineto(443.08593807,264.07153557)
}
}
{
\newrgbcolor{curcolor}{0 0 0}
\pscustom[linestyle=none,fillstyle=solid,fillcolor=curcolor]
{
}
}
{
\newrgbcolor{curcolor}{0 0 0}
\pscustom[linestyle=none,fillstyle=solid,fillcolor=curcolor]
{
\newpath
\moveto(450.47868683,262.96530242)
\lineto(450.47868683,262.06451256)
\lineto(459.72363535,262.06451256)
\lineto(459.72363535,262.96530242)
\lineto(450.47868683,262.96530242)
\moveto(450.47868683,260.01007956)
\lineto(450.47868683,259.1092897)
\lineto(459.72363535,259.1092897)
\lineto(459.72363535,260.01007956)
\lineto(450.47868683,260.01007956)
}
}
{
\newrgbcolor{curcolor}{0 0 0}
\pscustom[linestyle=none,fillstyle=solid,fillcolor=curcolor]
{
}
}
{
\newrgbcolor{curcolor}{0 0 0}
\pscustom[linestyle=none,fillstyle=solid,fillcolor=curcolor]
{
\newpath
\moveto(468.87894935,263.94510893)
\lineto(466.5400564,263.42359901)
\lineto(466.5400564,263.02851574)
\lineto(467.83592952,262.79146578)
\lineto(466.52425306,256.32790348)
\lineto(467.85173285,256.32790348)
\lineto(468.49966941,260.04168622)
\curveto(468.75252245,261.4955912)(469.25823084,262.34897312)(470.80695571,262.94949909)
\lineto(471.21784231,262.09611922)
\lineto(471.32846563,262.09611922)
\curveto(471.97640154,262.09611922)(472.38728879,262.52280976)(472.38728879,263.12333573)
\curveto(472.38728879,263.59743518)(472.03961498,263.94510893)(471.50230227,263.94510893)
\curveto(470.71213652,263.94510893)(469.70072154,263.21815427)(468.68930938,261.78005261)
\lineto(469.19501597,263.94510893)
\lineto(468.87894935,263.94510893)
}
}
{
\newrgbcolor{curcolor}{0 0 0}
\pscustom[linestyle=none,fillstyle=solid,fillcolor=curcolor]
{
\newpath
\moveto(478.51009177,265.74668864)
\lineto(478.51009177,261.49559265)
\lineto(474.33801244,261.49559265)
\lineto(474.33801244,260.5948028)
\lineto(478.51009177,260.5948028)
\lineto(478.51009177,256.32790348)
\lineto(479.42668495,256.32790348)
\lineto(479.42668495,260.5948028)
\lineto(483.58296095,260.5948028)
\lineto(483.58296095,261.49559265)
\lineto(479.42668495,261.49559265)
\lineto(479.42668495,265.74668864)
\lineto(478.51009177,265.74668864)
}
}
{
\newrgbcolor{curcolor}{0 0 0}
\pscustom[linestyle=none,fillstyle=solid,fillcolor=curcolor]
{
\newpath
\moveto(486.88561017,266.93193845)
\lineto(486.88561017,266.33141188)
\lineto(487.91282667,266.25239523)
\lineto(488.07085998,266.01534526)
\lineto(486.37990358,257.24449667)
\lineto(486.12705029,257.02325004)
\lineto(485.11563712,256.92843005)
\lineto(485.11563712,256.32790348)
\lineto(488.9558465,256.32790348)
\lineto(488.9558465,256.92843005)
\lineto(487.94443333,257.02325004)
\lineto(487.75479336,257.24449667)
\lineto(488.57656656,261.51139599)
\lineto(493.71264907,261.51139599)
\lineto(492.9066792,257.24449667)
\lineto(492.66962924,257.02325004)
\lineto(491.64241274,256.92843005)
\lineto(491.64241274,256.32790348)
\lineto(495.46681879,256.32790348)
\lineto(495.46681879,256.92843005)
\lineto(494.45540562,257.02325004)
\lineto(494.29737231,257.24449667)
\lineto(495.97252538,266.01534526)
\lineto(496.22537867,266.25239523)
\lineto(497.2683985,266.33141188)
\lineto(497.2683985,266.93193845)
\lineto(493.41238579,266.93193845)
\lineto(493.41238579,266.33141188)
\lineto(494.42379896,266.25239523)
\lineto(494.61343893,266.01534526)
\lineto(493.87068238,262.2383492)
\lineto(488.71879654,262.2383492)
\lineto(489.46155309,266.01534526)
\lineto(489.69860305,266.25239523)
\lineto(490.74162288,266.33141188)
\lineto(490.74162288,266.93193845)
\lineto(486.88561017,266.93193845)
}
}
{
\newrgbcolor{curcolor}{0 0 0}
\pscustom[linestyle=none,fillstyle=solid,fillcolor=curcolor]
{
\newpath
\moveto(502.41855585,266.14177191)
\curveto(502.90845862,266.2365918)(503.19291906,266.55265892)(503.19291906,266.96354511)
\curveto(503.19291906,267.43764456)(502.63980172,267.89594163)(501.8812426,267.89594163)
\curveto(500.41153431,267.89594163)(499.08405253,266.42622915)(498.56254314,263.70805897)
\lineto(497.36149,263.21815571)
\lineto(497.34568667,262.6966458)
\lineto(498.40450983,262.80726911)
\lineto(497.74076994,258.27171317)
\curveto(497.44050695,256.16987228)(497.17184928,254.7317663)(496.42909348,253.95740386)
\lineto(494.78554708,254.38409379)
\curveto(494.56430067,254.22606064)(494.4378738,254.0048136)(494.4378738,253.75196056)
\curveto(494.4378738,253.24625448)(494.78554758,252.91438403)(495.29125366,252.91438403)
\curveto(496.04981278,252.91438403)(496.91899757,253.51491179)(497.75657327,254.70016041)
\curveto(498.45191913,255.67996594)(498.73637998,256.86521818)(498.94182308,258.31912316)
\lineto(499.57395631,262.80726911)
\lineto(501.9286526,262.80726911)
\lineto(501.9286526,263.518419)
\lineto(499.70038295,263.518419)
\curveto(500.04805588,265.57284995)(500.52215676,266.8055122)(501.12268273,267.20059507)
\lineto(502.41855585,266.14177191)
}
}
{
\newrgbcolor{curcolor}{0 0 0}
\pscustom[linestyle=none,fillstyle=solid,fillcolor=curcolor]
{
\newpath
\moveto(505.35753651,260.18322485)
\lineto(502.96408835,259.50525241)
\lineto(502.96408835,259.08408771)
\lineto(504.45357325,259.2792616)
\lineto(504.55629635,259.1765385)
\lineto(503.52906538,253.84520976)
\lineto(503.36470842,253.6705805)
\lineto(502.29638821,253.60894664)
\lineto(502.29638821,253.16723732)
\lineto(505.36780882,253.16723732)
\lineto(505.36780882,253.59867433)
\lineto(504.47411787,253.65003588)
\lineto(504.38166708,253.80412052)
\lineto(505.60407194,260.18322485)
\lineto(505.35753651,260.18322485)
}
}
{
\newrgbcolor{curcolor}{0 0 0}
\pscustom[linestyle=none,fillstyle=solid,fillcolor=curcolor]
{
\newpath
\moveto(437.7128056,236.57373998)
\lineto(441.69524496,236.57373998)
\lineto(441.69524496,237.22167655)
\lineto(440.57320847,237.3006932)
\lineto(440.30455185,237.52193983)
\lineto(440.30455185,240.635196)
\curveto(440.79445462,240.58778605)(441.18953868,240.55617934)(441.48980166,240.55617934)
\curveto(444.03413538,240.55617934)(445.58286434,241.93107115)(445.58286434,243.95389547)
\curveto(445.58286434,246.35599935)(444.22377486,247.17777495)(441.18953838,247.17777495)
\lineto(437.58637895,247.17777495)
\lineto(437.58637895,246.54564172)
\lineto(438.61359546,246.45082173)
\lineto(438.88225208,246.2295751)
\lineto(438.88225208,237.52193983)
\lineto(438.61359546,237.3006932)
\lineto(437.7128056,237.22167655)
\lineto(437.7128056,236.57373998)
\moveto(440.30455185,241.39375588)
\lineto(440.30455185,246.41921507)
\curveto(440.58901152,246.4508217)(440.87347201,246.45082173)(441.12632505,246.45082173)
\curveto(443.14914937,246.45082173)(444.11315457,245.67645683)(444.11315457,243.98550213)
\curveto(444.11315457,242.24713748)(443.14914959,241.31473922)(441.34757168,241.31473922)
\curveto(441.11052196,241.31473922)(440.76284799,241.34634593)(440.30455185,241.39375588)
}
}
{
\newrgbcolor{curcolor}{0 0 0}
\pscustom[linestyle=none,fillstyle=solid,fillcolor=curcolor]
{
\newpath
\moveto(448.26301047,244.33317541)
\lineto(446.4298241,243.74845217)
\lineto(446.4298241,243.25854891)
\lineto(447.55186058,243.25854891)
\lineto(447.55186058,238.83361629)
\curveto(447.55186058,237.25328479)(448.31042182,236.41570667)(449.66950691,236.41570667)
\curveto(450.44386935,236.41570667)(451.2814476,236.77918396)(452.2296465,237.45872651)
\lineto(452.2296465,236.57373998)
\lineto(454.56853946,236.57373998)
\lineto(454.56853946,237.14265989)
\lineto(453.82578291,237.22167655)
\lineto(453.54132296,237.47452984)
\lineto(453.54132296,244.33317541)
\lineto(452.94079639,244.33317541)
\lineto(451.10761001,243.74845217)
\lineto(451.10761001,243.25854891)
\lineto(452.2296465,243.25854891)
\lineto(452.2296465,237.90121977)
\curveto(451.58171059,237.58515347)(450.98118278,237.42711985)(450.39646013,237.42711985)
\curveto(449.33763802,237.42711985)(448.86353704,237.96443423)(448.86353704,239.10227291)
\lineto(448.86353704,244.33317541)
\lineto(448.26301047,244.33317541)
}
}
{
\newrgbcolor{curcolor}{0 0 0}
\pscustom[linestyle=none,fillstyle=solid,fillcolor=curcolor]
{
\newpath
\moveto(457.07114505,248.20499145)
\lineto(455.12733536,247.63607154)
\lineto(455.12733536,247.14616829)
\lineto(456.35999516,247.14616829)
\lineto(456.35999516,236.57373998)
\lineto(458.3828215,236.57373998)
\curveto(461.40125467,236.57373998)(462.96578744,238.0118458)(462.96578744,240.73001598)
\curveto(462.96578744,242.87926682)(461.70151911,244.33317541)(459.83672794,244.33317541)
\curveto(459.14138208,244.33317541)(458.43023074,244.14353507)(457.67167162,243.78005883)
\lineto(457.67167162,248.20499145)
\lineto(457.07114505,248.20499145)
\moveto(457.67167162,237.3006932)
\lineto(457.67167162,243.24274558)
\curveto(458.19318101,243.40077873)(458.66728185,243.49559887)(459.06236473,243.49559887)
\curveto(460.64269623,243.49559887)(461.54348766,242.40516707)(461.54348766,240.4297527)
\curveto(461.54348766,238.24889523)(460.59528576,237.28488987)(458.54085481,237.28488987)
\curveto(458.46183824,237.28488987)(458.16157438,237.28488988)(457.67167162,237.3006932)
}
}
{
\newrgbcolor{curcolor}{0 0 0}
\pscustom[linestyle=none,fillstyle=solid,fillcolor=curcolor]
{
\newpath
\moveto(463.96238499,236.57373998)
\lineto(467.66036439,236.57373998)
\lineto(467.66036439,237.14265989)
\lineto(466.75957454,237.22167655)
\lineto(466.47511458,237.47452984)
\lineto(466.47511458,248.20499145)
\lineto(465.87458801,248.20499145)
\lineto(463.94658165,247.63607154)
\lineto(463.94658165,247.14616829)
\lineto(465.16343813,247.14616829)
\lineto(465.16343813,237.47452984)
\lineto(464.8947815,237.22167655)
\lineto(463.96238499,237.14265989)
\lineto(463.96238499,236.57373998)
}
}
{
\newrgbcolor{curcolor}{0 0 0}
\pscustom[linestyle=none,fillstyle=solid,fillcolor=curcolor]
{
\newpath
\moveto(470.39829145,247.76249819)
\curveto(469.89258537,247.76249819)(469.48169827,247.3674144)(469.48169827,246.845905)
\curveto(469.48169827,246.34019892)(469.89258537,245.94511515)(470.39829145,245.94511515)
\curveto(470.90399753,245.94511515)(471.31488464,246.34019892)(471.31488464,246.845905)
\curveto(471.31488464,247.35161108)(470.90399753,247.76249819)(470.39829145,247.76249819)
\moveto(470.33507813,244.25415875)
\lineto(468.51769509,243.68523884)
\lineto(468.51769509,243.19533559)
\lineto(469.62392824,243.19533559)
\lineto(469.62392824,237.47452984)
\lineto(469.35527162,237.22167655)
\lineto(468.51769509,237.14265989)
\lineto(468.51769509,236.57373998)
\lineto(472.07344452,236.57373998)
\lineto(472.07344452,237.14265989)
\lineto(471.22006466,237.22167655)
\lineto(470.9356047,237.47452984)
\lineto(470.9356047,244.25415875)
\lineto(470.33507813,244.25415875)
}
}
{
\newrgbcolor{curcolor}{0 0 0}
\pscustom[linestyle=none,fillstyle=solid,fillcolor=curcolor]
{
\newpath
\moveto(479.45829162,238.24889305)
\curveto(478.55750266,237.64836708)(477.76733443,237.33229986)(476.97716868,237.33229986)
\curveto(475.38103387,237.33229986)(474.32220911,238.53335502)(474.32220911,240.55617934)
\lineto(474.32220911,240.73001598)
\lineto(479.44248829,240.73001598)
\curveto(479.4582916,240.90385245)(479.45829162,241.06188609)(479.45829162,241.21991924)
\curveto(479.45829162,243.16372698)(478.33625341,244.33317541)(476.61369207,244.33317541)
\curveto(474.49604786,244.33317541)(472.96312266,242.70542995)(472.96312266,240.31912938)
\curveto(472.96312266,237.99604208)(474.40122785,236.43151001)(476.48726543,236.43151001)
\curveto(477.48287427,236.43151001)(478.51009272,236.74757725)(479.45829162,237.37970985)
\lineto(479.45829162,238.24889305)
\moveto(478.05179517,241.39375588)
\lineto(474.40122576,241.39375588)
\curveto(474.49604565,242.73703765)(475.39683661,243.62202552)(476.40824877,243.62202552)
\curveto(477.46707088,243.62202552)(478.05179517,242.92667768)(478.05179517,241.64660917)
\lineto(478.05179517,241.39375588)
}
}
{
\newrgbcolor{curcolor}{0 0 0}
\pscustom[linestyle=none,fillstyle=solid,fillcolor=curcolor]
{
}
}
{
\newrgbcolor{curcolor}{0 0 0}
\pscustom[linestyle=none,fillstyle=solid,fillcolor=curcolor]
{
\newpath
\moveto(491.76241928,247.9205315)
\curveto(490.68779386,247.84151492)(489.62896872,247.35161075)(488.72817977,246.51403506)
\curveto(487.77998087,245.64485273)(487.1162396,244.66504417)(486.68955009,243.47979554)
\curveto(486.05741749,241.70982426)(485.75715358,240.09788139)(485.75715358,238.7387963)
\curveto(485.75715358,236.22606922)(487.14784869,234.48770016)(489.1548697,234.32966701)
\lineto(489.28129634,234.9934069)
\curveto(487.84319468,235.21465331)(487.0530267,236.41570874)(487.0530267,238.48594301)
\curveto(487.0530267,239.60797837)(487.30588061,241.23572462)(487.92220989,243.29015557)
\curveto(488.23827619,244.34897768)(488.71237713,245.24976937)(489.40772299,246.0241318)
\curveto(490.00824896,246.70367435)(490.79841694,247.13036508)(491.63599263,247.2567916)
\lineto(491.76241928,247.9205315)
}
}
{
\newrgbcolor{curcolor}{0 0 0}
\pscustom[linestyle=none,fillstyle=solid,fillcolor=curcolor]
{
\newpath
\moveto(495.9964773,242.26293907)
\curveto(496.59700327,242.27874239)(496.99208714,242.6422195)(496.99208714,243.13212226)
\curveto(496.99208714,243.73264823)(496.3915597,244.23835542)(495.52237737,244.23835542)
\curveto(494.54257184,244.23835542)(493.49955023,243.74845128)(492.69358116,242.86346564)
\curveto(491.85600547,241.94687337)(491.3819047,240.69840791)(491.3819047,239.29191288)
\curveto(491.3819047,237.50613829)(492.33010594,236.41570667)(493.72079766,236.41570667)
\curveto(494.65319325,236.41570667)(495.69621499,236.92141429)(496.67602052,237.94862976)
\lineto(496.67602052,238.67558298)
\curveto(495.82264151,237.80640065)(494.93765341,237.36390652)(494.21070092,237.36390652)
\curveto(493.23089539,237.36390652)(492.63036784,238.13827111)(492.63036784,239.51315951)
\curveto(492.63036784,241.86785345)(493.72079894,243.46399221)(495.00086746,243.46399221)
\lineto(495.07988411,243.46399221)
\lineto(495.9964773,242.26293907)
}
}
{
\newrgbcolor{curcolor}{0 0 0}
\pscustom[linestyle=none,fillstyle=solid,fillcolor=curcolor]
{
\newpath
\moveto(501.28324028,240.42906135)
\lineto(498.88979212,239.75108891)
\lineto(498.88979212,239.32992421)
\lineto(500.37927703,239.5250981)
\lineto(500.48200013,239.422375)
\lineto(499.45476916,234.09104626)
\lineto(499.2904122,233.916417)
\lineto(498.22209199,233.85478314)
\lineto(498.22209199,233.41307382)
\lineto(501.29351259,233.41307382)
\lineto(501.29351259,233.84451083)
\lineto(500.39982165,233.89587238)
\lineto(500.30737086,234.04995702)
\lineto(501.52977572,240.42906135)
\lineto(501.28324028,240.42906135)
}
}
{
\newrgbcolor{curcolor}{0 0 0}
\pscustom[linestyle=none,fillstyle=solid,fillcolor=curcolor]
{
\newpath
\moveto(502.19695547,233.90297708)
\curveto(504.1249599,234.1400268)(505.43663828,235.70455826)(505.43663828,237.17426655)
\curveto(505.43663828,237.8538091)(505.07316103,238.24889305)(504.42522511,238.24889305)
\curveto(503.79309251,238.24889305)(503.30318863,237.79059581)(503.30318863,237.14265989)
\curveto(503.30318863,236.96882343)(503.35059871,236.82659313)(503.4454186,236.6843633)
\lineto(504.45683177,236.6843633)
\curveto(504.28299531,235.43590141)(503.55604065,234.66153672)(502.29177545,234.42448699)
\lineto(502.19695547,233.90297708)
}
}
{
\newrgbcolor{curcolor}{0 0 0}
\pscustom[linestyle=none,fillstyle=solid,fillcolor=curcolor]
{
\newpath
\moveto(512.33232291,242.26293907)
\curveto(512.93284888,242.27874239)(513.32793275,242.6422195)(513.32793275,243.13212226)
\curveto(513.32793275,243.73264823)(512.72740531,244.23835542)(511.85822298,244.23835542)
\curveto(510.87841745,244.23835542)(509.83539583,243.74845128)(509.02942677,242.86346564)
\curveto(508.19185107,241.94687337)(507.71775031,240.69840791)(507.71775031,239.29191288)
\curveto(507.71775031,237.50613829)(508.66595155,236.41570667)(510.05664327,236.41570667)
\curveto(510.98903886,236.41570667)(512.0320606,236.92141429)(513.01186613,237.94862976)
\lineto(513.01186613,238.67558298)
\curveto(512.15848712,237.80640065)(511.27349902,237.36390652)(510.54654653,237.36390652)
\curveto(509.566741,237.36390652)(508.96621345,238.13827111)(508.96621345,239.51315951)
\curveto(508.96621345,241.86785345)(510.05664455,243.46399221)(511.33671307,243.46399221)
\lineto(511.41572972,243.46399221)
\lineto(512.33232291,242.26293907)
}
}
{
\newrgbcolor{curcolor}{0 0 0}
\pscustom[linestyle=none,fillstyle=solid,fillcolor=curcolor]
{
\newpath
\moveto(512.81481835,233.90297708)
\curveto(514.74282278,234.1400268)(516.05450116,235.70455826)(516.05450116,237.17426655)
\curveto(516.05450116,237.8538091)(515.69102391,238.24889305)(515.04308799,238.24889305)
\curveto(514.41095539,238.24889305)(513.92105151,237.79059581)(513.92105151,237.14265989)
\curveto(513.92105151,236.96882343)(513.96846159,236.82659313)(514.06328148,236.6843633)
\lineto(515.07469465,236.6843633)
\curveto(514.90085819,235.43590141)(514.17390353,234.66153672)(512.90963834,234.42448699)
\lineto(512.81481835,233.90297708)
}
}
{
\newrgbcolor{curcolor}{0 0 0}
\pscustom[linestyle=none,fillstyle=solid,fillcolor=curcolor]
{
\newpath
\moveto(519.568273,244.22255209)
\lineto(517.97213659,243.70104217)
\lineto(517.97213659,243.3059589)
\lineto(519.18899306,243.21113892)
\curveto(519.45764941,243.03730245)(519.58407641,242.59480829)(519.66309298,241.8678558)
\lineto(520.05817625,238.16987639)
\curveto(520.07397957,237.98023661)(520.08978291,237.79059626)(520.08978291,237.60095648)
\curveto(520.08978291,237.22167692)(520.02656943,236.87400297)(519.86853628,236.57373998)
\lineto(520.62709616,236.57373998)
\curveto(523.56651275,239.89243613)(524.9572088,242.2313335)(524.9572088,243.32176223)
\curveto(524.9572088,243.89068157)(524.65694498,244.22255209)(524.11963227,244.22255209)
\curveto(523.85097591,244.22255209)(523.58231877,244.11192857)(523.32946573,243.90648547)
\lineto(523.62972901,241.97847912)
\curveto(523.21884282,240.88805038)(522.41287134,239.60797796)(521.2118194,238.07505641)
\lineto(520.8009328,242.40516905)
\curveto(520.67450628,243.66943425)(520.50066861,244.22255209)(519.59987966,244.22255209)
\lineto(519.568273,244.22255209)
}
}
{
\newrgbcolor{curcolor}{0 0 0}
\pscustom[linestyle=none,fillstyle=solid,fillcolor=curcolor]
{
\newpath
\moveto(529.16292519,240.42906135)
\lineto(526.76947703,239.75108891)
\lineto(526.76947703,239.32992421)
\lineto(528.25896193,239.5250981)
\lineto(528.36168503,239.422375)
\lineto(527.33445406,234.09104626)
\lineto(527.17009711,233.916417)
\lineto(526.1017769,233.85478314)
\lineto(526.1017769,233.41307382)
\lineto(529.1731975,233.41307382)
\lineto(529.1731975,233.84451083)
\lineto(528.27950655,233.89587238)
\lineto(528.18705577,234.04995702)
\lineto(529.40946062,240.42906135)
\lineto(529.16292519,240.42906135)
}
}
{
\newrgbcolor{curcolor}{0 0 0}
\pscustom[linestyle=none,fillstyle=solid,fillcolor=curcolor]
{
\newpath
\moveto(530.0766464,233.90297708)
\curveto(532.00465083,234.1400268)(533.31632922,235.70455826)(533.31632922,237.17426655)
\curveto(533.31632922,237.8538091)(532.95285196,238.24889305)(532.30491604,238.24889305)
\curveto(531.67278344,238.24889305)(531.18287956,237.79059581)(531.18287956,237.14265989)
\curveto(531.18287956,236.96882343)(531.23028964,236.82659313)(531.32510953,236.6843633)
\lineto(532.33652271,236.6843633)
\curveto(532.16268624,235.43590141)(531.43573159,234.66153672)(530.17146639,234.42448699)
\lineto(530.0766464,233.90297708)
}
}
{
\newrgbcolor{curcolor}{0 0 0}
\pscustom[linestyle=none,fillstyle=solid,fillcolor=curcolor]
{
\newpath
\moveto(536.83010105,244.22255209)
\lineto(535.23396464,243.70104217)
\lineto(535.23396464,243.3059589)
\lineto(536.45082111,243.21113892)
\curveto(536.71947746,243.03730245)(536.84590446,242.59480829)(536.92492103,241.8678558)
\lineto(537.3200043,238.16987639)
\curveto(537.33580762,237.98023661)(537.35161096,237.79059626)(537.35161096,237.60095648)
\curveto(537.35161096,237.22167692)(537.28839748,236.87400297)(537.13036433,236.57373998)
\lineto(537.88892421,236.57373998)
\curveto(540.8283408,239.89243613)(542.21903685,242.2313335)(542.21903685,243.32176223)
\curveto(542.21903685,243.89068157)(541.91877303,244.22255209)(541.38146032,244.22255209)
\curveto(541.11280396,244.22255209)(540.84414682,244.11192857)(540.59129378,243.90648547)
\lineto(540.89155706,241.97847912)
\curveto(540.48067087,240.88805038)(539.67469939,239.60797796)(538.47364745,238.07505641)
\lineto(538.06276085,242.40516905)
\curveto(537.93633433,243.66943425)(537.76249666,244.22255209)(536.86170771,244.22255209)
\lineto(536.83010105,244.22255209)
}
}
{
\newrgbcolor{curcolor}{0 0 0}
\pscustom[linestyle=none,fillstyle=solid,fillcolor=curcolor]
{
\newpath
\moveto(541.62047967,233.90297708)
\curveto(543.5484841,234.1400268)(544.86016248,235.70455826)(544.86016248,237.17426655)
\curveto(544.86016248,237.8538091)(544.49668523,238.24889305)(543.84874931,238.24889305)
\curveto(543.21661671,238.24889305)(542.72671282,237.79059581)(542.72671282,237.14265989)
\curveto(542.72671282,236.96882343)(542.77412291,236.82659313)(542.8689428,236.6843633)
\lineto(543.88035597,236.6843633)
\curveto(543.70651951,235.43590141)(542.97956485,234.66153672)(541.71529965,234.42448699)
\lineto(541.62047967,233.90297708)
}
}
{
\newrgbcolor{curcolor}{0 0 0}
\pscustom[linestyle=none,fillstyle=solid,fillcolor=curcolor]
{
\newpath
\moveto(553.38359018,244.19094543)
\lineto(553.06752356,243.84327215)
\curveto(552.5776208,244.08032188)(552.08771661,244.19094543)(551.64522379,244.19094543)
\curveto(550.08069561,244.19094543)(548.80062344,243.44818755)(547.99465438,242.12070909)
\curveto(547.45734167,241.23572345)(547.17288117,240.255915)(547.17288117,239.1970929)
\curveto(547.17288117,237.45872825)(548.01045905,236.43151001)(549.35374082,236.43151001)
\curveto(550.11229994,236.43151001)(550.87086142,236.87400415)(551.70843711,237.75898979)
\lineto(551.70843711,237.63256315)
\curveto(551.70843711,236.82659408)(552.00870106,236.43151001)(552.67244029,236.43151001)
\curveto(553.0991298,236.43151001)(553.74706766,236.62115034)(554.61624998,236.98462658)
\lineto(554.61624998,237.45872651)
\curveto(554.18956048,237.36390662)(553.90509992,237.3006932)(553.73126346,237.3006932)
\curveto(553.1465408,237.3006932)(552.87788359,237.52194034)(552.87788359,238.02764642)
\curveto(552.87788359,238.24889283)(552.9252937,238.61237018)(553.0359169,239.13387957)
\lineto(553.60483681,241.93106912)
\curveto(553.84188653,243.10051443)(554.04733023,243.85907581)(554.20536338,244.19094543)
\lineto(553.38359018,244.19094543)
\moveto(551.91388041,243.57461553)
\lineto(552.54601365,242.61061235)
\lineto(551.72424044,238.43853302)
\curveto(551.07630453,237.74318716)(550.45997338,237.39551318)(549.85944741,237.39551318)
\curveto(548.94285514,237.39551318)(548.48455763,238.02764768)(548.48455763,239.29191288)
\curveto(548.48455763,241.80463996)(549.84364593,243.57461553)(551.69263378,243.57461553)
\lineto(551.91388041,243.57461553)
}
}
{
\newrgbcolor{curcolor}{0 0 0}
\pscustom[linestyle=none,fillstyle=solid,fillcolor=curcolor]
{
\newpath
\moveto(554.30635654,233.90297708)
\curveto(556.23436097,234.1400268)(557.54603935,235.70455826)(557.54603935,237.17426655)
\curveto(557.54603935,237.8538091)(557.1825621,238.24889305)(556.53462618,238.24889305)
\curveto(555.90249358,238.24889305)(555.4125897,237.79059581)(555.4125897,237.14265989)
\curveto(555.4125897,236.96882343)(555.45999978,236.82659313)(555.55481967,236.6843633)
\lineto(556.56623285,236.6843633)
\curveto(556.39239638,235.43590141)(555.66544173,234.66153672)(554.40117653,234.42448699)
\lineto(554.30635654,233.90297708)
}
}
{
\newrgbcolor{curcolor}{0 0 0}
\pscustom[linestyle=none,fillstyle=solid,fillcolor=curcolor]
{
\newpath
\moveto(562.8613909,248.20499145)
\lineto(560.45928462,247.73089153)
\lineto(560.45928462,247.33580826)
\lineto(561.8341744,247.08295496)
\lineto(559.98518469,237.12685656)
\curveto(560.74374381,236.65275711)(561.54971528,236.43151001)(562.38729097,236.43151001)
\curveto(563.58834291,236.43151001)(564.69457808,236.9530209)(565.50054714,237.93282643)
\curveto(566.30651621,238.91263196)(566.73320694,240.17690076)(566.73320694,241.53598585)
\curveto(566.73320694,243.1637273)(565.86402257,244.22255209)(564.67877394,244.22255209)
\curveto(564.04664134,244.22255209)(563.27227657,243.82746803)(562.33988098,243.03730228)
\lineto(563.01942421,246.75108502)
\curveto(563.1142441,247.28839773)(563.24067098,247.76249863)(563.38290081,248.20499145)
\lineto(562.8613909,248.20499145)
\moveto(561.32846781,237.7115798)
\lineto(562.22925767,242.42097238)
\curveto(562.98781679,242.94248177)(563.55673802,243.25854891)(564.12565736,243.25854891)
\curveto(564.9000198,243.25854891)(565.43733382,242.54739781)(565.43733382,241.33054255)
\curveto(565.43733382,240.24011382)(565.05805337,239.05486216)(564.55234729,238.29630304)
\curveto(564.01503458,237.49033398)(563.39870348,237.07944657)(562.73496425,237.07944657)
\curveto(562.21345486,237.07944657)(561.75515732,237.2848903)(561.32846781,237.7115798)
}
}
{
\newrgbcolor{curcolor}{0 0 0}
\pscustom[linestyle=none,fillstyle=solid,fillcolor=curcolor]
{
\newpath
\moveto(566.71444049,233.90297708)
\curveto(568.64244492,234.1400268)(569.9541233,235.70455826)(569.9541233,237.17426655)
\curveto(569.9541233,237.8538091)(569.59064605,238.24889305)(568.94271013,238.24889305)
\curveto(568.31057753,238.24889305)(567.82067365,237.79059581)(567.82067365,237.14265989)
\curveto(567.82067365,236.96882343)(567.86808373,236.82659313)(567.96290362,236.6843633)
\lineto(568.97431679,236.6843633)
\curveto(568.80048033,235.43590141)(568.07352567,234.66153672)(566.80926047,234.42448699)
\lineto(566.71444049,233.90297708)
}
}
{
\newrgbcolor{curcolor}{0 0 0}
\pscustom[linestyle=none,fillstyle=solid,fillcolor=curcolor]
{
\newpath
\moveto(577.89282776,244.31737207)
\lineto(577.59256447,243.54300887)
\curveto(577.41872801,243.46399229)(577.26069439,243.41658222)(577.11846455,243.41658222)
\curveto(576.78659494,243.41658222)(576.45472425,243.57461573)(576.04383806,243.78005883)
\curveto(575.61714855,244.00130524)(575.22206438,244.20674876)(574.74796493,244.20674876)
\curveto(573.6575362,244.20674876)(572.61451527,243.1005145)(572.61451527,241.99428245)
\curveto(572.61451527,241.40955979)(572.914779,241.06188593)(573.35727182,241.06188593)
\curveto(573.67333812,241.06188593)(573.91038859,241.2357229)(574.10002837,241.56759252)
\lineto(573.26245184,242.57900569)
\curveto(573.49950156,242.9108753)(573.84717549,243.08471227)(574.25806168,243.08471227)
\curveto(574.58993129,243.08471227)(575.06403213,242.92667866)(575.64875479,242.62641568)
\lineto(577.21328454,242.97408896)
\lineto(571.77693874,236.87400327)
\lineto(572.64612193,236.39990334)
\lineto(573.02540187,237.07944657)
\curveto(573.19923834,237.17426646)(573.35727199,237.22167655)(573.53110846,237.22167655)
\curveto(574.28966758,237.22167655)(575.09563911,236.54213332)(575.99642806,236.54213332)
\curveto(577.15007006,236.54213332)(578.19309104,237.6641709)(578.19309104,238.75459963)
\curveto(578.19309104,239.32351897)(577.86122066,239.70279948)(577.41872784,239.70279948)
\curveto(577.15007148,239.70279948)(576.88141437,239.51315913)(576.66016796,239.13387957)
\lineto(577.52935115,238.16987639)
\curveto(577.27649811,237.79059683)(576.94462752,237.60095648)(576.54954464,237.60095648)
\curveto(576.24928166,237.60095648)(575.71196734,237.77479347)(574.9376049,238.1224664)
\lineto(573.40468181,237.72738313)
\lineto(578.85683094,243.81166549)
\lineto(577.89282776,244.31737207)
}
}
{
\newrgbcolor{curcolor}{0 0 0}
\pscustom[linestyle=none,fillstyle=solid,fillcolor=curcolor]
{
\newpath
\moveto(580.50753823,247.9205315)
\lineto(580.38111158,247.2567916)
\curveto(581.78760661,247.05134851)(582.59357789,245.85029317)(582.59357789,243.85907548)
\curveto(582.59357789,242.86346664)(582.38813404,241.29893387)(581.83501801,239.27610955)
\curveto(581.5031484,238.05925429)(580.99744074,237.04783908)(580.25468493,236.22606671)
\curveto(579.65415896,235.54652416)(578.86399099,235.10403011)(578.02641529,234.9934069)
\lineto(577.89998864,234.32966701)
\curveto(578.94300743,234.39288027)(580.00183257,234.88278444)(580.93422816,235.73616345)
\curveto(582.78321601,237.44292147)(583.88945102,240.24011609)(583.88945102,243.60622219)
\curveto(583.88945102,246.02412938)(582.51455923,247.74669503)(580.50753823,247.9205315)
}
}
{
\newrgbcolor{curcolor}{0 0 0}
\pscustom[linewidth=0.66161243,linecolor=curcolor]
{
\newpath
\moveto(572.64874321,89.8233904)
\lineto(611.40032865,89.8233904)
\lineto(611.40032865,30.04198537)
\lineto(572.64874321,30.04198537)
\closepath
}
}
{
\newrgbcolor{curcolor}{0 0 0}
\pscustom[linewidth=0.66686956,linecolor=curcolor]
{
\newpath
\moveto(548.16012762,108.26425569)
\lineto(572.03387939,90.10127144)
\lineto(572.03387939,30.4273484)
\lineto(548.16012762,48.59033265)
\closepath
}
}
{
\newrgbcolor{curcolor}{0 0 0}
\pscustom[linewidth=0.65791488,linecolor=curcolor]
{
\newpath
\moveto(548.78644383,108.61210906)
\lineto(587.24463762,108.61210906)
\lineto(610.77029932,90.46632875)
\lineto(572.31210553,90.46632875)
\closepath
}
}
{
\newrgbcolor{curcolor}{0 0 0}
\pscustom[linestyle=none,fillstyle=solid,fillcolor=curcolor]
{
\newpath
\moveto(590.99736113,57.28321439)
\curveto(594.88497662,58.95836578)(596.7813802,61.07601675)(596.7813802,64.04703997)
\curveto(596.7813802,67.46055601)(594.25284323,69.76784572)(590.20719459,69.76784572)
\curveto(587.39420452,69.76784572)(585.56101534,68.40875769)(585.56101534,66.82842619)
\curveto(585.56101534,65.7854074)(586.22475646,65.15327313)(587.45741503,65.15327313)
\curveto(587.64705481,65.15327313)(587.83669516,65.15327316)(588.02633494,65.18487979)
\lineto(589.29060141,68.31393929)
\lineto(589.48024137,68.31393929)
\curveto(592.29323144,68.31393929)(593.905174,66.67039007)(593.905174,63.8574)
\curveto(593.905174,61.23404971)(592.35644455,59.33764625)(589.32220807,58.19980757)
\lineto(589.32220807,53.58523498)
\lineto(590.99736113,53.58523498)
\lineto(590.99736113,57.28321439)
\moveto(590.14398127,51.49919532)
\curveto(589.16417574,51.49919532)(588.37400822,50.70902776)(588.37400822,49.6976156)
\curveto(588.37400822,48.68620344)(589.16417574,47.89603589)(590.14398127,47.89603589)
\curveto(591.15539343,47.89603589)(591.97716764,48.71781007)(591.97716764,49.6976156)
\curveto(591.97716764,50.70902776)(591.15539343,51.49919532)(590.14398127,51.49919532)
}
}
{
\newrgbcolor{curcolor}{0 0 0}
\pscustom[linewidth=0.79016654,linecolor=curcolor]
{
\newpath
\moveto(304.77861291,469.72746019)
\curveto(435.3980173,469.72746019)(512.4299784,504.89422267)(514.10458625,399.39393445)
}
}
{
\newrgbcolor{curcolor}{0 0 0}
\pscustom[linestyle=none,fillstyle=solid,fillcolor=curcolor]
{
\newpath
\moveto(513.97917878,407.29460461)
\lineto(510.76874772,410.40470969)
\lineto(514.10458625,399.39393445)
\lineto(517.08928386,410.50503566)
\lineto(513.97917878,407.29460461)
\closepath
}
}
{
\newrgbcolor{curcolor}{0 0 0}
\pscustom[linewidth=0.79016654,linecolor=curcolor]
{
\newpath
\moveto(513.97917878,407.29460461)
\lineto(510.76874772,410.40470969)
\lineto(514.10458625,399.39393445)
\lineto(517.08928386,410.50503566)
\lineto(513.97917878,407.29460461)
\closepath
}
}
{
\newrgbcolor{curcolor}{0 0 0}
\pscustom[linewidth=0.79016654,linecolor=curcolor]
{
\newpath
\moveto(514.10457835,216.02438673)
\lineto(514.10457835,187.55605329)
}
}
{
\newrgbcolor{curcolor}{0 0 0}
\pscustom[linestyle=none,fillstyle=solid,fillcolor=curcolor]
{
\newpath
\moveto(514.10457835,195.45771869)
\lineto(510.94391219,198.61838485)
\lineto(514.10457835,187.55605329)
\lineto(517.26524451,198.61838485)
\lineto(514.10457835,195.45771869)
\closepath
}
}
{
\newrgbcolor{curcolor}{0 0 0}
\pscustom[linewidth=0.79016654,linecolor=curcolor]
{
\newpath
\moveto(514.10457835,195.45771869)
\lineto(510.94391219,198.61838485)
\lineto(514.10457835,187.55605329)
\lineto(517.26524451,198.61838485)
\lineto(514.10457835,195.45771869)
\closepath
}
}
\end{pspicture}

			\label{fig:rep}
		\end{figure}

	Chaque serveur doit être capable de prouver son calcul, cette preuve est basée sur la preuve \textit{Zero Knowledge} de Fiat-Shamir dans
	\cite{journals/joc/FeigeFS88}. Un facteur de sécurité $k'$ est choisi, le serveur central envoie les clés $f(i)$ et une 
	valeur $v$ gé\-né\-ra\-tri\-ce  du groupe cyclique des résidus quadratiques dans \Mgrinv{Z}{n^{s+1}} aux $N_s$ 
	serveurs périphériques, il publie aussi les valeurs\footnote{Pour des raisons de simplicité, on note $f_j = f(j)$} 
	$v^{f_1}, \dots, v^{f_{N_s}}$. Soit $k$ le seuil, les serveurs tirent au hasard une 
	valeur $r_i$ de taille en bits égale à $(s+1)k+2k'$ et calculent $c_i = c^{f_i}$, $v_i = v^{f_i}$, $a = c^{r_i}$ et $b = v^{r_i}\mod{n^{s+1}}$. 
	Ils hachent l'ensemble $\{c,c_i,v,v_i,a,b\}$, soit $H_i$ la valeur hachée obtenue, ils publient les valeurs $c, c_i, v, v_i, a, b$ et 
	$z_i = r_i+H_if_i$. Finalement n'importe qui peut 
	vérifier $c^{z_i} \stackrel{?}{=}ac_i^{H_i}$ et $v^{z_i} \stackrel{?}{=}bv_i^{H_i}$. Si les deux égalités sont vraies il signifie 
	que $\log_{v}(v_i) = \log_{c}(c_i)$.
	La Figure \ref{fig:rep} illustre le protocole énoncé.

%	Inspiré dans le travail de  Fouque, Poupard et Stern on peut distribuer le calcul de $D'$. Comme déjà vue dans le chapitre 
%	sur l'implantation de Paillier, la partie plus couteuse est la exponentiation et il est intéressant de la repartir: Pour que 
%	le calcul de chaque serveur puisse être vérifié sans la connaissance de la fonction $f$ on introduit le protocole repartie pour la preuve 
%	d'égalité de logarithmes. 

	
%	\TODO{Agora nos temos chaves distribuidas, é hora de distribuir o calculo. C}
%	Jurik a étendu le système avec des méthodes reparties\cite{} pour l'appliquer 	
%	\subsection{Protocoles Répartis}
%	Des protocoles répartis on été proposées pour sécuriser et valider chaque sous étape d'une votation. 
%		\subsection{Preuve d'égalité de logarithmes}



%	\begin{algo}{Protocole d'exponentiation e preuve reapartie}
%		\begin{itemize}\renewcommand{\labelitemi}{} \renewcommand{\labelitemii}{$\cdot$}
%			\item{\bf Entités:}
%				\begin{itemize}
%					\item Serveur fiable: impossible d'être corrompu.
%					\item $N_s$ serveurs respectant les contraintes définies dans la page \pageref{item:restr} 
%				\end{itemize}
%			\item{\bf Entrée:} 
%				\begin{itemize}
%					\item $N_s$: le nombre de serveurs.
%					\item $k$: le nombre de serveurs nécessaires pour le déchiffrement.
%					\item $d$: la clé privé.
%				\end{itemize}
%			\item{\bf Serveur fiable:}
%				\begin{enumerate} %			
%				\renewcommand{\theenumi}{\arabic{enumi}}
%				\renewcommand{\theenumii}{\arabic{enumii}}
%				\renewcommand{\theenumiii}{\arabic{enumiii}}
%				\renewcommand{\labelenumi}{\theenumi.}
%				\renewcommand{\labelenumii}{\theenumi.\theenumii.}
%				\renewcommand{\labelenumiii}{\theenumi.\theenumii.\theenumiii.}
%				\makeatletter
%				\renewcommand{\p@enumii}{\theenumi.}
%				\renewcommand{\p@enumiii}{\theenumi.\theenumii.}
%				\makeatother
%				\item $L := \{\}$
%				\item Initialiser et envoyer les clés reparties.
%				\item De que un serveur répond avec une exponentiation:
%				\begin{enumerate}
%					\item Vérifier la preuve envoyé, si elle est valide, ajouter $(c_i,i)$ à la liste $L$.
%					\item Si $L$ est plus grand ou égal à $k$, sort de la boucle.
%				\end{enumerate}
%				\item Calculer: $c' = \prod{(c_i,i)\in L}c_i^{2\cdot N_s!\cdot \prod_{(c_j,j) \in L, j\neq i}\frac{-i}{i-j}}$
%				\item pour chaque $(c_i,i)$ dans $L$, calcule:
%				\begin{enumerate}
%					\item vérifier la preuve envoyé, si elle est valide, ajouter $(c_i,i)$ à la liste $L$.
%					\item si $L$ est plus grand ou égal à $k$, sort de la boucle.
%				\end{enumerate}
%
%				\item Trouver $p$ et $q$ premiers tels que $p = 2p'+1$ et $q = 2q'+1$ avec $p'$ et $q'$ premiers.
%				\item Trouver $k-1$ valeurs: $r_i\in_R\Mgr{Z}{(pq)^sp'q'}$, $1 \leq i \leq k-1$
%				\item Pour $j$ dans $\{1;N_s\}$:
%					\begin{enumerate}
%						\item Envoyer la clé $d+\sum_{i = 1}^{k-1}r_ij^i\mod{(pq)^sp'q'}$ au serveur $j$.
%					\end{enumerate}
%				\end{enumerate}
%
%			\item{\bf Serveurs Tiers:}
%				\begin{enumerate} %			
%				\renewcommand{\theenumi}{\arabic{enumi}}
%				\renewcommand{\theenumii}{\arabic{enumii}}
%				\renewcommand{\theenumiii}{\arabic{enumiii}}
%				\renewcommand{\labelenumi}{\theenumi.}
%				\renewcommand{\labelenumii}{\theenumi.\theenumii.}
%				\renewcommand{\labelenumiii}{\theenumi.\theenumii.\theenumiii.}
%				\makeatletter
%				\renewcommand{\p@enumii}{\theenumi.}
%				\renewcommand{\p@enumiii}{\theenumi.\theenumii.}
%				\makeatother
%				\item Trouver $p$ et $q$ premiers tels que $p = 2p'+1$ et $q = 2q'+1$ avec $p'$ et $q'$ premiers.
%				\item Trouver $k-1$ valeurs aléatoires: $r_i$, $1 \leq i \leq k-1$
%				\item Pour $j$ dans $\{1;N_s\}$:
%					\begin{enumerate}
%						\item Envoyer la clé $d+\sum_{i = 1}^{k-1}r_ij^i\mod{(pq)^sp'q'}$ au serveur $j$.
%					\end{enumerate}
%				\end{enumerate}
%
%%			\item{\bf Calcul:}
%%				\begin{enumerate} %			
%%				\renewcommand{\theenumi}{\arabic{enumi}}
%%				\renewcommand{\theenumii}{\arabic{enumii}}
%%				\renewcommand{\theenumiii}{\arabic{enumiii}}
%%				\renewcommand{\labelenumi}{\theenumi.}
%%				\renewcommand{\labelenumii}{\theenumi.\theenumii.}
%%				\renewcommand{\labelenumiii}{\theenumi.\theenumii.\theenumiii.}
%%				\makeatletter
%%				\renewcommand{\p@enumii}{\theenumi.}
%%				\renewcommand{\p@enumiii}{\theenumi.\theenumii.}
%%				\makeatother
%%				\item Trouver $p$ et $q$ premiers tels que $p = 2p'+1$ et $q = 2q'+1$ avec $p'$ et $q'$ premiers.
%%				\item Trouver $k-1$ valeurs aléatoires: $r_i$, $1 \leq i \leq k-1$
%%				\item Pour $j$ dans $\{1;N_s\}$:
%%					\begin{enumerate}
%%						\item Envoyer la clé $d+\sum_{i = 1}^{k-1}r_ij^i\mod{(pq)^sp'q'}$ au serveur $j$.
%%					\end{enumerate}
%%				\end{enumerate}
%		\end{itemize}
%	\end{algo}



%		Le schéma introduit dans \cite{Fouque00sharingdecryption} a été pris pour calculer le résultat d'une élection avec un seuil
%		en limitant 
%		Elle distribue le calcul d'une puissance (déchiffrement) sans publier  
		%\TODO{validacao do log de u e v, ver o paper do jurik, pg 23: Brics-jurik.pdf.Protocol for equality of discrete logs}\\

		%\TODO{introduzir a ideia de exponenciacao usando interpolacao de lagrange}\\
	%
%	\begin{gnuplot}[scale=1.1]
%	set pm3d
%	set surface
%	splot sin(sqrt((x*x+y*y)))/sqrt(x*x+y*y)
%	\end{gnuplot}	
%		\begin{figure}
%			\center
%			% GNUPLOT: LaTeX picture
\setlength{\unitlength}{0.240900pt}
\ifx\plotpoint\undefined\newsavebox{\plotpoint}\fi
\sbox{\plotpoint}{\rule[-0.200pt]{0.400pt}{0.400pt}}%
\begin{picture}(650,450)(0,0)
\sbox{\plotpoint}{\rule[-0.200pt]{0.400pt}{0.400pt}}%
\put(110.0,82.0){\rule[-0.200pt]{4.818pt}{0.400pt}}
\put(90,82){\makebox(0,0)[r]{-10}}
\put(669.0,82.0){\rule[-0.200pt]{4.818pt}{0.400pt}}
\put(110.0,148.0){\rule[-0.200pt]{4.818pt}{0.400pt}}
\put(90,148){\makebox(0,0)[r]{ 0}}
\put(669.0,148.0){\rule[-0.200pt]{4.818pt}{0.400pt}}
\put(110.0,213.0){\rule[-0.200pt]{4.818pt}{0.400pt}}
\put(90,213){\makebox(0,0)[r]{ 10}}
\put(669.0,213.0){\rule[-0.200pt]{4.818pt}{0.400pt}}
\put(110.0,279.0){\rule[-0.200pt]{4.818pt}{0.400pt}}
\put(90,279){\makebox(0,0)[r]{ 20}}
\put(669.0,279.0){\rule[-0.200pt]{4.818pt}{0.400pt}}
\put(110.0,344.0){\rule[-0.200pt]{4.818pt}{0.400pt}}
\put(90,344){\makebox(0,0)[r]{ 30}}
\put(669.0,344.0){\rule[-0.200pt]{4.818pt}{0.400pt}}
\put(110.0,410.0){\rule[-0.200pt]{4.818pt}{0.400pt}}
\put(90,410){\makebox(0,0)[r]{ 40}}
\put(669.0,410.0){\rule[-0.200pt]{4.818pt}{0.400pt}}
\put(110.0,82.0){\rule[-0.200pt]{0.400pt}{4.818pt}}
\put(110,41){\makebox(0,0){-1}}
\put(110.0,390.0){\rule[-0.200pt]{0.400pt}{4.818pt}}
\put(226.0,82.0){\rule[-0.200pt]{0.400pt}{4.818pt}}
\put(226,41){\makebox(0,0){ 0}}
\put(226.0,390.0){\rule[-0.200pt]{0.400pt}{4.818pt}}
\put(342.0,82.0){\rule[-0.200pt]{0.400pt}{4.818pt}}
\put(342,41){\makebox(0,0){ 1}}
\put(342.0,390.0){\rule[-0.200pt]{0.400pt}{4.818pt}}
\put(457.0,82.0){\rule[-0.200pt]{0.400pt}{4.818pt}}
\put(457,41){\makebox(0,0){ 2}}
\put(457.0,390.0){\rule[-0.200pt]{0.400pt}{4.818pt}}
\put(573.0,82.0){\rule[-0.200pt]{0.400pt}{4.818pt}}
\put(573,41){\makebox(0,0){ 3}}
\put(573.0,390.0){\rule[-0.200pt]{0.400pt}{4.818pt}}
\put(689.0,82.0){\rule[-0.200pt]{0.400pt}{4.818pt}}
\put(689,41){\makebox(0,0){ 4}}
\put(689.0,390.0){\rule[-0.200pt]{0.400pt}{4.818pt}}
\put(110.0,82.0){\rule[-0.200pt]{0.400pt}{79.015pt}}
\put(110.0,82.0){\rule[-0.200pt]{139.481pt}{0.400pt}}
\put(689.0,82.0){\rule[-0.200pt]{0.400pt}{79.015pt}}
\put(110.0,410.0){\rule[-0.200pt]{139.481pt}{0.400pt}}
\put(110,154){\usebox{\plotpoint}}
\put(122,152.67){\rule{1.445pt}{0.400pt}}
\multiput(122.00,153.17)(3.000,-1.000){2}{\rule{0.723pt}{0.400pt}}
\put(110.0,154.0){\rule[-0.200pt]{2.891pt}{0.400pt}}
\put(151,152.67){\rule{1.445pt}{0.400pt}}
\multiput(151.00,152.17)(3.000,1.000){2}{\rule{0.723pt}{0.400pt}}
\put(128.0,153.0){\rule[-0.200pt]{5.541pt}{0.400pt}}
\put(168,153.67){\rule{1.445pt}{0.400pt}}
\multiput(168.00,153.17)(3.000,1.000){2}{\rule{0.723pt}{0.400pt}}
\put(157.0,154.0){\rule[-0.200pt]{2.650pt}{0.400pt}}
\put(180,154.67){\rule{1.445pt}{0.400pt}}
\multiput(180.00,154.17)(3.000,1.000){2}{\rule{0.723pt}{0.400pt}}
\put(174.0,155.0){\rule[-0.200pt]{1.445pt}{0.400pt}}
\put(192,155.67){\rule{1.445pt}{0.400pt}}
\multiput(192.00,155.17)(3.000,1.000){2}{\rule{0.723pt}{0.400pt}}
\put(186.0,156.0){\rule[-0.200pt]{1.445pt}{0.400pt}}
\put(204,156.67){\rule{1.204pt}{0.400pt}}
\multiput(204.00,156.17)(2.500,1.000){2}{\rule{0.602pt}{0.400pt}}
\put(209,157.67){\rule{1.445pt}{0.400pt}}
\multiput(209.00,157.17)(3.000,1.000){2}{\rule{0.723pt}{0.400pt}}
\put(215,158.67){\rule{1.445pt}{0.400pt}}
\multiput(215.00,158.17)(3.000,1.000){2}{\rule{0.723pt}{0.400pt}}
\put(221,159.67){\rule{1.445pt}{0.400pt}}
\multiput(221.00,159.17)(3.000,1.000){2}{\rule{0.723pt}{0.400pt}}
\put(227,160.67){\rule{1.445pt}{0.400pt}}
\multiput(227.00,160.17)(3.000,1.000){2}{\rule{0.723pt}{0.400pt}}
\put(233,161.67){\rule{1.445pt}{0.400pt}}
\multiput(233.00,161.17)(3.000,1.000){2}{\rule{0.723pt}{0.400pt}}
\put(239,162.67){\rule{1.445pt}{0.400pt}}
\multiput(239.00,162.17)(3.000,1.000){2}{\rule{0.723pt}{0.400pt}}
\put(245,163.67){\rule{1.204pt}{0.400pt}}
\multiput(245.00,163.17)(2.500,1.000){2}{\rule{0.602pt}{0.400pt}}
\put(250,165.17){\rule{1.300pt}{0.400pt}}
\multiput(250.00,164.17)(3.302,2.000){2}{\rule{0.650pt}{0.400pt}}
\put(256,166.67){\rule{1.445pt}{0.400pt}}
\multiput(256.00,166.17)(3.000,1.000){2}{\rule{0.723pt}{0.400pt}}
\put(262,168.17){\rule{1.300pt}{0.400pt}}
\multiput(262.00,167.17)(3.302,2.000){2}{\rule{0.650pt}{0.400pt}}
\put(268,169.67){\rule{1.445pt}{0.400pt}}
\multiput(268.00,169.17)(3.000,1.000){2}{\rule{0.723pt}{0.400pt}}
\put(274,171.17){\rule{1.300pt}{0.400pt}}
\multiput(274.00,170.17)(3.302,2.000){2}{\rule{0.650pt}{0.400pt}}
\put(280,172.67){\rule{1.204pt}{0.400pt}}
\multiput(280.00,172.17)(2.500,1.000){2}{\rule{0.602pt}{0.400pt}}
\put(285,174.17){\rule{1.300pt}{0.400pt}}
\multiput(285.00,173.17)(3.302,2.000){2}{\rule{0.650pt}{0.400pt}}
\put(291,176.17){\rule{1.300pt}{0.400pt}}
\multiput(291.00,175.17)(3.302,2.000){2}{\rule{0.650pt}{0.400pt}}
\put(297,178.17){\rule{1.300pt}{0.400pt}}
\multiput(297.00,177.17)(3.302,2.000){2}{\rule{0.650pt}{0.400pt}}
\put(303,180.17){\rule{1.300pt}{0.400pt}}
\multiput(303.00,179.17)(3.302,2.000){2}{\rule{0.650pt}{0.400pt}}
\put(309,182.17){\rule{1.300pt}{0.400pt}}
\multiput(309.00,181.17)(3.302,2.000){2}{\rule{0.650pt}{0.400pt}}
\put(315,184.17){\rule{1.300pt}{0.400pt}}
\multiput(315.00,183.17)(3.302,2.000){2}{\rule{0.650pt}{0.400pt}}
\put(321,186.17){\rule{1.100pt}{0.400pt}}
\multiput(321.00,185.17)(2.717,2.000){2}{\rule{0.550pt}{0.400pt}}
\put(326,188.17){\rule{1.300pt}{0.400pt}}
\multiput(326.00,187.17)(3.302,2.000){2}{\rule{0.650pt}{0.400pt}}
\put(332,190.17){\rule{1.300pt}{0.400pt}}
\multiput(332.00,189.17)(3.302,2.000){2}{\rule{0.650pt}{0.400pt}}
\put(338,192.17){\rule{1.300pt}{0.400pt}}
\multiput(338.00,191.17)(3.302,2.000){2}{\rule{0.650pt}{0.400pt}}
\multiput(344.00,194.61)(1.132,0.447){3}{\rule{0.900pt}{0.108pt}}
\multiput(344.00,193.17)(4.132,3.000){2}{\rule{0.450pt}{0.400pt}}
\put(350,197.17){\rule{1.300pt}{0.400pt}}
\multiput(350.00,196.17)(3.302,2.000){2}{\rule{0.650pt}{0.400pt}}
\multiput(356.00,199.61)(0.909,0.447){3}{\rule{0.767pt}{0.108pt}}
\multiput(356.00,198.17)(3.409,3.000){2}{\rule{0.383pt}{0.400pt}}
\put(361,202.17){\rule{1.300pt}{0.400pt}}
\multiput(361.00,201.17)(3.302,2.000){2}{\rule{0.650pt}{0.400pt}}
\multiput(367.00,204.61)(1.132,0.447){3}{\rule{0.900pt}{0.108pt}}
\multiput(367.00,203.17)(4.132,3.000){2}{\rule{0.450pt}{0.400pt}}
\multiput(373.00,207.61)(1.132,0.447){3}{\rule{0.900pt}{0.108pt}}
\multiput(373.00,206.17)(4.132,3.000){2}{\rule{0.450pt}{0.400pt}}
\multiput(379.00,210.61)(1.132,0.447){3}{\rule{0.900pt}{0.108pt}}
\multiput(379.00,209.17)(4.132,3.000){2}{\rule{0.450pt}{0.400pt}}
\put(385,213.17){\rule{1.300pt}{0.400pt}}
\multiput(385.00,212.17)(3.302,2.000){2}{\rule{0.650pt}{0.400pt}}
\multiput(391.00,215.61)(1.132,0.447){3}{\rule{0.900pt}{0.108pt}}
\multiput(391.00,214.17)(4.132,3.000){2}{\rule{0.450pt}{0.400pt}}
\multiput(397.00,218.61)(0.909,0.447){3}{\rule{0.767pt}{0.108pt}}
\multiput(397.00,217.17)(3.409,3.000){2}{\rule{0.383pt}{0.400pt}}
\multiput(402.00,221.61)(1.132,0.447){3}{\rule{0.900pt}{0.108pt}}
\multiput(402.00,220.17)(4.132,3.000){2}{\rule{0.450pt}{0.400pt}}
\multiput(408.00,224.61)(1.132,0.447){3}{\rule{0.900pt}{0.108pt}}
\multiput(408.00,223.17)(4.132,3.000){2}{\rule{0.450pt}{0.400pt}}
\multiput(414.00,227.60)(0.774,0.468){5}{\rule{0.700pt}{0.113pt}}
\multiput(414.00,226.17)(4.547,4.000){2}{\rule{0.350pt}{0.400pt}}
\multiput(420.00,231.61)(1.132,0.447){3}{\rule{0.900pt}{0.108pt}}
\multiput(420.00,230.17)(4.132,3.000){2}{\rule{0.450pt}{0.400pt}}
\multiput(426.00,234.61)(1.132,0.447){3}{\rule{0.900pt}{0.108pt}}
\multiput(426.00,233.17)(4.132,3.000){2}{\rule{0.450pt}{0.400pt}}
\multiput(432.00,237.60)(0.774,0.468){5}{\rule{0.700pt}{0.113pt}}
\multiput(432.00,236.17)(4.547,4.000){2}{\rule{0.350pt}{0.400pt}}
\multiput(438.00,241.61)(0.909,0.447){3}{\rule{0.767pt}{0.108pt}}
\multiput(438.00,240.17)(3.409,3.000){2}{\rule{0.383pt}{0.400pt}}
\multiput(443.00,244.60)(0.774,0.468){5}{\rule{0.700pt}{0.113pt}}
\multiput(443.00,243.17)(4.547,4.000){2}{\rule{0.350pt}{0.400pt}}
\multiput(449.00,248.61)(1.132,0.447){3}{\rule{0.900pt}{0.108pt}}
\multiput(449.00,247.17)(4.132,3.000){2}{\rule{0.450pt}{0.400pt}}
\multiput(455.00,251.60)(0.774,0.468){5}{\rule{0.700pt}{0.113pt}}
\multiput(455.00,250.17)(4.547,4.000){2}{\rule{0.350pt}{0.400pt}}
\multiput(461.00,255.61)(1.132,0.447){3}{\rule{0.900pt}{0.108pt}}
\multiput(461.00,254.17)(4.132,3.000){2}{\rule{0.450pt}{0.400pt}}
\multiput(467.00,258.60)(0.774,0.468){5}{\rule{0.700pt}{0.113pt}}
\multiput(467.00,257.17)(4.547,4.000){2}{\rule{0.350pt}{0.400pt}}
\multiput(473.00,262.60)(0.627,0.468){5}{\rule{0.600pt}{0.113pt}}
\multiput(473.00,261.17)(3.755,4.000){2}{\rule{0.300pt}{0.400pt}}
\multiput(478.00,266.60)(0.774,0.468){5}{\rule{0.700pt}{0.113pt}}
\multiput(478.00,265.17)(4.547,4.000){2}{\rule{0.350pt}{0.400pt}}
\multiput(484.00,270.60)(0.774,0.468){5}{\rule{0.700pt}{0.113pt}}
\multiput(484.00,269.17)(4.547,4.000){2}{\rule{0.350pt}{0.400pt}}
\multiput(490.00,274.60)(0.774,0.468){5}{\rule{0.700pt}{0.113pt}}
\multiput(490.00,273.17)(4.547,4.000){2}{\rule{0.350pt}{0.400pt}}
\multiput(496.00,278.60)(0.774,0.468){5}{\rule{0.700pt}{0.113pt}}
\multiput(496.00,277.17)(4.547,4.000){2}{\rule{0.350pt}{0.400pt}}
\multiput(502.00,282.60)(0.774,0.468){5}{\rule{0.700pt}{0.113pt}}
\multiput(502.00,281.17)(4.547,4.000){2}{\rule{0.350pt}{0.400pt}}
\multiput(508.00,286.59)(0.599,0.477){7}{\rule{0.580pt}{0.115pt}}
\multiput(508.00,285.17)(4.796,5.000){2}{\rule{0.290pt}{0.400pt}}
\multiput(514.00,291.60)(0.627,0.468){5}{\rule{0.600pt}{0.113pt}}
\multiput(514.00,290.17)(3.755,4.000){2}{\rule{0.300pt}{0.400pt}}
\multiput(519.00,295.60)(0.774,0.468){5}{\rule{0.700pt}{0.113pt}}
\multiput(519.00,294.17)(4.547,4.000){2}{\rule{0.350pt}{0.400pt}}
\multiput(525.00,299.59)(0.599,0.477){7}{\rule{0.580pt}{0.115pt}}
\multiput(525.00,298.17)(4.796,5.000){2}{\rule{0.290pt}{0.400pt}}
\multiput(531.00,304.60)(0.774,0.468){5}{\rule{0.700pt}{0.113pt}}
\multiput(531.00,303.17)(4.547,4.000){2}{\rule{0.350pt}{0.400pt}}
\multiput(537.00,308.59)(0.599,0.477){7}{\rule{0.580pt}{0.115pt}}
\multiput(537.00,307.17)(4.796,5.000){2}{\rule{0.290pt}{0.400pt}}
\multiput(543.00,313.59)(0.599,0.477){7}{\rule{0.580pt}{0.115pt}}
\multiput(543.00,312.17)(4.796,5.000){2}{\rule{0.290pt}{0.400pt}}
\multiput(549.00,318.60)(0.627,0.468){5}{\rule{0.600pt}{0.113pt}}
\multiput(549.00,317.17)(3.755,4.000){2}{\rule{0.300pt}{0.400pt}}
\multiput(554.00,322.59)(0.599,0.477){7}{\rule{0.580pt}{0.115pt}}
\multiput(554.00,321.17)(4.796,5.000){2}{\rule{0.290pt}{0.400pt}}
\multiput(560.00,327.59)(0.599,0.477){7}{\rule{0.580pt}{0.115pt}}
\multiput(560.00,326.17)(4.796,5.000){2}{\rule{0.290pt}{0.400pt}}
\multiput(566.00,332.59)(0.599,0.477){7}{\rule{0.580pt}{0.115pt}}
\multiput(566.00,331.17)(4.796,5.000){2}{\rule{0.290pt}{0.400pt}}
\multiput(572.00,337.59)(0.599,0.477){7}{\rule{0.580pt}{0.115pt}}
\multiput(572.00,336.17)(4.796,5.000){2}{\rule{0.290pt}{0.400pt}}
\multiput(578.00,342.59)(0.599,0.477){7}{\rule{0.580pt}{0.115pt}}
\multiput(578.00,341.17)(4.796,5.000){2}{\rule{0.290pt}{0.400pt}}
\multiput(584.00,347.59)(0.599,0.477){7}{\rule{0.580pt}{0.115pt}}
\multiput(584.00,346.17)(4.796,5.000){2}{\rule{0.290pt}{0.400pt}}
\multiput(590.00,352.59)(0.487,0.477){7}{\rule{0.500pt}{0.115pt}}
\multiput(590.00,351.17)(3.962,5.000){2}{\rule{0.250pt}{0.400pt}}
\multiput(595.00,357.59)(0.599,0.477){7}{\rule{0.580pt}{0.115pt}}
\multiput(595.00,356.17)(4.796,5.000){2}{\rule{0.290pt}{0.400pt}}
\multiput(601.00,362.59)(0.491,0.482){9}{\rule{0.500pt}{0.116pt}}
\multiput(601.00,361.17)(4.962,6.000){2}{\rule{0.250pt}{0.400pt}}
\multiput(607.00,368.59)(0.599,0.477){7}{\rule{0.580pt}{0.115pt}}
\multiput(607.00,367.17)(4.796,5.000){2}{\rule{0.290pt}{0.400pt}}
\multiput(613.00,373.59)(0.491,0.482){9}{\rule{0.500pt}{0.116pt}}
\multiput(613.00,372.17)(4.962,6.000){2}{\rule{0.250pt}{0.400pt}}
\multiput(619.00,379.59)(0.599,0.477){7}{\rule{0.580pt}{0.115pt}}
\multiput(619.00,378.17)(4.796,5.000){2}{\rule{0.290pt}{0.400pt}}
\multiput(625.00,384.59)(0.491,0.482){9}{\rule{0.500pt}{0.116pt}}
\multiput(625.00,383.17)(4.962,6.000){2}{\rule{0.250pt}{0.400pt}}
\multiput(631.00,390.59)(0.487,0.477){7}{\rule{0.500pt}{0.115pt}}
\multiput(631.00,389.17)(3.962,5.000){2}{\rule{0.250pt}{0.400pt}}
\multiput(636.00,395.59)(0.491,0.482){9}{\rule{0.500pt}{0.116pt}}
\multiput(636.00,394.17)(4.962,6.000){2}{\rule{0.250pt}{0.400pt}}
\multiput(642.00,401.59)(0.491,0.482){9}{\rule{0.500pt}{0.116pt}}
\multiput(642.00,400.17)(4.962,6.000){2}{\rule{0.250pt}{0.400pt}}
\multiput(648.00,407.61)(0.462,0.447){3}{\rule{0.500pt}{0.108pt}}
\multiput(648.00,406.17)(1.962,3.000){2}{\rule{0.250pt}{0.400pt}}
\put(198.0,157.0){\rule[-0.200pt]{1.445pt}{0.400pt}}
\put(110.0,82.0){\rule[-0.200pt]{0.400pt}{79.015pt}}
\put(110.0,82.0){\rule[-0.200pt]{139.481pt}{0.400pt}}
\put(689.0,82.0){\rule[-0.200pt]{0.400pt}{79.015pt}}
\put(110.0,410.0){\rule[-0.200pt]{139.481pt}{0.400pt}}
\end{picture}

%			\caption{La solution trouvée quand a trois serveurs dans $\mathbb{R}$.}
%		\end{figure}

%	\TODO{dah o polinomio calculado em 1..n para todos os kras... mas o valor em 0 eh justamente a potencia.}

%		\TODO{falar de Fiat-Shamir}\\
	%	\TODO{protocolo de prova de escolha ``1-out-of-2''}\\
	%	\TODO{protocolo de multiplicacao mod $n^s$}\\





