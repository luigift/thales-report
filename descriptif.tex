


\iftoggle{rapport} {
 \chapter{Introduction}
}
\textbf{La perception de l'environment par machines est indispensable pour son integration à la vie quotidienne. Compétences telle comme se localiser, prise de décisions et capacité d'apprentissage sont nécessaire pour la réalisation de les plus simples tâches. Dans cet étude, on s'intéresse à la compréhension d'éléments constituent d'une scène, sujet récurrent dans le domaine de la vision par ordinateur. Plusieurs approches proposés dans la littérature explorent une sous-partie du pipiline de la réconnaissance pour faire face au difficile défi de représenter la forme visuelle des objets, pendent qu'autres s'intéresses à l'utilisation du système de reconnaissance pour tâches de recherche dans l'environment, *graspping*, par exemple.  Notre démarche correspond, initialement, à l'intégration de techniques de l'état de l'art pour arriver à un système fonctionnel de reconnaissance incorporé sur une plateforme mobile équipée d'un capteur RGB-D capable. Dans une première temps, la plateforme est capable d'acquérir une base de données d'images d'un objet et, ensuite, d'utiliser ces informations apprises pour vérifier si un objet candidat est ou non celui présenté auparavant, et, finalement, renforcer sa perception avec les information de son déplacement pour lever les possibles ambiguités.} 

\iftoggle{rapport} {
  \section{Thématiques du stage.} 
} {}

\begin{center}
    Vision par ordinateur - Reconnaissance d'objets multi-vue - Robotique
\end{center}
