%\begin{abstract}\centering % Colocar somente 4 linhas aqui depois....
% il s'agit d'étudier un système de chiffrement à clef
%publique inventé par Pascal Paillier en 1999. Un premier volet du
%travail consistera en une revue des propriétés des anneaux modulaires
%nécessaires à la compréhension du mécanisme de
%chiffrement/déchiffrement. Ensuite il s'agira de comprendre sur quels
%problèmes calculatoirement difficiles repose ce cryptosystème. Des
%notions de théorie de la complexité pourront être appréhendées à cette
%occasion. Une troisième partie sera dédiée à la mise en oeuvre
%informatique de ce système cryptographique. Pour cela,
%l'étudiant utilisera une bibliothèque de manipulation de grands
%entiers. Les performances du système seront alors évaluées. 
%Le stage s'ouvrira ensuite vers une direction plus large qui est
%l'investigation des systèmes de chiffrement ayant des propriétés
%homomorphiques, dont le système de Paillier fait partie, et
%l'application de ce type de fonctionnalité à la résolution de 
%problématiques de la vie courante (ex. calcul sur des serveurs
%distants, vote électronique). 
%Dans ce cadre, des variantes du système de Paillier seront considérées.
%\end{abstract}
\vspace*{\stretch{1}}
\thispagestyle{empty}
\selectlanguage{french}%
\begin{abstract}\centering
\noindent Dans ce projet il a été étudié le système de chiffrement à clé
publique inventé par Pascal Paillier en 1999. Ici sont discutées les principales
caractéristiques du point de vue mathématique et cryptographique qui 
régissent le système. 
Ensuite il est exposé une analyse du système 
étendu proposé par Ivan Damgård et Mads Jurik. Les deux cryptosystèmes
ont été implantés en \verb!Python! et un test de performance a été fait pour 
mesurer empiriquement leur fonctionnement.

\end{abstract}

{\bf Mots-clés:} Cryptographie à Cle Publique, Système de Paillier  , Système de Damgard-Jurik, Résiduosité, Chiffrement Homomorphe.
%Paillier, Damgård-Jurik, Homomorphisme, clé publique, ré\-si\-duo\-sité.
\selectlanguage{english}%
\begin{abstract}\centering
\noindent This project aims at the study of the public key cryptosystem 
invented by Pascal Paillier in 1999. On this report both the
mathematical and cryptographic characteristics relatives to the
system will be discussed 
and an analysis of the extended system proposed by Ivan Damgård and Mads Jurik
will be conducted.
Both systems were implemented in \verb!Python! and a benchmark was realized 
in order to empirically evaluate both systems and their optimizations.
\end{abstract}

{\bf Keywords:} Public Key Cryptography     , Paillier Cryptosystem, Damgard-Jurik           , Residuosity, Homomorphic Encryption.
%Paillier, Damgård-Jurik, Homomorphic, public key, re\-si\-duo\-si\-ty.
\selectlanguage{french}%
\vspace*{\stretch{1}}

%%
%Cryptographie à Cle Publique, Système de Paillier  , Système de Damgard-Jurik, Résiduosité, Chiffrement Homomorphe.
%Public Key Cryptography     , Paillier Cryptosystem, Damgard-Jurik           , Residuosity, Homomorphic Encryption.
