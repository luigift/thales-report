\section{Autres features}


\subsection{Features 2-Dimensions}
    \begin{itemize}
         \item 
    \end{itemize}



\subsection{Features 3-Dimensions}
Our stereo data is noisier and sparser than typical line scan data which
motivated the use of our new features

    \begin{itemize}
        \item Spherical harmonic invariants [5] : Spherical harmonic invariants and spin images have been
          successfully used for the problem of object recognition for
          densely sampled datasets, though their performance seems
          to degrade for noisier and sparser datasets

        \item Spin images [6], 
        \item Curvature maps [7] :  Conformal factors are
          based on conformal geometry, which is invariant to isometric
          transformations, and thus obtains good results on databases
          of watertight models. Its main drawback is that it can only
          be applied to manifold meshes which can be problematic
          in stereoXS
        \item Point Feature Histograms (PFH) [8]
        \item Conformal factors [9]
    \end{itemize}

[1] Radu Bogdan Rusu, Gary Bradski, Romain Thibaux, John Hsu,
Fast 3D Recognition and Pose Using the Viewpoint Feature Histogram

[5] G. Burel and H. Henocq, “Three-dimensional invariants and their ´
application to object recognition,” Signal Process., vol. 45, no. 1, pp.
1–22, 1995.
[6] A. Johnson and M. Hebert, “Using spin images for efficient object
recognition in cluttered 3D scenes,” IEEE Transactions on Pattern
Analysis and Machine Intelligence, May 1999.
[7] T. Gatzke, C. Grimm, M. Garland, and S. Zelinka, “Curvature Maps
for Local Shape Comparison,” in SMI ’05: Proceedings of the International
Conference on Shape Modeling and Applications 2005 (SMI’
05), 2005, pp. 246–255.
[8] R. B. Rusu, N. Blodow, and M. Beetz, “Fast Point Feature Histograms
(FPFH) for 3D Registration,” in ICRA, 2009.
[9] B.-C. M. and G. C., “Characterizing shape using conformal factors,”
in Eurographics Workshop on 3D Object Retrieval, 2008.
