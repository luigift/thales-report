\chapter{Chiffrement Homomorphe}

Le système de Paillier muni de la Propriété \ref{prop:mult} permet l'addition des données chiffrées,
cette caractéristique est souhaitable dans divers domaines de l'informatique, tels quels les systèmes
de vote électronique, réseaux de capteurs et tous qui est calcul dans la nuage. Par contre, 
l'addition n'est pas suffisante pour que les opérations - et par conséquent, les applications - puissent
manipuler les anneaux sans exposer les mots clairs, c'est-à-dire faire la multiplication 
et addition des données chiffrés sans avoir la clé privé.
%composer un système Turing-complet, et alors, on se pose dans un environnement assez limité.


\section{Les schémas homomorphes}
%Comme dit avant, un schéma cryptographique homomorphe possède un morphisme entre 
%le groupe algébri\-que des données clairs et le groupe algé\-bri\-que des mots chiffrés. Ou plus précisément:


Les schémas cryptographiques déjà décrits possèdent un morphisme entre le groupe algébrique 
des données claires et le groupe algébrique des mots chiffrés. Ils sont dits Homomorphes.
Une définition plus formelle des schémas homomorphes:

Un schéma cryptographique asymétrique $\mathcal{E}$ possède trois algorithmes principales 
\verb!KeyGen!$_\mathcal{E}$, \verb!Encript!$_\mathcal{E}$ et \verb!Decript!$_\mathcal{E}$, pour définir l'homomorphisme on introduit l'algorithme
\verb!Evaluate!$_\mathcal{E}$ qui est capable de exécuter un ensemble d'instructions avec des données chiffrées 
(On note $\mathcal{C}_\mathcal{E}$ l'ensemble de circuits 
logiques qui peuvent être traités par le schéma).

Un schéma homomorphe est dit {\bf correcte} si l'exécution de l'ensemble d'instructions avec \verb!Evaluate!$_\mathcal{E}$ sur les 
mots chiffrés et l'exécution directe sur les mots clairs produit la même valeur pour chaque combinaison vraie 
entre mots chiffrés et mots clairs. Un schéma homomorphe est dit {\bf compacte} si l'algorithme \verb!Decript!$_\mathcal{E}$ 
peut être décrit avec un circuit appartient a $\mathcal{C}_\mathcal{E}$ avec un nombre polynomial des portes logiques.
On dit qu'un schéma {\bf évalue de façon compacte} un ensemble de circuits $\mathcal{C}'$ si il est compacte et 
correcte pour cet ensemble. Finalement, on dit qu'un schéma est {\bf complètement homomorphe} si il évalue de façon 
compacte n'importe quel $\mathcal{C}'$. La complétude fonctionnelle peut être attendue avec les plus petites ensembles 
d'opérations (voir bibliographie \cite{wernick1942complete}):

\begin{table}[H]
\caption{Plus petits ensembles d'opérations qui fournissent la complétude fonctionnelle.}
\label{table:operations}
\center
\begin{tabular}{|c|m{6cm}|}
\hline
Nombre d'opérations & Plus petit ensemble \\
\hline
\hline
	1	& \ $\{$\verb!NON-ET!$\}$,$\{$\verb!NON-OU!$\}$ \\
		\hline
	2	& $\{\vee,\neg\}$, %
		$\{\wedge,\neg\}$, %
		$\{\rightarrow,\neg\}$, %
		$\{\leftarrow,\neg\}$, %
		$\{\rightarrow,\bot\}$, %
		$\{\leftarrow,\bot\}$, %
		$\{\leftarrow,\not\leftrightarrow\}$, %
		$\{\rightarrow,\not\leftrightarrow\}$, %
		$\{\rightarrow,\not\rightarrow\}$, %
		$\{\rightarrow,\not\leftarrow\}$, %
		$\{\leftarrow,\not\rightarrow\}$, %
		$\{\leftarrow,\not\leftarrow\}$, %
		$\{\not\rightarrow,\neg\}$, %
		$\{\not\leftarrow,\neg\}$, %
		$\{\not\rightarrow,\top\}$, %
		$\{\not\leftarrow,\top\}$, %
		$\{\not\leftarrow,\top\}$, %
		$\{\not\rightarrow,\leftrightarrow\}$, %
		$\{\not\leftarrow,\leftrightarrow\}$ \\
			\hline
	3 	& $\{\lor,\leftrightarrow,\bot\}$,%
		$\{\lor,\leftrightarrow,\not\leftrightarrow\}$,%
		$\{\lor,\not\leftrightarrow,\top\}$,%
		$\{\land,\leftrightarrow, \bot\}$,%
		$\{\land,\leftrightarrow,\not\leftrightarrow\}$,%
		$\{\land,\not\leftrightarrow,\top\}$\\
		\hline
\end{tabular}
\end{table}

Pour une définition plus formelle, voir la bibliographie \cite{homenc}.


Le système de Paillier est homomorphe pour $\mathcal{C}_\mathcal{E}= \{+\}$, par contre, il n'exis\-tent pas 
d'algorithmes linéaires (circuits) connus qui puissent implanter $+$ et $\cdot$ sans exposer la clé secrète ou 
le mot clair (une addition des entiers est l'opé\-ra\-tion $\oplus$ dans le bit de l'unité, il manque $\wedge$  
pour la complétude), et alors le système ne peut pas être appelé compacte.


%\begin{definition} {\bf Système Homomorphe:} Soient $\mathcal{E}: M \to C$ une fonction de chiffrement 
%et $\mathcal{D}: C \to M$ une fonction de déchiffrement, un cryptosystème est dit homomorphe 
%si $M$ et $C$ sont munis des opérations arithmétiques $\oplus$ et $\otimes$ respectivement tels que, pour $\forall x_1,x_2 \in M $ 
%et $\forall y_1,y_2 \in C$:
%	$$ \mathcal{E}(x_1\oplus x_2) = \mathcal{E}(x_1)\otimes \mathcal{E}(x_2) \text{ et } \mathcal{D}(y_1\otimes y_2) = \mathcal{D}(y_1)\oplus \mathcal{D}(y_2) $$
%\end{definition}
%
%\TODO{Pour une définition plus }

Si un système cryptographique est correcte mais n'est pas prouvé d'être 
complètement homomorphe, il est dit {\bf semi-homomorphe}. Il existent divers systèmes semi-homomorphes:

\begin{itemize}
	\item Paillier/Damgård–Jurik
	\item RSA
	\item Benaloh
	\item Boneh–Goh–Nissim
	\item Okamoto–Uchiyama
	\item Naccache–Stern
	\item ElGamal
	\item Goldwasser-Micali
\end{itemize}

\section{Bootstrap}

Une façon de manipuler les données sans restriction est les déchiffrer, calculer c'est qu'on veut et 
chiffrer une autre fois, ce chemin clairement expose les données chiffrées, alors il faut que
chaque partie de cet algorithme se passe dans le domaine des données chiffrées. Le {\it bootstrap} est la 
capacité de décrire chaque élément d'un des plus petits ensembles d'opérations (voir Table \ref{table:operations}) 
et un algorithme de ``réchiffrement'' avec $\mathcal{C}_\mathcal{E}$,
l'algorithme résultant appartient à $\mathcal{C}_\mathcal{E}$ et peut être traité avec \verb!Evaluate!$_\mathcal{E}$.
Un tel schéma est dit {\it bootstrapable}, un système {\it bootstrapable} est complètement homomorphe (voir \cite{homenc}).

Le {\it Bootstrap} est crucial dans les systèmes basées sur les réseaux. Comme le système introduit par Craig Gentry 
en 2009, dont la fonction \verb!Evaluate!$_\mathcal{E}$ ajoute du bruit au résultat de chaque opération. 
Dans le système de Gentry une séquence d'opérations augmente le bruit en fonction de leur taille. Même si le ré\-chif\-fre\-ment corrige le bruit,
il faut considérer que ors cet algorithme est accompli avec \verb!Evaluate!$_\mathcal{E}$, il lui même ajoute  du  bruit. Résultat: Le bruit
produit par la fonction de ré\-chif\-fre\-ment plus le bruit qui est ajouté par l'opération effectuée ne peut pas 
dépasser leurs limites de correction.

\section{Le système homomorphe de Gentry}

Le système de Gentry avant le {\it Bootstrapping} est un système symétrique, à savoir un système dont il existe une seule clé, 
un tel système est aussi dit ``à clé privée''. Un exemple connu de système à clé privée
est l'algorithme \verb!AES!, grâce à ça performance il est utilisé largement dans touts les domaines. 

	\begin{algo}{\verb!Keygen!$_\mathcal{E}$}
		\begin{itemize}\renewcommand{\labelitemi}{} \renewcommand{\labelitemii}{$\cdot$}
			\item{\bf Entrée:} 
				\begin{itemize}
					\item $\Lambda$: Le paramètre de sécurité.
				\end{itemize}
			\item{\bf Calcul:}
				\begin{enumerate} %			
				\renewcommand{\theenumi}{\arabic{enumi}}
				\renewcommand{\theenumii}{\arabic{enumii}}
				\renewcommand{\theenumiii}{\arabic{enumiii}}
				\renewcommand{\labelenumi}{\theenumi.}
				\renewcommand{\labelenumii}{\theenumi.\theenumii.}
				\renewcommand{\labelenumiii}{\theenumi.\theenumii.\theenumiii.}
				\makeatletter
				\renewcommand{\p@enumii}{\theenumi.}
				\renewcommand{\p@enumiii}{\theenumi.\theenumii.}
				\makeatother
					\item Calculer $N=\Lambda$, $P=\Lambda^2$, $Q=\Lambda^5$.
					\item Trouver $p$ un nombre impair aléatoire de taille $N$ bits.	
				\end{enumerate}
	%		\item{\bf Transmettre:}
	%			\begin{enumerate}[]
	%			\item $p$.
	%			\end{enumerate}

		\end{itemize}
	\end{algo}

%keygen(lambda):
%	N=lambda
%	P=lambda*lambda
%	Q=P*P*N
%	p = random(N)

	\begin{algo}{\verb!Encrypt!$_\mathcal{E}$}
		\begin{itemize}\renewcommand{\labelitemi}{} \renewcommand{\labelitemii}{$\cdot$}
			\item{\bf Entrée:} 
				\begin{itemize}
					\item $x\in{0,1}$.
					\item $pk$: la clé publique.
				\end{itemize}
			\item{\bf Calcul:}
				\begin{enumerate} %			
				\renewcommand{\theenumi}{\arabic{enumi}}
				\renewcommand{\theenumii}{\arabic{enumii}}
				\renewcommand{\theenumiii}{\arabic{enumiii}}
				\renewcommand{\labelenumi}{\theenumi.}
				\renewcommand{\labelenumii}{\theenumi.\theenumii.}
				\renewcommand{\labelenumiii}{\theenumi.\theenumii.\theenumiii.}
				\makeatletter
				\renewcommand{\p@enumii}{\theenumi.}
				\renewcommand{\p@enumiii}{\theenumi.\theenumii.}
				\makeatother
					\item Calculer $m'$ un nombre aléatoire de parité $x$ et de taille $N$ bits, et $q$ un nombre de taille $Q$ en bits.
					\item Calculer $c = m'+p\cdot q$.
					\item Retourner $c$, le mot chiffré. 
				\end{enumerate}
			\end{itemize}
	\end{algo}
%encrypt(x):
%	m'= 2*random(N-1)+x
%	q = random(Q)
%	c = m'+pq
%	return c

	\begin{algo}{\verb!Decrypt!$_\mathcal{E}$}
		\begin{itemize}\renewcommand{\labelitemi}{} \renewcommand{\labelitemii}{$\cdot$}
			\item{\bf Entrée:} 
				\begin{itemize}
					\item $c\in \mathbb{Z}$.
					\item $pk$: la clé publique.
				\end{itemize}
			\item{\bf Calcul:}
				\begin{enumerate} %			
				\renewcommand{\theenumi}{\arabic{enumi}}
				\renewcommand{\theenumii}{\arabic{enumii}}
				\renewcommand{\theenumiii}{\arabic{enumiii}}
				\renewcommand{\labelenumi}{\theenumi.}
				\renewcommand{\labelenumii}{\theenumi.\theenumii.}
				\renewcommand{\labelenumiii}{\theenumi.\theenumii.\theenumiii.}
				\makeatletter
				\renewcommand{\p@enumii}{\theenumi.}
				\renewcommand{\p@enumiii}{\theenumi.\theenumii.}
				\makeatother
					\item Retourner $m = (c\bmod{p})\mod{2}$. $c\in [-\frac{p-1}{2},\frac{p+1}{2}]$. 
				\end{enumerate}
	%		\item{\bf Transmettre:}
	%			\begin{enumerate}[]
	%			\item $c$.
	%			\end{enumerate}
		\end{itemize}
	\end{algo}


%decrypt(c):
%	return (c mod p) mod 2

	\begin{algo}{\verb!Evaluate!$_\mathcal{E}$}
		\begin{itemize}\renewcommand{\labelitemi}{} \renewcommand{\labelitemii}{$\cdot$}
			\item{\bf Entrée:} 
				\begin{itemize}
					\item $f$: une suite d'opérations $XOR$ et $AND$.
					\item $\{c_1,\dots,c_t\}$: les arguments de $f$
				\end{itemize}
			\item{\bf Calcul:}
				\begin{enumerate} %			
				\renewcommand{\theenumi}{\arabic{enumi}}
				\renewcommand{\theenumii}{\arabic{enumii}}
				\renewcommand{\theenumiii}{\arabic{enumiii}}
				\renewcommand{\labelenumi}{\theenumi.}
				\renewcommand{\labelenumii}{\theenumi.\theenumii.}
				\renewcommand{\labelenumiii}{\theenumi.\theenumii.\theenumiii.}
				\makeatletter
				\renewcommand{\p@enumii}{\theenumi.}
				\renewcommand{\p@enumiii}{\theenumi.\theenumii.}
				\makeatother
					\item Substituer les opérations $XOR$ par $\cdot$ et $AND$ par $+$.
					\item Calculer la fonction dans avec les arguments $\{c_1,\dots,c_t\}$
%					\item calculer $m = (c\bmod{p})\mod{2}$
				\end{enumerate}
	%		\item{\bf Transmettre:}
	%			\begin{enumerate}[]
	%			\item $c$.
	%			\end{enumerate}
		\end{itemize}
	\end{algo}

%evaluate(f,i1,i2,i3,..):
%	f' = replace(f,(and,or),(+,*))
Le {\it Bootstrapping} du système prendre en considération la difficulté de évaluer $c\bmod p$ dans le domaine 
chiffré en remplaçant le déchiffrement par le calcul de $\text{\verb!LSB!}(c)~XOR~\text{\verb!LSB!}(\lfloor c/p\rceil)$ et un vecteur
de valeurs sommant $1/p$ pour effectuer rapidement le calcul de $c/p$. La description avec {\it Bootstrap} est disponible sur \cite{homenc}.
%\section{Homomorphisme}




