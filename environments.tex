\newtheorem{theorem}{Théorème}
\newtheorem{lemma}{Lemme}
\newtheorem{remark}{Remarque}
\newtheorem{definition}{Définition}
\newtheorem{property}{Propriété}
\newtheorem{problem}{Problème}


%\newenvironment{rappel}%
%{%
%\begin{framed}\textbf{Rappel:}%
%}
%{%
%\end{framed}%
%}%

\newsavebox{\fmbox}
\newenvironment{rappel}
{\begin{center}\begin{lrbox}{\fmbox}\begin{minipage}{0.7\textwidth}\textbf{Rappel :}}
{{\vskip 1mm \vskip 0cm}\end{minipage}\end{lrbox}\fbox{\usebox{\fmbox}}\end{center}}

\newenvironment{algo}[1]
{\begin{center}\begin{lrbox}{\fmbox}\begin{minipage}{0.9\textwidth}\textbf{#1 :}}
{{\vskip 1mm \vskip 0cm}\end{minipage}\end{lrbox}\fbox{\usebox{\fmbox}}\end{center}}


%\newenvironment{algo}[1]
%{\begin{minipage}{0.9\textwidth}\textbf{#1 :}}
%{\end{minipage}\\}
%\newenvironment{rappel}{\begin{center}
%\fbox{
%\begin{minipage}{0.7\textwidth}
%	\textbf{Rappel:}}
%{\end{minipage}
%}
%\end{center}}
%   before}{  after}
%\newenvironment{proof}[1][Proof]{\begin{trivlist}
%\item[\hskip \labelsep {\bfseries #1}]}{\end{trivlist}}
%\newenvironment{definition}[1][Definition]{\begin{trivlist}
%\item[\hskip \labelsep {\bfseries #1}]}{\end{trivlist}}
%\newenvironment{example}[1][Example]{\begin{trivlist}
%\item[\hskip \labelsep {\bfseries #1}]}{\end{trivlist}}
%\newenvironment{remark}[1][Remark]{\begin{trivlist}
%\item[\hskip \labelsep {\bfseries #1}]}{\end{trivlist}}

%\newcommand{\qed}{\nobreak \ifvmode \relax \else
%      \ifdim\lastskip<1.5em \hskip-\lastskip
%      \hskip1.5em plus0em minus0.5em \fi \nobreak
%      \vrule height0.75em width0.5em depth0.25em\fi}

\newcommand{\Mgrinv}[2]{\ensuremath{\mathbb{#1}/#2\mathbb{#1}^*}}
\newcommand{\Mgr}[2]{\ensuremath{\mathbb{#1}/#2\mathbb{#1}}}
\newcommand{\benolah}[1]{\ensuremath{\llbracket#1\rrbracket}}

\newcommand{\PC}[2]{\emph{#1}[\ensuremath{#2}]}

\newcommand{\HRule}{\rule{\linewidth}{0.5mm}}

\DeclareMathOperator{\pgcd}{pgcd}
\DeclareMathOperator{\ppmc}{ppmc}

\floatname{algorithm}{Algorithme}

\definecolor{gray}{gray}{0.3}
\definecolor{blue}{rgb}{0,0,0.5}
\definecolor{green}{rgb}{0,0.5,0}
\definecolor{lightgreen}{rgb}{0,0.6,0}
\definecolor{purple}{rgb}{0.5,0,0.5}
\definecolor{darkred}{rgb}{0.5,0,0}
\definecolor{dkgreen}{rgb}{0,0.6,0}
\definecolor{mauve}{rgb}{0.58,0,0.82}
\definecolor{orange}{rgb}{1,0.5,0}

\renewcommand{\lstlistingname}{Code}

\newcommand{\TODO}[1]{ \textcolor{red}{\large TODO:} \textcolor{purple}{[#1]~}}
\lstset{ %
  language=python,                % the language of the code
  basicstyle=\footnotesize\ttfamily,           % the size of the fonts that are used for the code
  numbers=left,                   % where to put the line-numbers
  numberstyle=\tiny\color{gray},  % the style that is used for the line-numbers
  stepnumber=1,                   % the step between two line-numbers. If it's 1, each line 
                                  % will be numbered
  numbersep=5pt,                  % how far the line-numbers are from the code
  backgroundcolor=\color{white},      % choose the background color. You must add \usepackage{color}
  showspaces=false,               % show spaces adding particular underscores
  showstringspaces=false,         % underline spaces within strings
  showtabs=false,                 % show tabs within strings adding particular underscores
  frame=single,                   % adds a frame around the code
  rulecolor=\color{black},        % if not set, the frame-color may be changed on line-breaks within not-black text (e.g. commens (green here))
  tabsize=2,                      % sets default tabsize to 2 spaces
  captionpos=b,                   % sets the caption-position to bottom
  breaklines=true,                % sets automatic line breaking
  breakatwhitespace=false,        % sets if automatic breaks should only happen at whitespace
  title=\lstname,                   % show the filename of files included with \lstinputlisting;
                                  % also try caption instead of title
%
	emph={access,and,as,break,class,continue,def,del,elif,else,%
	except,exec,finally,for,from,global,if,import,in,is,%
	lambda,not,or,pass,print,raise,return,try,while,assert},
	emphstyle=\color{orange}\bfseries,
	emph={[2]self},
	emphstyle=[2]\color{blue},
	emph={[4]ArithmeticError,AssertionError,AttributeError,BaseException,%
	DeprecationWarning,EOFError,Ellipsis,EnvironmentError,Exception,%
	False,FloatingPointError,FutureWarning,GeneratorExit,IOError,%
	ImportError,ImportWarning,IndentationError,IndexError,KeyError,%
	KeyboardInterrupt,LookupError,MemoryError,NameError,None,%
	NotImplemented,NotImplementedError,OSError,OverflowError,%
	PendingDeprecationWarning,ReferenceError,RuntimeError,RuntimeWarning,%
	StandardError,StopIteration,SyntaxError,SyntaxWarning,SystemError,%
	SystemExit,TabError,True,TypeError,UnboundLocalError,UnicodeDecodeError,%
	UnicodeEncodeError,UnicodeError,UnicodeTranslateError,UnicodeWarning,%
	UserWarning,ValueError,Warning,ZeroDivisionError,abs,all,any,apply,%
	basestring,bool,buffer,callable,chr,classmethod,cmp,coerce,compile,%
	complex,copyright,credits,delattr,dict,dir,divmod,enumerate,eval,%
	execfile,exit,file,filter,float,frozenset,getattr,globals,hasattr,%
	hash,help,hex,id,input,int,intern,isinstance,issubclass,iter,len,%
	license,list,locals,long,map,max,min,object,oct,open,ord,pow,property,%
	quit,range,raw_input,reduce,reload,repr,reversed,round,set,setattr,%
	slice,sorted,staticmethod,str,sum,super,tuple,type,unichr,unicode,%
	vars,xrange,zip},
	emphstyle=[4]\color{purple}\bfseries,					  
%
%
  keywordstyle=\color{blue},          % keyword style
  commentstyle=\color{dkgreen},       % comment style
  stringstyle=\color{mauve},         % string literal style
  escapeinside={\%*}{*)},            % if you want to add a comment within your code
  morekeywords={self,*,...}               % if you want to add more keywords to the set
}
