
	\section{Les clés RSA}
%	\TODO{Revisar o francês!}
	\label{section:rsa}
	Depuis 2009, le standard \textit{FIPS 186-3} sert de référence pour les implantations des 
	protocoles \textit{RSA} et \textit{DSA}. Dans la page $52$ du document 
	 sont définies les contraintes pour la génération d'une paire \textit{RSA} $(p,q)$:
	\begin{quotation}
	$$\vdots$$
	\begin{enumerate}
	\setcounter{enumi}{1}
		\item The primes $p$ and $q$ \textbf{shall} be selected with the following cons\-train\-ts: 
			\begin{enumerate}
				\item $(p–1)$ and $(q–1)$ \textbf{shall} be relatively prime to the public exponent $e$. 
				\item The private prime factor $p$ \textbf{shall} be selected randomly and \textbf{shall} satisfy 
					$\sqrt{2}(2^{\frac{nlen}{2}-1}) \leq p\leq (2^{\frac{nlen}{2}}– 1)$, where $nlen$
					is the appropriate length for the desired \\$security\_strength$. 
				\item The private prime factor $q$ \textbf{shall} be selected randomly and \textbf{shall} satisfy 
					$\sqrt{2}(2^{\frac{nlen}{2}-1})\leq q\leq (2^{\frac{nlen}{2}}– 1)$, where $nlen$
					is the appropriate length for the desired \\$security\_strength$.
				\item $|p – q| > 2^{\frac{nlen}{2}} – 100$.
			\end{enumerate}
	\end{enumerate}
	$$ \vdots $$

		\hfill \em National Institute of Standards and Technology.\cite{Gallagher09fipspub}
	\end{quotation}

	Les items 2.b et 2.c permettent d'affirmer que $|p| = |q|$ en bits.
	Même si le choix entre $p$ et $q$ n'a pas été exprimé dans la description du système de Paillier, il est dit que 
	le choix doit respecter les recommandations usuelles, en particulier 2.b et 2.c garantissent la relation $p\not\equiv 1\mod{q}$, 
	et alors, le cardinal de $\mathbb{E}_n$ est $1$ et on a l'isomorphisme de $\mathcal{E}_g$ désiré pour $g \in \mathcal{B}$. 
	
	\begin{property}{Si $p$ et $q$ ont la même taille en bits, alors, $p \not\equiv 1\mod{q}$.}
		\begin{proof}
			On suppose que $p \equiv1\mod{q}$:
			\begin{align*}
				p&\equiv 1 \mod{q} \\
				&=1+qA, A\in\mathbb{Z}
			\end{align*}
			Comme $p$ et $q$ ont la même taille, $A$ doit être moins grand que $2$ ($A = 1$):
			$$p = 1+q$$
			On sait que, sauf 2 et 3, il n'existe pas deux premiers consécutifs. Alors, par l'absurde, 
			si $p$ et $q$ ont la même taille en bits, $p\not\equiv1\mod{q}$.
		\end{proof}
	\end{property}



