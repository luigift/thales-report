\section {Introduction}

L'acquisition d'images en different points de vue, augment la capacité de résoudre situations d'ambiguité et, donc, la capacité de reconnaissance. Un modèle qui calcule les probabilités conditionnelles à partir de celles données par un classificateur résoudre le problème. ( Transition equiprobable entre vues d'un même objet.) Cependant, incorporer les informations de déplacement du robot, elimine certaines hyphotèses et rendre l'ensembe du système de reconnaissance encore plus puissante.

\section {Chaînes de Markov Chachées}


\subsection{Estimation de l'odometrie}

Certains robots sont dotés de capteurs capables d'estimer de façon approximé sont déplacement. Dans le cadre de ce stage, le robot ciblé possède encodeurs capables d'estimer la rotation angulaire des roues. Une integration de cette information informe la position finale du robot. Cette position est dopté d'une incertitude. 
Pourtant, seulement un information du sense de déplacement vire des hyphotèses ambigous. *En gros*, la fiabilité de cet estimation joue un rolê important dans son *integration* au modèle markovien.



\subsection{Fusion de données }

L'estimation de l'odometries diverge au long du temps *devido* a l'acumulation d'erreurs mesure. Cette divergence est encore plus considerable dans notre cas à cause de la roue derrière qui produit de *devis* de trajectoire en certains mouvements. Dans l'autre côté, l'utilisation du senseur RGB-D estime la distance au centride de chaque objet. Une correspondace entre les objets de deux observations consecutives nous donné une autre répère de positionement. Un couplage des deux mesure, une provenant du encodeur moteur et l'autre du capteur infrarouge, fait que l'odometrie doive êtrte
