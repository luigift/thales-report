\section{Pipeline}

    \subsection{Segmentation}

        \subsubsection{Motivation}
            Simplifier l'étape de extraction de descripteurs

        \subsubsection{Algorithme}
            RANSAC : estimation de plans
            Average_3D : estimation de normales
            Enveloppe convexe et distance euclidienne : cluster d'objets
            
        \subsubsection{Restrictions}
            Tous les clusters segmentés sont considérés comme des objets. 
            \begin{itemize}
                \item Objets invisibles au infra-rouge (transaparents ou réflectives)
                \item Distance entre cluster et le sol 
                \item Distance entre deux points pour être considérés comme clusteurs differents 
                \item Objets de petit taille et/ou complexes 

        \subsubsection{Proposition d'Améilloration}
            ( Sans suggestion, besoin de recherche )

    \subsection {Extraction de Descripteurs}

        \subsubsection{Motivation}
            Réduction de la dimensionalité et exploration de l'invariance à transformées physiques

        \subsubsection{Algorithme}
            PFH  : Base construite, Publisher Online 
            VFH  : ---------------, Publisher Online 
            CVFH : ---------------, ----------------

        \subsubsection{Restrictions}
            Fiabilité dépend de la estimation de normales (raisonable)

        \subsubsection{Proposition d'Améilloration}
            Combinaison de features 

    \subsection {Classification}

        \subsubsection{Motivation}
            Comparaison entre une image de teste et une base de données

        \subsubsection{Algorithme}
            K-Nearest Neighbors : Construction de l'arbre, recherche de voisins
            Comparaison d'histogram Chi-squared

        \subsubsection{Restrictions}
            Fiabilité dépend Metrique de comparaison 
        
        \subsubsection{Proposition d'Améilloration}
            Possibilité d'apprentissage de distances
            Utilization d'une classificateur 'complet' (e.g. Overfeat)


    \subsection {Classification Multi-vue}

        \subsubsection{Motivation}
            Augmenter mes confiance d'après une ensemble d'observations

        \subsubsection{Algorithme}
            Chaînes de Markov cachées 

        \subsubsection{Restrictions}
            Matching entre objets dans la scène
        
        \subsubsection{Proposition d'Améilloration}
            Rédefinition de la matrice de transition pour prendre en compte le déplacement du robot
            Utilization d'un autre modèle de suivi : Filtre Bayesien
            Multi-objet multi-vue recognition (MOMVR)

    \subsection {Déplacement}

        \subsubsection{Motivation}
            Capacité d'acquirir de nouvelles points de vue

        \subsubsection{Algorithme}
            Aucun pour l'instant

        \subsubsection{Restrictions}
            Odometrie problematique
        
        \subsubsection{Proposition d'Améilloration}
            Avoid & Planification de trajectoires
            SLAM (odometrie)
            Optimization MOMVR (compromis entre déplacement et reconnaissance)
            
